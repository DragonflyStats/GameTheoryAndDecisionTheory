Cournot Duopoly with Heterogeneous items: Linear Demand and Linear
Costs
Let x1 and x2 be the quantities of heterogeneous or non-identical items produced by two
firms with associated costs C1(x1) = c1x1 and C2(x2) = c2x2 respectively.
Firm 1’s items sell at P1 = a1 − b11x1 − b12x2 each, firm 2’s at P2 = a2 − b21x1 − b22x2
each and it is assumed that all items produced are sold.
The profits made by the firms are then
π1 = P1x1 − c1x1 = (a1 − c1 − b11x1 − b12x2) x1
π2 = P2x2 − c2x2 = (a2 − c2 − b21x1 − b22x2) x2
respectively.
We’ll proceed with a specific example: P1 = 10 −
x1+x2
1000 , P2 = 9 −
x1+x2
1250 , C1(x1) =
2.5x1, C2(x2) = 2x2. The profits made by the firms are then
π1 = P1x1 − c1x1 =

7.5 −
x1 + x2
1000 
x1
π2 = P2x2 − c2x2 =

7 −
x1 + x2
1250 
x2
respectively.
Maximising π1 with respect to x1
∂π1
∂x1
= 7.5 −
2
1000
x1 −
1
1000
x2
set = 0
⇒ x1 = 3750 −
1
2
x2 (1)
Similarly maximising π2 with respect to x2 yields
x2 = 4375 −
1
2
x1 (2)
We’ll assume that Equations 1 and 2 give nonnegative values for x1 and x2 and so represent
Reaction Functions. Solving equations 1 and 2 simultaneously gives the equilibrium values
x
∗
1 = 2083.33
x
∗
2 = 3333.33
At these equilibrium values
P
∗
1 = 4.5833, P∗
2 = 4.67
π
∗
1 = 4340.28, π∗
2 = 8888.89
Stackelberg Duopoly
We’ll consider a Stackelberg duopoly in which Firm 1 is the Leader and Firm 2 is the
Follower. Irrespective of what the leader does, the follower will use the reaction function
(Eq. 2) as its best response.
Knowing this, the leader seeks to maximise
Π1 =

7.5 −
x1 + 4375 −
1
2
x1
1000 
x1 =

3.125 −
x1
2000

x1
as a function of x1.
dΠ1
dx1
= 3.125 −
x1
1000
set = 0
⇒ x1 = 3125 (3)
Denoting this optimal value by X∗
1
and the corresponding value of x2 by X∗
2
(substitute
Eq. 3 into Eq. 2) gives
X
∗
1 = 3125, X∗
2 = 2812.5
At these equilibrium values
P
∗
1 = 4.0625, P∗
2 = 4.25
and
Π
∗
1 = 4882.8125, Π
∗
2 = 6328.125 (4)
