Minimax (sometimes MinMax or MM) is a decision rule used in decision theory, game theory, statistics and philosophy for minimizing the possible loss for a worst case (maximum loss) scenario. When dealing with gains, it is referred to as "maximin" - to maximize the minimum gain.

A principle for decision-making by which, when presented with two various and conflicting strategies, one should, by the use of logic, determine and use the strategy that will minimize the maximum losses that could occur. This financial and business strategy strives to attain results that will cause the least amount of regret, should the strategy fail.

%=======================================================%


%- http://www.geeksforgeeks.org/minimax-algorithm-in-game-theory-set-1-introduction/
%- https://cs.stanford.edu/people/eroberts/courses/soco/projects/1998-99/game-theory/Minimax.html
%- http://www.economicsdiscussion.net/game-theory/4-strategies-of-the-game-theory-explained/3825

%=========================================================%

%- https://brilliant.org/wiki/minimax/

In game theory, minimax is a decision rule used to minimize the worst-case potential loss; in other words, a player considers all of the best opponent responses to his strategies, and selects the strategy such that the opponent's best strategy gives a payoff as large as possible.

The name "minimax" comes from minimizing the loss involved when the opponent selects the strategy that gives maximum loss, and is useful in analyzing the first player's decisions both when the players move sequentially and when the players move simultaneously. In the latter case, minimax may give a Nash equilibrium of the game if some additional conditions hold.

Minimax is also useful in combinatorial games, in which every position is assigned a payoff. The simplest example is assigning a "1" to a winning position and "-1" to a losing one, but as this is difficult to calculate for all but the simplest games, intermediate evaluations (specifically chosen for the game in question) are generally necessary. In this context, the goal of the first player is to maximize the evaluation of the position, and the goal of the second player is to minimize the evaluation of the position, so the minimax rule applies. This, in essence, is how computers approach games like chess and Go, though various computational improvements are possible to the "naive" implementation of minimax.

