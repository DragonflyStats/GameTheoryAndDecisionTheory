% !TEX TS-program = pdflatex
% !TEX encoding = UTF-8 Unicode

% This is a simple template for a LaTeX document using the "article" class.
% See "book", "report", "letter" for other types of document.

\documentclass[11pt]{article} % use larger type; default would be 10pt

\usepackage[utf8]{inputenc} % set input encoding (not needed with XeLaTeX)

%%% Examples of Article customizations
% These packages are optional, depending whether you want the features they provide.
% See the LaTeX Companion or other references for full information.

%%% PAGE DIMENSIONS
\usepackage{geometry} % to change the page dimensions
\geometry{a4paper} % or letterpaper (US) or a5paper or....
% \geometry{margin=2in} % for example, change the margins to 2 inches all round
% \geometry{landscape} % set up the page for landscape
%   read geometry.pdf for detailed page layout information

\usepackage{graphicx} % support the \includegraphics command and options

% \usepackage[parfill]{parskip} % Activate to begin paragraphs with an empty line rather than an indent

%%% PACKAGES
\usepackage{booktabs} % for much better looking tables
\usepackage{array} % for better arrays (eg matrices) in maths
\usepackage{paralist} % very flexible & customisable lists (eg. enumerate/itemize, etc.)
\usepackage{verbatim} % adds environment for commenting out blocks of text & for better verbatim
\usepackage{subfig} % make it possible to include more than one captioned figure/table in a single float
% These packages are all incorporated in the memoir class to one degree or another...

%%% HEADERS & FOOTERS
\usepackage{fancyhdr} % This should be set AFTER setting up the page geometry
\pagestyle{fancy} % options: empty , plain , fancy
\renewcommand{\headrulewidth}{0pt} % customise the layout...
\lhead{}\chead{}\rhead{}
\lfoot{}\cfoot{\thepage}\rfoot{}

%%% SECTION TITLE APPEARANCE
\usepackage{sectsty}
\allsectionsfont{\sffamily\mdseries\upshape} % (See the fntguide.pdf for font help)
% (This matches ConTeXt defaults)

%%% ToC (table of contents) APPEARANCE
\usepackage[nottoc,notlof,notlot]{tocbibind} % Put the bibliography in the ToC
\usepackage[titles,subfigure]{tocloft} % Alter the style of the Table of Contents
\renewcommand{\cftsecfont}{\rmfamily\mdseries\upshape}
\renewcommand{\cftsecpagefont}{\rmfamily\mdseries\upshape} % No bold!

%%% END Article customizations
\begin{document}
%%% The "real" document content comes below...

%----------------------------------------%
\section{The Nash Equilibrium}
Nash Equilibrium is an outcome reached that, once achieved, means no player can increase payoff by changing decisions unilaterally. It can also be thought of as ``no regrets", in the sense that once a decision is made, the player will have no regrets concerning decisions considering the consequences. (\textit{This is perhaps the best way to rationalize it while the concept is new to you.})

\smallskip

The Nash Equilibrium is reached over time, in most cases. However, once the Nash Equilibrium is reached, it will not be deviated from. (For this, we rely on the assumption of rationality).

\smallskip

After we learn how to find the Nash Equilibrium, take a look at how a unilateral move would affect the situation. Does it make any sense? It shouldn't, and that's why the Nash Equilibrium is described as "no regrets."

%----------------------------------------%
\subsection{Finding a Nash Equilibria}
Step One: Determine player one's best response to player two's actions.
When examining the choices that may maximize a player's payout, we must look at how player one should respond to each of the options player two has. An easy way to do this visually is to cover up the choices of player two. Consider the matrix portrayed at the beginning of this article as we apply this method.
\begin{verbatim}
Player one / Player two	Left	Right
Up	(1, -)	(4, -)
Down	(3, -)	(3, -)
\end{verbatim}
Player one has two possible choices to play: "up" or "down." Player two also has two choices to play: "left" or "right." In this step of determining Nash Equilibrium, we look at responses to player two's actions. If player two chooses to play "left," we can play "up" with the payoff of one, or play "down" with the payoff of three. Since three is greater than one, we will bold the 3 indicating the option to play "down" here.

\smallskip

If player two chooses to play "right," we can either choose to play 'up' for a payoff of four or play "down" for a playoff of three. Since four is greater than three, we bold the four to indicate the option to play "up" here. The bold outcomes are shown below on the full matrix.
\begin{verbatim}
Player one / Player two	Left	Right
Up	(1, 3)	(4, 2)
Down	(3, 2)	(3, 1)
\end{verbatim}
%----------------------------------------%
Step Two: Determine player two's best response to player one's actions.
As we did before with the player two payoffs for player one, we will hide the payoffs of player one when determining the best responses for player two. 
%-----------------------------------------%
\begin{verbatim}
Player one / Player two	Left	Right
Up	(-, 3)	(-, 2)
Down	(-, 2)	(-, 1)
\end{verbatim}
%-----------------------------------------%
Just as when looking at player one, each player has two choices to play. If player one chooses to play "up," we can play "left," with a payoff of three, or "right," with a payoff of two. Since three is greater than two, we bold the three to show the option to play "left" here. If player one chooses to play "down," we can play "left," for a payoff of two, or "right," for a payoff of one. Since two is greater than one, we bold the two indicating the option to play "left" here. The bold outcomes are shown below on the full matrix.
\begin{verbatim}
Player one / Player two	Left	Right
Up	(1, 3)	(4, 2)
Down	(3, 2)	(3, 1)
\end{verbatim}
Step Three: Determine which outcomes have both payoffs bold. That particular outcome is the Nash Equilibrium.
Now, we combine the bold options for both players onto the full matrix.

\begin{verbatim}
Player one / Player two	Left	Right
Up	(1, 3)	(4, 2)
Down	(3, 2)	(3, 1)
\end{verbatim}


Look for intersections where both payoffs are bold. In this case, we find the intersection of (Down , Left) with the payoff of (3, 2) fits our criteria. This indicates our Nash Equilibrium.

This method of finding Nash Equilibrium is well-suited to finding equilibria in games that are simultaneous since we are looking at how a player would respond independently of how the other acts. 
\subsection{Airlines Example}
This scenario of a simultaneous game is often played out in businesses such as airlines. Below is an example, similar to the game above, of how airline pricing may play out. The payouts are in thousands of dollars. Remember, these are the payouts, not the prices. The method we applied previously is already applied to show where the Nash Equilibrium appears.
\begin{verbatim}
Airline one / Airline two 	Low Price	High Price
Low Price	(3,000, 3,000)	(4,000, 2,000)
High Price	(2,000, 4,000)	(3,500, 3,500)
\end{verbatim}
\begin{itemize}
    \item Looking at just A1's choices we can see that if A2 chooses to play low price, we choose between Low Price for 3,000 or high price for 2,000. 
    \item We choose "low," since 3,000>2,000. We do the same thing for A2 playing High Price and see that we play "low" because 4,000>3,500. 
    \item Conversely, looking just at A2's choices, we can see that if A1 chooses to play low price, we choose between "low price" for 3,000 and "high price" for 2,000. 
    \item Since 3,000>2,000, we choose the "low price" option here. If A1 plays high price, we can charge a low price for 4,000 or high price for 3,500. Since 4,000>3,500, we choose to play "low price" here.
\end{itemize}


The Nash Equilibrium is that both airlines will charge a low price (shown when choices for each party are highlighted). If both airlines charged a high price, they would each be better off than they are at the Nash Equilibrium.

So why don't they agree to do this? First off, it's illegal to collude. Second, if this were to occur, a unilateral action on behalf of one airline to charge a low price would be beneficial, resulting in that airline making more money in turn. This logic also shows how the Nash Equilibrium is reached, and why it is not beneficial to deviate from it once it is reached. 

\end{document}
%----------------------------------------%