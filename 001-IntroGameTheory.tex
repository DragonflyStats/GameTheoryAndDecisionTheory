% !TEX TS-program = pdflatex
% !TEX encoding = UTF-8 Unicode

% This is a simple template for a LaTeX document using the "article" class.
% See "book", "report", "letter" for other types of document.

\documentclass[11pt]{article} % use larger type; default would be 10pt

\usepackage[utf8]{inputenc} % set input encoding (not needed with XeLaTeX)

%%% Examples of Article customizations
% These packages are optional, depending whether you want the features they provide.
% See the LaTeX Companion or other references for full information.

%%% PAGE DIMENSIONS
\usepackage{geometry} % to change the page dimensions
\geometry{a4paper} % or letterpaper (US) or a5paper or....
% \geometry{margin=2in} % for example, change the margins to 2 inches all round
% \geometry{landscape} % set up the page for landscape
%   read geometry.pdf for detailed page layout information

\usepackage{graphicx} % support the \includegraphics command and options

% \usepackage[parfill]{parskip} % Activate to begin paragraphs with an empty line rather than an indent

%%% PACKAGES
\usepackage{booktabs} % for much better looking tables
\usepackage{array} % for better arrays (eg matrices) in maths
\usepackage{paralist} % very flexible & customisable lists (eg. enumerate/itemize, etc.)
\usepackage{verbatim} % adds environment for commenting out blocks of text & for better verbatim
\usepackage{subfig} % make it possible to include more than one captioned figure/table in a single float
% These packages are all incorporated in the memoir class to one degree or another...

%%% HEADERS & FOOTERS
\usepackage{fancyhdr} % This should be set AFTER setting up the page geometry
\pagestyle{fancy} % options: empty , plain , fancy
\renewcommand{\headrulewidth}{0pt} % customise the layout...
\lhead{}\chead{}\rhead{}
\lfoot{}\cfoot{\thepage}\rfoot{}

%%% SECTION TITLE APPEARANCE
\usepackage{sectsty}
\allsectionsfont{\sffamily\mdseries\upshape} % (See the fntguide.pdf for font help)
% (This matches ConTeXt defaults)

%%% ToC (table of contents) APPEARANCE
\usepackage[nottoc,notlof,notlot]{tocbibind} % Put the bibliography in the ToC
\usepackage[titles,subfigure]{tocloft} % Alter the style of the Table of Contents
\renewcommand{\cftsecfont}{\rmfamily\mdseries\upshape}
\renewcommand{\cftsecpagefont}{\rmfamily\mdseries\upshape} % No bold!

%%% END Article customizations
\begin{document}
%%% The "real" document content comes below...

%\maketitle

\section{Introduction to Game Theory}

%===============================================%
%- http://www.investopedia.com/articles/financial-theory/08/game-theory-basics.asp

Game theory is the process of modeling the strategic interaction between two or more players in a situation containing set rules and outcomes. While used in a number of disciplines, game theory is most notably used as a tool within the study of economics. The economic application of game theory can be a valuable tool to aide in the fundamental analysis of industries, sectors and any strategic interaction between two or more firms. Here, we'll take an introductory look at game theory and the terms involved, and introduce you to a simple method of solving games, called backwards induction.

\subsection{Definitions} 
Any time we have a situation with two or more players that involves known payouts or quantifiable consequences, we can use game theory to help determine the most likely outcomes. 
Let's start out by defining a few terms commonly used in the study of game theory:

\begin{description}
\item[Game:] Any set of circumstances that has a result dependent on the actions of two of more decision makers ("players")
\item[Players:] A strategic decision maker within the context of the game
\item[Strategy:] A complete plan of action a player will take given the set of circumstances that might arise within the game
\item[Payoff:] The payout a player receives from arriving at a particular outcome. The payout can be in any quantifiable form, from dollars to utility.
\item[Information Set:] The information available at a given point in the game. The term information set is most usually applied when the game has a sequential component.
\item[Equilibrium:] The point in a game where both players have made their decisions and an outcome is reached.
\end{description}

\subsection{Assumptions}
As with any concept in economics, there is the assumption of rationality. There is also an assumption of maximization. It is assumed that players within the game are rational and will strive to maximize their payoffs in the game. 

\smallskip

When examining games that are already set up, it is assumed on your behalf that the payouts listed include the sum of all payoffs that are associated with that outcome. This will exclude any "what if" questions that may arise.

\smallskip

The number of players in a game can theoretically be infinite, but most games will be put into the context of \textbf{two players}. One of the simplest games is a sequential game involving two players.

\end{document}
