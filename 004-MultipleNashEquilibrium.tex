% !TEX TS-program = pdflatex
% !TEX encoding = UTF-8 Unicode

% This is a simple template for a LaTeX document using the "article" class.
% See "book", "report", "letter" for other types of document.

\documentclass[11pt]{article} % use larger type; default would be 10pt

\usepackage[utf8]{inputenc} % set input encoding (not needed with XeLaTeX)

%%% Examples of Article customizations
% These packages are optional, depending whether you want the features they provide.
% See the LaTeX Companion or other references for full information.

%%% PAGE DIMENSIONS
\usepackage{geometry} % to change the page dimensions
\geometry{a4paper} % or letterpaper (US) or a5paper or....
% \geometry{margin=2in} % for example, change the margins to 2 inches all round
% \geometry{landscape} % set up the page for landscape
%   read geometry.pdf for detailed page layout information

\usepackage{graphicx} % support the \includegraphics command and options

% \usepackage[parfill]{parskip} % Activate to begin paragraphs with an empty line rather than an indent

%%% PACKAGES
\usepackage{booktabs} % for much better looking tables
\usepackage{array} % for better arrays (eg matrices) in maths
\usepackage{paralist} % very flexible & customisable lists (eg. enumerate/itemize, etc.)
\usepackage{verbatim} % adds environment for commenting out blocks of text & for better verbatim
\usepackage{subfig} % make it possible to include more than one captioned figure/table in a single float
% These packages are all incorporated in the memoir class to one degree or another...

%%% HEADERS & FOOTERS
\usepackage{fancyhdr} % This should be set AFTER setting up the page geometry
\pagestyle{fancy} % options: empty , plain , fancy
\renewcommand{\headrulewidth}{0pt} % customise the layout...
\lhead{}\chead{}\rhead{}
\lfoot{}\cfoot{\thepage}\rfoot{}

%%% SECTION TITLE APPEARANCE
\usepackage{sectsty}
\allsectionsfont{\sffamily\mdseries\upshape} % (See the fntguide.pdf for font help)
% (This matches ConTeXt defaults)

%%% ToC (table of contents) APPEARANCE
\usepackage[nottoc,notlof,notlot]{tocbibind} % Put the bibliography in the ToC
\usepackage[titles,subfigure]{tocloft} % Alter the style of the Table of Contents
\renewcommand{\cftsecfont}{\rmfamily\mdseries\upshape}
\renewcommand{\cftsecpagefont}{\rmfamily\mdseries\upshape} % No bold!

%%% END Article customizations
\begin{document}
%%% The "real" document content comes below...

%\maketitle
Multiple Nash Equilibria

Here's a game that demonstrates multiple Nash Equilibria: Two drivers are traveling towards each other on a road. Should they drive on the left or the right side? They don't want to wreck...

Driver 2
Left	Right
Driver 1	Left	1,1	-1,-1
Right	-1,-1	1,1
Both (Left,Left) and (Right,Right) are Nash Equilibria. As long as they're on opposite sides of the road, the drivers are happy and don't want to deviate. Games like this are often solved by social convention--beforehand all the players agree on a strategy so that everyone is better off. Of course, everyone knows that the right side is the best side to drive on, so the game should look more like this:

Driver 2
Left	Right
Driver 1	Left	1,1	-1,-1
Right	-1,-1	2,2
In this case, the game itself gives the players a clue as to where the other player will be, even though there are two Nash Equilibria.

Here's a game with three Nash Equilibria and no dominated strategies:

2
a	b	c
1	A	1,1	2,0	3,0
B	0,2	3,3	0,0
C	0,3	0,0	10,10
The Nash Equilibria are (A,a), (B,b), and (C,c).


\end{document}