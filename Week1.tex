
\documentclass[]{report}
\voffset=-1.5cm
\oddsidemargin=0.0cm
\textwidth = 480pt


\usepackage{amsmath}
\usepackage{graphicx}
\usepackage{amssymb}
\usepackage{framed}
\usepackage{multicol}
%\usepackage[paperwidth=21cm, paperheight=29.8cm]{geometry}
%\usepackage[angle=0,scale=1,color=black,hshift=-0.4cm,vshift=15cm]{background}
%\usepackage{multirow}
\usepackage{enumerate}

\usepackage{amsmath,amsfonts,amssymb}
\usepackage{color}
\usepackage{multirow}
\usepackage{eurosym}
\usepackage{framed}

%\input def.tex
%\input dsdef.tex
%\input rgb.tex

%\newcommand \la{\lambda}
%\newcommand \al{a}
%\newcommand \be{b}
\newcommand \x{\overline{x}}
\newcommand \y{\overline{y}}

\begin{document}
%=========================================================================%
\section{What is Game Theory}

Game theory is a branch of applied mathematics that uses models to study interactions with formalised incentive structures ("games"). 

Unlike decision theory, which also studies formalised incentive structures, game theory encompasses decisions that are made in an environment where various players interact strategically. In other words, game theory studies choice of optimal behavior when costs and benefits of each option are not fixed, but depend upon the choices of other individuals.



Game theory is the science of strategic reasoning, in such a way that it studies the behaviour of rational game players who are trying to maximise their utility, profits, gains, etc., in interaction with other players, and therefore in a context of strategic interdependence.

Game theory has applications in a variety of fields, including economics,  international relations, evolutionary biology, political science, and  military strategy. Game theorists study the predicted and actual  behaviour of individuals in games, as well as optimal strategies.  Seemingly different situations can have similar incentive  structures, thus all exemplifying one particular game. 


\subsection{Types of Games}

In game theory, the unscrupulous diner's dilemma (or just diner's dilemma) is an n-player prisoner's dilemma. The situation imagined is that several individuals go out to eat, and prior to ordering, they agree to split the check equally between all of them. Each individual must now choose whether to order the expensive or inexpensive dish. It is presupposed that the expensive dish is better than the cheaper, but not by enough to warrant paying the difference when eating alone. Each individual reasons that the expense s/he adds to their bill by ordering the more expensive item is very small, and thus the improved dining experience is worth the money. However, having all reasoned thus, they all end up paying for the cost of the more expensive meal, which by assumption, is worse for everyone than having ordered and paid for the cheaper meal.


%=========================================================================%
\newpage
\subsection{Cooperative game}
\begin{itemize}
	\item A cooperative game is a game wherein two or more players do not compete, but rather strive toward a unique objective and therefore win or lose as a group.
	
	\item 	Cooperative games are rare, but still many exist. One example is "Stand Up", where a number of individuals sit down, link arms (all facing away from each other) and attempt to stand up. This objective becomes more difficult as the number of players increases.
	
	\item 	Another is the counting game, where the players, as a group, attempt to count to 20 with no two participants saying the same number twice. In a cooperative version of volleyball, the emphasis is on keeping the ball in the air for as long as possible.
	
	\item 	Cooperative games are rare in recreational gaming, where conflict between players is a powerful force. However, such scenarios can occur in real life (when the sense of the word "game" is extended beyond recreational games). For example, operation of a successful business is, at least in theory, a cooperative game, since all participants benefit if the business succeeds and suffer if it fails.
	
	\item 	Role-playing games are the most common form of cooperative game, though these games are not always purely cooperative. In such games, the players (who act through personae called "characters") usually strive toward intertwined and similar goals. However, each character has his or her own ambitions, and ultimately, individual goals. Hence conflict between characters often occurs in these games.
\end{itemize}
%=========================================================================%
\subsection{Winner's Curse}
The winner's curse is a phenomenon that may occur in common value auctions with incomplete information. In short, the winner's curse says that in such an auction, the winner will tend to overpay. The winner may overpay or be "cursed" in one of two ways: 1) the winning bid exceeds the value of the auctioned asset such that the winner is worse off in absolute terms; or 2) the value of the asset is less than the bidder anticipated, so the bidder may still have a net gain but will be worse off than anticipated.[1] However, an actual overpayment will generally occur only if the winner fails to account for the winner's curse when bidding (an outcome that, according to the revenue equivalence theorem, need never occur).
\section{Pure and Mixed Strategies}

%==================================================%

\begin{itemize}
\item A pure strategy determines all your moves during the game (and should therefore specify your moves for all possible other players' moves). 
\item A mixed strategy is a probability distribution over all possible pure strategies (some of which may get zero weight).
\end{itemize}

%==================================================%

%========================================%
\section{Pure and mixed strategies}
A pure strategy provides a complete definition of how a player will play a game. In particular, it determines the move a player will make for any situation he or she could face. A player's strategy set is the set of pure strategies available to that player.

A mixed strategy is an assignment of a probability to each pure strategy. This allows for a player to randomly select a pure strategy. Since probabilities are continuous, there are infinitely many mixed strategies available to a player.

Of course, one can regard a pure strategy as a degenerate case of a mixed strategy, in which that particular pure strategy is selected with probability 1 and every other strategy with probability 0.

A totally mixed strategy is a mixed strategy in which the player assigns a strictly positive probability to every pure strategy. (Totally mixed strategies are important for equilibrium refinement such as trembling hand perfect equilibrium.)

%=================================%


\subsection{What is a pure strategy?}

A pure strategy is an unconditional, defined choice that a person makes in a situation or game. For example, in the game of Rock-Paper-Scissors,if a player would choose to only play scissors for each and every independent trial, regardless of the other player’s strategy, choosing scissors would be the player’s pure strategy. The probability for choosing scissors equal to 1 and all other options (paper and rock) is chosen with the probability of 0. The set of all options (i.e. rock, paper, and scissors) available in this game is known as the strategy set.

%==================================================%

\subsection{What is a mixed strategy?}

A mixed strategy is an assignment of probability to all choices in the strategy set. Using the example of Rock-Paper-Scissors, if a person’s probability of employing each pure strategy is equal, then the probability distribution of the strategy set would be 1/3 for each option, or approximately 33%. In other words, a person using a mixed strategy incorporates more than one pure strategy into a game.

The definition of a mixed strategy does not rule out the possibility for an option(s)to never be chosen (eg. pscissors= 0.5, prock = 0.5, ppaper = 0). This means that in a way, a pure strategy can also be considered a mixed strategy at its extreme, with a binary probability assignment (setting one option to 1 and all others equal to 0). For this article, we shall say that pure strategies are not mixed strategies.



%=================================%
Dominance
%- https://en.wikipedia.org/wiki/Strategic_dominance#Terminology

In game theory, strategic dominance (commonly called simply dominance) occurs when one strategy is better than another strategy for one player, no matter how that player's opponents may play. Many simple games can be solved using dominance. The opposite, intransitivity, occurs in games where one strategy may be better or worse than another strategy for one player, depending on how the player's opponents may play.


\subsection{Terminology}
When a player tries to choose the "best" strategy among a multitude of options, that player may compare two strategies A and B to see which one is better. The result of the comparison is one of:


\begin{itemize}
\item B dominates A: choosing B always gives as good as or a better outcome than choosing A. There are 2 possibilities:
\item B strictly dominates A: choosing B always gives a better outcome than choosing A, no matter what the other player(s) do.
\item B weakly dominates A: There is at least one set of opponents' action for which B is superior, and all other sets of opponents' actions give B the same payoff as A.
\item B and A are intransitive: B neither dominates, nor is dominated by, A. Choosing A is better in some cases, while choosing B is better in other cases, depending on exactly how the opponent chooses to play. For example, B is "throw rock" while A is "throw scissors" in Rock, Paper, Scissors.
\item B is dominated by A: choosing B never gives a better outcome than choosing A, no matter what the other player(s) do. There are 2 possibilities:
\item B is weakly dominated by A: There is at least one set of opponents' actions for which B gives a worse outcome than A, while all other sets of opponents' actions give A the same payoff as B. (Strategy A weakly dominates B).
\item B is strictly dominated by A: choosing B always gives a worse outcome than choosing A, no matter what the other player(s) do. (Strategy A strictly dominates B).
\end{itemize}
This notion can be generalized beyond the comparison of two strategies.

Strategy B is strictly dominant if strategy B strictly dominates every other possible strategy.
Strategy B is weakly dominant if strategy B dominates all other strategies, but some (or all) strategies are only weakly dominated by B.
Strategy B is strictly dominated if some other strategy exists that strictly dominates B.
Strategy B is weakly dominated if some other strategy exists that weakly dominates B.




\end{document}
