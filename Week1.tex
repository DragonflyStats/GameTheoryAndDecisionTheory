
\documentclass[]{report}
\voffset=-1.5cm
\oddsidemargin=0.0cm
\textwidth = 480pt


\usepackage{amsmath}
\usepackage{graphicx}
\usepackage{amssymb}
\usepackage{framed}
\usepackage{multicol}
%\usepackage[paperwidth=21cm, paperheight=29.8cm]{geometry}
%\usepackage[angle=0,scale=1,color=black,hshift=-0.4cm,vshift=15cm]{background}
%\usepackage{multirow}
\usepackage{enumerate}

\usepackage{amsmath,amsfonts,amssymb}
\usepackage{color}
\usepackage{multirow}
\usepackage{eurosym}
\usepackage{framed}
\usepackage{fancyhdr}
\usepackage{listings}
\usepackage{eurosym}
\usepackage{vmargin}
\usepackage{amsmath}
\usepackage{fancyhdr}
\usepackage{listings}
\usepackage{framed}
\usepackage{graphics}
\usepackage{epsfig}
\usepackage{subfigure}
%\input def.tex
%\input dsdef.tex
%\input rgb.tex

%\newcommand \la{\lambda}
%\newcommand \al{a}
%\newcommand \be{b}
\newcommand \x{\overline{x}}
\newcommand \y{\overline{y}}

\pagestyle{fancy}
\setmarginsrb{20mm}{0mm}{20mm}{25mm}{12mm}{11mm}{0mm}{11mm}
\lhead{Operations Research 2} \rhead{Kevin O'Brien}
\chead{MS4315}
%\input{tcilatex}

\begin{document}
%=========================================================================%
\section{What is Game Theory}

Game theory is a branch of applied mathematics that uses models to study interactions with formalised incentive structures ("games"). 

Unlike decision theory, which also studies formalised incentive structures, game theory encompasses decisions that are made in an environment where various players interact strategically. In other words, game theory studies choice of optimal behavior when costs and benefits of each option are not fixed, but depend upon the choices of other individuals.



Game theory is the science of strategic reasoning, in such a way that it studies the behaviour of rational game players who are trying to maximise their utility, profits, gains, etc., in interaction with other players, and therefore in a context of strategic interdependence.

Game theory has applications in a variety of fields, including economics,  international relations, evolutionary biology, political science, and  military strategy. Game theorists study the predicted and actual  behaviour of individuals in games, as well as optimal strategies.  Seemingly different situations can have similar incentive  structures, thus all exemplifying one particular game. 


\subsection{Types of Games}

In game theory, the unscrupulous diner's dilemma (or just diner's dilemma) is an n-player prisoner's dilemma. The situation imagined is that several individuals go out to eat, and prior to ordering, they agree to split the check equally between all of them. Each individual must now choose whether to order the expensive or inexpensive dish. It is presupposed that the expensive dish is better than the cheaper, but not by enough to warrant paying the difference when eating alone. Each individual reasons that the expense s/he adds to their bill by ordering the more expensive item is very small, and thus the improved dining experience is worth the money. However, having all reasoned thus, they all end up paying for the cost of the more expensive meal, which by assumption, is worse for everyone than having ordered and paid for the cheaper meal.

%==========================================%
% Add Ins

\section{Two-Person Zero-Sum Games: Basic Concepts}
% - https://neos-guide.org/content/game-theory-basics

Game theory provides a mathematical framework for analyzing the decision-making processes and strategies of adversaries (or players) in different types of competitive situations. The simplest type of competitive situations are two-person, zero-sum games. These games involve only two players; they are called zero-sum games because one player wins whatever the other player loses.


\begin{framed}
A zero-sum game is a mathematical representation of a situation in which each participant's gain or loss of utility is exactly balanced by the losses or gains of the utility of the other participants.
\end{framed}

\section{Example: Odds and Evens}
Consider the simple game called \textbf{odds and evens}. Suppose that player 1 takes evens and player 2 takes odds. Then, each player simultaneously shows either one finger or two fingers. If the number of fingers matches, then the result is even, and player 1 wins the bet (\$2). If the number of fingers does not match, then the result is odd, and player 2 wins the bet (\$2). Each player has two possible strategies: show one finger or show two fingers. The payoff matrix shown below represents the payoff to player 1.

\[Image\]


\section{Basic Concepts of Two-Person Zero-Sum Games}
This game of odds and evens illustrates important concepts of simple games.

A two-person game is characterized by the strategies of each player and the payoff matrix.

\begin{itemize}
\item The payoff matrix shows the gain (positive or negative) for player 1 that would result from each combination of strategies for the two players. Note that the matrix for player 2 is the negative of the matrix for player 1 in a zero-sum game.
\item The entries in the payoff matrix can be in any units as long as they represent the utility (or value) to the player.
\item There are two key assumptions about the behavior of the players. The first is that both players are rational. The second is that both players are greedy meaning that they choose their strategies in their own interest (to promote their own wealth).
\end{itemize}

%= https://revisionworld.com/a2-level-level-revision/maths/decision-maths-0/game-theory
\subsection{A two person zero-sum game}
The starting point in the mathematical theory of games is that the outcome of a game is determined by the strategies of the players.
A  two person zero-sum game is a game in which the winnings of one player equal the losses of the other for every combination of strategies. Taking winnings to be positive and losses to be negative gives a zero sum in each case. Viewing a game from one player’s point of view, we could represent the outcomes (called pay-offs) for each combination of strategies in a matrix. This is called the pay-off matrix for that player.

\subsection{Example}
A and B are two players in a zero-sum game. A uses one of two strategies, W or X, and B uses one of the strategies Y or Z. The table shows the pay-off matrix for A.

The pay-off matrix shows that if B adopts strategy Y then the pay-off for A will be 2 by using strategy W and 5 using strategy X.

On the other hand, if B adopts strategy Z then the pay-off for A will be 02 using strategy W and 04 using strategy X.

The idea is that neither player knows in advance which strategy the other will use.

The situation could equally be represented by the pay-off matrix for B. This would show corresponding values with opposite signs since this is a zero-sum game.

%===========================%

\section{Payoff Matrix}
%- http://economicsmicro.blogspot.ie/2008/11/how-to-read-payoff-matrix-game-theory.html

How to read a payoff matrix : Game Theory
Eg – Payoff matrix for a new technology game 

\[IMAGE\]



\subsection{Explanation} 
\begin{enumerate}

\item There are 2 firms A and B and they want to decide whether to Start a new campaign. 
\item each firm will be affected by its competitor’s decision. 
\item The above table shows the payoff to both firms. This table is called \textbf{payoff matrix}. 
\item (a,b) -The first number in each cell is the payoff(profits) to A and second number in each cell 
is the payoff to B. 
\begin{itemize}
\item (10,5) shows the payoffs when both firms start a new campaign. 
Firm A’s profits are 10 and firm B’s are 5. 
\item (15,0) shows the payoffs when firm A starts a new campaign abd Firm B does not. 
Firm A’s profits are 15 and firm B’s are 0. 
\end{itemize}
\end{enumerate}
%-------------------------------------%

%=========================================================================%
\newpage
\subsection{Cooperative game}
\begin{itemize}
	\item A cooperative game is a game wherein two or more players do not compete, but rather strive toward a unique objective and therefore win or lose as a group.
	
	\item 	Cooperative games are rare, but still many exist. One example is "Stand Up", where a number of individuals sit down, link arms (all facing away from each other) and attempt to stand up. This objective becomes more difficult as the number of players increases.
	
	\item 	Another is the counting game, where the players, as a group, attempt to count to 20 with no two participants saying the same number twice. In a cooperative version of volleyball, the emphasis is on keeping the ball in the air for as long as possible.
	
	\item 	Cooperative games are rare in recreational gaming, where conflict between players is a powerful force. However, such scenarios can occur in real life (when the sense of the word "game" is extended beyond recreational games). For example, operation of a successful business is, at least in theory, a cooperative game, since all participants benefit if the business succeeds and suffer if it fails.
	
	\item 	Role-playing games are the most common form of cooperative game, though these games are not always purely cooperative. In such games, the players (who act through personae called "characters") usually strive toward intertwined and similar goals. However, each character has his or her own ambitions, and ultimately, individual goals. Hence conflict between characters often occurs in these games.
\end{itemize}
%=========================================================================%
\subsection{Winner's Curse}
The winner's curse is a phenomenon that may occur in common value auctions with incomplete information. In short, the winner's curse says that in such an auction, the winner will tend to overpay. The winner may overpay or be "cursed" in one of two ways: 1) the winning bid exceeds the value of the auctioned asset such that the winner is worse off in absolute terms; or 2) the value of the asset is less than the bidder anticipated, so the bidder may still have a net gain but will be worse off than anticipated.[1] However, an actual overpayment will generally occur only if the winner fails to account for the winner's curse when bidding (an outcome that, according to the revenue equivalence theorem, need never occur).
\section{Dominance}
%- https://en.wikipedia.org/wiki/Strategic_dominance#Terminology

Strategic dominance (commonly called simply dominance) occurs when one strategy is better than another strategy for one player, no matter how that player's opponents may play. Many simple games can be solved using dominance. The opposite, intransitivity, occurs in games where one strategy may be better or worse than another strategy for one player, depending on how the player's opponents may play.


\subsection{Terminology}
When a player tries to choose the "best" strategy among a multitude of options, that player may compare two strategies A and B to see which one is better. The result of the comparison is one of:


\begin{itemize}
\item B dominates A: choosing B always gives as good as or a better outcome than choosing A. There are 2 possibilities:
\item B strictly dominates A: choosing B always gives a better outcome than choosing A, no matter what the other player(s) do.
\item B weakly dominates A: There is at least one set of opponents' action for which B is superior, and all other sets of opponents' actions give B the same payoff as A.
\item B and A are intransitive: B neither dominates, nor is dominated by, A. Choosing A is better in some cases, while choosing B is better in other cases, depending on exactly how the opponent chooses to play. \textit{For example, B is "throw rock" while A is "throw scissors" in Rock, Paper, Scissors.}
\item B is dominated by A: choosing B never gives a better outcome than choosing A, no matter what the other player(s) do. There are 2 possibilities:
\item B is weakly dominated by A: There is at least one set of opponents' actions for which B gives a worse outcome than A, while all other sets of opponents' actions give A the same payoff as B. (Strategy A weakly dominates B).
\item B is strictly dominated by A: choosing B always gives a worse outcome than choosing A, no matter what the other player(s) do. (Strategy A strictly dominates B).
\end{itemize}
This notion can be generalized beyond the comparison of two strategies.

Strategy B is strictly dominant if strategy B strictly dominates every other possible strategy.
Strategy B is weakly dominant if strategy B dominates all other strategies, but some (or all) strategies are only weakly dominated by B.
Strategy B is strictly dominated if some other strategy exists that strictly dominates B.
Strategy B is weakly dominated if some other strategy exists that weakly dominates B.

\newpage
%=================================================================================================================%

%=================================%
\section{Normal Form of a Game}
 
The normal form is a description of a game. Unlike extensive form, normal-form representations are not graphical per se, but rather represent the game by way of a matrix. While this approach can be of greater use in identifying strictly dominated strategies and Nash equilibria, some information is lost as compared to extensive-form representations. The normal-form representation of a game includes all perceptible and conceivable strategies, and their corresponding payoffs, for each player.

In static games of complete, perfect information, a normal-form representation of a game is a specification of players' strategy spaces and payoff functions. A strategy space for a player is the set of all strategies available to that player, whereas a strategy is a complete plan of action for every stage of the game, regardless of whether that stage actually arises in play. A payoff function for a player is a mapping from the cross-product of players' strategy spaces to that player's set of payoffs (normally the set of real numbers, where the number represents a cardinal or ordinal utility—often cardinal in the normal-form representation) of a player, i.e. the payoff function of a player takes as its input a strategy profile (that is a specification of strategies for every player) and yields a representation of payoff as its output.

Contents  [hide] 
1	An example
1.1	Other representations
2	Uses of normal form
2.1	Dominated strategies
2.2	Sequential games in normal form
3	General formulation
4	References
An example[edit]
A normal-form game
Player 2

Player 1
Left	Right
Top	4, 3	−1, −1
Bottom	0, 0	3, 4
The matrix to the right is a normal-form representation of a game in which players move simultaneously (or at least do not observe the other player's move before making their own) and receive the payoffs as specified for the combinations of actions played. For example, if player 1 plays top and player 2 plays left, player 1 receives 4 and player 2 receives 3. In each cell, the first number represents the payoff to the row player (in this case player 1), and the second number represents the payoff to the column player (in this case player 2).

Other representations[edit]
Often, symmetric games (where the payoffs do not depend on which player chooses each action) are represented with only one payoff. This is the payoff for the row player. For example, the payoff matrices on the right and left below represent the same game.

Both players
Player 2

Player 1
Stag	Hare
Stag	3, 3	0, 2
Hare	2, 0	2, 2
Just row
Player 2

Player 1
Stag	Hare
Stag	3	0
Hare	2	2


%===========================%
\newpage
\subsection{The Play-Safe strategy}
The play-safe strategy for a player is the strategy for which the minimum pay-off is as high as possible. In the example above, the minimum pay-off for A using strategy W is 02, whereas the minimum pay-off using strategy X is 04. This means that strategy W is the play-safe strategy for player A.

\begin{itemize}
\item Notice that finding the play-safe strategy for player A involves comparing the minimum values in the rows of the pay-off matrix for A.
\item 
Finding the play-safe strategy for B will involve comparing the values in the columns, remembering that B’s pay-offs are the negatives of the ones in the pay-off matrix for A.
\item 
The minimum value for B using strategy Y is 05 and the minimum value using strategy Z is 2. This means that the play-safe strategy for B is strategy Z.
\item 
The situation is shown in the pay-off matrices for A and B.
\item

In this case, the maximum of the minimum pay-offs, for each player, is in the corresponding position in the two matrices. This represents the stable solution to the problem referred to as the saddle point (or minimax point).
\item 
The solution is stable in the sense that neither player can improve their pay-off by taking a different strategy, given that the other player doesn’t change.
\item
In other words, while B uses strategy Z, the best strategy for A is W and while A uses strategy W, the best strategy for B is Z.
\item
KEY POINT - If the sum of the two values used to determine the play-safe strategies is not zero then the values cannot correspond to the same cell position in the play-off matrices. This means that there is no saddle point and the game has no stable solution.
\end{itemize}
Example
The pay-off matrices for two players in a zero-sum game are given below. Show that there is no stable solution for the game.




\subsection{Optimal strategies for games that are not stable}
The repeated use of the same strategy over a series of games is called a pure strategy. This provides the best results for both players in a game which has a stable solution. In the case where no stable solution exists, a mixed strategy is used in which each of the strategies is employed with a given probability to find the optimal solution.


 

\subsection{Graphical representation}
Each graph corresponds to a strategy for Q (i.e. the opponent of P).

\begin{itemize}
\item The diagram shows graphs of 9p − 4 and −9p + 6 against values of p from 0 to 1.
The point of intersection corresponds to the probability that gives the optimal mixed strategy for P in the last example.
\item 
KEY POINT - When P’s opponent has more strategies there will be more graphs with several points of intersection. You will need to identify the one that represents the optimal mixed strategy for P. This will be the highest point on or below each of the graphs.
\item 
This diagram represents a situation where P’s opponent has three strategies to choose from.
The point representing the optimal mixed strategy for P is circled.
\item 
Notice how the problem of identifying the right vertex can be expressed as a linear programming problem in which the object is to maximise the expected gain for P subject to the constraints represented by the regions bounded by the straight line graphs.
\item 
This is, in fact, the approach used for higher dimensional problems. The conditions are formulated as a linear programming problem which may then be solved by the
simplex algorithm.
\item 
KEY POINT - If the pay-offs for one strategy are always better than the corresponding pay-offs for some other strategy then the weaker one can be ignored when determining the probabilities for a mixed strategy. In this way, the pay-off matrix is reduced by what is called a dominance argument.
\end{itemize}


\chapter{Mark Burke}
	\begin{center}
		\textbf{Game Theory %(  au Spaniel)
		}
	\end{center}

	
	\section{Analysis of (finite) Matrix Games}
	One of the best known games is \\
	
	{ \color{red} Prisoner's Dilemma (PD)} \vspace{3mm} \\
	
	\begin{center}
		{\color{blue}
			\begin{tabular}{c|c|c|c|}
				\multicolumn{2} {c} {} & \multicolumn{2}{c} {{\color{green}Player 2}} \\
				\cline{3-4}
				\multicolumn{2}{c|}{} & Keep Quiet         & Confess        \\
				\cline{2-4}
				\multirow{2} {*} {{\color{green}Player 1}}& Keep Quiet & (-1,-1) & (-12,0) \\
				\cline{2-4}
				& Confess &(0,-12)& (-8,-8) \\
				\cline{2-4}
				%C & (2,6) & (4,7)& (0,8) \\
				%\hline
			\end{tabular}
		}
	\end{center}
	
	The solution is $\langle$ confess, confess $\rangle$. (See IESDS). \\ Why don't the players coordinate to get $\langle$ Keep Quiet, Keep Quiet $\rangle$ ?
	
	The exact payoffs are irrelevant, the game can also be represented by the order of players' preferences  - most preferred (p1) to least preferred (p4) :
	\begin{center}
		{\color{blue}
			\begin{tabular}{c|c|c|c|}
				\multicolumn{2} {c} {} & \multicolumn{2}{c} {{\color{green}Player 2}} \\
				\cline{3-4}
				\multicolumn{2}{c|}{} & Keep Quiet         & Confess        \\
				\cline{2-4}
				\multirow{2} {*} {{\color{green}Player 1}}& Keep Quiet & (p2,p2) & (p4,p1) \\
				\cline{2-4}
				& Confess &(p1,p4)& (p3,p3) \\
				\cline{2-4}
				%C & (2,6) & (4,7)& (0,8) \\
				%\hline
			\end{tabular}
		}
	\end{center}
	
	Other examples of PD-like games are
	\begin{itemize}
		\item {\color{red} War} with strategies {\color{blue} Defend, Attack} respectively.
		\item {\color{red} Arms Race} with strategies {\color{blue} Pass, Build} respectively.
		\item {\color{red} Free Trade/ Protection} with strategies {\color{blue} No Tax, Tax} respectively.
		\item {\color{red} Advertising} with strategies {\color{blue} No Ads, Ads} respectively.
	\end{itemize}
	
	{ \color{red}Deadlock} is another game ( success is to fail!): \vspace{3mm} \\
	
	\begin{center}
		{\color{blue}
			\begin{tabular}{c|c|c|c|}
				\multicolumn{2} {c} {} & \multicolumn{2}{c} {{\color{green}Player 2}} \\
				\cline{3-4}
				\multicolumn{2}{c|}{} & Try         & Fail       \\
				\cline{2-4}
				\multirow{2} {*} {{\color{green}Player 1}}& Try & (0,0) & (-1,1) \\
				\cline{2-4}
				& Fail &(1,-1)& (0,0) \\
				\cline{2-4}
				%C & (2,6) & (4,7)& (0,8) \\
				%\hline
			\end{tabular}
		}
	\end{center}
	
	Using preferences, we can consider the more general version of deadlock
	\begin{center}
		{\color{blue}
			\begin{tabular}{c|c|c|c|}
				\multicolumn{2} {c} {} & \multicolumn{2}{c} {{\color{green}Player 2}} \\
				\cline{3-4}
				\multicolumn{2}{c|}{} & Left        & Right        \\
				\cline{2-4}
				\multirow{2} {*} {{\color{green}Player 1}}& Up & (p2,p2) & (p1,p4) \\
				\cline{2-4}
				& Down &(p4,p1)& (p3,p3) \\
				\cline{2-4}
				%C & (2,6) & (4,7)& (0,8) \\
				%\hline
			\end{tabular}
		}
	\end{center}
	The solution is $\langle$ Up, Left $\rangle$. (See IESDS). Neither player gets hir first choice unless the other makes a mistake.
	
\newpage	\subsection{Strict Dominance}
	
	\textit{Strategy X strictly dominates strategy Y for a player if X gives a bigger (more preferred) payoff than Y no matter what the other players do. Players never rationally choose strictly dominated strategies.}
	
	Reduce the matrix by \textbf{Iterated Elimination of Strictly Dominated Strategies} (IESDS) - see PD and Deadlock above. The order of elimination is irrelevant. Another example is the {\color{red} Dance Club} game:
	\begin{center}
		{\color{blue}
			\begin{tabular}{c|c|c|c|}
				\multicolumn{2} {c} {} & \multicolumn{2}{c} {{\color{green}Boon docks}} \\
				\cline{3-4}
				\multicolumn{2}{c|}{} &   Salsa       &  Hip Hop       \\
				\cline{2-4}
				\multirow{2} {*} {{\color{green}Downtown}}& Salsa & (80,0) & (60,40) \\
				\cline{2-4}
				& Hip Hop &(40,60)& (40,0) \\
				\cline{2-4}
				%C & (2,6) & (4,7)& (0,8) \\
				%\hline
			\end{tabular}
		}
	\end{center}
	The solution is $\langle$ Salsa, Hip Hop $\rangle$. Salsa dominates Hip Hop for Club Downtown, then Boonies choses Hip Hop.
	
	Some more examples
	
	\begin{center}
		{\color{blue}
			\begin{tabular}{c|c|c|c|c|}
				\multicolumn{2} {c} {} & \multicolumn{3}{c} {{\color{green}Player 2}} \\
				\cline{3-5}
				\multicolumn{2}{c|}{} & Left        & Centre & Right        \\
				\cline{2-5}
				\multirow{3} {*} {{\color{green}Player 1}}& Up & (13,3) & (1,4)  & (7, 3)\\
				\cline{2-5}
				& Middle &(4,1)& (3,3) & (6,2) \\
				\cline{2-5}
				& Down & (-1,9) & (2,8)& (8,-1) \\
				\cline{2-5}
			\end{tabular}
		}
	\end{center}
	Denoting strict dominance by $>$, then in the order given C $>$ R,  M $>$ D, C $>$ L, M $>$ U. Thus the solution is $\langle$ Middle, Centre $\rangle$.
	
	{\color{red} Cournot Duopoly game}. Firm 1 can produce $i$ units at a cost of \euro 1 each. Similarly firm 2 can produce $j$ units at a cost of \euro 1 each. The units sell on the market at a price of \euro $[8-2(i+j)]^{+}$ each, where $[ . ]^{+}$ is the positive part of $[.]$ The payoff to firm 1 is the profit gained  which is \euro $ [8-2(i+j)]^{+}i - i$. Similarly the payoff to firm 2 is \euro $[8-2 (i+j)]^{+}j-j$. The game matrix is
	\begin{center}
		{\color{blue}
			\begin{tabular}{c|c|c|c|c|c|}
				\multicolumn{2} {c} {} & \multicolumn{4}{c} {{\color{green}Firm 2}} \\
				\cline{3-6}
				\multicolumn{2}{c|}{} & $j = 0$        & $j = 1$ & $j = 2$  &  $j = 3$    \\
				\cline{2-6}
				\multirow{4} {*} {{\color{green}Firm 1}}& $i=0$ & (0,0) & (0,5)  & (0, 6)& (0,1)\\
				\cline{2-6}
				& $i=1$ &(5,0)& (3,3) & (1,2)& (-1,-3)\\
				\cline{2-6}
				& $i=2$ & (6,0) & (2,1)& (-2,-2)& (-2,-3) \\
				\cline{2-6}
				& $i=3$ & (1,0)& (-3,-1)& (-3,-2) & (-3,-3) \\
				\cline{2-6}
			\end{tabular}
		}
	\end{center}
	The solution is $\langle i=1, j=1\rangle$. What is the sequence of eliminations that leads to this?
	
	All routes lead to the same result (proof?):
	\begin{center}
		{\color{blue}
			\begin{tabular}{c|c|c|c|}
				\multicolumn{2} {c} {} & \multicolumn{2}{c} {{\color{green}Player 2}} \\
				\cline{3-4}
				\multicolumn{2}{c|}{} & Left        & Right        \\
				\cline{2-4}
				\multirow{3} {*} {{\color{green}Player 1}}& Up & (1,-1) & (4,2)  \\
				\cline{2-4}
				& Middle &(0,2)& (3,3)  \\
				\cline{2-4}
				& Down & (-2,-2) & (2,-1) \\
				\cline{2-4}
			\end{tabular}
		}
	\end{center}
	The solution is $\langle$ Up, Right $\rangle$.
	
	\subsection{Weak Dominance} \label{M-WDS}
	\textit{Strategy X weakly dominates strategy Y for a player if X gives at least as big a payoff as Y no matter what the other players do and there is one at least one X payoff than is strictly greater than the corresponding Y payoff.}
	
	\textbf{Iterated Elimination of Weakly Dominated Strategies} (IEWDS) is in general not a valid technique. Consider the game:
	\begin{center}
		{\color{blue}
			\begin{tabular}{c|c|c|c|}
				\multicolumn{2} {c} {} & \multicolumn{2}{c} {{\color{green}Player 2}} \\
				\cline{3-4}
				\multicolumn{2}{c|}{} & Left        & Right        \\
				\cline{2-4}
				\multirow{3} {*} {{\color{green}Player 1}}& Up & (0,1) & (-4,2)  \\
				\cline{2-4}
				& Middle &(0,3)& (3,3)  \\
				\cline{2-4}
				& Down & (-2,2) & (3,-1) \\
				\cline{2-4}
			\end{tabular}
		}
	\end{center}
	If we proceed as before and denoting weak dominance by $\geq$, we  might argue that
	M $\geq$ U, L $\geq$ R. The solution is then $\langle$ Middle, Left $\rangle$. Alternatively we might argue that M $\geq$ D, R $\geq$ L which leads to the solution $\langle$ Middle, Right $\rangle$.
	



	
\newpage
	\section{Best Response \& Nash Equilibrium}
	
	{ \color{red} Stag Hunt(SH)} - it requires cooperation to catch a stag! \vspace{3mm} \\
	
	\begin{center}
		{\color{blue}
			\begin{tabular}{c|c|c|c|}
				\multicolumn{2} {c} {} & \multicolumn{2}{c} {{\color{green}Player 2}} \\
				\cline{3-4}
				\multicolumn{2}{c|}{} & Stag         & Hare       \\
				\cline{2-4}
				\multirow{2} {*} {{\color{green}Player 1}}& Stag & ($3^*,3^*$) & (0,2) \\
				\cline{2-4}
				& Hare &(2,0)& ($1^*,1^*$) \\
				\cline{2-4}
				%C & (2,6) & (4,7)& (0,8) \\
				%\hline
			\end{tabular}
		}
	\end{center}
	For this game there is no SDS nor WDS.\\
	
	A \textbf{Nash equilibrium} (NE) is a set of strategies, one for each player, from which there is no incentive for any one player to deviate if all the other players play these strategies, i.e. no player can gain by changing, also called a ``No regrets'' choice. The \emph{Best Response} of a player to another player's choice of strategy is the strategy that gives the largest or best payoff. We'll denote this by placing an $^*$ beside the payoff, e.g. in the stag game above , ``stag'' with associated payoff 3 is the best response of player 1 to player 2 playing ``stag''. Hence if both parts of a payoff pair have asterisks beside them, this must be a pure strategy \textit{Nash} equilibrium (PSNE) - a pair of strategies where both players are playing deterministic strategies as opposed to a mixed strategy \textit{Nash} equilibrium (MSNE), where players are randomly mixing between the strategies available to them.\\
	
	Thus in the stag game, the PSNE solutions are $\langle$ stag, stag $\rangle$ and $\langle$ hare, hare $\rangle$. Notice that without efficient coordination, either solution is possible.
	
	Consider the ``Good buddies prisoner's dilemma'' game:
	\begin{center}
		{\color{blue}
			\begin{tabular}{c|c|c|c|}
				\multicolumn{2} {c} {} & \multicolumn{2}{c} {{\color{green}Player 2}} \\
				\cline{3-4}
				\multicolumn{2}{c|}{} & Keep Quiet        & Confess      \\
				\cline{2-4}
				\multirow{2} {*} {{\color{green}Player 1}}& Keep Quiet & (p1,p1) & (p4,p2) \\
				\cline{2-4}
				& Confess &(p2,p4)& (p3,p3) \\
				\cline{2-4}
				%C & (2,6) & (4,7)& (0,8) \\
				%\hline
			\end{tabular}
		}
	\end{center}
	This is identical to the stag hunt game.\\
	
	An alternative definition of a NE is ``mutual best response''. Some further examples:\\
	{ \color{red} Traffic Lights(TL)}   \vspace{3mm} \\
	
	\begin{center}
		{\color{blue}
			\begin{tabular}{c|c|c|c|}
				\multicolumn{2} {c} {} & \multicolumn{2}{c} {{\color{green}Car 2}} \\
				\cline{3-4}
				\multicolumn{2}{c|}{} & Go         & Stop      \\
				\cline{2-4}
				\multirow{2} {*} {{\color{green}Car 1}}& Go & (-5,-5) & (1,0) \\
				\cline{2-4}
				& Stop &(0,1)& (-1,-1) \\
				\cline{2-4}
				%C & (2,6) & (4,7)& (0,8) \\
				%\hline
			\end{tabular}
		}
	\end{center}
	The PSNE solutions are $\langle$ go, stop $\rangle$ and $\langle$ stop, go $\rangle$.\\
	
	{\color{red} Generals, Armies \& Battles}. Each general commands 3 armies. No battle occurs if either general puts 0 armies in the field. Otherwise the general with more armies wins the day. With $i$ and$j$ standing for the number of armies of General 1 and General 2 respectively in the battlefield, one possible game matrix is
	
	\begin{center}
		{\color{blue}
			\begin{tabular}{c|c|c|c|c|c|}
				\multicolumn{2} {c} {} & \multicolumn{4}{c} {{\color{green}General 2}} \\
				\cline{3-6}
				\multicolumn{2}{c|}{} & $j= 0$        & $j = 1$ & $j = 2$  &  $j = 3$    \\
				\cline{2-6}
				\multirow{4} {*} {{\color{green}General 1}}& $i=0$ & (0,0) & (0,0)  & (0, 0)& (0,0)\\
				\cline{2-6}
				& $i=1$ &(0,0)& (0,0) & (-1,1)& (-1,1)\\
				\cline{2-6}
				& $i=2$ & (0,0) & (1,-1)& (0,0)& (-1,1) \\
				\cline{2-6}
				& $i=3$ & (0,0)& (1,-1)& (1,-1) & (0,0) \\
				\cline{2-6}
			\end{tabular}
		}
	\end{center}
	What are the PSNE(s) ?
	
	\subsection{Dominance \& NE} \label{D-NE}
	
	If IESDS results in a unique solution then it is a (unique) NE. [proof by  appeal to ``no regrets'']. IESDS does not remove any NE.\\
	
	IEWDS may lose NE. It is necessary to check using e.g. Best Responses. If you have a choice eliminate SDS before WDS. Examples:
	\begin{itemize}
		\item The example of Section \ref{M-WDS} has two NE, each obtained by a different sequence of IEWDS.
		\item Consider the game
		\begin{center}
			{\color{blue}
				\begin{tabular}{c|c|c|c|}
					\multicolumn{2} {c} {} & \multicolumn{2}{c} {{\color{green}Player 2}} \\
					\cline{3-4}
					\multicolumn{2}{c|}{} & Left        & Right      \\
					\cline{2-4}
					\multirow{2} {*} {{\color{green}Player 1}}& Up & (2,3) & (4,3) \\
					\cline{2-4}
					& Down &(3,3)& (1,1) \\
					\cline{2-4}
					%C & (2,6) & (4,7)& (0,8) \\
					%\hline
				\end{tabular}
			}
		\end{center}
		Using IEWDS, L $\geq$ R, then D $>$ U. Hence the solution is $\langle$ Down, Left $\rangle$. But using Best Responses, another NE is $\langle$ Up, Right $\rangle$.
		\item The game
		\begin{center}
			{\color{blue}
				\begin{tabular}{c|c|c|c|c|}
					\multicolumn{2} {c} {} & \multicolumn{3}{c} {{\color{green}Player 2}} \\
					\cline{3-5}
					\multicolumn{2}{c|}{} & Left        & Centre & Right        \\
					\cline{2-5}
					\multirow{3} {*} {{\color{green}Player 1}}& Up & (2,2) & (4,2)  & (4, 3)\\
					\cline{2-5}
					& Middle &(2,4)& (5,5) & (7,3) \\
					\cline{2-5}
					& Down & (3,4) & (3,7)& (6,6) \\
					\cline{2-5}
				\end{tabular}
			}
		\end{center}
		Using IEWDS C $\geq$ L, then M $>$ U, M $>$ D and C$>$ R. Hence the solution is \\ $\langle$ Middle, Centre $\rangle$. This is the only NE.
	\end{itemize}
	
	
\newpage

	\section{MB2-MSNE}
	
	{ \color{red} Matching Pennies(MP)} is an example of a game with no PSNE.  \vspace{3mm} \\
	
	\begin{center}
		{\color{blue}
			\begin{tabular}{c|c|c|c|}
				\multicolumn{2} {c} {} & \multicolumn{2}{c} {{\color{green}Player 2}} \\
				\cline{3-4}
				\multicolumn{2}{c|}{} & Heads         & Tails      \\
				\cline{2-4}
				\multirow{2} {*} {{\color{green}Player 1}}& Heads & (1,-1) & (-1,1) \\
				\cline{2-4}
				& Tails &(-1,1)& (1,-1) \\
				\cline{2-4}
				%C & (2,6) & (4,7)& (0,8) \\
				%\hline
			\end{tabular}
		}
	\end{center}
	Other names for this game are
	\begin{itemize}
		\item Goalkeeeper v. Penalty Taker
		\item Offense v. Defense (American Football)
		\item Fastball v. Curveball (Baseball)
		\item Attack A or B v. Defend A or B
	\end{itemize}
	\textit{Nash} proved that every finite \footnote{finite number of players, finite number of pure strategies} game has a NE. \\ This game has the MSNE
	$\langle$ 1/2 Heads + 1/2 Tails, 1/2 Heads + 1/2 Tails $\rangle$.\\
	
	More generally, MSNEs can be found using the \textit{Bishop-Cannings} theorem. Consider the following game:
	\begin{center}
		{\color{blue}
			\begin{tabular}{c|c|c|c|}
				\multicolumn{2} {c} {} & \multicolumn{2}{c} {{\color{green}Player 2}} \\
				\cline{3-4}
				\multicolumn{2}{c|}{} & Left         & Right     \\
				\cline{2-4}
				\multirow{2} {*} {{\color{green}Player 1}}& Up & (3,-3) & (-2,2) \\
				\cline{2-4}
				& Down &(-1,1)& (0,0) \\
				\cline{2-4}
				%C & (2,6) & (4,7)& (0,8) \\
				%\hline
			\end{tabular}
		}
	\end{center}
	Again note that there is no PSNE. We'll use the notation $R_1 \langle s_1, s_2 \rangle $ and $R_2 \langle s_1, s_2\rangle $ to stand for the payoffs to Player 1 and 2 respectively when Player 1 plays strategy $s_1$ and Player 2 strategy $s_2$.
	\\ In the mixed strategy game let $p$ be the probability that Player 1 plays Up, and similarly $q$ the probability that Player 2 plays Left. Then the payoff to player 2  by playing Left against Player 1's mixed strategy is
	$$ R_2 \langle p \textrm{Up} + (1-p)\textrm{Down}, \textrm{Left}\rangle  = -3p + 1(1-p) $$
	Again the payoff to player 2 by playing Right against Player 1's mixed strategy is
	$$ R_2 \langle p \textrm{Up} + (1-p)\textrm{Down}, \textrm{Right}\rangle = 2p + 0(1-p) $$
	If the two payoff values are different then Player 2 has a definite preference between the two and so would not want to randomise. If for instance Player 2 were to choose Left, then Player 1 would abandon his mixed strategy and choose Up instead. Similarly if Player 2 were to choose Right instead, Player 1 would change from randomising to playing Down. In either case, the mixed strategies go out the window and no NE exists. To get a MSNE requires that Player 2 be indifferent between the two payoffs i.e. requires that the two payoffs be the same and since the the payoff at any mixed strategy is a linear combination of the pure strategy payoffs, this must also have the same value. Equating the two payoffs gives
	\begin{eqnarray*}
		% \nonumber to remove numbering (before each equation)
		-3p +1(1-p) &=& 2p + 0(1-p) \\
		\Rightarrow p &=& 1/6
	\end{eqnarray*}
	and the value of the payoff is
	$$ v_2 = R_2 \left\langle \frac{1}{6} \textrm{Up} + \frac{5}{6}\textrm{Down}, \textrm{Left or Right}\right\rangle  = \frac{1}{3} $$
	Similarly the payoffs to Player 1 by playing Up (respectively Down) against Player 2's mixed strategy are
	$$ R_1 \langle \textrm{Up}, q \textrm{Left} + (1-q)\textrm{Right}\rangle = 3q + (-2)(1-q) $$
	$$ R_1 \langle \textrm{Down}, q \textrm{Left} + (1-q)\textrm{Right} \rangle = (-1)q + 0(1-q) $$
	respectively. To be indifferent between the two requires
	\begin{eqnarray*}
		% \nonumber to remove numbering (before each equation)
		3q + (-2)(1-q) &=& (-1)q + 0(1-q) \\
		\Rightarrow q &=& 1/3
	\end{eqnarray*}
	and the value of the payoff is
	$$ v_1 = R_1 \left\langle \textrm{Up or Down}, \frac{1}{3} \textrm{Left} + \frac{2}{3}\textrm{Right}\right\rangle = -\frac{1}{3} $$
	
	Exercise: Show that the stag hunt game has a MSNE at $\langle$ 1/2 stag + 1/2 hare, 1/2 stag + 1/2 hare $\rangle$.\\
	
	We can have partial MSNE where (at least) one player has a pure strategy and (at least) one player has a mixed strategy.\\
	
	Examples of games with MSNE:
	\begin{itemize}
		\item { \color{red} Chicken} (aka {\color{magenta} Snowdrift}) :  \vspace{3mm} \\
		
		\begin{center}
			{\color{blue}
				\begin{tabular}{c|c|c|c|}
					\multicolumn{2} {c} {} & \multicolumn{2}{c} {{\color{green}Player 2}} \\
					\cline{3-4}
					\multicolumn{2}{c|}{} & Continue {\color{magenta}  / Stay in car }       & Swerve   {\color{magenta}  / Shovel }   \\
					\cline{2-4}
					\multirow{2} {*} {{\color{green}Player 1}}& Continue {\color{magenta}  / Stay in car }& (-10,-10) & (2,-2) \\
					\cline{2-4}
					& Swerve {\color{magenta}  / Shovel } &(-2,2)& (0,0) \\
					\cline{2-4}
					%C & (2,6) & (4,7)& (0,8) \\
					%\hline
				\end{tabular}
			}
		\end{center}
		PSNEs occur at $\langle$ Continue, Swerve $\rangle$ and at $\langle$ Swerve, Continue $\rangle$. There is also a MSNE at $\langle$ 1/5 continue + 4/5 swerve, 1/5 continue + 4/5 swerve $\rangle$. Show that $v_1 = v_2 = -2/5$.
		
		\item { \color{red} Battle of the Sexes}:  \vspace{3mm} \\
		
		\begin{center}
			{\color{blue}
				\begin{tabular}{c|c|c|c|}
					\multicolumn{2} {c} {} & \multicolumn{2}{c} {{\color{green}Her}} \\
					\cline{3-4}
					\multicolumn{2}{c|}{} & Ballet         & Fight      \\
					\cline{2-4}
					\multirow{2} {*} {{\color{green}Him}}& Ballet & (1,2) & (-1,1) \\
					\cline{2-4}
					& Fight &(-1,1)& (2,1) \\
					\cline{2-4}
					%C & (2,6) & (4,7)& (0,8) \\
					%\hline
				\end{tabular}
			}
		\end{center}
		PSNEs occur at $\langle$ Ballet, Ballet $\rangle$ and at $\langle$ Fight, Fight $\rangle$. Show that there is a MSNE at $\langle$ 1/3 ballet + 2/3 fight, 2/3 ballet + 1/3 fight $\rangle$ with $v_1= 2/3 = v_2$. Compare and contast this with the payoffs of the PSNEs.
	\end{itemize}
	\subsection{MSNE and Dominance}
	
	A SDS cannot be played with positive probability in a MSNE (otherwise a higher payoff can be obtained by not playing the SDS when the strategy says to play it).
	
\bigskip

	\subsection{Strict Dominance in Mixed Strategies}
	Consider the game:
	\begin{center}
		{\color{blue}
			\begin{tabular}{c|c|c|c|}
				\multicolumn{2} {c} {} & \multicolumn{2}{c} {{\color{green}Player 2}} \\
				\cline{3-4}
				\multicolumn{2}{c|}{} & Left        & Right        \\
				\cline{2-4}
				\multirow{3} {*} {{\color{green}Player 1}}& Up & (3,-1) & (-1,1)  \\
				\cline{2-4}
				& Middle &(0,0)& (0,0)  \\
				\cline{2-4}
				& Down & (-1,2) & (2,-1) \\
				\cline{2-4}
			\end{tabular}
		}
	\end{center}
	This game has no SDS dominated by a pure strategy nor any PSNE. If a mixture of two pure strategies dominates another, then that strategy is a SDS.\\ In the above game 1/2 Up + 1/2 Down $>$ Middle. Remove Middle to get
	\begin{center}
		{\color{blue}
			\begin{tabular}{c|c|c|c|}
				\multicolumn{2} {c} {} & \multicolumn{2}{c} {{\color{green}Player 2}} \\
				\cline{3-4}
				\multicolumn{2}{c|}{} & Left         & Right      \\
				\cline{2-4}
				\multirow{2} {*} {{\color{green}Player 1}}& Up & (3,-1) & (-1,1) \\
				\cline{2-4}
				& Down &(-1,2)& (2,-1) \\
				\cline{2-4}
				%C & (2,6) & (4,7)& (0,8) \\
				%\hline
			\end{tabular}
		}
	\end{center}
	Show that a MSNE exists at $\langle$ (3/5)Up + (2/5)Down, (3/7)Left + (4/7)Right $\rangle$.\\
	
	Another example:
	\begin{center}
		{\color{blue}
			\begin{tabular}{c|c|c|c|c|}
				\multicolumn{2} {c} {} & \multicolumn{3}{c} {{\color{green}Player 2}} \\
				\cline{3-5}
				\multicolumn{2}{c|}{} & Left        & Centre & Right        \\
				\cline{2-5}
				\multirow{3} {*} {{\color{green}Player 1}}& Up & (-3,6) & (9,1)  & (0, 2)\\
				\cline{2-5}
				& Middle &(3,-4)& (2,4) & (4,1) \\
				\cline{2-5}
				& Down & (4,7) & (3,2)& (-3,2) \\
				\cline{2-5}
			\end{tabular}
		}
	\end{center}
	Using IESDS, we get (1/4)L + (3/4)C $>$ R, D $>$ M, L $>$ C, D $>$ U. Thus the solution is $\langle$ Down, Left $\rangle$.
	

	
	
	\newpage
	\subsection{Atypical Matrix Games}
	Almost all matrix games have an odd number of solutions (Wilson 1971). Examples of non generic games follow. Weak dominance is usually the culprit.
	\begin{itemize}
		\item \textit{}
		\begin{center}
			{\color{blue}
				\begin{tabular}{c|c|c|c|}
					\multicolumn{2} {c} {} & \multicolumn{2}{c} {{\color{green}Player 2}} \\
					\cline{3-4}
					\multicolumn{2}{c|}{} & Left         & Right      \\
					\cline{2-4}
					\multirow{2} {*} {{\color{green}Player 1}}& Up & (1,1) & (0,0) \\
					\cline{2-4}
					& Down &(0,0)& (0,0) \\
					\cline{2-4}
					%C & (2,6) & (4,7)& (0,8) \\
					%\hline
				\end{tabular}
			}
		\end{center}
		There are two PSNEs.
		\item\textit{}
		\begin{center}
			{\color{blue}
				\begin{tabular}{c|c|c|c|}
					\multicolumn{2} {c} {} & \multicolumn{2}{c} {{\color{green}Player 2}} \\
					\cline{3-4}
					\multicolumn{2}{c|}{} & Left         & Right      \\
					\cline{2-4}
					\multirow{2} {*} {{\color{green}Player 1}}& Up & (2,2) & (9,0) \\
					\cline{2-4}
					& Down &(2,3)& (5,-1) \\
					\cline{2-4}
					%C & (2,6) & (4,7)& (0,8) \\
					%\hline
				\end{tabular}
			}
		\end{center}
		Using IESDS, L $>$ R. Player 1 can choose Up or Down as pure strategies or any mixture of the two. All strategies yield a payoff of 2: an infinite number of strategies.\\
		If tempted to use IEWDS, U $\geq$ D, leading to the solution $\langle$ Up, Right $\rangle$. Caveat emptor!
		\item Example 2 of Section \ref{D-NE} had two PSNEs. It also has an infinite number of MSNEs at $\langle$ Up, $q$Left + $(1-q)$Right $\rangle$ for $ q \leq 3/4$.
		\item { \color{red} \textit{Selten}'s Horse}:  \vspace{3mm} \\
		
		\begin{center}
			{\color{blue}
				\begin{tabular}{c|c|c|c|}
					\multicolumn{2} {c} {} & \multicolumn{2}{c} {{\color{green}Player 2}} \\
					\cline{3-4}
					\multicolumn{2}{c|}{} & Left         & Right      \\
					\cline{2-4}
					\multirow{2} {*} {{\color{green}Player 1}}& Up & (3,1) & (0,0) \\
					\cline{2-4}
					& Down &(2,2)& (2,2) \\
					\cline{2-4}
					%C & (2,6) & (4,7)& (0,8) \\
					%\hline
				\end{tabular}
			}
		\end{center}
		There are two PSNEs at $\langle$ Up,Left $\rangle$ and $\langle$ Down,Right $\rangle$ and an infinite number of MSNEs at $\langle$ Down, $q$Left + $(1-q)$Right $\rangle$ for $ q \leq 2/3$.
		
		\item { \color{red} Take or Share} - TV game show:  \vspace{3mm} \\
		
		\begin{center}
			{\color{blue}
				\begin{tabular}{c|c|c|c|}
					\multicolumn{2} {c} {} & \multicolumn{2}{c} {{\color{green}Player 2}} \\
					\cline{3-4}
					\multicolumn{2}{c|}{} & Share         & Take     \\
					\cline{2-4}
					\multirow{2} {*} {{\color{green}Player 1}}& Share & (4,4) & (0,8) \\
					\cline{2-4}
					& Take &(8,0)& (0,0) \\
					\cline{2-4}
					%C & (2,6) & (4,7)& (0,8) \\
					%\hline
				\end{tabular}
			}
		\end{center}
		There are three PSNEs at all but $\langle$ Share, Share $\rangle$.
		If one player chooses Take, then the other is indifferent between Share and Take which leads to an infinity of MSNEs.
	\end{itemize}
	
	\section{Minimax Game Solution}
	An alternative to ``solving '' matrix games using the concept of \textit{Nash} Equilibrium is the Minimax approach. It was originally devised for 2-player ``Zero Sum'' or ``Constant Sum'' games whereby what one player gains the other player loses. Each player attempts to maximise hir payoff assuming that hir opponent is attempting to minimise it.\\
	
	Consider the constant sum game
\end{document}
