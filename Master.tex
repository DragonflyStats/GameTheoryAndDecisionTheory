

\section{Class Notes}

- Class Notes 1 (Matrix Games & Nash Equilibria)
- Class Notes 2 (Minimax Solutions)
- Class Notes 3 (Cournot & Stackelberg Duopoly)
- Class Notes 3a (Homogeneous Goods: Example)
- Class Notes 3b (Heterogeneous Goods: Example)

%===========================================%

% - https://ocw.mit.edu/courses/economics/14-12-economic-applications-of-game-theory-fall-2012/lecture-notes/
% -  



Game theory is "the study of mathematical models of conflict and cooperation between intelligent rational decision-makers." Game theory is mainly used in economics, political science, and psychology, as well as logic, computer science and biology.




Using game theory, real-world scenarios for such situations as pricing competition and 
product releases (and many more) can be laid out and their outcomes predicted. 
Companies that use (and stick to) this device to determine the Nash Equilibrium see a 
huge benefit in their budgeting strategies.

%----------------------------------------%
\section{Whose Turn Is It?}
While sequential games are played by turn, simultaneous games are played with each player making their decision at the same time. With simultaneous games, we no longer use the common introductory method of backward induction. Proponents of game theory often tabulate the different outcomes in what is called a matrix (shown below).

Player one / Player two	Left	Right
Up	(1, 3)	(4, 2)
Down	(3, 2)	(3, 1)
This matrix is referred to as normal form. Player one's choices are shown on the left vertical axis and player two's choices are shown on the top horizontal axis. The payoffs for each player are in their corresponding intersections and are displayed as follows (player one, player two).


%%- http://policonomics.com/lp-game-theory2-dominant-strategy/

Game theory II: Dominant strategies

Summary

In this LP we learn everything there is about simultaneous games. 

These games, used when considering a game where players move or play their strategies simultaneously, are commonly used in many fields. From military strategies to collusion agreements, the analysis of these situations as simultaneous games can help us discover the best way to act.

Simultaneous games
Nash equilibria and dominant strategies:

Prisoner’s dilemma
Nash equilibrium
Dominant strategies
Mixed strategies:

Battle of the sexes
Mixed strategies
Continuous strategies:

Cournot duopoly
Dominant strategies are considered as better than other strategies, no matter what other players might do. In game theory, there are two kinds of strategic dominance:

-a strictly dominant strategy is that strategy that always provides greater utility to a the player, no matter what the other player’s strategy is;

-a weakly dominant strategy is that strategy that provides at least the same utility for all the other player’s strategies, and strictly greater for some strategy.

 

Prisoner's dilemma - Nash and Pareto equilibriaA dominant strategy equilibrium is reached when each player chooses their own dominant strategy. In the prisoner’s dilemma, the dominant strategy for both players is to confess, which means that confess-confess is the dominant strategy equilibrium (underlined in red), even if this equilibrium is not a Pareto optimal equilibrium (underlined in green).

It must be noted that any dominant strategy equilibrium is always a Nash equilibrium. However, not all Nash equilibria are dominant strategy equilibria.

 
\section{IEDS}
The elimination of dominated strategies is commonly used to simplify the analysis of any game. The way to proceed is to eliminate for each player every strategy that seems ‘unreasonable’, which will greatly reduce the number of equilibria. This method is quite easy to use when only strictly dominated strategies are in place, but the elimination of weakly dominated strategies can turn problematic, ending up with a game that does not resembles the original one from a strategic point of view.

\section{Bismarck Sea}
Battle of the Bismarck Sea - Game matrix

A good example of elimination of dominated strategy is the analysis of the Battle of the Bismarck Sea. In this game, as depicted in the adjacent game matrix, Kenney has no dominant strategy (the sum of the payoffs of the first strategy equals the sum of the second strategy), but the Japanese do have a weakly dominating strategy, which is to go North (the payoffs are equal for one strategy but strictly better for the other). Since only one of them has a dominant strategy, there is no dominant strategy equilibrium. We must then proceed by eliminating dominated strategies. As we’ve already mentioned, for the Japanese strategy ‘go North’ weakly dominates strategy ‘go South’. Therefore, we eliminate the strategy ‘go South’ for the Japanese, who will go North. Now that we only consider the Japanese going North, Kenney’s strategy ‘go North’ is strictly dominant over strategy ‘go South’, which will be eliminated. Therefore, North-North is the weak-dominance equilibrium.

However, as it was mentioned before, it can be easily seen that the game has lost its strategic nature.

Now, what if there isn’t a clear equilibrium? The Battle of the Sexes, as we’ll see next, is a good example of this particular case.
	Game theory II: Nash equilibria Game theory II: Battle of the sexes 
Game theory
Nash equilibrium
John Nash
Prisoner’s dilemma
Battle of the sexes
 
Game theory III
 
D.3 Prisoner’s dilemma
D.6 Battle of the sexes
D.5 Dominant strategies and Nash equilibrium



