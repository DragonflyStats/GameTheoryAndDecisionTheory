%This is a latex file

\documentclass[12pt]{article}
\usepackage{graphicx}
\usepackage{amsmath,amsfonts,amssymb}
\usepackage{color}
\usepackage{multirow}
\usepackage{eurosym}
\setlength{\oddsidemargin}{0.0mm} \setlength{\textwidth}{160.0mm}
\setlength{\topmargin}{-10.0mm} \setlength{\textheight}{250mm}
\setlength{\parindent}{0mm} \setlength{\parskip}{2mm}
\pagestyle{empty}
%\input def.tex
%\input dsdef.tex
\input rgb.tex

%\newcommand \la{\lambda}
%\newcommand \al{a}
%\newcommand \be{b}
\newcommand \x{\overline{x}}
\newcommand \y{\overline{y}}

\begin{document}
\begin{center}
\textbf{Game Theory %(  au Spaniel)
}
\end{center}
Game Theory (GT) is the study of strategic interdependence. The typical ``game'' consists of players, actions, strategies and payoffs. The standard modes of analysis are once-played games in either
\begin{enumerate}
\item[(i)] matrix or strategic form where players' choice of actions are made simultaneously,
\item[] \hspace*{50mm} or
\item[(ii)] extensive form where choice of actions are made sequentially.
\end{enumerate}

\section{Analysis of (finite) Matrix Games}
One of the best known games is \\

{ \color{red} Prisoner's Dilemma (PD)} \vspace{3mm} \\

\begin{center}
{\color{blue}
\begin{tabular}{c|c|c|c|}
  \multicolumn{2} {c} {} & \multicolumn{2}{c} {{\color{green}Player 2}} \\
\cline{3-4}
\multicolumn{2}{c|}{} & Keep Quiet         & Confess        \\
 \cline{2-4}
\multirow{2} {*} {{\color{green}Player 1}}& Keep Quiet & (-1,-1) & (-12,0) \\
 \cline{2-4}
& Confess &(0,-12)& (-8,-8) \\
\cline{2-4}
%C & (2,6) & (4,7)& (0,8) \\
%\hline
\end{tabular}
}
\end{center}

The solution is $\langle$ confess, confess $\rangle$. (See IESDS). \\ Why don't the players coordinate to get $\langle$ Keep Quiet, Keep Quiet $\rangle$ ?

The exact payoffs are irrelevant, the game can also be represented by the order of players' preferences  - most preferred (p1) to least preferred (p4) :
\begin{center}
{\color{blue}
\begin{tabular}{c|c|c|c|}
  \multicolumn{2} {c} {} & \multicolumn{2}{c} {{\color{green}Player 2}} \\
\cline{3-4}
\multicolumn{2}{c|}{} & Keep Quiet         & Confess        \\
 \cline{2-4}
\multirow{2} {*} {{\color{green}Player 1}}& Keep Quiet & (p2,p2) & (p4,p1) \\
 \cline{2-4}
& Confess &(p1,p4)& (p3,p3) \\
\cline{2-4}
%C & (2,6) & (4,7)& (0,8) \\
%\hline
\end{tabular}
}
\end{center}

Other examples of PD-like games are
\begin{itemize}
  \item {\color{red} War} with strategies {\color{blue} Defend, Attack} respectively.
  \item {\color{red} Arms Race} with strategies {\color{blue} Pass, Build} respectively.
  \item {\color{red} Free Trade/ Protection} with strategies {\color{blue} No Tax, Tax} respectively.
  \item {\color{red} Advertising} with strategies {\color{blue} No Ads, Ads} respectively.
\end{itemize}

{ \color{red}Deadlock} is another game ( success is to fail!): \vspace{3mm} \\

\begin{center}
{\color{blue}
\begin{tabular}{c|c|c|c|}
  \multicolumn{2} {c} {} & \multicolumn{2}{c} {{\color{green}Player 2}} \\
\cline{3-4}
\multicolumn{2}{c|}{} & Try         & Fail       \\
 \cline{2-4}
\multirow{2} {*} {{\color{green}Player 1}}& Try & (0,0) & (-1,1) \\
 \cline{2-4}
& Fail &(1,-1)& (0,0) \\
\cline{2-4}
%C & (2,6) & (4,7)& (0,8) \\
%\hline
\end{tabular}
}
\end{center}

Using preferences, we can consider the more general version of deadlock
\begin{center}
{\color{blue}
\begin{tabular}{c|c|c|c|}
  \multicolumn{2} {c} {} & \multicolumn{2}{c} {{\color{green}Player 2}} \\
\cline{3-4}
\multicolumn{2}{c|}{} & Left        & Right        \\
 \cline{2-4}
\multirow{2} {*} {{\color{green}Player 1}}& Up & (p2,p2) & (p1,p4) \\
 \cline{2-4}
& Down &(p4,p1)& (p3,p3) \\
\cline{2-4}
%C & (2,6) & (4,7)& (0,8) \\
%\hline
\end{tabular}
}
\end{center}
The solution is $\langle$ Up, Left $\rangle$. (See IESDS). Neither player gets hir first choice unless the other makes a mistake.

\subsection{Strict Dominance}

\textit{Strategy X strictly dominates strategy Y for a player if X gives a bigger (more preferred) payoff than Y no matter what the other players do. Players never rationally choose strictly dominated strategies.}

Reduce the matrix by \textbf{Iterated Elimination of Strictly Dominated Strategies} (IESDS) - see PD and Deadlock above. The order of elimination is irrelevant. Another example is the {\color{red} Dance Club} game:
\begin{center}
{\color{blue}
\begin{tabular}{c|c|c|c|}
  \multicolumn{2} {c} {} & \multicolumn{2}{c} {{\color{green}Boon docks}} \\
\cline{3-4}
\multicolumn{2}{c|}{} &   Salsa       &  Hip Hop       \\
 \cline{2-4}
\multirow{2} {*} {{\color{green}Downtown}}& Salsa & (80,0) & (60,40) \\
 \cline{2-4}
& Hip Hop &(40,60)& (40,0) \\
\cline{2-4}
%C & (2,6) & (4,7)& (0,8) \\
%\hline
\end{tabular}
}
\end{center}
The solution is $\langle$ Salsa, Hip Hop $\rangle$. Salsa dominates Hip Hop for Club Downtown, then Boonies choses Hip Hop.

Some more examples

\begin{center}
{\color{blue}
\begin{tabular}{c|c|c|c|c|}
  \multicolumn{2} {c} {} & \multicolumn{3}{c} {{\color{green}Player 2}} \\
\cline{3-5}
\multicolumn{2}{c|}{} & Left        & Centre & Right        \\
 \cline{2-5}
\multirow{3} {*} {{\color{green}Player 1}}& Up & (13,3) & (1,4)  & (7, 3)\\
 \cline{2-5}
& Middle &(4,1)& (3,3) & (6,2) \\
\cline{2-5}
& Down & (-1,9) & (2,8)& (8,-1) \\
\cline{2-5}
\end{tabular}
}
\end{center}
Denoting strict dominance by $>$, then in the order given C $>$ R,  M $>$ D, C $>$ L, M $>$ U. Thus the solution is $\langle$ Middle, Centre $\rangle$.

{\color{red} Cournot Duopoly game}. Firm 1 can produce $i$ units at a cost of \euro 1 each. Similarly firm 2 can produce $j$ units at a cost of \euro 1 each. The units sell on the market at a price of \euro $[8-2(i+j)]^{+}$ each, where $[ . ]^{+}$ is the positive part of $[.]$ The payoff to firm 1 is the profit gained  which is \euro $ [8-2(i+j)]^{+}i - i$. Similarly the payoff to firm 2 is \euro $[8-2 (i+j)]^{+}j-j$. The game matrix is
\begin{center}
{\color{blue}
\begin{tabular}{c|c|c|c|c|c|}
  \multicolumn{2} {c} {} & \multicolumn{4}{c} {{\color{green}Firm 2}} \\
\cline{3-6}
\multicolumn{2}{c|}{} & $j = 0$        & $j = 1$ & $j = 2$  &  $j = 3$    \\
 \cline{2-6}
\multirow{4} {*} {{\color{green}Firm 1}}& $i=0$ & (0,0) & (0,5)  & (0, 6)& (0,1)\\
 \cline{2-6}
& $i=1$ &(5,0)& (3,3) & (1,2)& (-1,-3)\\
\cline{2-6}
& $i=2$ & (6,0) & (2,1)& (-2,-2)& (-2,-3) \\
\cline{2-6}
& $i=3$ & (1,0)& (-3,-1)& (-3,-2) & (-3,-3) \\
\cline{2-6}
\end{tabular}
}
\end{center}
The solution is $\langle i=1, j=1\rangle$. What is the sequence of eliminations that leads to this?

All routes lead to the same result (proof?):
\begin{center}
{\color{blue}
\begin{tabular}{c|c|c|c|}
  \multicolumn{2} {c} {} & \multicolumn{2}{c} {{\color{green}Player 2}} \\
\cline{3-4}
\multicolumn{2}{c|}{} & Left        & Right        \\
 \cline{2-4}
\multirow{3} {*} {{\color{green}Player 1}}& Up & (1,-1) & (4,2)  \\
 \cline{2-4}
& Middle &(0,2)& (3,3)  \\
\cline{2-4}
& Down & (-2,-2) & (2,-1) \\
\cline{2-4}
\end{tabular}
}
\end{center}
The solution is $\langle$ Up, Right $\rangle$.

\subsection{Weak Dominance} \label{M-WDS}
\textit{Strategy X weakly dominates strategy Y for a player if X gives at least as big a payoff as Y no matter what the other players do and there is one at least one X payoff than is strictly greater than the corresponding Y payoff.}

\textbf{Iterated Elimination of Weakly Dominated Strategies} (IEWDS) is in general not a valid technique. Consider the game:
\begin{center}
{\color{blue}
\begin{tabular}{c|c|c|c|}
  \multicolumn{2} {c} {} & \multicolumn{2}{c} {{\color{green}Player 2}} \\
\cline{3-4}
\multicolumn{2}{c|}{} & Left        & Right        \\
 \cline{2-4}
\multirow{3} {*} {{\color{green}Player 1}}& Up & (0,1) & (-4,2)  \\
 \cline{2-4}
& Middle &(0,3)& (3,3)  \\
\cline{2-4}
& Down & (-2,2) & (3,-1) \\
\cline{2-4}
\end{tabular}
}
\end{center}
If we proceed as before and denoting weak dominance by $\geq$, we  might argue that
 M $\geq$ U, L $\geq$ R. The solution is then $\langle$ Middle, Left $\rangle$. Alternatively we might argue that M $\geq$ D, R $\geq$ L which leads to the solution $\langle$ Middle, Right $\rangle$.

\subsection{Best Response \& Nash Equilibrium}

{ \color{red} Stag Hunt(SH)} - it requires cooperation to catch a stag! \vspace{3mm} \\

\begin{center}
{\color{blue}
\begin{tabular}{c|c|c|c|}
  \multicolumn{2} {c} {} & \multicolumn{2}{c} {{\color{green}Player 2}} \\
\cline{3-4}
\multicolumn{2}{c|}{} & Stag         & Hare       \\
 \cline{2-4}
\multirow{2} {*} {{\color{green}Player 1}}& Stag & ($3^*,3^*$) & (0,2) \\
 \cline{2-4}
& Hare &(2,0)& ($1^*,1^*$) \\
\cline{2-4}
%C & (2,6) & (4,7)& (0,8) \\
%\hline
\end{tabular}
}
\end{center}
For this game there is no SDS nor WDS.\\

A \textbf{Nash equilibrium} (NE) is a set of strategies, one for each player, from which there is no incentive for any one player to deviate if all the other players play these strategies, i.e. no player can gain by changing, also called a ``No regrets'' choice. The \emph{Best Response} of a player to another player's choice of strategy is the strategy that gives the largest or best payoff. We'll denote this by placing an $^*$ beside the payoff, e.g. in the stag game above , ``stag'' with associated payoff 3 is the best response of player 1 to player 2 playing ``stag''. Hence if both parts of a payoff pair have asterisks beside them, this must be a pure strategy \textit{Nash} equilibrium (PSNE) - a pair of strategies where both players are playing deterministic strategies as opposed to a mixed strategy \textit{Nash} equilibrium (MSNE), where players are randomly mixing between the strategies available to them.\\

Thus in the stag game, the PSNE solutions are $\langle$ stag, stag $\rangle$ and $\langle$ hare, hare $\rangle$. Notice that without efficient coordination, either solution is possible.

Consider the ``Good buddies prisoner's dilemma'' game:
\begin{center}
{\color{blue}
\begin{tabular}{c|c|c|c|}
  \multicolumn{2} {c} {} & \multicolumn{2}{c} {{\color{green}Player 2}} \\
\cline{3-4}
\multicolumn{2}{c|}{} & Keep Quiet        & Confess      \\
 \cline{2-4}
\multirow{2} {*} {{\color{green}Player 1}}& Keep Quiet & (p1,p1) & (p4,p2) \\
 \cline{2-4}
& Confess &(p2,p4)& (p3,p3) \\
\cline{2-4}
%C & (2,6) & (4,7)& (0,8) \\
%\hline
\end{tabular}
}
\end{center}
This is identical to the stag hunt game.\\

An alternative definition of a NE is ``mutual best response''. Some further examples:\\
{ \color{red} Traffic Lights(TL)}   \vspace{3mm} \\

\begin{center}
{\color{blue}
\begin{tabular}{c|c|c|c|}
  \multicolumn{2} {c} {} & \multicolumn{2}{c} {{\color{green}Car 2}} \\
\cline{3-4}
\multicolumn{2}{c|}{} & Go         & Stop      \\
 \cline{2-4}
\multirow{2} {*} {{\color{green}Car 1}}& Go & (-5,-5) & (1,0) \\
 \cline{2-4}
 & Stop &(0,1)& (-1,-1) \\
\cline{2-4}
%C & (2,6) & (4,7)& (0,8) \\
%\hline
\end{tabular}
}
\end{center}
The PSNE solutions are $\langle$ go, stop $\rangle$ and $\langle$ stop, go $\rangle$.\\

{\color{red} Generals, Armies \& Battles}. Each general commands 3 armies. No battle occurs if either general puts 0 armies in the field. Otherwise the general with more armies wins the day. With $i$ and$j$ standing for the number of armies of General 1 and General 2 respectively in the battlefield, one possible game matrix is

\begin{center}
{\color{blue}
\begin{tabular}{c|c|c|c|c|c|}
  \multicolumn{2} {c} {} & \multicolumn{4}{c} {{\color{green}General 2}} \\
\cline{3-6}
\multicolumn{2}{c|}{} & $j= 0$        & $j = 1$ & $j = 2$  &  $j = 3$    \\
 \cline{2-6}
\multirow{4} {*} {{\color{green}General 1}}& $i=0$ & (0,0) & (0,0)  & (0, 0)& (0,0)\\
 \cline{2-6}
& $i=1$ &(0,0)& (0,0) & (-1,1)& (-1,1)\\
\cline{2-6}
& $i=2$ & (0,0) & (1,-1)& (0,0)& (-1,1) \\
\cline{2-6}
& $i=3$ & (0,0)& (1,-1)& (1,-1) & (0,0) \\
\cline{2-6}
\end{tabular}
}
\end{center}
What are the PSNE(s) ?

\subsection{Dominance \& NE} \label{D-NE}

If IESDS results in a unique solution then it is a (unique) NE. [proof by  appeal to ``no regrets'']. IESDS does not remove any NE.\\

IEWDS may lose NE. It is necessary to check using e.g. Best Responses. If you have a choice eliminate SDS before WDS. Examples:
\begin{itemize}
  \item The example of Section \ref{M-WDS} has two NE, each obtained by a different sequence of IEWDS.
  \item Consider the game
  \begin{center}
{\color{blue}
\begin{tabular}{c|c|c|c|}
  \multicolumn{2} {c} {} & \multicolumn{2}{c} {{\color{green}Player 2}} \\
\cline{3-4}
\multicolumn{2}{c|}{} & Left        & Right      \\
 \cline{2-4}
\multirow{2} {*} {{\color{green}Player 1}}& Up & (2,3) & (4,3) \\
 \cline{2-4}
 & Down &(3,3)& (1,1) \\
\cline{2-4}
%C & (2,6) & (4,7)& (0,8) \\
%\hline
\end{tabular}
}
\end{center}
Using IEWDS, L $\geq$ R, then D $>$ U. Hence the solution is $\langle$ Down, Left $\rangle$. But using Best Responses, another NE is $\langle$ Up, Right $\rangle$.
  \item The game
  \begin{center}
{\color{blue}
\begin{tabular}{c|c|c|c|c|}
  \multicolumn{2} {c} {} & \multicolumn{3}{c} {{\color{green}Player 2}} \\
\cline{3-5}
\multicolumn{2}{c|}{} & Left        & Centre & Right        \\
 \cline{2-5}
\multirow{3} {*} {{\color{green}Player 1}}& Up & (2,2) & (4,2)  & (4, 3)\\
 \cline{2-5}
& Middle &(2,4)& (5,5) & (7,3) \\
\cline{2-5}
& Down & (3,4) & (3,7)& (6,6) \\
\cline{2-5}
\end{tabular}
}
\end{center}
Using IEWDS C $\geq$ L, then M $>$ U, M $>$ D and C$>$ R. Hence the solution is \\ $\langle$ Middle, Centre $\rangle$. This is the only NE.
\end{itemize}



\subsection{MSNE}

{ \color{red} Matching Pennies(MP)} is an example of a game with no PSNE.  \vspace{3mm} \\

\begin{center}
{\color{blue}
\begin{tabular}{c|c|c|c|}
  \multicolumn{2} {c} {} & \multicolumn{2}{c} {{\color{green}Player 2}} \\
\cline{3-4}
\multicolumn{2}{c|}{} & Heads         & Tails      \\
 \cline{2-4}
\multirow{2} {*} {{\color{green}Player 1}}& Heads & (1,-1) & (-1,1) \\
 \cline{2-4}
 & Tails &(-1,1)& (1,-1) \\
\cline{2-4}
%C & (2,6) & (4,7)& (0,8) \\
%\hline
\end{tabular}
}
\end{center}
 Other names for this game are
 \begin{itemize}
   \item Goalkeeeper v. Penalty Taker
   \item Offense v. Defense (American Football)
   \item Fastball v. Curveball (Baseball)
   \item Attack A or B v. Defend A or B
 \end{itemize}
 \textit{Nash} proved that every finite \footnote{finite number of players, finite number of pure strategies} game has a NE. \\ This game has the MSNE
 $\langle$ 1/2 Heads + 1/2 Tails, 1/2 Heads + 1/2 Tails $\rangle$.\\

 More generally, MSNEs can be found using the \textit{Bishop-Cannings} theorem. Consider the following game:
 \begin{center}
{\color{blue}
\begin{tabular}{c|c|c|c|}
  \multicolumn{2} {c} {} & \multicolumn{2}{c} {{\color{green}Player 2}} \\
\cline{3-4}
\multicolumn{2}{c|}{} & Left         & Right     \\
 \cline{2-4}
\multirow{2} {*} {{\color{green}Player 1}}& Up & (3,-3) & (-2,2) \\
 \cline{2-4}
 & Down &(-1,1)& (0,0) \\
\cline{2-4}
%C & (2,6) & (4,7)& (0,8) \\
%\hline
\end{tabular}
}
\end{center}
Again note that there is no PSNE. We'll use the notation $R_1 \langle s_1, s_2 \rangle $ and $R_2 \langle s_1, s_2\rangle $ to stand for the payoffs to Player 1 and 2 respectively when Player 1 plays strategy $s_1$ and Player 2 strategy $s_2$.
\\ In the mixed strategy game let $p$ be the probability that Player 1 plays Up, and similarly $q$ the probability that Player 2 plays Left. Then the payoff to player 2  by playing Left against Player 1's mixed strategy is
$$ R_2 \langle p \textrm{Up} + (1-p)\textrm{Down}, \textrm{Left}\rangle  = -3p + 1(1-p) $$
Again the payoff to player 2 by playing Right against Player 1's mixed strategy is
$$ R_2 \langle p \textrm{Up} + (1-p)\textrm{Down}, \textrm{Right}\rangle = 2p + 0(1-p) $$
If the two payoff values are different then Player 2 has a definite preference between the two and so would not want to randomise. If for instance Player 2 were to choose Left, then Player 1 would abandon his mixed strategy and choose Up instead. Similarly if Player 2 were to choose Right instead, Player 1 would change from randomising to playing Down. In either case, the mixed strategies go out the window and no NE exists. To get a MSNE requires that Player 2 be indifferent between the two payoffs i.e. requires that the two payoffs be the same and since the the payoff at any mixed strategy is a linear combination of the pure strategy payoffs, this must also have the same value. Equating the two payoffs gives
\begin{eqnarray*}
% \nonumber to remove numbering (before each equation)
  -3p +1(1-p) &=& 2p + 0(1-p) \\
  \Rightarrow p &=& 1/6
\end{eqnarray*}
and the value of the payoff is
$$ v_2 = R_2 \left\langle \frac{1}{6} \textrm{Up} + \frac{5}{6}\textrm{Down}, \textrm{Left or Right}\right\rangle  = \frac{1}{3} $$
Similarly the payoffs to Player 1 by playing Up (respectively Down) against Player 2's mixed strategy are
$$ R_1 \langle \textrm{Up}, q \textrm{Left} + (1-q)\textrm{Right}\rangle = 3q + (-2)(1-q) $$
$$ R_1 \langle \textrm{Down}, q \textrm{Left} + (1-q)\textrm{Right} \rangle = (-1)q + 0(1-q) $$
respectively. To be indifferent between the two requires
\begin{eqnarray*}
% \nonumber to remove numbering (before each equation)
  3q + (-2)(1-q) &=& (-1)q + 0(1-q) \\
  \Rightarrow q &=& 1/3
\end{eqnarray*}
and the value of the payoff is
$$ v_1 = R_1 \left\langle \textrm{Up or Down}, \frac{1}{3} \textrm{Left} + \frac{2}{3}\textrm{Right}\right\rangle = -\frac{1}{3} $$

Exercise: Show that the stag hunt game has a MSNE at $\langle$ 1/2 stag + 1/2 hare, 1/2 stag + 1/2 hare $\rangle$.\\

We can have partial MSNE where (at least) one player has a pure strategy and (at least) one player has a mixed strategy.\\

Examples of games with MSNE:
\begin{itemize}
  \item { \color{red} Chicken} (aka {\color{magenta} Snowdrift}) :  \vspace{3mm} \\

\begin{center}
{\color{blue}
\begin{tabular}{c|c|c|c|}
  \multicolumn{2} {c} {} & \multicolumn{2}{c} {{\color{green}Player 2}} \\
\cline{3-4}
\multicolumn{2}{c|}{} & Continue {\color{magenta}  / Stay in car }       & Swerve   {\color{magenta}  / Shovel }   \\
 \cline{2-4}
\multirow{2} {*} {{\color{green}Player 1}}& Continue {\color{magenta}  / Stay in car }& (-10,-10) & (2,-2) \\
 \cline{2-4}
 & Swerve {\color{magenta}  / Shovel } &(-2,2)& (0,0) \\
\cline{2-4}
%C & (2,6) & (4,7)& (0,8) \\
%\hline
\end{tabular}
}
\end{center}
PSNEs occur at $\langle$ Continue, Swerve $\rangle$ and at $\langle$ Swerve, Continue $\rangle$. There is also a MSNE at $\langle$ 1/5 continue + 4/5 swerve, 1/5 continue + 4/5 swerve $\rangle$. Show that $v_1 = v_2 = -2/5$.

  \item { \color{red} Battle of the Sexes}:  \vspace{3mm} \\

\begin{center}
{\color{blue}
\begin{tabular}{c|c|c|c|}
  \multicolumn{2} {c} {} & \multicolumn{2}{c} {{\color{green}Her}} \\
\cline{3-4}
\multicolumn{2}{c|}{} & Ballet         & Fight      \\
 \cline{2-4}
\multirow{2} {*} {{\color{green}Him}}& Ballet & (1,2) & (-1,1) \\
 \cline{2-4}
 & Fight &(-1,1)& (2,1) \\
\cline{2-4}
%C & (2,6) & (4,7)& (0,8) \\
%\hline
\end{tabular}
}
\end{center}
PSNEs occur at $\langle$ Ballet, Ballet $\rangle$ and at $\langle$ Fight, Fight $\rangle$. Show that there is a MSNE at $\langle$ 1/3 ballet + 2/3 fight, 2/3 ballet + 1/3 fight $\rangle$ with $v_1= 2/3 = v_2$. Compare and contast this with the payoffs of the PSNEs.
\end{itemize}
\subsection{MSNE and Dominance}

A SDS cannot be played with positive probability in a MSNE (otherwise a higher payoff can be obtained by not playing the SDS when the strategy says to play it).

\subsection{Strict Dominance in Mixed Strategies}
Consider the game:
\begin{center}
{\color{blue}
\begin{tabular}{c|c|c|c|}
  \multicolumn{2} {c} {} & \multicolumn{2}{c} {{\color{green}Player 2}} \\
\cline{3-4}
\multicolumn{2}{c|}{} & Left        & Right        \\
 \cline{2-4}
\multirow{3} {*} {{\color{green}Player 1}}& Up & (3,-1) & (-1,1)  \\
 \cline{2-4}
& Middle &(0,0)& (0,0)  \\
\cline{2-4}
& Down & (-1,2) & (2,-1) \\
\cline{2-4}
\end{tabular}
}
\end{center}
This game has no SDS dominated by a pure strategy nor any PSNE. If a mixture of two pure strategies dominates another, then that strategy is a SDS.\\ In the above game 1/2 Up + 1/2 Down $>$ Middle. Remove Middle to get
\begin{center}
{\color{blue}
\begin{tabular}{c|c|c|c|}
  \multicolumn{2} {c} {} & \multicolumn{2}{c} {{\color{green}Player 2}} \\
\cline{3-4}
\multicolumn{2}{c|}{} & Left         & Right      \\
 \cline{2-4}
\multirow{2} {*} {{\color{green}Player 1}}& Up & (3,-1) & (-1,1) \\
 \cline{2-4}
 & Down &(-1,2)& (2,-1) \\
\cline{2-4}
%C & (2,6) & (4,7)& (0,8) \\
%\hline
\end{tabular}
}
\end{center}
Show that a MSNE exists at $\langle$ (3/5)Up + (2/5)Down, (3/7)Left + (4/7)Right $\rangle$.\\

Another example:
  \begin{center}
{\color{blue}
\begin{tabular}{c|c|c|c|c|}
  \multicolumn{2} {c} {} & \multicolumn{3}{c} {{\color{green}Player 2}} \\
\cline{3-5}
\multicolumn{2}{c|}{} & Left        & Centre & Right        \\
 \cline{2-5}
\multirow{3} {*} {{\color{green}Player 1}}& Up & (-3,6) & (9,1)  & (0, 2)\\
 \cline{2-5}
& Middle &(3,-4)& (2,4) & (4,1) \\
\cline{2-5}
& Down & (4,7) & (3,2)& (-3,2) \\
\cline{2-5}
\end{tabular}
}
\end{center}
Using IESDS, we get (1/4)L + (3/4)C $>$ R, D $>$ M, L $>$ C, D $>$ U. Thus the solution is $\langle$ Down, Left $\rangle$.

\subsection{Atypical Matrix Games}
Almost all matrix games have an odd number of solutions (Wilson 1971). Examples of non generic games follow. Weak dominance is usually the culprit.
\begin{itemize}
  \item \textit{}
  \begin{center}
{\color{blue}
\begin{tabular}{c|c|c|c|}
  \multicolumn{2} {c} {} & \multicolumn{2}{c} {{\color{green}Player 2}} \\
\cline{3-4}
\multicolumn{2}{c|}{} & Left         & Right      \\
 \cline{2-4}
\multirow{2} {*} {{\color{green}Player 1}}& Up & (1,1) & (0,0) \\
 \cline{2-4}
 & Down &(0,0)& (0,0) \\
\cline{2-4}
%C & (2,6) & (4,7)& (0,8) \\
%\hline
\end{tabular}
}
\end{center}
There are two PSNEs.
  \item\textit{}
  \begin{center}
{\color{blue}
\begin{tabular}{c|c|c|c|}
  \multicolumn{2} {c} {} & \multicolumn{2}{c} {{\color{green}Player 2}} \\
\cline{3-4}
\multicolumn{2}{c|}{} & Left         & Right      \\
 \cline{2-4}
\multirow{2} {*} {{\color{green}Player 1}}& Up & (2,2) & (9,0) \\
 \cline{2-4}
 & Down &(2,3)& (5,-1) \\
\cline{2-4}
%C & (2,6) & (4,7)& (0,8) \\
%\hline
\end{tabular}
}
\end{center}
Using IESDS, L $>$ R. Player 1 can choose Up or Down as pure strategies or any mixture of the two. All strategies yield a payoff of 2: an infinite number of strategies.\\
If tempted to use IEWDS, U $\geq$ D, leading to the solution $\langle$ Up, Right $\rangle$. Caveat emptor!
  \item Example 2 of Section \ref{D-NE} had two PSNEs. It also has an infinite number of MSNEs at $\langle$ Up, $q$Left + $(1-q)$Right $\rangle$ for $ q \leq 3/4$.
\item { \color{red} \textit{Selten}'s Horse}:  \vspace{3mm} \\

\begin{center}
{\color{blue}
\begin{tabular}{c|c|c|c|}
  \multicolumn{2} {c} {} & \multicolumn{2}{c} {{\color{green}Player 2}} \\
\cline{3-4}
\multicolumn{2}{c|}{} & Left         & Right      \\
 \cline{2-4}
\multirow{2} {*} {{\color{green}Player 1}}& Up & (3,1) & (0,0) \\
 \cline{2-4}
 & Down &(2,2)& (2,2) \\
\cline{2-4}
%C & (2,6) & (4,7)& (0,8) \\
%\hline
\end{tabular}
}
\end{center}
There are two PSNEs at $\langle$ Up,Left $\rangle$ and $\langle$ Down,Right $\rangle$ and an infinite number of MSNEs at $\langle$ Down, $q$Left + $(1-q)$Right $\rangle$ for $ q \leq 2/3$.

\item { \color{red} Take or Share} - TV game show:  \vspace{3mm} \\

\begin{center}
{\color{blue}
\begin{tabular}{c|c|c|c|}
  \multicolumn{2} {c} {} & \multicolumn{2}{c} {{\color{green}Player 2}} \\
\cline{3-4}
\multicolumn{2}{c|}{} & Share         & Take     \\
 \cline{2-4}
\multirow{2} {*} {{\color{green}Player 1}}& Share & (4,4) & (0,8) \\
 \cline{2-4}
 & Take &(8,0)& (0,0) \\
\cline{2-4}
%C & (2,6) & (4,7)& (0,8) \\
%\hline
\end{tabular}
}
\end{center}
There are three PSNEs at all but $\langle$ Share, Share $\rangle$.
If one player chooses Take, then the other is indifferent between Share and Take which leads to an infinity of MSNEs.
\end{itemize}

%\section{Analysis of Extensive Form Games}

\end{document}

 \begin{figure}[h]
\centering
%\begin{minipage}[l]{80mm}
{\includegraphics[height=80mm,width=90mm]{Log-OGY.jpg}}
%\end{minipage}
%\qquad
%\begin{minipage}[l]{80mm}
%{\includegraphics[height=85mm,width=90mm]{rec-fb3.pdf}}
%\end{minipage}
\caption{The trajectory of the Logistic Map with OGY control starting from $x_0 = 0.43$} \label{fig:trajOGY}
\vspace{5mm}
\end{figure}
