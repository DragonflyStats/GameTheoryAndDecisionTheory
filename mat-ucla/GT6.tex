
\documentclass[]{report}
\voffset=-1.5cm
\oddsidemargin=0.0cm
\textwidth = 480pt


\usepackage{amsmath}
\usepackage{graphicx}
\usepackage{amssymb}
\usepackage{framed}
\usepackage{multicol}
%\usepackage[paperwidth=21cm, paperheight=29.8cm]{geometry}
%\usepackage[angle=0,scale=1,color=black,hshift=-0.4cm,vshift=15cm]{background}
%\usepackage{multirow}
\usepackage{enumerate}

\usepackage{amsmath,amsfonts,amssymb}
\usepackage{color}
\usepackage{multirow}
\usepackage{eurosym}
\usepackage{framed}

%\input def.tex
%\input dsdef.tex
%\input rgb.tex

%\newcommand \la{\lambda}
%\newcommand \al{a}
%\newcommand \be{b}
\newcommand \x{\overline{x}}
\newcommand \y{\overline{y}}
\begin{document}

\section{6. Recursive and Stochastic Games}
6.1 Matrix Games with Games as Components. We consider now matrix games
in which the outcome of a particular choice of pure strategies of the players may be that
the players have to play another game. Let us take a simple example.
Let G1 and G2 denote 2 × 2 games with matrices
G1 =
 0 3
2 –1
and G2 =
 0 1
4 3
and let G denote the 2 × 2 game whose matrix is represented by
G =
 G1 4
5 G2

.
The game G is played in the usual manner with Player I choosing a row and Player II
choosing a column. If the entry in the chosen row and column is a number, II pays I that
amount and the game is over. If I chooses row 1 and II chooses column 1, then the game
G1 is played. If I chooses row 2 and II chooses column 2, then G2 is played.
We may analyze the game G by first analyzing G1 and G2.
G1 : Optimal for I is (1/2,1/2)
Optimal for II is (2/3,1/3)
Val(G1)=1.
G2 : Optimal for I is (0,1)
Optimal for II is (0,1)
Val(G2)=3.
If after playing the game G the players end up playing G1, then they can expect a payoff
of the value of G1, namely 1, on the average. If the players end up playing G2, they can
expect an average payoff of the value of G2, namely 3. Therefore, the game G can be
considered equivalent to the game with matrix
 1 4
5 3
G :
Optimal for I is (2/5,3/5)
Optimal for II is (1/5,4/5)
Val(G)=17/5.
This method of solving the game G may be summarized as follows. If the matrix of
a game G has other games as components, the solution of G is the solution of the game
whose matrix is obtained by replacing each game in the matrix of G by its value.
II – 61
Decomposition. This example may be written as a 4×4 matrix game. The four pure
strategies of Player I may be denoted {(1, 1),(1, 2),(2, 1),(1, 2)}, where (i, j) represents:
use row i in G, and if this results in Gi being played use row j. A similar notation may
be used for Player II. The 4 × 4 game matrix becomes
G =
⎛
⎜⎝
0 3
2 –1
4 4
4 4
5 5
5 5
0 1
4 3
⎞
⎟⎠
We can solve this game by the methods of Chapter 4.
Conversely, suppose we are given a game G and suppose after some rearrangement of
the rows and of the columns the matrix may be decomposed into the form
G =
 G11 G12
G21 G22
where G11 and G22 are arbitrary matrices and G12 and G21 are constant matrices. (A
constant matrix has the same numerical value for all of its entries.) Then we can solve
G by the above method, pretending that as the first move the players choose a row and
column from the 2 × 2 decomposed matrix. See Exercise 1(b).
%=========================================================================%
\subsection{6.2 Multistage Games.} Of course, a game that is the component of some matrix
game may itself have other games as components, in which case one has to iterate the
above method to obtain the solution. This works if there are a finite number of stages.
Example 1. The Inspection Game. (M. Dresher (1962)) Player II must try to perform
some forbidden action in one of the next n time periods. Player I is allowed to inspect II
secretly just once in the next n time periods. If II acts while I is inspecting, II loses 1 unit
to I. If I is not inspecting when II acts, the payoff is zero.
Let Gn denote this game. We obtain the iterative equations
Gn =

act wait
inspect 1 0
wait 0 Gn−1

for n = 2, 3,...
with boundary condition G1 = (1). We may solve iteratively.
Val(G1)=1
Val(G2) = Val  1 0
0 1
= 1/2
Val(G3) = Val  1 0
0 1/2

= 1/3
.
.
.
Val(Gn) = Val  1 0
0 1/(n − 1)
= 1/n
II – 62
since inductively, Val(Gn) = 1
n−1 /(1 + 1
n−1 )=1/n. The optimal strategy in the game Gn
for both players is (1/n,(n − 1)/n). For other games of this sort, see the book by Garnaev
(2000).
%============================================================%
Example 2. Guess it! (Rufus Isaacs (1955); see also Martin Gardner (1978), p. 40.) As
a more complex example of a multistage game, consider the following game loosely related
to the game Cluedo. From a deck with m + n + 1 distinct cards, m cards are dealt to
Player I, n cards are dealt to Player II, and the remaining card, called the “target card”,
is placed face down on the table. Players knows their own cards but not those of their
opponent. The objective is to guess correctly the target card. Players alternate moves,
with Player I starting. At each move, a player may either
(1) guess at the target card, in which case the game ends, with the winner being the player
who guessed if the guess is correct, and his opponent if the guess is incorrect, or
(2) ask if the other player holds a certain card. If the other player has the card, that card
must be shown and is removed from play.
With a deck of say 11 cards and each player receiving 5 cards, this is a nice playable
game that illustrates need for bluffing in a clear way. If a player asks about a card that is
in his own hand, he knows what the answer will be. We call such a play a bluff . If a player
asks about a card not in his hand, we say he is honest. If a player is always honest and
the card he asks about is the target card, the other player will know that the requested
card is the target card and so will win. Thus a player must bluff occasionally. Bluffing
may also lure the opponent into a wrong guess at the target card.
%====================================================================%
Let us denote this game with Player I to move by Gm,n. The game Gm,0 is easy to
play. Player I can win immediately. Since his opponent has no cards, he can tell what the
target card is. Similarly, the game G0,n is easy to solve. If Player I does not make a guess
immediately, his opponent will win on the next move. However, his probability of guessing
correctly is only 1/(n + 1). Valuing 1 for a win and 0 for a loss from Player I’s viewpoint,
the value of the game is just the probability I wins under optimal play. We have
Val(Gm,0) = 1 for all m \geq 0, and Val(G0,n) = 1
n + 1
for all n \geq 0. (1)
If Player I asks for a card that Player II has, that card is removed from play and it is
Player II’s turn to move, holding n − 1 cards to her opponent’s m cards. This is exactly
the game Gn−1,m but with Player II to move. We denote this game by Gn−1,m. Since the
probability that Player I wins is one minus the probability that Player II wins, we have
Val(Gn,m)=1 − Val(Gn,m) for all m and n. (2)
Suppose Player I asks for a card that Player II does not have. Player II must immediately
decide whether or not Player I was bluffing. If they decides Player I was honest, she
will announce the card Player I asked for as her guess at the target card, and win if she
was right and lose if they was wrong. If they decides Player I was bluffing and they is wrong,
Player I will win on his turn. If they is correct, the card Player I asked for is removed from
his hand, and the game played next is Gn,m−1.
II – 63
Using such considerations, we may write the game as a multistage game in which
a stage consists of three pure strategies for Player I (honest, bluff, guess) and two pure
strategies for Player II (ignore the asked card, call the bluff by guessing the asked card).
The game matrix becomes, for $m \geq 1$ and $n \geq 1$,
Gm,n =
⎛
⎝
ignore call
honest n
n+1Gn−1,m + 1
n+1
n
n+1Gn−1,m
bluff Gn,m−1 1
guess 1
n+1
1
n+1
⎞
⎠ (3)
This assumes that if Player I asks honestly, he chooses among the n + 1 unknown
cards with probability 1/(n +1) each; also if he bluffs, he chooses among his m cards with
probability 1/m each. That this may be done follows from the invariance considerations
of Section 3.6.
As an example, the upper left entry of the matrix is found as follows. With probability
n/(n + 1), Player I asks a card that is in Player II’s hand and the game becomes Gn−1,m;
with probability 1/(n + 1), Player I asks the target card, Player II ignors it and Player I
wins on his next turn, i.e. gets 1. The upper right entry is similar, except this time if the
asked card is the target card, Player II guesses it and Player I gets 0.
It is reasonable to assume that if $m \geq 1$ and $n \geq 1$, Player I should not guess,
because the probability of winning is too small. In fact if $m \geq 1$ and $n \geq 1$, there is a
strategy for Player I that dominates guessing, so that the last row of the matrix may be
deleted. This strategy is: On the first move, ask any of the m + n + 1 cards with equal
probability 1/(m + n + 1) (i.e. use row 1 with probability (n + 1)/(m + n + 1) and row
2 with probability m/(m + n + 1)), and if Player II doesn’t guess at her turn, then guess
at the next turn. We must show that Player I wins with probability at least 1/(n + 1)
whether or not Player II guesses at her next turn. If Player II guesses, her probability of
win is exactly 1/(m + 1) whether or not the asked card is one of hers. So Player I’s win
probability is m/(m + 1) \geq 1/2 \geq 1/(n + 1). If Player II does not guess, then at Player
I’s next turn, Player II has at most n cards (she may have n − 1) so again Player I’s win
probability is at least 1/(n + 1).
So the third row may be removed in (3) and the games reduce to
Gm,n =

ignore call
honest n
n+1Gn−1,m + 1
n+1
n
n+1Gn−1,m
bluff Gn,m−1 1

(4)
for $m \geq 1$ and $n \geq 1$. These 2 by 2 games are easily solved recursively, using the boundary
conditions (1). One can find the value and optimal strategies of Gm,n after one finds the
values of Gn,m−1 and Gn−1,m and uses (2). For example, the game G1,1 reduces to the
game with matrix  3/4 1/4
0 1
. The value of this game is 1/2, an optimal strategy for
II – 64
Player I is (2/3,1/3) (i.e. bluff with probability 1/3), and the optimal strategy of player
II is (1/2,1/2).
One can also show that for all $m \geq 1$ and $n \geq 1$ these games do not have saddle points.
In fact, one can show more: that Val(Gm,n) is nondecreasing in m and nonincreasing in n.
(The more cards you have in your hand, the better.) Let Vm,n = Val(Gm,n). Then using
Val  a b
c d
= (ad − bc)/(a + d − b − c), we have after some algebra
Vm,n = Val  n
n+1 (1 − Vn−1,m) + 1
n+1
n
n+1 (1 − Vn−1,m)
(1 − Vn,m−1) 1
= 1 + n(1 − Vn−1,m)Vn,m−1
1+(n + 1)Vn,m−1
.
for $m \geq 1$ and $n \geq 1$. This provides a simple direct way to compute the values recursively.
The following table gives the computed values as well as the optimal strategies for the
players for small values of m and n.
m\n 123456
0.5000 0.5000 0.4000 0.3750 0.3333 0.3125
1 0.3333 0.2500 0.2000 0.1667 0.1429 0.1250
0.5000 0.5000 0.4000 0.3750 0.3333 0.3125
0.6667 0.5556 0.5111 0.4500 0.4225 0.3871
2 0.5000 0.3333 0.2667 0.2143 0.1818 0.1563
0.3333 0.3333 0.2889 0.2500 0.2301 0.2055
0.6875 0.6250 0.5476 0.5126 0.4667 0.4411
3 0.5000 0.3750 0.2857 0.2361 0.1967 0.1701
0.3750 0.3250 0.2762 0.2466 0.2167 0.1984
0.7333 0.6469 0.5966 0.5431 0.5121 0.4749
4 0.5556 0.3947 0.3134 0.2511 0.2122 0.1806
0.3333 0.3092 0.2634 0.2342 0.2118 0.1899
0.7500 0.6809 0.6189 0.5810 0.5389 0.5112
5 0.5714 0.4255 0.3278 0.2691 0.2229 0.1917
0.3333 0.2908 0.2566 0.2284 0.2062 0.1885
0.7714 0.6972 0.6482 0.6024 0.5704 0.5353
6 0.6000 0.4410 0.3488 0.2808 0.2362 0.2003
0.3143 0.2834 0.2461 0.2236 0.2028 0.1854
%======================================================================%
Table of values and optimal strategies of Gm,n for 1 ≤ m, n ≤ 6. The top number in each
box is the value, the middle number is the probability with which Player I should bluff,
and the bottom number is the probability with which Player II should call the asked card.
II – 65
6.3 Recursive Games. -Optimal Strategies. In some games with games as
components, it may happen that the original game comes up again. Such games are called
recursive. A simple example is
G =
 G 1
1 0
This is an infinite game. If the players always play the first row and first column, the game
will be played forever. No matter how unlikely such a possibility is, the mathematical
definition is not complete until we say what the payoff is if G is played forever. Let us say
that II pays I Q units if they both choose their first pure strategy forever, and write
G =
 G 1
1 0
, Q.
We are not automatically assured the existence of a value or the existence of optimal
strategies in infinite games. However, it is easy to see that the value of G exists and is
equal to 1 no matter what the value of the number Q is. The analysis can be made as
follows.
II can restrict her losses to at most 1 by choosing the second column. If $Q \geq 1$, I can
guarantee winning at least 1 by playing his first row forever. But if Q < 1, this won’t work.
It turns out that an optimal strategy for I, guaranteeing him at least 1, does not exist in
this case. However, for any  > 0 there is a strategy for I that guarantees him an average
gain of at least 1 − . Such a strategy, that guarantees a player an average payoff within
 of the value, is called -optimal. In this case, the strategy that continually uses the
mixed strategy (1 − , ) (top row with probability 1 −  and bottom row with probability
) is -optimal for I. The use of such a strategy by I insures that he will eventually choose
row 2, so that the payoff is bound to be 0 or 1 and never Q. The best that Player II can
do against this strategy is to choose column 2 immediately, hoping that I chooses row 2.
The expected payoff would then be 1 · (1 − )+0 ·  = 1 − .
In summary, for the game G above, the value is 1; Player II has an optimal strategy,
namely column 2; If $Q \geq 1$, the first row forever is optimal for I; if Q < 1, there is no
optimal strategy for I, but the strategy (1 − , ) forever is -optimal for I.
Consider now the game
G0 =
 G0 5
1 0
, Q.
For this game, the value depends on Q. If $Q \geq 1$, the first row forever is optimal for I,
and the value is Q if 1 ≤ Q ≤ 5, and the value is 5 if $Q \geq 5$. For Q < 1, the value is 1;
however, in contrast to the game G above, I has an optimal strategy for the game G0, for
example (1/2, 1/2) forever. II’s optimal strategy is the first column forever if Q < 5, the
second column if Q > 5 and anything if Q = 5.
In analogy to what we did for games with games as components, we might attempt
to find the value v of such a game by replacing G0 by v in the matrix and solving the
II – 66
equation
v = Val  v 5
1 0
for v. Here there are many solutions to this equation. The set of all solutions to this
equation is the set of numbers v in the interval 1 ≤ v ≤ 5. (Check this!)
This illustrates a general result that the equation, given by equating v to the value of
the game obtained by replacing the game in the matrix by v, always has a solution equal
to the value of the game. It may have more solutions but the value of the game is that
solution that is closest to Q. For more information on these points, consult the papers of
Everett (1957) and of Milnor and Shapley (1957).
Example 3. Let
G =
⎛
⎝
G 1 0
1 0 G
0 G 1
⎞
⎠ , Q.
Then, if the value of G is v,
v = Val
⎛
⎝
v 1 0
1 0 v
0 v 1
⎞
⎠ = 1 + v
3 .
This equation has a unique solution, v = 1/2. This must be the value for all Q. The
strategy (1/3,1/3,1/3) forever is optimal for both players.
Example 4. The basic game of Dare is played as follows. Player I, the leader, and Player
II, the challenger, simultaneously “pass” or “dare”. If both pass, the payoff is zero (and
the game is over). If I passes and II dares, I wins 1. If I dares and II passes, I wins 3. If
both dare, the basic game is played over with the roles of the players reversed (the leader
becomes the challenger and vice versa). If the players keep daring forever, let the payoff
be zero. We might write
G =

pass dare
pass 0 1
dare 3 −GT

where −GT represents the game with the roles of the players reversed. (Its matrix is the
negative of the transpose of the matrix G.) The value of −GT is the negative of the value
of G.
If v represents the value of G, then v \geq 0 because of the top row. Therefore the matrix
for G with −GT replaced by −v does not have a saddle point, and we have
v = Val  0 1
3 −v

= 3
4 + v
.
This gives the quadratic equation, v2 + 4v − 3 = 0. The only nonnegative solution is
v = √7 − 2 = .64575 ···. The optimal strategy for I is ((5 − √7)/3, (
√7 − 2)/3) and the
optimal strategy for II is (3 − √7,
√
7 − 2).
II – 67
Example 5. Consider the following three related games.
G1 =
 G2 0
0 G3

G2 =
 G1 1
1 0
G3 =
 G1 2
2 0
and suppose the payoff if the games are played forever is Q. Let us attempt to solve these
games. Let v1 = Val(G1), v2 = Val(G2), and v3 = Val(G3). Player I can guarantee that
v1 > 0, v2 > 0 and v3 > 0 by playing (1/2, 1/2) forever. In addition, v2 ≤ 1 and v3 ≤ 2,
which implies v1 < 1. Therefore none of the games has a saddle point and we may write
v1 = v2v3
v2 + v3
, v2 = 1
2 − v1
, v3 = 4
4 − v1
.
Substituting the latter two equations into the former, we obtain
v1
2 − v1
+
4v1
4 − v1
= 4
(2 − v1)(4 − v1)
5v2
1 − 12v1 +4=0
(5v1 − 2)(v1 − 2) = 0
Since 0 < v1 < 1, this implies that v1 = 2/5. Hence
Game value opt. for I= opt. for II
G1 2/5 (16/25, 9/25)
G2 5/8 (5/8, 3/8)
G3 10/9 (5/9, 4/9)
independent of the value of Q.
\subsection{6.4 Stochastic Movement Among Games.} We may generalize the notion of a
recursive game by allowing the choice of the next game played to depend not only upon the
pure strategy choices of the players, but also upon chance. Let G1,...,Gn be games and
let p1,...,pn be probabilities that sum to one. We use the notation, p1G1 +···+pnGn, to
denote the situation where the game to be played next is chosen at random, with game Gi
being chosen with probability pi, i = 1,...,n. Since, for a given number z, the 1×1 matrix
(z) denotes the trivial game in which II pays I z, we may, for example, use 1
2G1 + 1
2 (3) to
represent the situation where G1 is played if a fair coin comes up heads, and II pays I 3
otherwise.
Example 6. Let G1 and G2 be related as follows.
G1 =
 1
2G2 + 1
2 (0) 1
2 0
G2 =
 2
3G1 + 1
3 (−2) 0
0 −1

The game must eventually end (with probability 1). In fact, the players could not play
forever even if they wanted to. Even if they choose the first row and first column forever,
%- II – 68
eventually the game would end with a payoff of 0 or −2. Thus we do not need to specify
any payoff if play continues forever. To solve, let vi = Val(Gi) for i = 1, 2. Then 0 ≤ v1 ≤ 1
and −1 ≤ v2 ≤ 0, so neither game has a saddle point. Hence,
v1 = Val  1
2 v2 1
2 0
= 4
6 − v2
and
v2 = Val  2
3 v1 − 2
3 0
0 −1

= −2(1 − v1)
5 − 2v1
Thus
v1 = 4
6 + 2(1−v1)
5−2v1
= 2(5 − 2v1)
16 − 7v1
.
This leads to the quadratic equation, 7v2
1−20v1+10 = 0, with solution, v1 = (10−
√30)/7 =
.646 ···. We may substitute back into the equation for v2 to find v2 = −(2√30 − 10)/5 =
−.191 ···. From these values one can easily find the optimal strategies for the two games.
Example 7. A coin with probability 2/3 of heads is tossed. Both players must guess
whether the coin will land heads or tails. If I is right and II is wrong, I wins 1 if the coin
is heads and 4 if the coin is tails and the game is over. If I is wrong and II is right, there
is no payoff and the game is over. If both players are right, the game is played over. But
if both players are wrong, the game is played over with the roles of the players reversed.
If the game never ends, the payoff is Q.
If we denote this game by G, then
G =
 2
3G + 1
3 (−GT ) 2
3 (1) + 1
3 (0)
2
3 (0) + 1
3 (4) 2
3 (−GT ) + 1
3G

If we let its value be denoted by v, then
v = Val  1
3 v 2
3 4
3 −1
3 v

If v \geq 2, then there is a saddle at the upper right corner with v = 2/3. This contradiction
shows that v < 2 and there is no saddle. Therefore,
v = 8 + v2
18
or v2 − 18v +8=0.
This has a unique solution less than two,
v = 9 − √
73 = .456 ···
from which we may calculate the optimal strategy for I:
(
13 − √73
6 ,
√
73 − 7
6 )=(.743 ··· , .256 ···)
II – 69
and the optimal strategy for II:
(
11 − √
73
6 ,
√
73 − 5
6 )=(.409 ··· , .591 ···).
The value and optimal strategies are independent of Q.
\subsection{6.5 Stochastic Games.} If to the features of the games of the previous section is
added the possibility of a payoff at each stage until the game ends, the game is called
a Stochastic Game. This seems to be the proper level of generality for theoretical
treatment of multistage games. It is an area of intense contemporary research. %See for example the books of Filar and Vrieze (1997) and % %Maitra and Sudderth (1996). Stochastic games were introduced by Shapley in (1953) in a beautiful paper that has been reprinted
% in Raghavan et al. (1991), and more recently in Kuhn (1997). In this section, we present
% Shapley’s main result.
\begin{framed}
A Stochastic Game, G, consists of a finite set of positions or states, {1, 2,...,N}, one
of which is specified as the starting position. We denote by G(k) the game in which k is the
starting position. Associated with each state, k, is a matrix game, A(k) = 
a(k)
ij 

. 
\end{framed}

If the
stochastic game is in state k, the players simultaneously choose a row and column of A(k),
say i and j. As a result, two things happen. First, Player I wins the amount a(k)
ij from
Player II. Second, with probabilities that depend on i, j and k, the game either stops, or
it moves to another state (possibly the same one). The probability that the game stops is
denoted by s
(k)
ij , and the probability that the next state is  is denoted by P(k)
ij (), where
s
(k)
ij +

N
=1
P(k)
ij () = 1 (5)
for all i, j and k.
The payoffs accumulate throughout the game until it stops. To make sure the game
eventually stops, we make the assumption that all the stopping probabilities are positive.
Let s denote the smallest of these probabilities.
s = min
i,j,k
s
(k)
ij > 0 (6)
Under this assumption, the probability is one that the game ends in a finite number of
moves. This assumption also makes the expected accumulated payoff finite no matter how
the game is played, since if M denotes the largest of the absolute values of the payoffs,
M = maxi,j,k |a(k)
ij |, then the total expected payoff to either player is bounded by
M + (1 − s)M + (1 − s)
2M + ··· = M/s. (7)
Player I wishes to maximize the total accumulated payoff and Player II to minimize
it. We use a modification of the notation of the previous section to describe this game.
II – 70
G(k) =

a
(k)
ij +

N
=1
P(k)
ij ()G()

. (8)
Note that the probabilities in each component of this matrix sum to less than one. It is
understood that with the remaining probability, s
(k)
ij , the game ends. It should be noted
that in contrast to the previous section, a payoff does not end the game. After a payoff is
made, it is then decided at random whether the game ends and, if not, which state should
be played next.
Since no upper bound can be placed on the length of the game, this is an infinite
game. A strategy for a player must specify for every n how to choose an action if the game
reaches stage n. In general, theory does not guarantee a value. Moreover, the choice of
what to do at stage n may depend on what happened at all previous stages, so the space
of possible strategies is extremely complex.
Nevertheless, in stochastic games, the value and optimal strategies for the players
exist for every starting position. Moreover, optimal strategies exist that have a very simple
form. Strategies that prescribe for a player a probability distribution over his choices that
depends only on the game, Gk, being played and not on the stage n or past history are
called \textbf{\textit{stationary strategies}}. The following theorem states that there exist stationary
optimal strategies.
\begin{framed}
Theorem 1. (Shapley (1952)) Each game G(k) has a value, v(k). These values are the
unique solution of the set of equations,
v(k) = Val 
a(k)
ij +

N
=1
P(k)
ij () v()

for k = 1,...,N. (9)
\end{framed}
Each player has a stationary optimal strategy that in state k uses the optimal mixed
strategy for the game with matrix
A(k)
(v) = 
a(k)
ij +

N
=1
P(k)
ij () v()

(10)
where v represents the vector of values, v = (v(1),...,v(N)).
In equations (9), we see the same principle as in the earlier sections: the value of a
game is the value of the matrix game (8) with the games replaced by their values. A proof
of this theorem may be found in Appendix 2.
%==========================================================%
\subsection{Example 8.} As a very simple example, consider the following stochastic game with
one state, call it G.
G =
 1 + (3/5)G 3 + (1/5)G
1 + (4/5)G 2 + (2/5)G

II – 71
\begin{itemize}
\item From Player II’s viewpoint, column 1 is better than column 2 in terms of immediate payoff,
but column 2 is more likely to end the game sooner than column 1, so that it should entail
smaller future payoffs. Which column should they choose?
\item Assume that all strategies are active, i.e. that the game does not have a saddle point.
We must check when we are finished to see if the assumption was correct. Then
v = Val  1 + (3/5)v 3 + (1/5)v
1 + (4/5)v 2 + (2/5)v

= (1 + (4/5)v)(3 + (1/5)v) − (1 + (3/5)v)(2 + (2/5)v)
1 + (4/5)v + 3 + (1/5)v − 1 − (3/5)v − 2 − (2/5)v
=1+ v − (2/25)v2
This leads to
(2/25)v2 = 1.
\item Solving this quadratic equation gives two possible solutions v = ±
25/2 = ±(5/2)√
2.
\item 
Since the value is obviously positive, we must use the plus sign. This is v = (5/2)√
2 =
3.535. If we put this value into the matrix above, it becomes
 1 + (3/2)\sqrt{2} 3 + (1/2)\sqrt{2}
1+2\sqrt{2} 2+ \sqrt{2}
\end{itemize}
The optimal strategy for Player I in this matrix is p = (√
2 − 1, 2 − √
2) = (.414, .586),
and the optimal strategy for Player II is q = (1 − \sqrt{2}/2,
\sqrt{2}/2) = (.293, .707). Since these
are probability vectors, our assumption is correct and these are the optimal strategies, and
v = (5/2)\sqrt{2} is the value of the stochastic game.
6.6 Approximating the solution. For a general stochastic game with many states,
equations (9) become a rather complex system of simultaneous nonlinear equations. We
cannot hope to solve such systems in general. However, there is a simple iterative method
of approximating the solution. This is based on Shapley’s proof of Theorem 1, and is called
Shapley iteration.
First we make a guess at the solution, call it v0 = (v0(1),...,v0(N)). Any guess will
do. We may use all zero’s as the initial guess, v0 = 0 = (0,..., 0). Then given vn, we
define inductively, vn+1, by the equations,
vn+1(k) = Val 
a
(k)
ij +

N
=1
P(k)
ij () vn()

for k = 1,...,N. (11)
With v0 = 0, the vn(k) have an easily understood interpretation. vn(k) is the value of the
stochastic game starting in state k if there is forced stopping if the game reaches stage n.
In particular, v1(k) = Val(Ak) for all k.
The proof of Theorem 1 shows that vn(k) converges to the true value, v(k), of the
stochastic game starting at k. Two useful facts should be noted. First, the convergence
II – 72
is at an exponential rate: the maximum error goes down at least as fast as (1 − s)n.
(See Corollary 1 of Appendix 2.) Second, the maximum error at stage n + 1 is at most
the maximum change from stage n to n + 1 multiplied by (1 − s)/s. (See Corollary 2 of
Appendix 2.)
Let us take an example of a stochastic game with two positions. The corresponding
games G(1) and G(2), are related as follows.
G(1) =
 4 + .3G(1) 0 + .4G(2)
1 + .4G(2) 3 + .5G(1)
G(2) =
 0 + .5G(1) −5
−4 1+ .5G(2)
Using v0 = (0, 0) as the initial guess, we find v1 = (2, −2), since
v1(1) = Val  4 0
1 3
= 2 v1(2) = Val  0 −5
−4 1
= −2.
The next iteration gives
v2(1) = Val  4.6 −.8
.2 4
= 2.0174 v2(2) = Val  1 −5
−4 0
= −2.
Continuing, we find
v3(1) = 2.0210 v3(2) = −1.9983
v4(1) = 2.0220 v4(2) = −1.9977
v5(1) = 2.0224 v5(2) = −1.9974
v6(1) = 2.0225 v6(2) = −1.9974
The smallest stopping probability is .5, so the rate of convergence is at least (.5)n and the
maximum error of v6 is at most .0002.
The optimal strategies using v6 are easily found. For game G(1), the optimal strategies
are p(1) = (.4134, .5866) for Player I and q(1) = (.5219, .4718) for Player II. For game G(2),
the optimal strategies are p(2) = (.3996, .6004) for Player I and q(2) = (.4995, .5005) for
Player II.
6.7 Exercises
1.(a) Solve the system of games
G =
 0 G1
G2 G3

G1 =
 4 3
1 2
G2 =
 0 6
5 1
G3 =
 0 −2
−2 0
.
(b). Solve the games with matrices
(b1) (b2)
⎛
⎜⎝
0602
0305
5020
1040
⎞
⎟⎠
⎛
⎜⎜⎜⎝
31522
13522
44122
11163
11147
⎞
⎟⎟⎟⎠
II – 73
2. The Inspection Game. Let Gm,n denote the inspection game in which I is
allowed m inspections in the n time periods. (Thus, for 1 ≤ n ≤ m, Val(Gm,n) = 1, while
for $n \geq 1$, Val(G0,n) = 0.) Find the iterative structure of the games and solve.
3. A Game of Endurance. II must count from n down to zero by subtracting
either one or two at each stage. I must guess at each stage whether II is going to subtract
one or two. If I ever guesses incorrectly at any stage, the game is over and there is no
payoff. Otherwise, if I guesses correctly at each stage, he wins 1 from II. Let Gn denote
this game, and use the initial conditions G0 = (1) and G1 = (1). Find the recursive
structure of the games and solve. (In the solution, you may use the notation Fn to denote
the Fibonacci sequence, 1, 1, 2, 3, 5, 8, 13,... , with definition F0 = 1, F1 = 1, and for
n \geq 2, Fn = Fn−1 + Fn−2.)
4. Solve the sequence of games, G0, G1,..., where
G0 =
 3 2
1 G1

,...,Gn =
 n + 3 n + 2
n + 1 Gn+1
,...
Assume that if play continues forever, the payoff is zero.
5. (a) In the game “Guess it!”, G1,n, with m = 1 and arbitrary n, show that Player
I’s optimal strategy if to bluff with probability 1/(n + 2).
(b) Show that Player II’s optimal strategy in G1,n is to call the asked card with
probability V1,n, the value of G1,n.
6. Recursive Games. (a) Solve the game G =
 G 2
0 1
, Q.
(b) Solve the game G =
⎛
⎝
G 1 1
1 0 G
1 G 0
⎞
⎠ , Q.
7. Consider the following three related games.
G1 =
 G2 1
1 0
G2 =
 G3 0
0 2
G3 =
 G1 1
1 0
and suppose the payoff is Q if the games are played forever. Solve.
8. Consider the following three related games.
G1 =
⎛
⎝
G1 G2 G3
G2 G3 G1
G3 G1 G2
⎞
⎠ G2 =
 G1 0
0 2
G3 =
 G2 1
1 0
and suppose the payoff is Q if the games are played forever. Solve.
II – 74
9. There is one point to go in the match. The player that wins the last point while
serving wins the match. The server has two strategies, high and low. The receiver has
two strategies, near and far. The probability the server wins the point is given in the
accompanying table.
near far
high .8 .5
low .6 .7
If the server misses the point, the roles of the players are interchanged and the win probabilities
for given pure strategies are the same for the new server. Find optimal strategies
for server and receiver, and find the probability the server wins the match.
10. Player I tosses a coin with probability p of heads. For each k = 1, 2,..., if I tosses
k heads in a row he may stop and challenge II to toss the same number of heads; then II
tosses the coin and wins if and only if he tosses k heads in a row. If I tosses tails before
challenging II, then the game is repeated with the roles of the players reversed. If neither
player ever challenges, the game is a draw.
(a) Solve when p = 1/2.
(b) For arbitrary p, what are the optimal strategies of the players? Find the limit as p → 1
of the probability that I wins.
11. Solve the following stochastic game.
G =
 4 1 + (1/3)G
0 1 + (2/3)G

.
12. Consider the following stochastic game with two positions.
G(1) =
 2 2 + (1/2)G(2)
0 4 + (1/2)G(2)
G(2) =
 −4 0
−2 + (1/2)G(1) −4 + (1/2)G(1)
(a) Solve the equations (9) exactly for the values v(1) and v(2).
(b) Carry out Shapley iteration to find v2 starting with the initial guess v0 = (0, 0),
and compare with the exact values found in (a).

