\section{5. The Extensive Form of a Game}
The strategic form of a game is a compact way of describing the mathematical aspects
of a game. In addition, it allows a straightforward method of analysis, at least in principle.
However, the flavor of many games is lost in such a simple model. Another mathematical
model of a game, called the extensive form, is built on the basic notions of position and
move, concepts not apparent in the strategic form of a game. In the extensive form, we
may speak of other characteristic notions of games such as bluffing, signaling, sandbagging,
and so on. Three new concepts make their appearance in the extensive form of a game:
the game tree, chance moves, and information sets.
%====================================================================%
\section{5.1 The Game Tree.} The extensive form of a game is modelled using a directed
graph. A directed graph is a pair (T,F) where T is a nonempty set of vertices and F is
a function that gives for each x \in T a subset, F(x) of T called the followers of x. When
a directed graph is used to represent a game, the vertices represent positions of the game.
The followers, F(x), of a position, x, are those positions that can be reached from x in one
move.
A path from a vertex t0 to a vertex t1 is a sequence, x0, x1,...,xn, of vertices
such that x0 = t0, xn = t1 and xi is a follower of xi−1 for i = 1,...,n. For the extensive
form of a game, we deal with a particular type of directed graph called a tree.
\begin{framed}
Definition. A tree is a directed graph, (T,F) in which there is a special vertex, t0, called
the root or the initial vertex, such that for every other vertex t \in T, there is a unique path
beginning at t0 and ending at t.
\end{framed]
The existence and uniqueness of the path implies that a tree is connected, has a unique
initial vertex, and has no circuits or loops.
In the extensive form of a game, play starts at the initial vertex and continues along
one of the paths eventually ending in one of the terminal vertices. At terminal vertices,
the rules of the game specify the payoff. For n-person games, this would be an n-tuple of
payoffs. Since we are dealing with two-person zero-sum games, we may take this payoff to
be the amount Player I wins from Player II. For the nonterminal vertices there are three
possibilities. Some nonterminal vertices are assigned to Player I who is to choose the move
at that position. Others are assigned to Player II. However, some vertices may be singled
out as positions from which a chance move is made.


%====================================================================%
Chance Moves. Many games involve chance moves. Examples include the rolling of
dice in board games like monopoly or backgammon or gambling games such as craps, the
dealing of cards as in bridge or poker, the spinning of the wheel of fortune, or the drawing
of balls out of a cage in lotto. In these games, chance moves play an important role. Even
in chess, there is generally a chance move to determine which player gets the white pieces
(and therefore the first move which is presumed to be an advantage). It is assumed that
the players are aware of the probabilities of the various outcomes resulting from a chance
move.
II – 49
Information. Another important aspect we must consider in studying the extensive
form of games is the amount of information available to the players about past moves of
the game. In poker for example, the first move is the chance move of shuffling and dealing
the cards, each player is aware of certain aspects of the outcome of this move (the cards he
received) but they is not informed of the complete outcome (the cards received by the other
players). This leads to the possibility of “bluffing.”

%====================================================================%
\section{5.2 Basic Endgame in Poker.} 
\begin{itemize}
\itme One of the simplest and most useful mathematical
models of a situation that occurs in poker is called the “classical betting situation” by
Friedman (1971) and “basic endgame” by Cutler (1976). These papers provide explicit
situations in the game of stud poker and of lowball stud for which the model gives a very
accurate description. \item This model is also found in the exercises of the book of Ferguson
(1967).\item Since this is a model of a situation that occasionally arises in the last round of
betting when there are two players left, we adopt the terminology of Cutler and call it
Basic Endgame in poker. This will also emphasize what we feel is an important feature of
the game of poker, that like chess, go, backgammon and other games, there is a distinctive
phase of the game that occurs at the close, where special strategies and tactics that are
analytically tractable become important.
\end{itemize}

%====================================================================%
Basic Endgame is played as follows. Both players put 1 dollar, called the ante, in the
center of the table. The money in the center of the table, so far two dollars, is called the
pot. Then Player I is dealt a card from a deck. It is a winning card with probability 1/4
and a losing card with probability 3/4. Player I sees this card but keeps it hidden from
Player II. (Player II does not get a card.) Player I then checks or bets. If they checks, then
his card is inspected; if they has a winning card they wins the pot and hence wins the 1 dollar
ante from II, and otherwise they loses the 1 dollar ante to II. If I bets, they puts 2 dollars more
into the pot. Then Player II – not knowing what card Player I has – must fold or call. If
they folds, they loses the 1 dollar ante to I no matter what card I has. If II calls, they adds 2
dollars to the pot. Then Player I’s card is exposed and I wins 3 dollars (the ante plus the
bet) from II if they has a winning card, and loses 3 dollars to II otherwise.
Let us draw the tree for this game. There are at most three moves in this game: (1)
the chance move that chooses a card for I, (2) I’s move in which they checks or bets, and (3)
II’s move in which they folds or calls. To each vertex of the game tree, we attach a label
indicating which player is to move from that position. Chance moves we generally refer to
as moves by nature and use the label N. See Figure 1.

%====================================================================%
Each edge is labelled to identify the move. (The arrows are omitted for the sake of
clarity. Moves are assumed to proceed down the page.) Also, the moves leading from a
vertex at which nature moves are labelled with the probabilities with which they occur.
At each terminal vertex, we write the numerical value of I’s winnings (II’ s losses).
There is only one feature lacking from the above figure. From the tree we should be
able to reconstruct all the essential rules of the game. That is not the case with the tree of
Figure 1 since we have not indicated that at the time II makes their decision they does not
know which card I has received. That is, when it is II’s turn to move, they does not know at
which of their two possible positions they is. We indicate this on the diagram by encircling the
II – 50
3 1
1
−3 1
−1
N
I
II II
I
winning losing
1/4 3/4
bet check bet check
call fold call fold
Figure 1.
two positions in a closed curve, and we say that these two vertices constitute an information
set. The two vertices at which I is to move constitute two separate information sets since
he is told the outcome of the chance move. To be complete, this must also be indicated
on the diagram by drawing small circles about these vertices. We may delete one of the
labels indicating II’s vertices since they belong to the same information set. It is really
the information set that must be labeled. The completed game tree becomes
3 1
1
−3 1
−1
N
I I
II
winning losing
1/4 3/4
bet check bet check
call fold call fold
Figure 2.
The diagram now contains all the essential rules of the game.

%====================================================================%
\section{ 5.3 The Kuhn Tree.} The game tree with all the payoffs, information sets, and labels
for the edges and vertices included is known as the Kuhn Tree. We now give the formal
definition of a Kuhn tree.
Not every set of vertices can form an information set. In order for a player not to
be aware of which vertex of a given information set the game has come to, each vertex in
that information set must have the same number of edges leaving it. Furthermore, it is
important that the edges from each vertex of an information set have the same set of labels.
The player moving from such an information set really chooses a label. It is presumed that
a player makes just one choice from each information set.

%====================================================================%
II – 51
\begin{framed}
Definition. A finite two-person zero-sum game in extensive form is given by
\begin{enumerate}
\item a finite tree with vertices T,
\item a payoff function that assigns a real number to each terminal vertex,
\item a set T0 of non-terminal vertices (representing positions at which chance moves
occur) and for each t \in T0, a probability distribution on the edges leading from t,
\item a partition of the rest of the vertices (not terminal and not in T0) into two groups
of information sets T11, T12,...,T1k1 (for Player I) and T21, T22,...,T2k2 (for Player II),
and
\item for each information set Tjk for player j, a set of labels Ljk, and for each t \in Tjk,
a one-to-one mapping of Ljk onto the set of edges leading from t.
\end{enumerate}
\end{framed}
The information structure in a game in extensive form can be quite complex. It may
involve lack of knowledge of the other player’s moves or of some of the chance moves. It
may indicate a lack of knowledge of how many moves have already been made in the game
(as is the case With Player II in Figure 3).
0
1 –1 1 –1 0 –2 2 1
1 2
I
II II
I
A B
a b c d
e
DE de DE a b c
D E
Figure 3.
%===========================================================%
It may describe situations in which one player has forgotten a move they has made
earlier (as is the case With Player I in Figures 3 or 4). In fact, one way to try to model
the game of bridge as a two-person zero-sum game involves the use of this idea. In bridge,
there are four individuals forming two teams or partnerships of two players each. The
interests of the members of a partnership are identical, so it makes sense to describe this
as a two-person game. But the members of one partnership make bids alternately based
on cards that one member knows and the other does not. This may be described as a
single player who alternately remembers and forgets the outcomes of some of the previous
random moves. Games in which players remember all past information they once knew
and all past moves they made are called games of perfect recall.
%===========================================================%
A kind of degenerate situation exists when an information set contains two vertices
which are joined by a path, as is the case with I’s information set in Figure 5.
II – 52
1
–1 0 0 –1
2
I
II
I
f g
ab ab
cd cd
Figure 4.
We take it as a convention that a player makes one choice from each information set
during a game. That choice is used no matter how many times the information set is
reached. In Figure 5, if I chooses option a there is no problem. If I chooses option b, then
in the lower of I’s two vertices the a is superfluous, and the tree is really equivalent to
Figure 6. Instead of using the above convention, we may if we like assume in the definition
of a game in extensive form that no information set contains two vertices joined by a path.
2 0
0 1
2
I
II
a b
cd cd
a b
2 0
1
2
I
I
II
a b
cd cd
b
Figure 5. Figure 6.
Games in which both players know the rules of the game, that is, in which both players
know the Kuhn tree, are called games of complete information. Games in which one or
both of the players do not know some of the payoffs, or some of the probabilities of chance
moves, or some of the information sets, or even whole branches of the tree, are called
games with incomplete information, or pseudogames. We assume in the following
that we are dealing with games of complete information.
%=================================================================%
\subsection{5.4 The Representation of a Strategic Form Game in Extensive Form.}
\begin{itemize} 
\item The
notion of a game in strategic form is quite simple. It is described by a triplet (X, Y, A) as in
Section 1. The extensive form of a game on the other hand is quite complex. It is described
%II – 53
by the game tree with each non-terminal vertex labeled as a chance move or as a move
of one of the players, with all information sets specified, with probability distributions
given for all chance moves, and with a payoff attached to each terminal vertex. 
\itme It would
seem that the theory of games in extensive is much more comprehensive than the theory
of games in strategic form. 
\item However, by taking a game in extensive form and considering
only the strategies and average payoffs, we may reduce the game to strategic form.
First, let us check that a game in strategic form can be put into extensive form. In the
strategic form of a game, the players are considered to make their choices simultaneously,
while in the extensive form of a game simultaneous moves are not allowed.
\item  However,
simultaneous moves may be made sequentially as follows. We let one of the players, say
Player I, move first, and then let player II move without knowing the outcome of I’s move.
\item This lack of knowledge may be described by the use of an appropriate information set.
The example below illustrates this.
\end{itemize}
 3 01
−120
3 0 1 –1 2 0
I
II
1 2
1 23 1 23
\noindent \textbf{Matrix Form Equivalent Extensive Form}
Player I has 2 pure strategies and Player II has 3. We pretend that Player I moves first by
choosing row 1 or row 2. Then Player II moves, not knowing the choice of Player I. This is
indicated by the information set for Player II. Then Player II moves by choosing column
1, 2 or 3, and the appropriate payoff is made.
%=============================================================================%
\subsection{5.5 Reduction of a Game in Extensive Form to Strategic Form.} To go in
the reverse direction, from the extensive form of a game to the strategic form, requires the
consideration of pure strategies and the usual convention regarding random payoffs.
Pure strategies. Given a game in extensive form, we first find X and Y , the sets of
pure strategies of the players to be used in the strategic form. A pure strategy for Player
I is a rule that tells him exactly what move to make in each of their information sets. Let
T11,...,T1k1 be the information sets for Player I and let L11,...,L1k1 be the corresponding
sets of labels. A pure strategy for I is a k1-tuple x = (xl, ..., xk1 ) where for each i, xi is one
of the elements of L1i. If there are mi elements in L1i, the number of such kl-tuples and
hence the number of I s pure strategies is the product m1m2 ··· mk. The set of all such
strategies is X. Similarly, if T21,...,T2k2 represent II’s information sets and L21,...,L2k2
the corresponding sets of labels, a pure strategy for II is a k2-tuple, y = (y1,...,yk2 ) where
yj \in L2j for each j. Player II has $n_1,n_2 ··· nk2$ pure strategies if there are nj elements in
L2j . Y denotes the set of these strategies.
Random payoffs. A referee, given $x \in X $and $y \in Y $ , could play the game, playing the
appropriate move from x whenever the game enters one of I’s information sets, playing the
II – 54
appropriate move from y whenever the game enters one of II’s information sets, and playing
the moves at random with the indicated probabilities at each chance move. The actual
outcome of the game for given $x \in X $and $y \in Y $ depends on the chance moves selected,
and is therefore a random quantity. Strictly speaking, random payoffs were not provided
for in our definition of games in normal form. However, we are quite used to replacing
random payoffs by their average values (expected values) when the randomness is due to
the use of mixed strategies by the players. We adopt the same convention in dealing with
random payoffs when the randomness is due to the chance moves. The justification of this
comes from utility theory.
Convention. If for fixed pure strategies of the players, $x \in X $and $y \in Y $ , the payoff is
a random quantity, we replace the payoff by the average value, and denote this average
value by $A(x, y)$.
%==========================================================================%
For example, if for given strategies $x \in X $and $y \in Y $ , Player I wins 3 with probability
1/4, wins 1 with probability 1/4, and loses 1 with probability 1/2, then their average payoff
is 1
4 (3) + 1
4 (1) + 1
2 (−1) = 1/2 so we let $A(x, y)$=1/2.
Therefore, given a game in extensive form, we say (X, Y, A) is the equivalent strategic
form of the game if X and Y are the pure strategy spaces of players I and II respectively,
and if $A(x, y)$ is the average payoff for $x \in X $and $y \in Y $ .
5.6 Example. Let us find the equivalent strategic form to Basic Endgame in Poker
described in the Section 5.2, whose tree is given in Figure 2. Player I has two information
sets. In each set they must make a choice from among two options. they therefore has 2·2=4
pure strategies. We may denote them by
\begin{itemize}
\item (b, b): bet with a winning card or a losing card.
\item (b, c): bet with a winning card, check with a losing card.
\item (c, b): check with a winning card, bet with a losing card.
\item (c, c): check with a winning card or a losing card.
\end{itemize}
Therefore, X = {(b, b),(b, c),(c, b),(c, c)}. We include in X all pure strategies whether
good or bad (in particular, (c, b) seems a rather perverse sort of strategy.)
Player II has only one information set. Therefore, Y = {c, f}, where
c: if I bets, call.
f: if I bets, fold.
Now we find the payoff matrix. Suppose I uses (b, b) and II uses c. Then if I gets a
winning card (which happens with probability 1/4), they bets, II calls, and I wins 3 dollars.
But if I gets a losing card (which happens with probability 3/4), they bets, II calls, and I
loses 3 dollars. I’s average or expected winnings is
A((b, b), c) = 1
4
(3) + 3
4
(−3) = −3
2
.
II – 55
This gives the upper left entry in the following matrix. The other entries may be computed
similarly and are left as exercises.
⎛
⎜⎜⎝
c f
(b, b) −3/2 1
(b, c) 0 −1/2
(c, b) −2 1
(c, c) −1/2 −1/2
⎞
⎟⎟⎠
Let us solve this 4 by 2 game. The third row is dominated by the first row, and the
fourth row is dominated by the second row. In terms of the original form of the game, this
says something you may already have suspected: that if I gets a winning card, it cannot
be good for him to check. By betting they will win at least as much, and maybe more. With
the bottom two rows eliminated the matrix becomes  −3/2 1
0 −1/2

, whose solution is
easily found. The value is V = −1/4. I’s optimal strategy is to mix (b, b) and (b, c) with
probabilities 1/6 and 5/6 respectively, while II’s optimal strategy is to mix c and f with
equal probabilities 1/2 each. The strategy (b, b) is Player I’s bluffing strategy. Its use
entails betting with a losing hand. The strategy (b, c) is Player I’s “honest” strategy, bet
with a winning hand and check with a losing hand. I’s optimal strategy requires some
bluffing and some honesty.
%===========================================================%
In Exercise 4, there are six information sets for I each with two choices. The number
of I’s pure strategies is therefore 26 = 64. II has 2 information sets each with two choices.
Therefore, II has 22 = 4 pure strategies. The game matrix for the equivalent strategic
form has dimension 64 by 4. Dominance can help reduce the dimension to a 2 by 3 game!
(See Exercise 10(d).)
%===========================================================%
\section{5.7 Games of Perfect Information.} 
\begin{itemize}
\item Now that a game in extensive form has been
defined, we may make precise the notion of a game of perfect information.
Definition. A game of perfect information is a game in extensive form in which each
information set of every player contains a single vertex.
\item In a game of perfect information, each player when called upon to make a move knows
the exact position in the tree. In particular, each player knows all the past moves of the
game including the chance ones.
\item  Examples include tic-tac-toe, chess, backgammon, craps,
etc.
Games of perfect information have a particularly simple mathematical structure. The
main result is that every game of perfect information when reduced to strategic form has
a saddle point; both players have optimal pure strategies.
\item Moreover, the saddle point can
be found by removing dominated rows and columns. This has an interesting implication
for the game of chess for example. Since there are no chance moves, every entry of the
game matrix for chess must be either +1 (a win for Player I), or −1 (a win for Player
% II – 56
II), or 0 (a draw). \item A saddle point must be one of these numbers. Thus, either Player
I can guarantee himself a win, or Player II can guarantee himself a win, or both players
can assure themselves at least a draw. From the game-theoretic viewpoint, chess is a very
simple game. One needs only to write down the matrix of the game. If there is a row of all
+1’s, Player I can win. If there is a column of all −1’s, then Player II can win. Otherwise,
there is a row with all +1’s and 0’s and a column with all −1’s and 0’s, and so the game is
drawn with best play. Of course, the real game of chess is so complicated, there is virtually
no hope of ever finding an optimal strategy. In fact, it is not yet understood how humans
can play the game so well.
\end{itemize
%==============================================================%
\subsection{5.8 Behavioral Strategies.} 
\begin{itemize}
\item For games in extensive form, it is useful to consider a
different method of randomization for choosing among pure strategies. All a player really
needs to do is to make one choice of an edge for each of their information sets in the game.
A behavioral strategy is a strategy that assigns to each information set a probability
distributions over the choices of that set.
\item For example, suppose the first move of a game is the deal of one card from a deck of
52 cards to Player I. After seeing their card, Player I either bets or passes, and then Player
II takes some action. Player I has 52 information sets each with 2 choices of action, and
so they has 252 pure strategies. Thus, a mixed strategy for I is a vector of 252 components
adding to 1. On the other hand, a behavioral strategy for I simply given by the probability
of betting for each card they may receive, and so is specified by only 52 numbers.
\item In general, the dimension of the space of behavioral strategies is much smaller than
the dimension of the space of mixed strategies. The question arises – Can we do as well
with behavioral strategies as we can with mixed strategies? The answer is we can if both
players in the game have perfect recall. The basic theorem, due to Kuhn in 1953 says that
in finite games with perfect recall, any distribution over the payoffs achievable by mixed
strategies is achievable by behavioral strategies as well.
\item To see that behavioral strategies are not always sufficient, consider the game of imperfect
recall of Figure 4. Upon reducing the game to strategic form, we find the matrix
\end{itemize}
⎛
⎜⎜⎝
a b
(f, c) 1 −1
(f, d)10
(g, c)02
(g, d) −1 2
⎞
⎟⎟⎠
The top and bottom rows may be removed by domination, so it is easy to see that the
unique optimal mixed strategies for I and II are (0, 2/3, 1/3, 0) and (2/3, 1/3) respectively.
The value is 2/3. However, Player I’s optimal strategy is not achievable by behavioral
strategies. A behavioral strategy for I is given by two numbers, pf , the probability of choice
f in the first information set, and pc, the probability of choice c in the second information
set. This leads to the mixed strategy, (pf pc, pf (1 − pc),(1 − pf )pc,(1 − pf )(1 − pc)). The
strategy (0, 2/3, 1/3, 0) is not of this form since if the first component is zero, that is if
II – 57
pf pc = 0, then either pf = 0 or pc = 0, so that either the second or third component must
be zero also.
If the rules of the game require players to use behavioral strategies, as is the case for
certain models of bridge, then the game may not have a value. This means that if Player I
is required to announce their behavioral strategy first, then they is at a distinct disadvantage.
The game of Figure 4 is an example of this. (see Exercise 11.)
\subsection{5.9 Exercises.}
\subsubsection{1. The Silver Dollar.} Player II chooses one of two rooms in which to hide a silver
dollar. Then, Player I, not knowing which room contains the dollar, selects one of the
rooms to search. However, the search is not always successful. In fact, if the dollar is
in room #1 and I searches there, then (by a chance move) they has only probability 1/2 of
finding it, and if the dollar is in room #2 and I searches there, then they has only probability
1/3 of finding it. Of course, if they searches the wrong room, they certainly won’t find it. If
he does find the coin, they keeps it; otherwise the dollar is returned to Player II. Draw the
game tree.
\subsubsection{2. Two Guesses for the Silver Dollar.} Draw the game tree for problem 1, if
when I is unsuccessful in their first attempt to find the dollar, they is given a second chance
to choose a room and search for it with the same probabilities of success, independent of
his previous search. (Player II does not get to hide the dollar again.)
\subsubsection{3. A Statistical Game.} Player I has two coins. One is fair (probability 1/2 of heads
and 1/2 of tails) and the other is biased with probability 1/3 of heads and 2/3 of tails.
Player I knows which coin is fair and which is biased. they selects one of the coins and tosses
it. The outcome of the toss is announced to II. Then II must guess whether I chose the
fair or biased coin. If II is correct there is no payoff. If II is incorrect, they loses 1. Draw
the game tree.
\subsubsection{4. A Forgetful Player.} A fair coin (probability 1/2 of heads and 1/2 of tails) is
tossed and the outcome is shown to Player I. On the basis of the outcome of this toss, I
decides whether to bet 1 or 2. Then Player II hearing the amount bet but not knowing
the outcome of the toss, must guess whether the coin was heads or tails. Finally, Player I
(or, more realistically, their partner), remembering the amount bet and II’s guess, but not
remembering the outcome of the toss, may double or pass. II wins if their guess is correct
and loses if their guess is incorrect. The absolute value of the amount won is [the amount
bet (+1 if the coin comes up heads)] (×2 if I doubled). Draw the game tree.
\subsubsection{5. A One-Shot Game of Incomplete Information.} Consider the two games
G1 =
 6 0
0 0
and G2 =
 3 0
0 6
. One of these games is chosen to be played at random
with probability 1/3 for G1 and probability 2/3 for G2. The game chosen is revealed to
Player I but not to Player II. Then Player I selects a row, 1 or 2, and simultaneously
Player II chooses a column, 1 or 2, with payoff determined by the selected game. Draw
the game tree. (If the game chosen by nature is played repeatedly with Player II learning
only the pure strategy choices of Player I and not the payoffs, this is called a repeated
%%II – 58
game of incomplete information. There is a large literature concerning such games; see for
example, the books of Aumann and Maschler (1995) and Sorin (2002).)
\subsubsection{6. Basic Endgame in Poker.} Generalize Basic Endgame in poker by letting the
probability of receiving a winning card be an arbitrary number p, 0 ≤ p ≤ 1, and by letting
the bet size be an arbitrary number b > 0. (In Figure 2, 1/4 is replaced by p and 3/4 is
replaced by 1 − p. Also 3 is replaced by 1 + b and −3 is replaced by −(1 + b).) Find the
value and optimal strategies. (Be careful. For p ≥ (2 + b)/(2 + 2b) there is a saddle point.
When you are finitheyd, note that for p < (2+b)/(2+2b), Player II’s optimal strategy does
not depend on p!) For other generalizations, see Ferguson and Ferguson (2007).
7. (a) Find the equivalent strategic form of the game with the game tree:
3
3 0 3 –3 –3 0 0 0 3
N
I II
II
1/3 2/3
a b c d e
f g d e ab c f g
(b) Solve the game.
8. (a). Find the equivalent strategic form of the game with the game tree:
0 2 1 –1 2 –2 0 4
N
I I
II II
1/2 1/2
A B C D
ab cd ab cd
(b). Solve the game.
%------------------------------------------------------------%
9. Coin A has probability 1/2 of heads and 1/2 of tails. Coin B has probability 1/3 of
heads and 2/3 of tails. Player I must predict “heads” or “tails”. If they predicts heads, coin
A is tossed. If they predicts tails, coin B is tossed. Player II is informed as to whether I’s
prediction was right or wrong (but they is not informed of the prediction or the coin that
was used), and then must guess whether coin A or coin B was used. If II guesses correctly
%%- II – 59
they wins 1 dollar from I. If II guesses incorrectly and I’s prediction was right, I wins 2
dollars from II. If both are wrong there is no payoff.
(a) Draw the game tree.
(b) Find the equivalent strategic form of the game.
(c) Solve.
%------------------------------------------------------------%
10. Find the equivalent strategic form and solve the game of
(a) Exercise 1.
(b) Exercise 2.
(c) Exercise 3.
(d) Exercise 4.
(e) Exercise 5.
%------------------------------------------------------------%
11. Suppose, in the game of Figure 4, that Player I is required to use behavioral
strategies. Show that if Player I is required to announce their behavioral strategy first, he
can only achieve a lower value of 1/2. Whereas, if Player II is required to announce her
strategy first, Player I has a behavioral strategy reply that achieves the upper value of 2/3
at least.
%-------------------------------------------------------------%
12. (Beasley (1990), Chap. 6.) Player I draws a card at random from a full deck of
52 cards. After looking at the card, they bets either 1 or 5 that the card they drew is a face
card (king, queen or jack, probability 3/13). Then Player II either concedes or doubles. If
they concedes, they pays I the amount bet (no matter what the card was). If they doubles,
the card is shown to her, and Player I wins twice their bet if the card is a face card, and
loses twice their bet otherwise.
(a) Draw the game tree. (You may argue first that Player I always bets 5 with a face card
and Player II always doubles if Player I bets 1.)
(b) Find the equivalent normal form.
(c) Solve.
II – 60
\end{document}
