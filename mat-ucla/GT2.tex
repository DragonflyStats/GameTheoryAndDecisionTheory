2. Matrix Games — Domination
A finite two-person zero-sum game in strategic form, (X, Y, A), is sometimes called
a matrix game because the payoff function A can be represented by a matrix. If X =
{x1,...,xm} and Y = {y1,...,yn}, then by the game matrix or payoff matrix we mean
the matrix
A =
⎛
⎝
a11 ··· a1n
.
.
. .
.
.
am1 ··· amn
⎞
⎠ where $ a_{ij}$  = A(xi , yj ),
In this form, Player I chooses a row, Player II chooses a column, and II pays I the entry
in the chosen row and column. Note that the entries of the matrix are the winnings of the
row chooser and losses of the column chooser.
A mixed strategy for Player I may be represented by an m-tuple, p = (p1, p2,...,pm)
T
of probabilities that add to 1. If I uses the mixed strategy p = (p1, p2,...,pm)
T and II
chooses column j, then the (average) payoff to I is m
i=1 pi$ a_{ij}$  . Similarly, a mixed strategy
for Player II is an n-tuple q = (q1, q2,...,qn)
T. If II uses q and I uses row i the payoff
to I is n
j=1 $ a_{ij}$  qj . More generally, if I uses the mixed strategy p and II uses the mixed
strategy q, the (average) payoff to I is pTAq = m
i=1
n
j=1 pi$ a_{ij}$  qj .
Note that the pure strategy for Player I of choosing row i may be represented as the
mixed strategy ei, the unit vector with a 1 in the ith position and 0’s elsewhere. Similarly,
the pure strategy for II of choosing the jth column may be represented by ej. In the
following, we shall be attempting to ‘solve’ games. This means finding the value, and at
least one optimal strategy for each player. Occasionally, we shall be interested in finding
all optimal strategies for a player.

%====================================================================%
\subsection{2.1 Saddle points.} Occasionally it is easy to solve the game. If some entry $ a_{ij}$  of
the matrix A has the property that
(1) $ a_{ij}$  is the minimum of the ith row, and
(2) $ a_{ij}$  is the maximum of the jth column,
then we say $ a_{ij}$  is a saddle point. If $ a_{ij}$  is a saddle point, then Player I can then win at
least $ a_{ij}$  by choosing row i, and Player II can keep her loss to at most $a_{ij}$  by choosing
column j. Hence $ a_{ij}$  is the value of the game.
Example 1.
A =
⎛
⎝
4 1 −3
325
016
⎞
⎠
The central entry, 2, is a saddle point, since it is a minimum of its row and maximum
of its column. Thus it is optimal for I to choose the second row, and for II to choose the
second column. The value of the game is 2, and (0, 1, 0) is an optimal mixed strategy for
both players.
II – 9
For large $ m \times n$  matrices it is tedious to check each entry of the matrix to see if it
has the saddle point property. It is easier to compute the minimum of each row and the
maximum of each column to see if there is a match. Here is an example of the method.
row min
A =
⎛
⎜⎝
3210
0120
1021
3122
⎞
⎟⎠
0
0
0
1
col max 3 2 2 2
row min
B =
⎛
⎜⎝
3110
0120
1021
3122
⎞
⎟⎠
0
0
0
1
col max 3 1 2 2
In matrix A, no row minimum is equal to any column maximum, so there is no saddle
point. However, if the 2 in position a12 were changed to a 1, then we have matrix B. Here,
the minimum of the fourth row is equal to the maximum of the second column; so b42 is a
saddle point.
2.2 Solution of All 2 by 2 Matrix Games. Consider the general 2 × 2 game
matrix
A =
 a b
d c
.


%====================================================================%
To solve this game (i.e. to find the value and at least one optimal strategy for each player)
we proceed as follows:
\begin{emumerate}
\item Test for a saddle point.
\item If there is no saddle point, solve by finding equalizing strategies.
\end{enumerate}

We now prove the method of finding equalizing strategies of Section 1.2 works whenever
there is no saddle point by deriving the value and the optimal strategies.
Assume there is no saddle point. If a \geq b, then b<c, as otherwise b is a saddle point.
Since b<c, we must have c>d, as otherwise c is a saddle point. Continuing thus, we see
that d<a and a>b. In other words, if a \geq b, then a>b<c>d<a. By symmetry, if
a ≤ b, then a<b>c<d>a. This shows that
If there is no saddle point, then either a>b, b<c, c>d and d<a, or a<b, b>c,
c<d and d>a.

%====================================================================%
In equations (1), (2) and (3) below, we develop formulas for the optimal strategies
and value of the general 2 × 2 game. If I chooses the first row with probability p (i.e. uses
the mixed strategy (p, 1 − p)), we equate his average return when II uses columns 1 and 2.
ap + d(1 − p) = bp + c(1 − p).
Solving for p, we find
p = c − d
(a − b)+(c − d)
. (1)
II – 10
Since there is no saddle point, (a−b) and (c−d) are either both positive or both negative;
hence, 0 <p< 1. Player I’s average return using this strategy is
v = ap + d(1 − p) = ac − bd
a − b + c − d
.
If II chooses the first column with probability q (i.e. uses the strategy (q, 1−q)), we equate
his average losses when I uses rows 1 and 2.
aq + b(1 − q) = dq + c(1 − q)
Hence,
q = c − b
a − b + c − d
. (2)
Again, since there is no saddle point, 0 <q< 1. Player II’s average loss using this strategy
is
aq + b(1 − q) = ac − bd
a − b + c − d = v, (3)
the same value achievable by I. This shows that the game has a value, and that the players
have optimal strategies. (something the minimax theorem says holds for all finite games).
Example 2.
A =
 −2 3
3 −4

p = −4 − 3
−2 − 3 − 4 − 3 = 7/12
q = same
v = 8 − 9
−2 − 3 − 4 − 3
= 1/12
Example 3.
A =
 0 −10
1 2  p = 2 − 1
0 + 10 + 2 − 1
= 1/11
q = 2 + 10
0 + 10 + 2 − 1
= 12/11.
But q must be between zero and one. What happened? The trouble is we “forgot to test
this matrix for a saddle point, so of course it has one”. (J. D. Williams The Compleat
Strategyst Revised Edition, 1966, McGraw-Hill, page 56.) The lower left corner is a saddle
point. So p = 0 and q = 1 are optimal strategies, and the value is v = 1.
2.3 Removing Dominated Strategies. Sometimes, large matrix games may be
reduced in size (hopefully to the 2×2 case) by deleting rows and columns that are obviously
bad for the player who uses them.
Definition. We say the ith row of a matrix A = ($ a_{ij}$  ) dominates the kth row if
$ a_{ij}  \geq a_{kj}$ for all j. We say the ith row of A strictly dominates the kth row if $ a_{ij}$  > akj
for all j. Similarly, the jth column of A dominates (strictly dominates) the kth column if
$ a_{ij}$  ≤ aik (resp. $ a_{ij}$  < aik) for all i.

%====================================================================%
Anything Player I can achieve using a dominated row can be achieved at least as well
using the row that dominates it. Hence dominated rows may be deleted from the matrix.
A similar argument shows that dominated columns may be removed. To be more precise,
removal of a dominated row or column does not change the value of a game. However, there
may exist an optimal strategy that uses a dominated row or column (see Exercise 9). If so,
removal of that row or column will also remove the use of that optimal strategy (although
there will still be at least one optimal strategy left). However, in the case of removal of a
strictly dominated row or column, the set of optimal strategies does not change.

%====================================================================%
We may iterate this procedure and successively remove several rows and columns. As
an example, consider the matrix, A.
The last column is dominated by the middle
column. Deleting the last column we obtain:
A =
⎛
⎝
204
123
412
⎞
⎠
Now the top row is dominated by the bottom
row. (Note this is not the case in the original
matrix). Deleting the top row we obtain:
⎛
⎝
2 0
1 2
4 1
⎞
⎠
This 2 × 2 matrix does not have a saddle point, so p = 3/4,
q = 1/4 and v = 7/4. I’s optimal strategy in the original game is
(0, 3/4, 1/4); II’s is (1/4, 3/4, 0).
 1 2
4 1
A row (column) may also be removed if it is dominated by a probability combination
of other rows (columns).
%-------------------------------------------------------%
If for some 0 <p< 1, pai1j +(1−p)ai2j \geq akj for all j, then the kth row is dominated
by the mixed strategy that chooses row i1 with probability p and row i2 with probability
1 − p. Player I can do at least as well using this mixed strategy instead of choosing row
k. (In addition, any mixed strategy choosing row k with probability pk may be replaced
by the one in which k’s probability is split between i1 and i2. That is, i1’s probability is
increased by ppk and i2’s probability is increased by (1 − p)pk.) A similar argument may
be used for columns.
Consider the matrix A =
⎛
⎝
046
574
963
⎞
⎠.
The middle column is dominated by the outside columns taken with probability 1/2
each. With the central column deleted, the middle row is dominated by the combination
of the top row with probability 1/3 and the bottom row with probability 2/3. The reduced
matrix,  0 6
9 3
, is easily solved. The value is V = 54/12 = 9/2.
Of course, mixtures of more than two rows (columns) may be used to dominate and
remove other rows (columns). For example, the mixture of columns one two and three
with probabilities 1/3 each in matrix B =
⎛
⎝
1353
4022
3735
⎞
⎠ dominates the last column,
II – 12
and so the last column may be removed.
Not all games may be reduced by dominance. In fact, even if the matrix has a saddle
point, there may not be any dominated rows or columns. The 3 × 3 game with a saddle
point found in Example 1 demonstrates this.

%====================================================================%
2.4 Solving 2 × n and m × 2 games. Games with matrices of size 2 × n or m × 2
may be solved with the aid of a graphical interpretation. Take the following example.
p
1 − p
 2315
4160
Suppose Player I chooses the first row with probability p and the second row with probability
1−p. If II chooses Column 1, I’s average payoff is 2p+4(1−p). Similarly, choices of
Columns 2, 3 and 4 result in average payoffs of 3p+(1−p), p+6(1−p), and 5p respectively.
We graph these four linear functions of p for 0 ≤ p ≤ 1. For a fixed value of p, Player I can
be sure that his average winnings is at least the minimum of these four functions evaluated
at p. This is known as the lower envelope of these functions. Since I wants to maximize
his guaranteed average winnings, he wants to find p that achieves the maximum of this
lower envelope. According to the drawing, this should occur at the intersection of the lines
for Columns 2 and 3. This essentially, involves solving the game in which II is restricted
to Columns 2 and 3. The value of the game  3 1
1 6
is v = 17/7 , I’s optimal strategy is
(5/7, 2/7), and II’s optimal strategy is (5/7, 2/7). Subject to the accuracy of the drawing,
we conclude therefore that in the original game I’s optimal strategy is (5/7, 2/7) , II’s is
(0, 5/7, 2/7, 0) and the value is 17/7.
Fig 2.1
0
1
2
3
4
5
6
col. 3
col. 1
col. 2
col. 4
0 1 5/7 p
The accuracy of the drawing may be checked: Given any guess at a solution to a
game, there is a sure-fire test to see if the guess is correct, as follows. If I uses the strategy
(5/7, 2/7), his average payoff if II uses Columns 1, 2, 3 and 4, is 18/7, 17/7, 17/7, and 25/7
II – 13
respectively. Thus his average payoff is at least 17/7 no matter what II does. Similarly,
if II uses (0, 5/7, 2/7, 0), her average loss is (at most) 17/7. Thus, 17/7 is the value, and
these strategies are optimal.
We note that the line for Column 1 plays no role in the lower envelope (that is, the
lower envelope would be unchanged if the line for Column 1 were removed from the graph).
This is a test for domination. Column 1 is, in fact, dominated by Columns 2 and 3 taken
with probability 1/2 each. The line for Column 4 does appear in the lower envelope, and
hence Column 4 cannot be dominated.

%====================================================================%
As an example of a m × 2 game, consider the matrix associated with Figure 2.2. If
q is the probability that II chooses Column 1, then II’s average loss for I’s three possible
choices of rows is given in the accompanying graph. Here, Player II looks at the largest
of her average losses for a given q. This is the upper envelope of the function. II wants
to find q that minimizes this upper envelope. From the graph, we see that any value of q
between 1/4 and 1/3 inclusive achieves this minimum. The value of the game is 4, and I
has an optimal pure strategy: row 2.
Fig 2.2
⎛
⎝
q 1 − q
1 5
4 4
6 2
⎞
⎠
0
1
2
3
4
5
6
row 1
row 2
row 3
0 1 1/4 1/2
q
These techniques work just as well for 2 × ∞ and ∞ × 2 games.

%====================================================================%
2.5 Latin Square Games. A Latin square is an n × n array of n different letters
such that each letter occurs once and only once in each row and each column. The 5 × 5
array at the right is an example. If in a Latin square each letter is assigned a numerical
value, the resulting matrix is the matrix of a Latin square game. Such games have simple
solutions. The value is the average of the numbers in a row, and the strategy that chooses
each pure strategy with equal probability 1/n is optimal for both players. The reason is
not very deep. The conditions for optimality are satisfied.
II – 14
⎛
⎜⎜⎜⎝
abcde
beacd
cadeb
dceba
edbac
⎞
⎟⎟⎟⎠
a = 1, b = 2, c = d = 3, e = 6
⎛
⎜⎜⎜⎝
12336
26133
31362
33621
63213
⎞
⎟⎟⎟⎠
In the example above, the value is V = (1+2+3+3+6)/5 = 3, and the mixed strategy
p = q = (1/5, 1/5, 1/5, 1/5, 1/5) is optimal for both players. The game of matching pennies
is a Latin square game. Its value is zero and (1/2, 1/2) is optimal for both players.
2.6 Exercises.
1. Solve the game with matrix  −1 −3
−2 2
, that is find the value and an optimal
(mixed) strategy for both players.
2. Solve the game with matrix  0 2
t 1

for an arbitrary real number t. (Don’t forget
to check for a saddle point!) Draw the graph of v(t), the value of the game, as a function
of t, for −∞ <t< ∞.
3. Show that if a game with m×n matrix has two saddle points, then they have equal
values.
4. Reduce by dominance to 2 × 2 games and solve.
(a)
⎛
⎜⎝
5410
432 −1
0 −143
1 −212
⎞
⎟⎠ (b)
⎛
⎝
10 0 7 1
2 647
6 335
⎞
⎠.
5. (a) Solve the game with matrix  3240
−2 1 −4 5
.
(b) Reduce by dominance to a 3 × 2 matrix game and solve:
⎛
⎝
085
846
12 −4 3
⎞
⎠.
6. Players I and II choose integers i and j respectively from the set {1, 2,...,n} for
some n \geq 2. Player I wins 1 if |i − j| = 1. Otherwise there is no payoff. If n = 7, for
example, the game matrix is
⎛
⎜⎜⎜⎜⎜⎜⎜⎝
0100000
1010000
0101000
0010100
0001010
0000101
0000010
⎞
⎟⎟⎟⎟⎟⎟⎟⎠
II – 15
(a) Using dominance to reduce the size of the matrix, solve the game for n = 7 (i.e.
find the value and one optimal strategy for each player).
(b) See if you can solve the game for arbitrary n.
%------------------------------------------%
7. In general, the sure-fire test may be stated thus: For a given game, conjectured
optimal strategies (p1,...,pm) and (q1,...,qn) are indeed optimal if the minimum of I’s
average payoffs using (p1,...,pm) is equal to the maximum of II’s average payoffs using
(q1,...,qn). Show that for the game with the following matrix the mixed strategies
p = (6/37, 20/37, 0, 11/37) and q = (14/37, 4/37, 0, 19/37, 0) are optimal for I and II respectively.
What is the value?
⎛
⎜⎝
58316
42635
24641
13253
⎞
⎟⎠
8. Given that p = (52/143, 50/143, 41/143) is optimal for I in the game with the
following matrix, what is the value?
⎛
⎝
0 5 −2
−304
6 −4 0
⎞
⎠
9. Player I secretly chooses one of the numbers, 1, 2 and 3, and Player II tries to guess
which. If II guesses correctly, she loses nothing; otherwise, she loses the absolute value of
the difference of I’s choice and her guess. Set up the matrix and reduce it by dominance
to a 2 by 2 game and solve. Note that II has an optimal pure strategy that was eliminated
by dominance. Moreover, this strategy dominates the optimal mixed strategy in the 2 by
2 game.
10. Magic Square Games. A magic square is an n × n array of the first n integers
with the property that all row and column sums are equal. Show how to solve all games
with magic square game matrices. Solve the example,
⎛
⎜⎝
16 3 2 13
5 10 11 8
9 6 7 12
4 15 14 1
⎞
⎟⎠ .
(This is the magic square that appears in Albrecht D¨urer’s engraving, Melencolia. See
http://freemasonry.bcy.ca/art/melencolia.html)

%====================================================================%
11. In an article, “Normandy: Game and Reality” by W. Drakert in Moves, No. 6
(1972), an analysis is given of the invasion of Europe at Normandy in World War II. Six
possible attacking configurations (1 to 6) by the Allies and six possible defensive strategies
(A to F) by the Germans were simulated and evaluated, 36 simulations in all. The following
II – 16
table gives the estimated value to the Allies of each hypothetical battle in some numerical
units.
⎛
⎜⎜⎜⎜⎜⎜⎝
ABCDEF
1 13 29 8 12 16 23
2 18 22 21 22 29 31
3 18 22 31 31 27 37
4 11 22 12 21 21 26
5 18 16 19 14 19 28
6 23 22 19 23 30 34
⎞
⎟⎟⎟⎟⎟⎟⎠
(a) Assuming this is a matrix of a six by six game, reduce by dominance and solve.
(b) The historical defense by the Germans was B, and the historical attack by the Allies
was 1. Criticize these choices.
II – 17

