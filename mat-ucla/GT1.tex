GAME THEORY
Thomas S. Ferguson
Part II. Two-Person Zero-Sum Games
1. The Strategic Form of a Game.
1.1 Strategic Form.
1.2 Example: Odd or Even.
1.3 Pure Strategies and Mixed Strategies.
1.4 The Minimax Theorem.
1.5 Exercises.
2. Matrix Games. Domination.
2.1 Saddle Points.
2.2 Solution of All 2 by 2 Matrix Games.
2.3 Removing Dominated Strategies.
2.4 Solving 2 × n and m × 2 Games.
2.5 Latin Square Games.
2.6 Exercises.
3. The Principle of Indifference.
3.1 The Equilibrium Theorem.
3.2 Nonsingular Game Matrices.
3.3 Diagonal Games.
3.4 Triangular Games.
3.5 Symmetric Games.
3.6 Invariance.
3.7 Exercises.
II – 1
4. Solving Finite Games.
4.1 Best Responses.
4.2 Upper and Lower Values of a Game.
4.3 Invariance Under Change of Location and Scale.
4.4 Reduction to a Linear Programming Problem.
4.5 Description of the Pivot Method for Solving Games.
4.6 A Numerical Example.
4.7 Approximating the Solution: Fictitious Play.
4.8 Exercises.
5. The Extensive Form of a Game.
5.1 The Game Tree.
5.2 Basic Endgame in Poker.
5.3 The Kuhn Tree.
5.4 The Representation of a Strategic Form Game in Extensive Form.
5.5 Reduction of a Game in Extensive Form to Strategic Form.
5.6 Example.
5.7 Games of Perfect Information.
5.8 Behavioral Strategies.
5.9 Exercises.
6. Recursive and Stochastic Games.
6.1 Matrix Games with Games as Components.
6.2 Multistage Games.
6.3 Recursive Games. -Optimal Strategies.
6.4 Stochastic Movement Among Games.
6.5 Stochastic Games.
6.6 Approximating the Solution.
6.7 Exercises.
7. Infinite Games.
7.1 The Minimax Theorem for Semi-Finite Games.
II – 2
7.2 Continuous Games.
7.3 Concave and Convex Games.
7.4 Solving Games.
7.5 Uniform[0,1] Poker Models.
7.6 Exercises.
References.
II – 3
Part II. Two-Person Zero-Sum Games
1. The Strategic Form of a Game.
The individual most closely associated with the creation of the theory of games is
John von Neumann, one of the greatest mathematicians of the 20th century. Although
others preceded him in formulating a theory of games - notably Emile Borel - it was von ´
Neumann who published in 1928 the paper that laid the foundation for the theory of
two-person zero-sum games. Von Neumann’s work culminated in a fundamental book on
game theory written in collaboration with Oskar Morgenstern entitled Theory of Games
and Economic Behavior, 1944. Other discussions of the theory of games relevant for our
present purposes may be found in the text book, Game Theory by Guillermo Owen, 2nd
edition, Academic Press, 1982, and the expository book, Game Theory and Strategy by
Philip D. Straffin, published by the Mathematical Association of America, 1993.
The theory of von Neumann and Morgenstern is most complete for the class of games
called two-person zero-sum games, i.e. games with only two players in which one player
wins what the other player loses. In Part II, we restrict attention to such games. We will
refer to the players as Player I and Player II.
1.1 Strategic Form. The simplest mathematical description of a game is the strategic
form, mentioned in the introduction. For a two-person zero-sum game, the payoff
function of Player II is the negative of the payoff of Player I, so we may restrict attention
to the single payoff function of Player I, which we call here A.
Definition 1. The strategic form, or normal form, of a two-person zero-sum game is given
by a triplet (X, Y, A), where
(1) X is a nonempty set, the set of strategies of Player I
(2) Y is a nonempty set, the set of strategies of Player II
(3) A is a real-valued function defined on X × Y . (Thus, A(x, y) is a real number for
every x ∈ X and every y ∈ Y .)
The interpretation is as follows. Simultaneously, Player I chooses x ∈ X and Player
II chooses y ∈ Y , each unaware of the choice of the other. Then their choices are made
known and I wins the amount A(x, y) from II. Depending on the monetary unit involved,
A(x, y) will be cents, dollars, pesos, beads, etc. If A is negative, I pays the absolute value
of this amount to II. Thus, A(x, y) represents the winnings of I and the losses of II.
This is a very simple definition of a game; yet it is broad enough to encompass the
finite combinatorial games and games such as tic-tac-toe and chess. This is done by being
sufficiently broadminded about the definition of a strategy. A strategy for a game of chess,
II – 4
for example, is a complete description of how to play the game, of what move to make in
every possible situation that could occur. It is rather time-consuming to write down even
one strategy, good or bad, for the game of chess. However, several different programs for
instructing a machine to play chess well have been written. Each program constitutes one
strategy. The program Deep Blue, that beat then world chess champion Gary Kasparov
in a match in 1997, represents one strategy. The set of all such strategies for Player I is
denoted by X. Naturally, in the game of chess it is physically impossible to describe all
possible strategies since there are too many; in fact, there are more strategies than there
are atoms in the known universe. On the other hand, the number of games of tic-tac-toe
is rather small, so that it is possible to study all strategies and find an optimal strategy
for each player. Later, when we study the extensive form of a game, we will see that many
other types of games may be modeled and described in strategic form.
To illustrate the notions involved in games, let us consider the simplest non-trivial
case when both X and Y consist of two elements. As an example, take the game called
Odd-or-Even.
1.2 Example: Odd or Even. Players I and II simultaneously call out one of the
numbers one or two. Player I’s name is Odd; he wins if the sum of the numbers is odd.
Player II’s name is Even; she wins if the sum of the numbers is even. The amount paid to
the winner by the loser is always the sum of the numbers in dollars. To put this game in
strategic form we must specify X, Y and A. Here we may choose X = {1, 2}, Y = {1, 2},
and A as given in the following table.
II (even) y
I (odd) x

1 2
1 −2 +3
2 +3 −4

A(x, y) = I’s winnings = II’s losses.
It turns out that one of the players has a distinct advantage in this game. Can you
tell which one it is?
Let us analyze this game from Player I’s point of view. Suppose he calls ‘one’ 3/5ths
of the time and ‘two’ 2/5ths of the time at random. In this case,
1. If II calls ‘one’, I loses 2 dollars 3/5ths of the time and wins 3 dollars 2/5ths of the
time; on the average, he wins −2(3/5) + 3(2/5) = 0 (he breaks even in the long run).
2. If II call ‘two’, I wins 3 dollars 3/5ths of the time and loses 4 dollars 2/5ths of the time;
on the average he wins 3(3/5) − 4(2/5) = 1/5.
That is, if I mixes his choices in the given way, the game is even every time II calls
‘one’, but I wins 20/c on the average every time II calls ‘two’. By employing this simple
strategy, I is assured of at least breaking even on the average no matter what II does. Can
Player I fix it so that he wins a positive amount no matter what II calls?
II – 5
Let p denote the proportion of times that Player I calls ‘one’. Let us try to choose p
so that Player I wins the same amount on the average whether II calls ‘one’ or ‘two’. Then
since I’s average winnings when II calls ‘one’ is −2p + 3(1 − p), and his average winnings
when II calls ‘two’ is 3p − 4(1 − p) Player I should choose p so that
−2p + 3(1 − p)=3p − 4(1 − p)
3 − 5p = 7p − 4
12p = 7
p = 7/12.
Hence, I should call ‘one’ with probability 7/12, and ‘two’ with probability 5/12. On the
average, I wins −2(7/12) + 3(5/12) = 1/12, or 8 1
3 cents every time he plays the game, no
matter what II does. Such a strategy that produces the same average winnings no matter
what the opponent does is called an equalizing strategy.
Therefore, the game is clearly in I’s favor. Can he do better than 8 1
3 cents per game
on the average? The answer is: Not if II plays properly. In fact, II could use the same
procedure:
call ‘one’ with probability 7/12
call ‘two’ with probability 5/12.
If I calls ‘one’, II’s average loss is −2(7/12) + 3(5/12) = 1/12. If I calls ‘two’, II’s average
loss is 3(7/12) − 4(5/12) = 1/12.
Hence, I has a procedure that guarantees him at least 1/12 on the average, and II has
a procedure that keeps her average loss to at most 1/12. 1/12 is called the value of the
game, and the procedure each uses to insure this return is called an optimal strategy or
a minimax strategy.
If instead of playing the game, the players agree to call in an arbitrator to settle this
conflict, it seems reasonable that the arbitrator should require II to pay 8 1
3 cents to I. For
I could argue that he should receive at least 8 1
3 cents since his optimal strategy guarantees
him that much on the average no matter what II does. On the other hand II could argue
that she should not have to pay more than 8 1
3 cents since she has a strategy that keeps
her average loss to at most that amount no matter what I does.
1.3 Pure Strategies and Mixed Strategies. It is useful to make a distinction
between a pure strategy and a mixed strategy. We refer to elements of X or Y as pure
strategies. The more complex entity that chooses among the pure strategies at random in
various proportions is called a mixed strategy. Thus, I’s optimal strategy in the game of
Odd-or-Even is a mixed strategy; it mixes the pure strategies one and two with probabilities
7/12 and 5/12 respectively. Of course every pure strategy, x ∈ X, can be considered as
the mixed strategy that chooses the pure strategy x with probability 1.
In our analysis, we made a rather subtle assumption. We assumed that when a player
uses a mixed strategy, he is only interested in his average return. He does not care about his
II – 6
maximum possible winnings or losses — only the average. This is actually a rather drastic
assumption. We are evidently assuming that a player is indifferent between receiving 5
million dollars outright, and receiving 10 million dollars with probability 1/2 and nothing
with probability 1/2. I think nearly everyone would prefer the $5,000,000 outright. This is
because the utility of having 10 megabucks is not twice the utility of having 5 megabucks.
The main justification for this assumption comes from utility theory and is treated
in Appendix 1. The basic premise of utility theory is that one should evaluate a payoff by
its utility to the player rather than on its numerical monetary value. Generally a player’s
utility of money will not be linear in the amount. The main theorem of utility theory
states that under certain reasonable assumptions, a player’s preferences among outcomes
are consistent with the existence of a utility function and the player judges an outcome
only on the basis of the average utility of the outcome.
However, utilizing utility theory to justify the above assumption raises a new difficulty.
Namely, the two players may have different utility functions. The same outcome may be
perceived in quite different ways. This means that the game is no longer zero-sum. We
need an assumption that says the utility functions of two players are the same (up to
change of location and scale). This is a rather strong assumption, but for moderate to
small monetary amounts, we believe it is a reasonable one.
A mixed strategy may be implemented with the aid of a suitable outside random
mechanism, such as tossing a coin, rolling dice, drawing a number out of a hat and so
on. The seconds indicator of a watch provides a simple personal method of randomization
provided it is not used too frequently. For example, Player I of Odd-or-Even wants an
outside random event with probability 7/12 to implement his optimal strategy. Since
7/12 = 35/60, he could take a quick glance at his watch; if the seconds indicator showed
a number between 0 and 35, he would call ‘one’, while if it were between 35 and 60, he
would call ‘two’.
1.4 The Minimax Theorem. A two-person zero-sum game (X, Y, A) is said to be
a finite game if both strategy sets X and Y are finite sets. The fundamental theorem
of game theory due to von Neumann states that the situation encountered in the game of
Odd-or-Even holds for all finite two-person zero-sum games. Specifically,
The Minimax Theorem. For every finite two-person zero-sum game,
(1) there is a number V , called the value of the game,
(2) there is a mixed strategy for Player I such that I’s average gain is at least V no
matter what II does, and
(3) there is a mixed strategy for Player II such that II’s average loss is at most V no
matter what I does.
This is one form of the minimax theorem to be stated more precisely and discussed in
greater depth later. If V is zero we say the game is fair. If V is positive, we say the game
favors Player I, while if V is negative, we say the game favors Player II.
II – 7
1.5 Exercises.
1. Consider the game of Odd-or-Even with the sole change that the loser pays the
winner the product, rather than the sum, of the numbers chosen (who wins still depends
on the sum). Find the table for the payoff function A, and analyze the game to find the
value and optimal strategies of the players. Is the game fair?
2. Player I holds a black Ace and a red 8. Player II holds a red 2 and a black 7. The
players simultaneously choose a card to play. If the chosen cards are of the same color,
Player I wins. Player II wins if the cards are of different colors. The amount won is a
number of dollars equal to the number on the winner’s card (Ace counts as 1.) Set up the
payoff function, find the value of the game and the optimal mixed strategies of the players.
3. Sherlock Holmes boards the train from London to Dover in an effort to reach the
continent and so escape from Professor Moriarty. Moriarty can take an express train and
catch Holmes at Dover. However, there is an intermediate station at Canterbury at which
Holmes may detrain to avoid such a disaster. But of course, Moriarty is aware of this too
and may himself stop instead at Canterbury. Von Neumann and Morgenstern (loc. cit.)
estimate the value to Moriarty of these four possibilities to be given in the following matrix
(in some unspecified units).
Holmes
Moriarty 
Canterbury Dover
Canterbury 100 −50
Dover 0 100
What are the optimal strategies for Holmes and Moriarty, and what is the value? (Historically,
as related by Dr. Watson in “The Final Problem” in Arthur Conan Doyle’s The
Memoires of Sherlock Holmes, Holmes detrained at Canterbury and Moriarty went on to
Dover.)
4. The entertaining book The Compleat Strategyst by John Williams contains many
simple examples and informative discussion of strategic form games. Here is one of his
problems.
“I know a good game,” says Alex. “We point fingers at each other; either
one finger or two fingers. If we match with one finger, you buy me one Daiquiri,
If we match with two fingers, you buy me two Daiquiris. If we don’t match I let
you off with a payment of a dime. It’ll help pass the time.”
Olaf appears quite unmoved. “That sounds like a very dull game — at least
in its early stages.” His eyes glaze on the ceiling for a moment and his lips flutter
briefly; he returns to the conversation with: “Now if you’d care to pay me 42
cents before each game, as a partial compensation for all those 55-cent drinks I’ll
have to buy you, then I’d be happy to pass the time with you.
Olaf could see that the game was inherently unfair to him so he insisted on a side
payment as compensation. Does this side payment make the game fair? What are the
optimal strategies and the value of the game?
II – 8
2. Matrix Games — Domination
A finite two-person zero-sum game in strategic form, (X, Y, A), is sometimes called
a matrix game because the payoff function A can be represented by a matrix. If X =
{x1,...,xm} and Y = {y1,...,yn}, then by the game matrix or payoff matrix we mean
the matrix
A =
⎛
⎝
a11 ··· a1n
.
.
. .
.
.
am1 ··· amn
⎞
⎠ where aij = A(xi , yj ),
In this form, Player I chooses a row, Player II chooses a column, and II pays I the entry
in the chosen row and column. Note that the entries of the matrix are the winnings of the
row chooser and losses of the column chooser.
A mixed strategy for Player I may be represented by an m-tuple, p = (p1, p2,...,pm)
T
of probabilities that add to 1. If I uses the mixed strategy p = (p1, p2,...,pm)
T and II
chooses column j, then the (average) payoff to I is m
i=1 piaij . Similarly, a mixed strategy
for Player II is an n-tuple q = (q1, q2,...,qn)
T. If II uses q and I uses row i the payoff
to I is n
j=1 aij qj . More generally, if I uses the mixed strategy p and II uses the mixed
strategy q, the (average) payoff to I is pTAq = m
i=1
n
j=1 piaij qj .
Note that the pure strategy for Player I of choosing row i may be represented as the
mixed strategy ei, the unit vector with a 1 in the ith position and 0’s elsewhere. Similarly,
the pure strategy for II of choosing the jth column may be represented by ej. In the
following, we shall be attempting to ‘solve’ games. This means finding the value, and at
least one optimal strategy for each player. Occasionally, we shall be interested in finding
all optimal strategies for a player.
2.1 Saddle points. Occasionally it is easy to solve the game. If some entry aij of
the matrix A has the property that
(1) aij is the minimum of the ith row, and
(2) aij is the maximum of the jth column,
then we say aij is a saddle point. If aij is a saddle point, then Player I can then win at
least aij by choosing row i, and Player II can keep her loss to at most aij by choosing
column j. Hence aij is the value of the game.
Example 1.
A =
⎛
⎝
4 1 −3
325
016
⎞
⎠
The central entry, 2, is a saddle point, since it is a minimum of its row and maximum
of its column. Thus it is optimal for I to choose the second row, and for II to choose the
second column. The value of the game is 2, and (0, 1, 0) is an optimal mixed strategy for
both players.
II – 9
For large m × n matrices it is tedious to check each entry of the matrix to see if it
has the saddle point property. It is easier to compute the minimum of each row and the
maximum of each column to see if there is a match. Here is an example of the method.
row min
A =
⎛
⎜⎝
3210
0120
1021
3122
⎞
⎟⎠
0
0
0
1
col max 3 2 2 2
row min
B =
⎛
⎜⎝
3110
0120
1021
3122
⎞
⎟⎠
0
0
0
1
col max 3 1 2 2
In matrix A, no row minimum is equal to any column maximum, so there is no saddle
point. However, if the 2 in position a12 were changed to a 1, then we have matrix B. Here,
the minimum of the fourth row is equal to the maximum of the second column; so b42 is a
saddle point.
2.2 Solution of All 2 by 2 Matrix Games. Consider the general 2 × 2 game
matrix
A =
 a b
d c
.
To solve this game (i.e. to find the value and at least one optimal strategy for each player)
we proceed as follows.
1. Test for a saddle point.
2. If there is no saddle point, solve by finding equalizing strategies.
We now prove the method of finding equalizing strategies of Section 1.2 works whenever
there is no saddle point by deriving the value and the optimal strategies.
Assume there is no saddle point. If a ≥ b, then b<c, as otherwise b is a saddle point.
Since b<c, we must have c>d, as otherwise c is a saddle point. Continuing thus, we see
that d<a and a>b. In other words, if a ≥ b, then a>b<c>d<a. By symmetry, if
a ≤ b, then a<b>c<d>a. This shows that
If there is no saddle point, then either a>b, b<c, c>d and d<a, or a<b, b>c,
c<d and d>a.
In equations (1), (2) and (3) below, we develop formulas for the optimal strategies
and value of the general 2 × 2 game. If I chooses the first row with probability p (i.e. uses
the mixed strategy (p, 1 − p)), we equate his average return when II uses columns 1 and 2.
ap + d(1 − p) = bp + c(1 − p).
Solving for p, we find
p = c − d
(a − b)+(c − d)
. (1)
II – 10
Since there is no saddle point, (a−b) and (c−d) are either both positive or both negative;
hence, 0 <p< 1. Player I’s average return using this strategy is
v = ap + d(1 − p) = ac − bd
a − b + c − d
.
If II chooses the first column with probability q (i.e. uses the strategy (q, 1−q)), we equate
his average losses when I uses rows 1 and 2.
aq + b(1 − q) = dq + c(1 − q)
Hence,
q = c − b
a − b + c − d
. (2)
Again, since there is no saddle point, 0 <q< 1. Player II’s average loss using this strategy
is
aq + b(1 − q) = ac − bd
a − b + c − d = v, (3)
the same value achievable by I. This shows that the game has a value, and that the players
have optimal strategies. (something the minimax theorem says holds for all finite games).
Example 2.
A =
 −2 3
3 −4

p = −4 − 3
−2 − 3 − 4 − 3 = 7/12
q = same
v = 8 − 9
−2 − 3 − 4 − 3
= 1/12
Example 3.
A =
 0 −10
1 2  p = 2 − 1
0 + 10 + 2 − 1
= 1/11
q = 2 + 10
0 + 10 + 2 − 1
= 12/11.
But q must be between zero and one. What happened? The trouble is we “forgot to test
this matrix for a saddle point, so of course it has one”. (J. D. Williams The Compleat
Strategyst Revised Edition, 1966, McGraw-Hill, page 56.) The lower left corner is a saddle
point. So p = 0 and q = 1 are optimal strategies, and the value is v = 1.
2.3 Removing Dominated Strategies. Sometimes, large matrix games may be
reduced in size (hopefully to the 2×2 case) by deleting rows and columns that are obviously
bad for the player who uses them.
Definition. We say the ith row of a matrix A = (aij ) dominates the kth row if
aij ≥ akj for all j. We say the ith row of A strictly dominates the kth row if aij > akj
for all j. Similarly, the jth column of A dominates (strictly dominates) the kth column if
aij ≤ aik (resp. aij < aik) for all i.
II – 11
Anything Player I can achieve using a dominated row can be achieved at least as well
using the row that dominates it. Hence dominated rows may be deleted from the matrix.
A similar argument shows that dominated columns may be removed. To be more precise,
removal of a dominated row or column does not change the value of a game. However, there
may exist an optimal strategy that uses a dominated row or column (see Exercise 9). If so,
removal of that row or column will also remove the use of that optimal strategy (although
there will still be at least one optimal strategy left). However, in the case of removal of a
strictly dominated row or column, the set of optimal strategies does not change.
We may iterate this procedure and successively remove several rows and columns. As
an example, consider the matrix, A.
The last column is dominated by the middle
column. Deleting the last column we obtain:
A =
⎛
⎝
204
123
412
⎞
⎠
Now the top row is dominated by the bottom
row. (Note this is not the case in the original
matrix). Deleting the top row we obtain:
⎛
⎝
2 0
1 2
4 1
⎞
⎠
This 2 × 2 matrix does not have a saddle point, so p = 3/4,
q = 1/4 and v = 7/4. I’s optimal strategy in the original game is
(0, 3/4, 1/4); II’s is (1/4, 3/4, 0).
 1 2
4 1
A row (column) may also be removed if it is dominated by a probability combination
of other rows (columns).
If for some 0 <p< 1, pai1j +(1−p)ai2j ≥ akj for all j, then the kth row is dominated
by the mixed strategy that chooses row i1 with probability p and row i2 with probability
1 − p. Player I can do at least as well using this mixed strategy instead of choosing row
k. (In addition, any mixed strategy choosing row k with probability pk may be replaced
by the one in which k’s probability is split between i1 and i2. That is, i1’s probability is
increased by ppk and i2’s probability is increased by (1 − p)pk.) A similar argument may
be used for columns.
Consider the matrix A =
⎛
⎝
046
574
963
⎞
⎠.
The middle column is dominated by the outside columns taken with probability 1/2
each. With the central column deleted, the middle row is dominated by the combination
of the top row with probability 1/3 and the bottom row with probability 2/3. The reduced
matrix,  0 6
9 3
, is easily solved. The value is V = 54/12 = 9/2.
Of course, mixtures of more than two rows (columns) may be used to dominate and
remove other rows (columns). For example, the mixture of columns one two and three
with probabilities 1/3 each in matrix B =
⎛
⎝
1353
4022
3735
⎞
⎠ dominates the last column,
II – 12
and so the last column may be removed.
Not all games may be reduced by dominance. In fact, even if the matrix has a saddle
point, there may not be any dominated rows or columns. The 3 × 3 game with a saddle
point found in Example 1 demonstrates this.
2.4 Solving 2 × n and m × 2 games. Games with matrices of size 2 × n or m × 2
may be solved with the aid of a graphical interpretation. Take the following example.
p
1 − p
 2315
4160
Suppose Player I chooses the first row with probability p and the second row with probability
1−p. If II chooses Column 1, I’s average payoff is 2p+4(1−p). Similarly, choices of
Columns 2, 3 and 4 result in average payoffs of 3p+(1−p), p+6(1−p), and 5p respectively.
We graph these four linear functions of p for 0 ≤ p ≤ 1. For a fixed value of p, Player I can
be sure that his average winnings is at least the minimum of these four functions evaluated
at p. This is known as the lower envelope of these functions. Since I wants to maximize
his guaranteed average winnings, he wants to find p that achieves the maximum of this
lower envelope. According to the drawing, this should occur at the intersection of the lines
for Columns 2 and 3. This essentially, involves solving the game in which II is restricted
to Columns 2 and 3. The value of the game  3 1
1 6
is v = 17/7 , I’s optimal strategy is
(5/7, 2/7), and II’s optimal strategy is (5/7, 2/7). Subject to the accuracy of the drawing,
we conclude therefore that in the original game I’s optimal strategy is (5/7, 2/7) , II’s is
(0, 5/7, 2/7, 0) and the value is 17/7.
Fig 2.1
0
1
2
3
4
5
6
col. 3
col. 1
col. 2
col. 4
0 1 5/7 p
The accuracy of the drawing may be checked: Given any guess at a solution to a
game, there is a sure-fire test to see if the guess is correct, as follows. If I uses the strategy
(5/7, 2/7), his average payoff if II uses Columns 1, 2, 3 and 4, is 18/7, 17/7, 17/7, and 25/7
II – 13
respectively. Thus his average payoff is at least 17/7 no matter what II does. Similarly,
if II uses (0, 5/7, 2/7, 0), her average loss is (at most) 17/7. Thus, 17/7 is the value, and
these strategies are optimal.
We note that the line for Column 1 plays no role in the lower envelope (that is, the
lower envelope would be unchanged if the line for Column 1 were removed from the graph).
This is a test for domination. Column 1 is, in fact, dominated by Columns 2 and 3 taken
with probability 1/2 each. The line for Column 4 does appear in the lower envelope, and
hence Column 4 cannot be dominated.
As an example of a m × 2 game, consider the matrix associated with Figure 2.2. If
q is the probability that II chooses Column 1, then II’s average loss for I’s three possible
choices of rows is given in the accompanying graph. Here, Player II looks at the largest
of her average losses for a given q. This is the upper envelope of the function. II wants
to find q that minimizes this upper envelope. From the graph, we see that any value of q
between 1/4 and 1/3 inclusive achieves this minimum. The value of the game is 4, and I
has an optimal pure strategy: row 2.
Fig 2.2
⎛
⎝
q 1 − q
1 5
4 4
6 2
⎞
⎠
0
1
2
3
4
5
6
row 1
row 2
row 3
0 1 1/4 1/2
q
These techniques work just as well for 2 × ∞ and ∞ × 2 games.
2.5 Latin Square Games. A Latin square is an n × n array of n different letters
such that each letter occurs once and only once in each row and each column. The 5 × 5
array at the right is an example. If in a Latin square each letter is assigned a numerical
value, the resulting matrix is the matrix of a Latin square game. Such games have simple
solutions. The value is the average of the numbers in a row, and the strategy that chooses
each pure strategy with equal probability 1/n is optimal for both players. The reason is
not very deep. The conditions for optimality are satisfied.
II – 14
⎛
⎜⎜⎜⎝
abcde
beacd
cadeb
dceba
edbac
⎞
⎟⎟⎟⎠
a = 1, b = 2, c = d = 3, e = 6
⎛
⎜⎜⎜⎝
12336
26133
31362
33621
63213
⎞
⎟⎟⎟⎠
In the example above, the value is V = (1+2+3+3+6)/5 = 3, and the mixed strategy
p = q = (1/5, 1/5, 1/5, 1/5, 1/5) is optimal for both players. The game of matching pennies
is a Latin square game. Its value is zero and (1/2, 1/2) is optimal for both players.
2.6 Exercises.
1. Solve the game with matrix  −1 −3
−2 2
, that is find the value and an optimal
(mixed) strategy for both players.
2. Solve the game with matrix  0 2
t 1

for an arbitrary real number t. (Don’t forget
to check for a saddle point!) Draw the graph of v(t), the value of the game, as a function
of t, for −∞ <t< ∞.
3. Show that if a game with m×n matrix has two saddle points, then they have equal
values.
4. Reduce by dominance to 2 × 2 games and solve.
(a)
⎛
⎜⎝
5410
432 −1
0 −143
1 −212
⎞
⎟⎠ (b)
⎛
⎝
10 0 7 1
2 647
6 335
⎞
⎠.
5. (a) Solve the game with matrix  3240
−2 1 −4 5
.
(b) Reduce by dominance to a 3 × 2 matrix game and solve:
⎛
⎝
085
846
12 −4 3
⎞
⎠.
6. Players I and II choose integers i and j respectively from the set {1, 2,...,n} for
some n ≥ 2. Player I wins 1 if |i − j| = 1. Otherwise there is no payoff. If n = 7, for
example, the game matrix is
⎛
⎜⎜⎜⎜⎜⎜⎜⎝
0100000
1010000
0101000
0010100
0001010
0000101
0000010
⎞
⎟⎟⎟⎟⎟⎟⎟⎠
II – 15
(a) Using dominance to reduce the size of the matrix, solve the game for n = 7 (i.e.
find the value and one optimal strategy for each player).
(b) See if you can solve the game for arbitrary n.
7. In general, the sure-fire test may be stated thus: For a given game, conjectured
optimal strategies (p1,...,pm) and (q1,...,qn) are indeed optimal if the minimum of I’s
average payoffs using (p1,...,pm) is equal to the maximum of II’s average payoffs using
(q1,...,qn). Show that for the game with the following matrix the mixed strategies
p = (6/37, 20/37, 0, 11/37) and q = (14/37, 4/37, 0, 19/37, 0) are optimal for I and II respectively.
What is the value?
⎛
⎜⎝
58316
42635
24641
13253
⎞
⎟⎠
8. Given that p = (52/143, 50/143, 41/143) is optimal for I in the game with the
following matrix, what is the value?
⎛
⎝
0 5 −2
−304
6 −4 0
⎞
⎠
9. Player I secretly chooses one of the numbers, 1, 2 and 3, and Player II tries to guess
which. If II guesses correctly, she loses nothing; otherwise, she loses the absolute value of
the difference of I’s choice and her guess. Set up the matrix and reduce it by dominance
to a 2 by 2 game and solve. Note that II has an optimal pure strategy that was eliminated
by dominance. Moreover, this strategy dominates the optimal mixed strategy in the 2 by
2 game.
10. Magic Square Games. A magic square is an n × n array of the first n integers
with the property that all row and column sums are equal. Show how to solve all games
with magic square game matrices. Solve the example,
⎛
⎜⎝
16 3 2 13
5 10 11 8
9 6 7 12
4 15 14 1
⎞
⎟⎠ .
(This is the magic square that appears in Albrecht D¨urer’s engraving, Melencolia. See
http://freemasonry.bcy.ca/art/melencolia.html)
11. In an article, “Normandy: Game and Reality” by W. Drakert in Moves, No. 6
(1972), an analysis is given of the invasion of Europe at Normandy in World War II. Six
possible attacking configurations (1 to 6) by the Allies and six possible defensive strategies
(A to F) by the Germans were simulated and evaluated, 36 simulations in all. The following
II – 16
table gives the estimated value to the Allies of each hypothetical battle in some numerical
units.
⎛
⎜⎜⎜⎜⎜⎜⎝
ABCDEF
1 13 29 8 12 16 23
2 18 22 21 22 29 31
3 18 22 31 31 27 37
4 11 22 12 21 21 26
5 18 16 19 14 19 28
6 23 22 19 23 30 34
⎞
⎟⎟⎟⎟⎟⎟⎠
(a) Assuming this is a matrix of a six by six game, reduce by dominance and solve.
(b) The historical defense by the Germans was B, and the historical attack by the Allies
was 1. Criticize these choices.
II – 17
3. The Principle of Indifference.
For a matrix game with m × n matrix A, if Player I uses the mixed strategy p =
(p1,...,pm)
T and Player II uses column j, Player I’s average payoff is m
i=1 piaij . If V is
the value of the game, an optimal strategy, p, for I is characterized by the property that
Player I’s average payoff is at least V no matter what column j Player II uses, i.e.

m
i=1
piaij ≥ V for all j = 1, . . . , n. (1)
Similarly, a strategy q = (q1,...,qn)
T is optimal for II if and only if

n
j=1
aij qj ≤ V for all i = 1, . . . , m. (2)
When both players use their optimal strategies the average payoff, 
i

j piaij qj , is exactly
V . This may be seen from the inequalities
V = 
n
j=1
V qj ≤ 
n
j=1
(

m
i=1
piaij )qj = 
m
i=1

n
j=1
piaij qj
= 
m
i=1
pi(

n
j=1
aij qj ) ≤ 
m
i=1
piV = V.
(3)
Since this begins and ends with V we must have equality throughout.
3.1 The Equilibrium Theorem. The following simple theorem – the Equilibrium
Theorem – gives conditions for equality to be achieved in (1) for certain values of j, and
in (2) for certain values of i.
Theorem 3.1. Consider a game with m×n matrix A and value V . Let p = (p1,...,pm)
T
be any optimal strategy for I and q = (q1,...,qn)
T be any optimal strategy for II. Then

n
j=1
aij qj = V for all i for which pi > 0 (4)
and

m
i=1
piaij = V for all j for which qj > 0. (5)
Proof. Suppose there is a k such that pk > 0 and n
j=1 akj qj = V . Then from (2),
n
j=1 akj qj < V . But then from (3) with equality throughout
V = 
m
i=1
pi(

n
j=1
aij qj ) < 
m
i=1
piV = V.
II – 18
The inequality is strict since it is strict for the kth term of the sum. This contradiction
proves the first conclusion. The second conclusion follows analogously.
Another way of stating the first conclusion of this theorem is: If there exists an optimal
strategy for I giving positive probability to row i, then every optimal strategy of II gives
I the value of the game if he uses row i.
This theorem is useful in certain classes of games for helping direct us toward the
solution. The procedure this theorem suggests for Player 1 is to try to find a solution to
the set of equations (5) formed by those j for which you think it likely that qj > 0. One
way of saying this is that Player 1 searches for a strategy that makes Player 2 indifferent
as to which of the (good) pure strategies to use. Similarly, Player 2 should play in such a
way to make Player 1 indifferent among his (good) strategies. This is called the Principle
of Indifference.
Example. As an example of this consider the game of Odd-or-Even in which both
players simultaneously call out one of the numbers zero, one, or two. The matrix is
Even
Odd
⎛
⎝
0 1 −2
1 −2 3
−2 3 −4
⎞
⎠
Again it is difficult to guess who has the advantage. If we play the game a few times we
might become convinced that Even’s optimal strategy gives positive weight (probability)
to each of the columns. If this assumption is true, Odd should play to make Player 2
indifferent; that is, Odd’s optimal strategy p must satisfy
p2 − 2p3 = V
p1 − 2p2 + 3p3 = V
−2p1 + 3p2 − 4p3 = V,
(6)
for some number, V — three equations in four unknowns. A fourth equation that must be
satisfied is
p1 + p2 + p3 = 1. (7)
This gives four equations in four unknowns. This system of equations is solved as follows.
First we work with (6); add the first equation to the second.
p1 − p2 + p3 = 2V (8)
Then add the second equation to the third.
−p1 + p2 − p3 = 2V (9)
Taken together (8) and (9) imply that V = 0. Adding (7) to (9), we find 2p2 = 1, so that
p2 = 1/2. The first equation of (6) implies p3 = 1/4 and (7) implies p1 = 1/4. Therefore
p = (1/4, 1/2, 1/4)T (10)
II – 19
is a strategy for I that keeps his average gain to zero no matter what II does. Hence the
value of the game is at least zero, and V = 0 if our assumption that II’s optimal strategy
gives positive weight to all columns is correct. To complete the solution, we note that if the
optimal p for I gives positive weight to all rows, then II’s optimal strategy q must satisfy
the same set of equations (6) and (7) with p replaced by q (because the game matrix here
is symmetric). Therefore,
q = (1/4, 1/2, 1/4)T (11)
is a strategy for II that keeps his average loss to zero no matter what I does. Thus the
value of the game is zero and (10) and (11) are optimal for I and II respectively. The game
is fair.
3.2 Nonsingular Game Matrices. Let us extend the method used to solve this
example to arbitrary nonsingular square matrices. Let the game matrix A be m × m,
and suppose that A is nonsingular. Assume that I has an optimal strategy giving positive
weight to each of the rows. (This is called the all-strategies-active case.) Then by the
principle of indifference, every optimal strategy q for II satisfies (4), or

m
j=1
aij qj = V for i = 1,...,m. (12)
This is a set of m equations in m unknowns, and since A is nonsingular, we may solve
for the qi. Let us write this set of equations in vector notation using q to represent the
column vector of II’s strategy, and 1 = (1, 1,..., 1)T to represent the column vector of all
1’s:
Aq = V 1 (13)
We note that V cannot be zero since (13) would imply that A was singular. Since A is
non-singular, A−1 exists. Multiplying both sides of (13) on the left by A−1 yields
q = V A−11. (14)
If the value of V were known, this would give the unique optimal strategy for II. To find
V , we may use the equation m
j=1 qj = 1, or in vector notation 1Tq = 1. Multiplying both
sides of (14) on the left by 1T yields 1 = 1Tq = V 1TA−11. This shows that 1TA−11 cannot
be zero so we can solve for V :
V = 1/1
TA−11. (15)
The unique optimal strategy for II is therefore
q = A−11/1
TA−11. (16)
However, if some component, qj , turns out to be negative, then our assumption that I has
an optimal strategy giving positive weight to each row is false.
However, if qj ≥ 0 for all j, we may seek an optimal strategy for I by the same method.
The result would be
p
T = 1
TA−1/1
T
A−11. (17)
II – 20
Now, if in addition pi ≥ 0 for all i, then both p and q are optimal since both guarantee an
average payoff of V no matter what the other player does. Note that we do not require the
pi to be strictly positive as was required by our original “all-strategies-active” assumption.
We summarize this discussion as a theorem.
Theorem 3.2. Assume the square matrix A is nonsingular and 1TA−11 = 0. Then the
game with matrix A has value V = 1/1TA−11 and optimal strategies pT = V 1TA−1 and
q = V A−11, provided both p ≥ 0 and q ≥ 0.
If the value of a game is zero, this method cannot work directly since (13) implies
that A is singular. However, the addition of a positive constant to all entries of the matrix
to make the value positive, may change the game matrix into being nonsingular. The
previous example of Odd-or-Even is a case in point. The matrix is singular so it would
seem that the above method would not work. Yet if 1, say, were added to each entry of
the matrix to obtain the matrix A below, then A is nonsingular and we may apply the
above method. Let us carry through the computations. By some method or another A−1
is obtained.
A =
⎛
⎝
1 2 −1
2 −1 4
−1 4 −3
⎞
⎠ A−1 = 1
16
⎛
⎝
13 −2 −7
−246
−765
⎞
⎠
Then 1TA−11 , the sum of the elements of A−1, is found to be 1, so from (15), V = 1.
Therefore, we compute pT = 1TA−1 = (1/4, 1/2, 1/4)T, and q = A−11 = (1/4, 1/2, 1/4)T.
Since both are nonnegative, both are optimal and 1 is the value of the game with matrix
A.
What do we do if either p or q has negative components? A complete answer to
questions of this sort is given in the comprehensive theorem of Shapley and Snow (1950).
This theorem shows that an arbitrary m × n matrix game whose value is not zero may be
solved by choosing some suitable square submatrix A, and applying the above methods
and checking that the resulting optimal strategies are optimal for the whole matrix, A.
Optimal strategies obtained in this way are called basic, and it is noted that every optimal
strategy is a probability mixture of basic optimal strategies. Such a submatrix, A, is called
an active submatrix of the game.See Karlin (1959, Vol. I, Section 2.4) for a discussion and
proof. The problem is to determine which square submatrix to use. The simplex method
of linear programming is simply an efficient method not only for solving equations of the
form (13), but also for finding which square submatrix to use. This is described in Section
4.4.
3.3 Diagonal Games. We apply these ideas to the class of diagonal games - games
whose game matrix A is square and diagonal,
A =
⎛
⎜⎜⎝
d1 0 ... 0
0 d2 ... 0 .
.
. .
.
. ... .
.
.
0 0 ... dm
⎞
⎟⎟⎠
(18)
II – 21
Suppose all diagonal terms are positive, di > 0 for all i. (The other cases are treated in
Exercise 2.) One may apply Theorem 3.2 to find the solution, but it is as easy to proceed
directly. The set of equations (12) becomes
pidi = V for i = 1,...,m (19)
whose solution is simply
pi = V /di for i = 1, . . . , m. (20)
To find V , we sum both sides over i to find
1 = V 
m
i=1
1/di or V = (
m
i=1
1/di)
−1. (21)
Similarly, the equations for Player II yield
qi = V /di for i = 1,...,m. (22)
Since V is positive from (21), we have pi > 0 and qi > 0 for all i, so that (20) and (22)
give optimal strategies for I and II respectively, and (21) gives the value of the game.
As an example, consider the game with matrix C.
C =
⎛
⎜⎝
1000
0200
0030
0004
⎞
⎟⎠
From (20) and (22) the optimal strategy is proportional to the reciprocals of the diagonal
elements. The sum of these reciprocals is 1 + 1/2+1/3+1/4 = 25/12. Therefore, the
value is V = 12/25, and the optimal strategies are p = q = (12/25, 6/25, 4/25, 3/25)T.
3.4 Triangular Games. Another class of games for which the equations (12) are
easy to solve are the games with triangular matrices - matrices with zeros above or below
the main diagonal. Unlike for diagonal games, the method does not always work to solve
triangular games because the resulting p or q may have negative components. Nevertheless,
it works often enough to merit special mention. Consider the game with triangular matrix
T .
T =
⎛
⎜⎝
1 −2 3 −4
0 1 −2 3
001 −2
0001
⎞
⎟⎠
The equations (12) become
p1 = V
−2p1 + p2 = V
3p1 − 2p2 + p3 = V
−4p1 + 3p2 − 2p3 + p4 = V .
II – 22
These equations may be solved one at a time from the top down to give
p1 = V p2 = 3V p3 = 4V p4 = 4V.
Since pi = 1, we find V = 1/12 and p = (1/12, 1/4, 1/3, 1/3). The equations for the q’s
are
q1 − 2q2 + 3q3 − 4q4 = V
q2 − 2q3 + 3q4 = V
q3 − 2q4 = V
q4 = V .
The solution is
q1 = 4V q2 = 4V q3 = 3V q4 = V.
Since the p’s and q’s are non-negative, V = 1/12 is the value, p = (1/12, 1/4, 1/3, 1/3) is
optimal for I, and q = (1/3, 1/3, 1/4, 1/12) is optimal for II.
3.5 Symmetric Games. A game is symmetric if the rules do not distinguish between
the players. For symmetric games, both players have the same options (the game matrix
is square), and the payoff if I uses i and II uses j is the negative of the payoff if I uses j
and II uses i. This means that the game matrix should be skew-symmetric: A = −AT,
or aij = −aji for all i and j.
Definition 3.1. A finite game is said to be symmetric if its game matrix is square and
skew-symmetric.
Speaking more generally, we may say that a game is symmetric if after some rearrangement
of the rows or columns the game matrix is skew-symmetric.
The game of paper-scissors-rock is an example. In this game, Players I and II simultaneously
display one of the three objects: paper, scissors, or rock. If they both choose the
same object to display, there is no payoff. If they choose different objects, then scissors win
over paper (scissors cut paper), rock wins over scissors (rock breaks scissors), and paper
wins over rock (paper covers rock). If the payoff upon winning or losing is one unit, then
the matrix of the game is as follows.
II
I
⎛
⎝
paper scissors rock
paper 0 −1 1
scissors 1 0 −1
rock −1 10
⎞
⎠
This matrix is skew-symmetric so the game is symmetric. The diagonal elements of
the matrix are zero. This is true of any skew-symmetric matrix, since aii = −aii implies
aii = 0 for all i.
A contrasting example is the game of matching pennies. The two players simultaneously
choose to show a penny with either the heads or the tails side facing up. One of the
II – 23
players, say Player I, wins if the choices match. The other player, Player II, wins if the
choices differ. Although there is a great deal of symmetry in this game, we do not call it
a symmetric game. Its matrix is
II
I

heads tails
heads 1 −1
tails −1 1
This matrix is not skew-symmetric.
We expect a symmetric game to be fair, that is to have value zero, V = 0. This is
indeed the case.
Theorem 3.3. A finite symmetric game has value zero. Any strategy optimal for one
player is also optimal for the other.
Proof. Let p be an optimal strategy for I. If II uses the same strategy the average payoff
is zero, because
p
TAp = 

piaijpj = 

pi(−aji)pj = −


pjajipi = −p
T
Ap (23)
implies that pTAp = 0. This shows that the value V ≤ 0. A symmetric argument shows
that V ≥ 0. Hence V = 0. Now suppose p is optimal for I. Then m
i=1 piaij ≥ 0 for all
j. Hence m
j=1 aijpj = −m
j=1 pjaji ≤ 0 for all i, so that p is also optimal for II. By
symmetry, if q is optimal for II, it is optimal for I also.
Mendelsohn Games. (N. S. Mendelsohn (1946)) In Mendelsohn games, two players
simultaneously choose a positive integer. Both players want to choose an integer larger
but not too much larger than the opponent. Here is a simple example. The players choose
an integer between 1 and 100. If the numbers are equal there is no payoff. The player that
chooses a number one larger than that chosen by his opponent wins 1. The player that
chooses a number two or more larger than his opponent loses 2. Find the game matrix
and solve the game.
Solution. The payoff matrix is
⎛
⎜⎜⎜⎜⎜⎜⎝
12345 ···
1 0 −1222 ···
2 10 −122 ···
3 −210 −1 2 ···
4 −2 −210 −1 ···
5 −2 −2 −210 ··· .
.
. .
.
. ...
⎞
⎟⎟⎟⎟⎟⎟⎠
(24)
The game is symmetric so the value is zero and the players have identical optimal strategies.
We see that row 1 dominates rows 4, 5, 6,... so we may restrict attention to the upper left
II – 24
3 × 3 submatrix. We suspect that there is an optimal strategy for I with p1 > 0, p2 > 0
and p3 > 0. If so, it would follow from the principle of indifference (since q1 = p1 > 0,
q2 = p2 > 0 q3 = p3 > 0 is optimal for II) that
p2 − 2p3 = 0
−p1 + p3 = 0
2p1 − p2 = 0. (25)
We find p2 = 2p3 and p1 = p3 from the first two equations, and the third equation is
redundant. Since p1 + p2 + p3 = 1, we have 4p3 = 1; so p1 = 1/4, p2 = 1/2, and p3 = 1/4.
Since p1, p2 and p3 are positive, this gives the solution: p = q = (1/4, 1/2, 1/4, 0, 0,...)
T is
optimal for both players.
3.6 Invariance. Consider the game of matching pennies: Two players simultaneously
choose heads or tails. Player I wins if the choices match and Player II wins otherwise.
There doesn’t seem to be much of a reason for either player to choose heads instead of
tails. In fact, the problem is the same if the names of heads and tails are interchanged. In
other words, the problem is invariant under interchanging the names of the pure strategies.
In this section, we make the notion of invariance precise. We then define the notion of
an invariant strategy and show that in the search for a minimax strategy, a player may
restrict attention to invariant strategies. Use of this result greatly simplifies the search for
minimax strategies in many games. In the game of matching pennies for example, there is
only one invariant strategy for either player, namely, choose heads or tails with probability
1/2 each. Therefore this strategy is minimax without any further computation.
We look at the problem from Player II’s viewpoint. Let Y denote the pure strategy
space of Player II, assumed finite. A transformation, g of Y into Y is said to be onto Y
if the range of g is the whole of Y , that is, if for every y1 ∈ Y there is y2 ∈ Y such that
g(y2) = y1. A transformation, g, of Y into itself is said to be one-to-one if g(y1) = g(y2)
implies y1 = y2.
Definition 3.2. Let G = (X, Y, A) be a finite game, and let g be a one-to-one transformation
of Y onto itself. The game G is said to be invariant under g if for every x ∈ X
there is a unique x ∈ X such that
A(x, y) = A(x
, g(y)) for all y ∈ Y . (26)
The requirement that x be unique is not restrictive, for if there were another point
x ∈ X such that
A(x, y) = A(x, g(y)) for all y ∈ Y , (27)
then, we would have A(x
, g(y)) = A(x, g(y)) for all y ∈ Y , and since g is onto,
A(x
, y) = A(x, y) for all y ∈ Y . (28)
Thus the strategies x and x have identical payoffs and we could remove one of them from
X without changing the problem at all.
II – 25
To keep things simple, we assume without loss of generality that all duplicate pure
strategies have been eliminated. That is, we assume
A(x
, y) = A(x, y) for all y ∈ Y implies that x = x, and
A(x, y
) = A(x, y) for all x ∈ X implies that y = y. (29)
Unicity of x in Definition 3.2 follows from this assumption.
The given x in Definition 3.2 depends on g and x only. We denote it by x = g(x).
We may write equation (26) defining invariance as
A(x, y) = A(g(x), g(y)) for all x ∈ X and y ∈ Y . (26
)
The mapping g is a one-to-one transformation of X since if g(x1) = g(x2), then
A(x1, y) = A(g(x1), g(y)) = A(g(x2), g(y)) = A(x2, y) (30)
for all y ∈ Y , which implies x1 = x2 from assumption (29). Therefore the inverse, g−1, of
g, defined by g−1(g(x)) = g(g−1(x)) = x, exists. Moreover, any one-to-one transformation
of a finite set is automatically onto, so g is a one-to-one transformation of X onto itself.
Lemma 1. If a finite game, G = (X, Y, A), is invariant under a one-to-one transformation,
g, then G is also invariant under g−1.
Proof. We are given A(x, y) = A(g(x), g(y)) for all x ∈ X and all y ∈ Y . Since true for
all x and y, it is true if y is replaced by g−1(y) and x is replaced by g−1(x). This gives
A(g−1(x), g−1(y)) = A(x, y) for all x ∈ X and all y ∈ Y . This shows that G is invariant
under g−1.
Lemma 2. If a finite game, G = (X, Y, A), is invariant under two one-to-one transformations,
g1 and g2, then G is also invariant under under the composition transformation,
g2g1, defined by g2g1(y) = g2(g1(y)).
Proof. We are given A(x, y) = A(g1(x), g1(y)) for all x ∈ X and all y ∈ Y , and A(x, y) =
A(g2(x), g2(y)) for all x ∈ X and all y ∈ Y . Therefore,
A(x, y) = A(g2(g1(x)), g2(g1(y))) = A(g2(g1(x)), g2g1(y)) for all y ∈ Y and x ∈ X.
(31)
which shows that G is invarant under g2g1.
Furthermore, these proofs show that
g2g1 = g2 g1, and g−1 = g−1. (32)
Thus the class of transformations, g on Y , under which the problem is invariant forms a
group, G, with composition as the multiplication operator. The identity element, e of the
group is the identity transformation, e(y) = y for all y ∈ Y . The set, G of corresponding
transformations g on X is also a group, with identity e(x) = x for all x ∈ X. Equations
(32) say that G is isomorphic to G; as groups, they are indistinguishable.
This shows that we could have analyzed the problem from Player I’s viewpoint and
arrived at the same groups G and G.
II – 26
Definition 3.3. A finite game G = (X, Y, A) is said to be invariant under a group, G of
transformations, if (26
) holds for all g ∈ G.
We now define what it means for a mixed strategy, q, for Player II to be invariant
under a group G. Let m denote the number of elements in X and n denote the number of
elements in Y .
Definition 3.4. Given that a finite game G = (X, Y, A) is invariant under a group, G,
of one-to-one transformations of Y , a mixed strategy, q = (q(1),...,q(n)), for Player II is
said to be invariant under G if
q(g(y)) = q(y) for all y ∈ Y and all g ∈ G. (33)
Similarly a mixed strategy p = (p(1),...,p(m)), for Player I is said to be invariant under
G (or G) if
p(g(x)) = p(x) for all x ∈ X and all g ∈ G. (34)
Two points y1 and y2 in Y are said to be equivalent if there exists a g in G such that
g(y2) = y1. It is an easy exercise to show that this is an equivalence relation. The set of
points, Ey = {y : g(y
) = y for some g ∈ G}, is called an equivalence class, or an orbit.
Thus, y1 and y2 are equivalent if they lie in the same orbit. Definition 3.4 says that a mixed
strategy q for Player II is invariant if it is constant on orbits, that is, if it assigns the same
probability to all pure strategies in the orbit. The power of this notion is contained in the
following theorem.
Theorem 3.4. If a finite game G = (X, Y, A) is invariant under a group G, then there
exist invariant optimal strategies for the players.
Proof. It is sufficient to show that Player II has an invariant optimal strategy. Since the
game is finite, there exists a value, V , and an optimal mixed strategy for player II, q∗.
This is to say that


y∈Y
A(x, y)q∗(y) ≤ V for all x ∈ X. (35)
We must show that there is an invariant strategy q˜ that satisfies this same condition. Let
N = |G| be the number of elements in the group G. Define
q˜(y) = 1
N


g∈G
q∗(g(y)) (36)
(This takes each orbit and replaces each probability by the average of the probabilities in
the orbit.) Then q˜ is invariant since for any g ∈ G,
q˜(g
(y)) = 1
N


g∈G
q∗(g(g
(y)))
= 1
N


g∈G
q∗(g(y)) = ˜q(y)
(37)
II – 27
since applying g to Y = {1, 2,...,n} is just a reordering of the points of Y . Moreover, q˜
satisfies (35) since


y∈Y
A(x, y)˜q(y) = 

y∈Y
A(x, y) 1
N


g∈G
q∗(g(y))
= 1
N


g∈G


y∈Y
A(x, y)q∗(g(y))
= 1
N


g∈G


y∈Y
A(g(x), g(y))q∗(g(y))
= 1
N


g∈G


y∈Y
A(g(x), y)q∗(y)
≤
1
N


g∈G
V = V.
(38)
In matching pennies, X = Y = {1, 2}, and A(1, 1) = A(2, 2) = 1 and A(1, 2) =
A(2, 1) = −1. The Game G = (X, Y, A) is invariant under the group G = {e, g}, where e is
the identity transformation, and g is the transformation, g(1) = 2, g(2) = 1. The (mixed)
strategy (q(1), q(2)) is invariant under G if q(1) = q(2). Since q(1) + q(2) = 1, this implies
that q(1) = q(2) = 1/2 is the only invariant strategy for Player II. It is therefore minimax.
Similarly, p(1) = p(2) = 1/2 is the only invariant, and hence minimax, strategy for Player
I.
Similarly, the game of paper-scissors-rock is invariant under the group G = {e, g, g2},
where g(paper)=scissors, g(scissors)=rock and g(rock)=paper. The unique invariant, and
hence minimax, strategy gives probability 1/3 to each of paper, scissors and rock.
Colonel Blotto Games. For more interesting games reduced by invariance, we
consider a class of tactical military games called Blotto Games, introduced by Tukey
(1949). There are many variations of these games; just google “Colonel Blotto Games”
to get a sampling. Here, we describe the discrete version treated in Williams (1954),
Karlin(1959) and Dresher (1961).
Colonel Blotto has 4 regiments with which to occupy two posts. The famous Lieutenant
Kije has 3 regiments with which to occupy the same posts. The payoff is defined as
follows. The army sending the most units to either post captures it and all the regiments
sent by the other side, scoring one point for the captured post and one for each captured
regiment. If the players send the same number of regiments to a post, both forces withdraw
and there is no payoff.
Colonel Blotto must decide how to split his forces between the two posts. There are
5 pure strategies he may employ, namely, X = {(4, 0),(3, 1),(2, 2),(1, 3),(0, 4)}, where
(n1, n2) represents the strategy of sending n1 units to post number 1, and n2 units to post
II – 28
number two. Lieutenant Kije has 4 pure strategies, Y = {(3, 0),(2, 1),(1, 2),(0, 3)}. The
payoff matrix is
⎛
⎜⎜⎜⎜⎝
(3, 0) (2, 1) (1, 2) (0, 3)
(4, 0) 4 2 1 0
(3, 1) 1 3 0 −1
(2, 2) −22 2 −2
(1, 3) −10 3 1
(0, 4) 0 1 2 4
⎞
⎟⎟⎟⎟⎠
(39)
Unfortunately, the 5 by 4 matrix game cannot be reduced by removing dominated
strategies. So it seems that to solve it, we must use the simplex method. However, there is
an invariance in this problem that simplifies it considerably. This involves the symmetry
between the posts. This leads to the group, G = {e, g}, where
g((3, 0)) = (0, 3) g((0, 3)) = (3, 0) g((2, 1)) = (1, 2) g((1, 2)) = (2, 1)
and the corresponding group, G = {e, g}, where
g((4, 0)) = (0, 4) g((0, 4)) = (4, 0) g((3, 1)) = (1, 3) g((1, 3)) = (3, 1)
and g((2, 2)) = (2, 2)
The orbits for Kije are {(3, 0),(0, 3)} and {(2, 1),(1, 2)}. Therefore a strategy, q, is
invariant if q((3, 0)) = q((0, 3)) and q((2, 1)) = q((1, 2)). Similarly, the orbits for Blotto
are {(4, 0),(0, 4)}, {(3, 1),(1, 3)} and {(2, 2)}. So a strategy, p, for Blotto is invariant if
p((4, 0)) = p((0, 4)) and p((3, 1)) = p((1, 3)).
We may reduce Kije’s strategy space to two elements, defined as follows:
(3, 0)∗: use (3, 0) and (0, 3) with probability 1/2 each.
(2, 1)∗: use (2, 1) and (1, 2) with probability 1/2 each.
Similarly, Blotto’s strategy space reduces to three elements:
(4, 0)∗: use (4, 0) and (0, 4) with probability 1/2 each.
(3, 1)∗: use (3, 1) and (1, 3) with probability 1/2 each.
(2, 2): use (2, 2).
With these strategy spaces, the payoff matrix becomes
⎛
⎝
(3, 0)∗ (2, 1)∗
(4, 0)∗ 2 1.5
(3, 1)∗ 0 1.5
(2, 2) −2 2
⎞
⎠ (40)
As an example of the computations used to arrive at these payoffs, consider the upper
left entry. If Blotto uses (4,0) and (0,4) with probability 1/2 each, and if Kije uses (3,0)
II – 29
and (0,3) with probability 1/2 each, then the four corners of the matrix (39) occur with
probability 1/4 each, so the expected payoff is the average of the four numbers, 4, 0, 0, 4,
namely 2.
To complete the analysis, we solve the game with matrix (40). We first note that
the middle row is dominated by the top row (even though there was no domination in the
original matrix). Removal of the middle row reduces the game to a 2 by 2 matrix game
whose solution is easily found. The mixed strategy (8/9,0,1/9) is optimal for Blotto, the
mixed strategy (1/9,8/9) is optimal for Kije, and the value is V = 14/9.
Returning now to the original matrix (39), we find that (4/9,0,1/9,0,4/9) is optimal
for Blotto, (1/18,4/9,4/9,1/18) is optimal for Kije, and V = 14/9 is the value.
3.7 Exercises.
1. Consider the game with matrix
⎛
⎝
−2 2 −1
111
301
⎞
⎠.
(a) Note that this game has a saddle point.
(b) Show that the inverse of the matrix exists.
(c) Show that II has an optimal strategy giving positive weight to each of his columns.
(d) Why then, don’t equations (16) give an optimal strategy for II?
2. Consider the diagonal matrix game with matrix (18).
(a) Suppose one of the diagonal terms is zero. What is the value of the game?
(b) Suppose one of the diagonal terms is positive and another is negative. What is
the value of the game?
(c) Suppose all diagonal terms are negative. What is the value of the game?
3. Player II chooses a number j ∈ {1, 2, 3, 4}, and Player I tries to guess what number
II has chosen. If he guesses correctly and the number was j, he wins 2j dollars from II.
Otherwise there is no payoff. Set up the matrix of this game and solve.
4. Player II chooses a number j ∈ {1, 2, 3, 4} and I tries to guess what it is. If
he guesses correctly, he wins 1 from II. If he overestimates he wins 1/2 from II. If he
underestimates, there is no payoff. Set up the matrix of this game and solve.
5. Player II chooses a number j ∈ {1, 2,...,n} and I tries to guess what it is. If he
guesses correctly, he wins 1. If he guesses too high, he loses 1. If he guesses too low, there
is no payoff. Set up the matrix and solve.
6. Player II chooses a number j ∈ {1, 2,...,n}, n ≥ 2, and Player I tries to guess
what it is by guessing some i ∈ {1, 2,...,n}. If he guesses correctly, i.e. i = j, he wins 1.
If i>j, he wins bi−j for some number b < 1. Otherwise, if i<j, he wins nothing. Set up
II – 30
the matrix and solve. Hint: If An = (aij ) denotes the game matrix, then show the inverse
matrix is A−1
n = (aij ), where aij =
 1 if i = j
−b if i = j + 1
0 otherwise
, and use Theorem 3.2.
7. The Pascal Matrix Game. The Pascal matrix of order n is the n × n matrix
Bn of elements bij , where
bij =
i − 1
j − 1

if i ≥ j, and bij = 0 if i<j.
The ith row of Bn consists of the binomial coefficients in the expansion of (x + y)i
. Call
and Velleman (1993) show that the inverse of Bn is the the matrix An with entries aij ,
where aij = (−1)i+j bij . Using this, find the value and optimal strategies for the matrix
game with matrix An.
8. Solve the games with the following matrices.
(a)
⎛
⎝
1 −1 −1
021
003
⎞
⎠ (b)
⎛
⎜⎝
21 1 1
1 3/21 1
114/3 1
11 15/4
⎞
⎟⎠
(c)
⎛
⎜⎝
2002
0300
0043
1101
⎞
⎟⎠
9. Another Mendelsohn game. Two players simultaneously choose an integer
between 1 and n inclusive, where n ≥ 5. If the numbers are equal there is no payoff. The
player that chooses a number one larger than that chosen by his opponent wins 2. The
player that chooses a number two or more larger than that chosen by his opponent loses
1.
(a) Set up the game matrix.
(b) It turns out that the optimal strategy satisfies pi > 0 for i = 1,..., 5, and pi = 0 for all
other i. Solve for the optimal p. (It is not too difficult since you can argue that p1 = p5
and p2 = p4 by symmetry of the equations.) Check that in fact the strategy you find is
optimal.
10. Silverman Games. (See R. T. Evans (1979) and Heuer and Leopold-Wildburger
(1991).) Two players simultaneously choose positive integers. As in Mendelsohn games, a
player wants to choose an integer larger but not too much larger than the opponent, but in
Silverman games “too much larger” is determined multiplicatively rather than additively.
Solve the following example: The player whose number is larger but less than three times
as large as the opponent’s wins 1. But the player whose number is three times as large or
larger loses 2. If the numbers are the same, there is no payoff.
(a) Note this is a symmetric game, and show that dominance reduces the game to a 3 by
II – 31
3 matrix.
(b) Solve.
11. Solve the following games.
(a)
⎛
⎝
0 1 −2
−103
2 −3 0
⎞
⎠ (b)
⎛
⎝
0 1 −2
−201
1 −2 0
⎞
⎠
(c)
⎛
⎜⎜⎜⎝
1 4 −1 5
4 −151
−1514
514 −1
2222
⎞
⎟⎟⎟⎠
12. Run the original Blotto matrix (39) through the Matrix Game Solver, on the
web at: http://www.math.ucla.edu/˜tom/gamesolve.html, and note that it gives different
optimal strategies than those found in the text. What does this mean? Show that (3, 1)∗ is
strictly dominated in (40). This means that no optimal strategy can give weight to (3, 1)∗.
Is this true for the solution found?
13. (a) Suppose Blotto has 2 units and Kije just 1 unit, with 2 posts to capture.
Solve.
(b) Suppose Blotto has 3 units and Kije 2 units, with 2 posts to capture. Solve.
14. (a) Suppose there are 3 posts to capture. Blotto has 4 units and Kije has 3. Solve.
(Reduction by invariance leads to a 4 by 3 matrix, reducible further by domination to 2
by 2.)
(b) Suppose there are 4 posts to capture. Blotto has 4 units and Kije has 3. Solve.
(A 5 by 3 reduced matrix, reducible by domination to 4 by 3. But you may as well use
the Matrix Game Solver to solve it.)
15. Battleship. The game of Battleship, sometimes called Salvo, is played on two
square boards, usually 10 by 10. Each player hides a fleet of ships on his own board and
tries to sink the opponent’s ships before the opponent sinks his. (For one set of rules, see
http://www.kielack.de/games/destroya.htm, and while you are there, have a game.)
For simplicity, consider a 3 by 3 board and suppose that Player I hides a destroyer
(length 2 squares) horizontally or vertically on this board. Then Player II shoots by calling
out squares of the board, one at a time. After each shot, Player I says whether the shot
was a hit or a miss. Player II continues until both squares of the destroyer have been hit.
The payoff to Player I is the number of shots that Player II has made. Let us label the
squares from 1 to 9 as follows.
1 2 3
4 5 6
7 8 9
II – 32
The problem is invariant under rotations and reflections of the board. In fact, of
the 12 possible positions for the destroyer, there are only two distinct invariant choices
available to Player I: the strategy, [1, 2]∗, that chooses one of [1,2], [2,3], [3,6], [6,9], [8,9],
[7,8], [4,7], and [1,4], at random with probability 1/8 each, and the strategy, [2, 5]∗, that
chooses one of [2,5], [5,6], [5,8], and [4,5], at random with probability 1/4 each. This means
that invariance reduces the game to a 2 by n game where n is the number of invariant
strategies of Player II. Domination may reduce it somewhat further. Solve the game.
16. Dresher’s Guessing Game. Player I secretly writes down one of the numbers
1, 2,...,n. Player II must repeatedly guess what I’s number is until she guesses correctly,
losing 1 for each guess. After each guess, Player I must say whether the guess is correct,
too high, or too low. Solve this game for n = 3. (This game was investigated by Dresher
(1961) and solved for n ≤ 11 by Johnson (1964). A related problem is treated in Gal
(1974).)
17. Thievery. Player I wants to steal one of m ≥ 2 items from Player II. Player II
can only guard one item at a time. Item i is worth ui > 0, for i = 1,...,m. This leads to
a matrix game with m × m matrix,
A =
⎛
⎜⎜⎜⎜⎝
0 u1 u1 ... u1
u2 0 u2 ... u2
u3 u3 0 ... u3
.
.
. .
.
. .
.
. ... .
.
.
um um um ... 0
⎞
⎟⎟⎟⎟⎠
Solve!
Hint: It might be expected that for some k ≤ m Player I will give all his probability
to stealing one of the k most expensive items. Order the items from most expensive to
least expensive, u1 ≥ u2 ≥ ... ≥ um > 0, and use the principle of indifference on the upper
left k × k submatrix of A for some k.
18. Player II chooses a number j ∈ {1, 2,...,n}, n ≥ 2, and Player I tries to guess
what it is by guessing some i ∈ {1, 2,...,n}. If he guesses correctly, i.e. i = j, he wins 2.
If he misses by exactly 1, i.e. |i − j| = 1, then he loses 1. Otherwise there is no payoff.
Solve. Hint: Let An denote the n by n payoff matrix, and show that A−1 n = Bn = (bij ),
where bij = i(n + 1 − j)/(n + 1) for i ≤ j, and bij = bji for i>j.
19. The Number Hides Game. The Number Hides Game, introduced by Ruckle
(1983) and solved by Baston, Bostock and Ferguson (1989), may be described as follows.
From the set S = {1, 2,...,k}, Player I chooses an interval of m1 consecutive integers
and Player II chooses an interval of m2 consecutive integers. The payoff to Player I is
the number of integers in the intersection of the two intervals. When k = n + 1 and
m1 = m2 = 2, this game is equivalent to the game with n × n matrix An = (aij ), where
aij =
 2 if i = j
1 if |i − j| = 1
0 otherwise.
II – 33
[In this form, the game is also a special case of the Helicopter versus Submarine
Game, solved in the book of Garnaev (2000), in which the payoff for |i − j| = 1 is allowed
to be an arbitrary number a, 0 ≤ a ≤ 1.] Since A−1 n is just Bn of the previous exercise
with bij replaced by (−1)i+j bij , the solution can be derived as in that exercise. Instead,
just show the following.
(a) For n odd, the value is Vn = 4/(n + 1). There is an optimal equalizing strategy
(the same for both players) that is proportional to (1, 0, 1, 0,..., 0, 1).
(b) For n even, the value is 4(n+1)/(n(n+2)). There is an optimal equalizing strategy
(the same for both players) that is proportional to (k, 1, k − 1, 2, k − 2, 3,..., 2, k − 1, 1, k),
where k = n/2.
II – 34
4. Solving Finite Games.
Consider an arbitrary finite two-person zero-sum game, (X, Y, A), with m × n matrix,
A. Let us take the strategy space X to be the first m integers, X = {1, 2,...,m}, and
similarly, Y = {1, 2,...,n}. A mixed strategy for Player I may be represented by a column
vector, (p1, p2,...,pm)
T of probabilities that add to 1. Similarly, a mixed strategy for Player
II is an n-tuple q = (q1, q2,...,qn)
T. The sets of mixed strategies of players I and II will
be denoted respectively by X∗ and Y ∗,
X∗ = {p = (p1,...,pm)
T : pi ≥ 0, for i = 1,...,m and m
1 pi = 1}
Y ∗ = {q = (q1,...,qn)
T : qj ≥ 0, for j = 1,...,n and n
1 qj = 1}
The m-dimensional unit vector ek ∈ X∗ with a one for the kth component and zeros
elsewhere may be identified with the pure strategy of choosing row k. Thus, we may
consider the set of Player I’s pure strategies, X, to be a subset of X∗. Similarly, Y may
be considered to be a subset of Y ∗. We could if we like consider the game (X, Y, A) in
which the players are allowed to use mixed strategies as a new game (X∗, Y ∗, A), where
A(p, q) = pTAq, though we would no longer call this game a finite game.
In this section, we give an algorithm for solving finite games; that is, we show how to
find the value and at least one optimal strategy for each player. Occasionally, we shall be
interested in finding all optimal strategies for a player.
4.1 Best Responses. Suppose that Player II chooses a column at random using
q ∈ Y ∗. If Player I chooses row i, the average payoff to I is

n
j=1
aij qj = (Aq)i, (1)
the ith component of the vector Aq. Similarly, if Player I uses p ∈ X∗ and Player II
chooses column j, Then I’s average payoff is

n
i=1
piaij = (pTA)j , (2)
the jth component of the vector pTA. More generally, if Player I uses p ∈ X∗ and Player
II uses q ∈ Y ∗, the average payoff to I becomes

m
i=1
⎛
⎝
n
j=1
aij qj
⎞
⎠ pi = 
m
i=1

n
j=1
piaij qj = p
TAq. (3)
Suppose it is known that Player II is going to use a particular strategy q ∈ Y ∗. Then
Player I would choose that row i that maximizes (1); or, equivalently, he would choose
that p ∈ X∗ that maximizes (3). His average payoff would be
max
1≤i≤m

n
j=1
aij qj = max
p∈X∗ pTAq (4)
II – 35
To see that these quantities are equal, note that the left side is the maximum of pTAq over
p ∈ X∗, and so, since X ⊂ X∗ , must be less than or equal to the right side. The reverse
inequality follows since (3) is an average of the quantities in (1) and so must be less than
or equal to the largest of the values in (1).
Any p ∈ X∗ that achieves the maximum of (3) is called a best response or a Bayes
strategy against q. In particular, any row i that achieves the maximum of (1) is a (pure)
Bayes strategy against q. There always exist pure Bayes strategies against q for every
q ∈ Y ∗ in finite games.
Similarly, if it is known that Player I is going to use a particular strategy p ∈ X∗, then
Player II would choose that column j that minimizes (2), or, equivalently, that q ∈ Y ∗
that minimizes (3). Her average payoff would be
min
1≤j≤n

m
i=1
piaij = min
q∈Y ∗ p
TAq. (5)
Any q ∈ Y ∗ that achieves the minimum in (5) is called a best response or a Bayes strategy
for Player II against p.
The notion of a best response presents a practical way of playing a game: Make a
guess at the probabilities that you think your opponent will play his/her various pure
strategies, and choose a best response against this. This method is available in quite
complex situations. In addition, it allows a player to take advantage of an opponent’s
perceived weaknesses. Of course this may be a dangerous procedure. Your opponent may
be better at this type of guessing than you. (See Exercise 1.)
4.2 Upper and Lower Values of a Game. Suppose now that II is required to
announce her choice of a mixed strategy q ∈ Y ∗ before I makes his choice. This changes
the game to make it apparently more favorable to I. If II announces q, then certainly I
would use a Bayes strategy against q and II would lose the quantity (4) on the average.
Therefore, II would choose to announce that q that minimizes (4). The minimum of (4)
over all q ∈ Y ∗ is denoted by V and called the upper value of the game (X, Y, A).
V = min
q∈Y ∗ max
1≤i≤m

n
j=1
aij qj = min
q∈Y ∗ max
p∈X∗ p
TAq. (6)
Any q ∈ Y ∗ that achieves the minimum in (6) is called a minimax strategy for II. It
minimizes her maximum loss. There always exists a minimax strategy in finite games: the
quantity (4), being the maximum of m linear functions of q, is a continuous function of q
and since Y ∗ is a closed bounded set, this function assumes its minimum over Y ∗ at some
point of Y ∗.
In words, V as the smallest average loss that Player II can assure for herself no matter
what I does.
A similar analysis may be carried out assuming that I must announce his choice of a
mixed strategy p ∈ X∗ before II makes her choice. If I announces p, then II would choose
II – 36
that column with the smallest average payoff, or equivalently that q ∈ Y ∗ that minimizes
the average payoff (5). Given that (5) is the average payoff to I if he announces p, he
would therefore choose p to maximize (5) and obtain on the average
V = max
p∈X∗ min
1≤j≤n

m
i=1
piaij = max
p∈X∗ min
q∈Y ∗ p
TAq. (7)
The quantity V is called the lower value of the game. It is the maximum amount that I can
guarantee himself no matter what II does. Any p ∈ X∗ that achieves the maximum in (7)
is called a minimax strategy for I. Perhaps maximin strategy would be more appropriate
terminology in view of (7), but from symmetry (either player may consider himself Player
II for purposes of analysis) the same word to describe the same idea may be preferable
and it is certainly the customary terminology. As in the analysis for Player II, we see that
Player I always has a minimax strategy. The existence of minimax strategies in matrix
games is worth stating as a lemma.
Lemma 1. In a finite game, both players have minimax strategies.
It is easy to argue that the lower value is less than or equal to the upper value. For if
V < V and if I can assure himself of winning at least V , Player II cannot assure herself of
not losing more than V , an obvious contradiction. It is worth stating this fact as a lemma
too.
Lemma 2. The lower value is less than or equal to the upper value,
V ≤ V.
This lemma also follows from the general mathematical principle that for any realvalued
function, f(x, y), and any sets, X∗ and Y ∗,
max
x∈X∗ min
y∈Y ∗ f(x, y) ≤ min
y∈Y ∗ max
x∈X∗ f(x, y).
To see this general principle, note that miny f(x, y
) ≤ f(x, y) ≤ maxx f(x
, y) for every
fixed x and y. Then, taking maxx on the left does not change the inequality, nor does
taking miny on the right, which gives the result.
If V < V , the average payoff should fall between V and V . Player II can keep it from
getting larger than V and Player I can keep it from getting smaller than V . When V = V ,
a very nice stable situation exists.
Definition. If V = V , we say the value of the game exists and is equal to the common
value of V and V , denoted simply by V . If the value of the game exists, we refer to
minimax strategies as optimal strategies.
The Minimax Theorem, stated in Chapter 1, may be expressed simply by saying that
for finite games, V = V .
II – 37
The Minimax Theorem. Every finite game has a value, and both players have minimax
strategies.
We note one remarkable corollary of this theorem. If the rules of the game are changed
so that Player II is required to announce her choice of a mixed strategy before Player I
makes his choice, then the apparent advantage given to Player I by this is illusory. Player
II can simply announce her minimax strategy.
4.3 Invariance under Change of Location and Scale. Another simple observation
is useful in this regard. This concerns the invariance of the minimax strategies under
the operations of adding a constant to each entry of the game matrix, and of multiplying
each entry of the game matrix by a positive constant. The game having matrix A = (aij )
and the game having matrix A = (a
ij ) with a
ij = aij + b, where b is an arbitrary real
number, are very closely related. In fact, the game with matrix A is equivalent to the
game in which II pays I the amount b, and then I and II play the game with matrix A.
Clearly any strategies used in the game with matrix A give Player I b plus the payoff
using the same strategies in the game with matrix A. Thus, any minimax strategy for
either player in one game is also minimax in the other, and the upper (lower) value of the
game with matrix A is b plus the upper (lower) value of the game with matrix A.
Similarly, the game having matrix A = (a
ij ) with a
ij = caij , where c is a positive
constant, may be considered as the game with matrix A with a change of scale (a change
of monetary unit if you prefer). Again, minimax strategies do not change, and the upper
(lower) value of A is c times the upper (lower) value of A. We combine these observations
as follows. (See Exercise 2.)
Lemma 3. If A = (aij ) and A = (a
ij ) are matrices with a
ij = caij + b, where c > 0,
then the game with matrix A has the same minimax strategies for I and II as the game
with matrix A
. Also, if V denotes the value of the game with matrix A, then the value
V  of the game with matrix A satisfies V  = cV + b.
4.4 Reduction to a Linear Programming Problem. There are several nice proofs
of the Minimax Theorem. The simplest proof from scratch seems to be that of G. Owen
(1982). However, that proof is by contradiction using induction on the size of the matrix.
It gives no insight into properties of the solution or on how to find the value and optimal
strategies. Other proofs, based on the Separating Hyperplane Theorem or the Brouwer
Fixed Point Theorem, give some insight but are based on nontrivial theorems not known
to all students.
Here we use a proof based on linear programming. Although based on material not
known to all students, it has the advantage of leading to a simple algorithm for solving
finite games. For a background in linear programming, the book by Chv´atal (1983) can be
recommended. A short course on Linear Programming more in tune with the material as
it is presented here may be found on the web at http://www.math.ucla.edu/˜tom/LP.pdf.
A Linear Program is defined as the problem of choosing real variables to maximize or
minimize a linear function of the variables, called the objective function, subject to linear
II – 38
constraints on the variables. The constraints may be equalities or inequalities. A standard
form of this problem is to choose y1,... ,yn, to
maximize b1y1 + ··· + bnyn, (8)
subject to the constraints
a11y1 + ··· + a1nyn ≤ c1
.
.
.
am1y1 + ··· + amnyn ≤ cm
(9)
and
yj ≥ 0 for j = 1,...,n.
Let us consider the game problem from Player I’s point of view. He wants to choose
p1,..., pm to maximize (5) subject to the constraint p ∈ X∗. This becomes the mathematical
program: choose p1,...,pm to
maximize min
1≤j≤n

m
i=1
piaij (10)
subject to the constraints
p1 + ··· + pm = 1 (11)
and
pi ≥ 0 for i = 1, . . . , m.
Although the constraints are linear, the objective function is not a linear function of
the p’s because of the min operator, so this is not a linear program. However, it can be
changed into a linear program through a trick. Add one new variable v to Player I’s list of
variables, restrict it to be less than the objective function, v ≤ min1≤j≤n
m
i=1 piaij , and
try to make v as large as possible subject to this new constraint. The problem becomes:
Choose v and p1,...,pm to
maximize v (12)
subject to the constraints
v ≤ 
m
i=1
piai1
.
.
.
v ≤ 
m
i=1
piain
p1+ ··· + pm = 1
(13)
II – 39
and
pi ≥ 0 for i = 1, . . . , m.
This is indeed a linear program. For solving such problems, there exists a simple algorithm
known as the simplex method.
In a similar way, one may view the problem from Player II’s point of view and arrive
at a similar linear program. II’s problem is: choose w and q1,...,qn to
minimize w (14)
subject to the constraints
w ≥ 
n
j=1
a1j qj
.
.
.
w ≥ 
n
j=1
amj qj
q1+ ··· + qn = 1
(15)
and
qj ≥ 0 for j = 1, . . . , n.
In Linear Programming, there is a theory of duality that says these two programs,
(12)-(13), and (14)-(15), are dual programs. And there is a remarkable theorem, called the
Duality Theorem, that says dual programs have the same value. The maximum Player I
can achieve in (14) is equal to the minimum that Player II can achieve in (12). But this is
exactly the claim of the Minimax Theorem. In other words, the Duality Theorem implies
the Minimax Theorem.
There is another way to transform the linear program, (12)-(13), into a linear program
that is somewhat simpler for computations when it is known that the value of the game
is positive. So suppose v > 0 and let xi = pi/v. Then the constraint p1 + ··· + pm = 1
becomes x1 + ··· + xm = 1/v, which looks nonlinear. But maximizing v is equivalent to
minimizing 1/v, so we can remove v from the problem by minimizing x1 +···+xm instead.
The problem, (12)-(13), becomes: choose x1,...,xm to
minimize x1 + ··· + xm (16)
subject to the constraints
1 ≤ 
m
i=1
xiai1
.
.
.
1 ≤ 
m
i=1
xiain
(17)
II – 40
and
xi ≥ 0 for i = 1, . . . , m.
When we have solved this problem, the solution of the original game may be easily found.
The value will be v = 1/(x1 + ··· + xm) and the optimal strategy for Player I will be
pi = vxi for i = 1, . . . .m.
4.5 Description of the Pivot Method for Solving Games The following algorithm
for solving finite games is essentially the simplex method for solving (16)-(17) as
described in Williams (1966).
Step 1. Add a constant to all elements of the game matrix if necessary to insure that the
value is positive. (If you do, you must remember at the end to subtract this constant from
the value of the new matrix game to get the value of the original matrix game.)
Step 2. Create a tableau by augmenting the game matrix with a border of −1’s along
the lower edge, +1’s along the right edge, and zero in the lower right corner. Label I’s
strategies on the left from x1 to xm and II’s strategies on the top from y1 to yn.
y1 y2 ··· yn
x1 a11 a12 ··· a1n 1
x2 a21 a22 ··· a2n 1
.
.
. .
.
. .
.
. .
.
. .
.
.
xm am1 am2 ··· amn 1
−1 −1 ··· −1 0
Step 3. Select any entry in the interior of the tableau to be the pivot, say row p column q,
subject to the properties:
a. The border number in the pivot column, a(m + 1, q), must be negative.
b. The pivot, a(p, q), itself must be positive.
c. The pivot row, p, must be chosen to give the smallest of the ratios the border
number in the pivot row to the pivot, a(p, n+1)/a(p, q), among all positive pivots for that
column.
Step 4. Pivot as follows:
a. Replace each entry, a(i, j), not in the row or column of the pivot by a(i, j)−a(p, j)·
a(i, q)/a(p, q).
b. Replace each entry in the pivot row, except for the pivot, by its value divided by
the pivot value.
c. Replace each entry in the pivot column, except for the pivot, by the negative of its
value divided by the pivot value.
d. Replace the pivot value by its reciprocal.
This may be represented symbolically by
p r
c q −→ 1/p r/p
−c/p q − (rc/p)
II – 41
where p stands for the pivot, r represents any number in the same row as the pivot, c
represents any number in the same column as the pivot, and q is an arbitrary entry not in
the same row or column as the pivot.
Step 5. Exchange the label on the left of the pivot row with the label on the top of the
pivot column.
Step 6. If there are any negative numbers remaining in the lower border row, go back to
step 3.
Step 7. Otherwise, a solution may now be read out:
a. The value, v, is the reciprocal of the number in the lower right corner. (If you
subtracted a number from each entry of the matrix in step 1, it must be added to v here.)
b. I’s optimal strategy is constructed as follows. Those variables of Player I that end
up on the left side receive probability zero. Those that end up on the top receive the value
of the bottom edge in the same column divided by the lower right corner.
c. II’s optimal strategy is constructed as follows. Those variables of Player II that
end up on the top receive probability zero. Those that end up on the left receive the value
of the right edge in the same row divided by the lower right corner.
4.6 A Numerical Example. Let us illustrate these steps using an example. Let’s
take a three-by-three matrix since that is the simplest example one we cannot solve using
previous methods. Consider the matrix game with the following matrix,
B =
⎛
⎝
2 −1 6
0 1 −1
−22 1
⎞
⎠
We might check for a saddle point (there is none) and we might check for domination
(there is none). Is the value positive? We might be able to guess by staring at the matrix
long enough, but why don’t we simply make the first row positive by adding 2 to each
entry of the matrix:
B =
⎛
⎝
418
231
043
⎞
⎠
The value of this game is at least one since Player I can guarantee at least 1 by using the
first (or second) row. We will have to remember to subtract 2 from the value of B to get
the value of B. This completes Step 1 of the algorithm.
In Step 2, we set up the tableau for the matrix B as follows.
y1 y2 y3
x1 418 1
x2 231 1
x3 043 1
−1 −1 −1 0
II – 42
In Step 3, we must choose the pivot. Since all three columns have a negative number
in the lower edge, we may choose any of these columns as the pivot column. Suppose we
choose column 1. The pivot row must have a positive number in this column, so it must be
one of the top two rows. To decide which row, we compute the ratios of border numbers
to pivot. For the first row it is 1/4; for the second row it is 1/2. the former is smaller, so
the pivot is in row 1. We pivot about the 4 in the upper left corner.
Step 4 tells us how to pivot. The pivot itself gets replaced by its reciprocal, namely
1/4. The rest of the numbers in the pivot row are simply divided by the pivot, giving 1/4,
2, and 1/4. Then the rest of the numbers in the pivot column are divided by the pivot and
changed in sign. The remaining nine numbers are modified by subtracting r · c/p for the
corresponding r and c. For example, from the 1 in second row third column we subtract
8 × 2/4 = 4, leaving −3. The complete pivoting operation is
y1 y2 y3
x1 418 1
x2 231 1
x3 043 1
−1 −1 −1 0
−→
x1 y2 y3
y1 1/4 1/4 2 1/4
x2 −1/2 5/2 −3 1/2
x3 043 1
1/4 −3/4 1 1/4
In the fifth step, we interchange the labels of the pivot row and column. Here we
interchange x1 and y1. This has been done in the display.
For Step 6, we check for negative entries in the lower edge. Since there is one, we
return to Step 3.
This time, we must pivot in column 2 since it has the unique negative number in the
lower edge. All three numbers in this column are positive. We find the ratios of border
numbers to pivot for rows 1, 2 and 3 to be 1, 1/5, and 1/4. The smallest occurs in the
second row, so we pivot about the 5/2 in the second row, second column. Completing
Steps 4 and 5, we obtain
x1 y2 y3
y1 1/4 1/4 2 1/4
x2 −1/2 5/2 −3 1/2
x3 043 1
1/4 −3/4 1 1/4
−→
x1 x2 y3
y1 .3 −.1 2.3 .2
y2 −.2 .4 −1.2 .2
x3 .8 −1.6 7.8 .2
.1 .3 .1 .4
At Step 6 this time, all values on the lower edge are non-negative so we pass to Step
7. We may now read the solution to the game with matrix B
.
The value is the reciprocal of .4, namely 5/2.
Since x3 is on the left in the final tableau, the optimal p3 is zero. The optimal p1
and p2 are the ratios, .1/.4 and .3/.4, namely 1/4 and 3/4. Therefore, I’s optimal mixed
strategy is (p1, p2, p3)=(.25, .75, 0).
II – 43
Since y3 is on the top in the final tableau, the optimal q3 is zero. The optimal q1
and q2 are the ratios, .2/.4 and .2/.4, namely 1/2 and 1/2. Therefore, II’s optimal mixed
strategy is (q1, q2, q3)=(.5, .5, 0).
The game with matrix B has the same optimal mixed strategies but the value is
5/2 − 2=1/2.
Remarks. 1. The reason that the pivot row is chosen according to the rule in Step
3(c) is so that the numbers in the resulting right edge of the tableau stay non-negative. If
after pivoting you find a negative number in the last column, you have made a mistake,
either in the choice of the pivot row, or in your numerical calculations for pivoting.
2. There may be ties in comparing the ratios to choose the pivot row. The rule given
allows you to choose among those rows with the smallest ratios. The smallest ratio may
be zero.
3. The value of the number in the lower right corner never decreases. (Can you see
why this is?) In fact, the lower right corner is always equal to the sum of the values in the
lower edge corresponding to Player I’s labels along the top. Similarly, it is also the sum
of the values on the right edge corresponding to Player II’s labels on the left. This gives
another small check on your arithmetic.
4. One only pivots around numbers in the main body of the tableau, never in the
lower or right edges.
5. This method gives one optimal strategy for each player. If other optimal strategies
exist, there will be one or more zeros in the bottom edge or right edge in the final tableau.
Other optimal basic strategies can be found by pivoting further, in a column with a zero
in the bottom edge or a row with a zero in the right edge, in such a way that the bottom
row and right edge stay nonnegative.
4.7 Approximating the Solution: Fictitious Play.
As an alternative to the simplex method, the method of fictitious play may be used to
approximate the value and optimal strategies of a finite game. It is a sequential procedure
that approximates the value of a game as closely as desired, giving upper and lower bounds
that converge to the value and strategies for the players that achieve these bounds.
The advantage of the simplex method is that it gives answers that are accurate,
generally to machine accuracy, and for small size problems is extremely fast. The advantage
of the method of fictitious play is its simplicity, both to program and understand, and the
fact that you can stop it at any time and obtain answers whose accuracy you know. The
simplex method only gives answers when it is finished. For large size problems, say a
matrix 50 by 50 or greater, the method of fictitious play will generally give a sufficiently
accurate answer in a shorter time than the simplex method. For very large problems, it is
the only way to proceed.
Let A(i, j) be an m by n payoff matrix. The method starts with an arbitrary initial
pure strategy 1 ≤ i1 ≤ m for Player I. Alternatively from then on, each player chooses his
II – 44
next pure strategy as a best reply assuming the other player chooses among his previous
choices at random equally likely. For example, if i1,...,ik have already been chosen by
Player I for some k ≥ 1, then jk is chosen as that j that minimizes the expectation
(1/k)
k
=1 A(i, j). Similarly, if j1,...,jk have already been chosen, ik+1 is then chosen
as that i that maximizes the expectation (1/k)
k
=1 A(i, j). To be specific, we define
sk(j) = 

k
=1
A(i, j) and tk(i) = 

k
=1
A(i, j) (1)
and then define
jk = argmin sk(j) and ik+1 = argmax tk(i) (2)
If the maximum of tk(i) is assumed at several different values of i, then it does not matter
which of these is taken as ik+1. To be specific, we choose ik+1 as the smallest value of i
that maximizes tk(i). Similarly jk is taken as the smallest j that minimizes sk(j). In this
way, the sequences ik and jk are defined deterministically once i1 is given.
Notice that Vk = (1/k)tk(ik+1) is an upper bound to the value of the game since
Player II can use the strategy that chooses j randomly and equally likely from j1,...,jk
and keep Player I’s expected return to be at most Vk. Similarly, Vk = (1/k)sk(jk) is a
lower bound to the value of the game. It is rather surprising that these upper and lower
bounds to the value converge to the value of the game as k tends to infinity.
Theorem. If V denotes the value of the game, then Vk → V , Vk → V , and Vk ≤ V ≤ Vk,
for all k.
This approximation method was suggested by George Brown (1951), and the proof
of convergence was provided by Julia Robinson (1951). The convergence of Vk and Vk to
V is slow. It is thought to be of order at least 1/
√
k. In addition, the convergence is not
monotone. See the example below.
A modification of this method in which i1 and j1 are initially arbitrarily chosen, and
then the selection of future ik and jk is made simultaneously by the players rather than
sequentially, is often used, but it is not as fast.
It should be mentioned that as a practical matter, choosing at each stage a best reply
to an opponent’s imagined strategy of choosing among his previous choices at random is
not a good idea. See Exercise 7. On the other hand, Alfredo Ba˜nos (1968) describes a
sequential method for Player I, say, to choose mixed strategies such that liminf of the
average payoff is at least the value of the game no matter what Player II does. This
choice of mixed strategies is based only upon Player I’s past pure strategy choices and
the past observed payoffs, but not otherwise on the payoff matrix or upon the opponent’s
pure strategy choices. It would be nice to devise a practical method of choosing a mixed
strategy depending on all the information contained in the previous plays of the game that
performs well.
II – 45
EXAMPLE. Take as an example the game with matrix
A =
⎛
⎝
2 −1 6
0 1 −1
−22 1
⎞
⎠
This is the game solved in Section 4.6. It has value .5, and optimal mixed strategies,
(.25, .75, 0) and (.5, .5, 0) for Player I and Player II respectively. It is easy to set up a
program to perform the calculations. In particular, the computations, (1), may be made
recursively in the simpler form
sk(j) = sk−1(j) + A(ik, j) and tk(i) = tk−1(i) + A(i, jk) (3)
We take the initial i1 = 1, and find
k ik sk(1) sk(2) sk(3) Vk jk tk(1) tk(2) tk(3) Vk
1 1 2 −1 6 −1 2 −1 1 2 2
2 3 0 1 70 1 1 1 00.5
3 1 2 0 13 0 2 0 2 2 0.6667
4 2 2 1 12 0.25 2 −1 3 4 1
5 3 0 3 13 0 1 1 3 2 0.6
6 2 0 4 12 0 1 3 3 00.5
7 1 2 3 18 0.2857 1 5 3 −2 0.7143
8 1 4 2 24 0.25 2 4 4 00.5
9 1 6 1 30 0.1111 2 3 5 2 0.5556
10 2 6 2 29 0.2 2 2 6 4 0.6
11 2 6 3 28 0.2727 2 1 7 6 0.6364
12 2 6 4 27 0.3333 2 0 8 8 0.6667
13 2 6 5 26 0.3846 2 −1 9 10 0.7692
14 3 4 7 27 0.2857 1 1 9 8 0.6429
15 2 4 8 26 0.2667 1 3 9 6 0.6
The initial choice of i1 = 1 gives (s1(1), s1(2), s1(3)) as the first row of A, which has a
minimum at s1(2). Therefore, j1 = 2. The second column of A has t1(3) as the maximum,
so i2 = 3. Then the third row of A is added to the s1 to produce the s2 and so on. The
minimums of the sk and the maximums of the tk are indicated in boldface. The largest
of the Vk found so far occurs at k = 13 and has value sk(jk)/k = 5/13 = 0.3846 .... This
value can be guaranteed to Player I by using the mixed strategy (5/13,6/13,2/13), since in
the first 13 of the ik there are 5 1’s, 6 2’s and 2 3’s. The smallest of the Vk occurs several
times and has value .5. It can be achieved by Player II using the first and second columns
equally likely. So far we know that .3846 ≤ V ≤ .5, although we know from Section 4.6
that V = .5.
Computing further, we can find that V91 = 44/91 = .4835 ... and is achieved by
the mixed strategy (25/91, 63/91, 3/91). From row 9 on, the difference between the boldface
numbers in each row seems to be bounded between 4 and 6. This implies that the
convergence is of order 1/k.
II – 46
4.8 Exercises.
1. Consider the game with matrix A. Past experience in playing the game with Player
II enables Player I to arrive at a set of probabilities reflecting his belief of the column that II
will choose. I thinks that with probabilities 1/5, 1/5, 1/5, and 2/5, II will choose columns
1, 2, 3, and 4 respectively.
A =
⎛
⎝
0724
1482
9 3 −1 6
⎞
⎠ .
(a) Find for I a Bayes strategy (best response) against (1/5, 1/5, 1/5, 2/5).
(b) Suppose II guesses correctly that I is going to use a Bayes strategy against
(1/5, 1/5, 1/5, 2/5). Instruct II on the strategy she should use - that is, find II’s Bayes
strategy against I’s Bayes strategy against (1/5, 1/5, 1/5, 2/5).
2. The game with matrix A has value zero, and (6/11, 3/11, 2/11) is optimal for I.
A =
⎛
⎝
0 −1 1
2 0 −2
−330
⎞
⎠ B =
⎛
⎝
537
951
−1 11 5
⎞
⎠
(a) Find the value of the game with matrix B and an optimal strategy for I.
(b) Find an optimal strategy for II in both games.
3. A game without a value. Let X = {1, 2, 3,...}, let Y = {1, 2, 3,...} and
A(i, j) =  +1 if i>j
0 if i = j
−1 if i<j
This is the game “The player that chooses the larger integer wins”. Here we may take for
the space X∗ of mixed strategies of Player I
X∗ = {(p1, p2,...) : pi ≥ 0 for all i, and 
∞
i=1
pi = 1}.
Similarly,
Y ∗ = {(q1, q2,...) : qj ≥ 0 for all j, and 
∞
j=1
qj = 1}.
The payoff for given p ∈ X∗ and q ∈ Y ∗ is
A(p, q) = 
∞
i=1

∞
j=1
piA(i, j)qj .
II – 47
(a) Show that for all q ∈ Y ∗, sup1≤i<∞
∞
j=1 A(i, j)qj = +1.
(b) Conclude that V = +1.
(c) Using symmetry, argue that V = −1.
(d) What are I’s minimax strategies in this game?
4. Use the method presented in Section 4.5 to solve the game with matrix
A =
⎛
⎝
012
2 −1 −2
3 −3 0
⎞
⎠ .
Either argue that the value is positive, or add +1 to the elements of the matrix. To go
easy on the homework grader, make the first pivot in the second column.
5. An Example In Which the Lower Value is Greater than the Upper
Value? Consider the infinite game with strategy spaces X = Y = {0, 1, 2,...}, and payoff
function,
A(i, j) =  0 if i = j
4j if i>j
−4i if i<j.
Note that the game is symmetric. Let p = (p0, p1, p2,...)
T = (1/2, 1/4, 1/8,...)
T be be a
mixed strategy for Player I, pi = 2−(i+1).
(a) Show that if Player I uses this strategy, his average return, ∞
i=0 piA(i, j), is equal
to 1/2 for all pure strategies j for Player II.
(b) So p is an equalizer strategy that guarantees Player I at least 1/2. So the lower
value is at least 1/2. Perhaps he can do better. In fact he can, but ...Wait a minute! The
game is symmetric. Shouldn’t the value be zero? Worse, suppose Player II uses the same
strategy. By symmetry, she can keep Player I’s winnings down to −1/2 no matter what
pure strategy he chooses. So the upper value is at most −1/2. What is wrong? What
if both players use the mixed strategy, p? We haven’t talked much about infinite games,
but what restrictions would you place on infinite games to avoid such absurd examples?
Should the restrictions be placed on the payoff function, A, or on the notion of a mixed
strategy?
6. Carry out the fictitious play algorithm on the matrix A =
 1 −1
0 2
through step
k = 4. Find the upper and lower bounds on the value of the game that this gives.
7. Suppose the game with matrix,  √
2 0
0 1
is played repeatedly. On the first round
the players make any choices.
(a) Thereafter Player I makes a best response to his opponent’s imagined strategy of
choosing among her previous choice at random. If Player II knows this, what should she
do? What are the limiting average frequencies of the choices of the players?
(b) Suppose Player II is required to play a best response to her opponent’s previous
choices. What should Player I do, and what would his limiting average payoff be?
II – 48
5. The Extensive Form of a Game
The strategic form of a game is a compact way of describing the mathematical aspects
of a game. In addition, it allows a straightforward method of analysis, at least in principle.
However, the flavor of many games is lost in such a simple model. Another mathematical
model of a game, called the extensive form, is built on the basic notions of position and
move, concepts not apparent in the strategic form of a game. In the extensive form, we
may speak of other characteristic notions of games such as bluffing, signaling, sandbagging,
and so on. Three new concepts make their appearance in the extensive form of a game:
the game tree, chance moves, and information sets.
5.1 The Game Tree. The extensive form of a game is modelled using a directed
graph. A directed graph is a pair (T,F) where T is a nonempty set of vertices and F is
a function that gives for each x ∈ T a subset, F(x) of T called the followers of x. When
a directed graph is used to represent a game, the vertices represent positions of the game.
The followers, F(x), of a position, x, are those positions that can be reached from x in one
move.
A path from a vertex t0 to a vertex t1 is a sequence, x0, x1,...,xn, of vertices
such that x0 = t0, xn = t1 and xi is a follower of xi−1 for i = 1,...,n. For the extensive
form of a game, we deal with a particular type of directed graph called a tree.
Definition. A tree is a directed graph, (T,F) in which there is a special vertex, t0, called
the root or the initial vertex, such that for every other vertex t ∈ T, there is a unique path
beginning at t0 and ending at t.
The existence and uniqueness of the path implies that a tree is connected, has a unique
initial vertex, and has no circuits or loops.
In the extensive form of a game, play starts at the initial vertex and continues along
one of the paths eventually ending in one of the terminal vertices. At terminal vertices,
the rules of the game specify the payoff. For n-person games, this would be an n-tuple of
payoffs. Since we are dealing with two-person zero-sum games, we may take this payoff to
be the amount Player I wins from Player II. For the nonterminal vertices there are three
possibilities. Some nonterminal vertices are assigned to Player I who is to choose the move
at that position. Others are assigned to Player II. However, some vertices may be singled
out as positions from which a chance move is made.
Chance Moves. Many games involve chance moves. Examples include the rolling of
dice in board games like monopoly or backgammon or gambling games such as craps, the
dealing of cards as in bridge or poker, the spinning of the wheel of fortune, or the drawing
of balls out of a cage in lotto. In these games, chance moves play an important role. Even
in chess, there is generally a chance move to determine which player gets the white pieces
(and therefore the first move which is presumed to be an advantage). It is assumed that
the players are aware of the probabilities of the various outcomes resulting from a chance
move.
II – 49
Information. Another important aspect we must consider in studying the extensive
form of games is the amount of information available to the players about past moves of
the game. In poker for example, the first move is the chance move of shuffling and dealing
the cards, each player is aware of certain aspects of the outcome of this move (the cards he
received) but he is not informed of the complete outcome (the cards received by the other
players). This leads to the possibility of “bluffing.”
5.2 Basic Endgame in Poker. One of the simplest and most useful mathematical
models of a situation that occurs in poker is called the “classical betting situation” by
Friedman (1971) and “basic endgame” by Cutler (1976). These papers provide explicit
situations in the game of stud poker and of lowball stud for which the model gives a very
accurate description. This model is also found in the exercises of the book of Ferguson
(1967). Since this is a model of a situation that occasionally arises in the last round of
betting when there are two players left, we adopt the terminology of Cutler and call it
Basic Endgame in poker. This will also emphasize what we feel is an important feature of
the game of poker, that like chess, go, backgammon and other games, there is a distinctive
phase of the game that occurs at the close, where special strategies and tactics that are
analytically tractable become important.
Basic Endgame is played as follows. Both players put 1 dollar, called the ante, in the
center of the table. The money in the center of the table, so far two dollars, is called the
pot. Then Player I is dealt a card from a deck. It is a winning card with probability 1/4
and a losing card with probability 3/4. Player I sees this card but keeps it hidden from
Player II. (Player II does not get a card.) Player I then checks or bets. If he checks, then
his card is inspected; if he has a winning card he wins the pot and hence wins the 1 dollar
ante from II, and otherwise he loses the 1 dollar ante to II. If I bets, he puts 2 dollars more
into the pot. Then Player II – not knowing what card Player I has – must fold or call. If
she folds, she loses the 1 dollar ante to I no matter what card I has. If II calls, she adds 2
dollars to the pot. Then Player I’s card is exposed and I wins 3 dollars (the ante plus the
bet) from II if he has a winning card, and loses 3 dollars to II otherwise.
Let us draw the tree for this game. There are at most three moves in this game: (1)
the chance move that chooses a card for I, (2) I’s move in which he checks or bets, and (3)
II’s move in which she folds or calls. To each vertex of the game tree, we attach a label
indicating which player is to move from that position. Chance moves we generally refer to
as moves by nature and use the label N. See Figure 1.
Each edge is labelled to identify the move. (The arrows are omitted for the sake of
clarity. Moves are assumed to proceed down the page.) Also, the moves leading from a
vertex at which nature moves are labelled with the probabilities with which they occur.
At each terminal vertex, we write the numerical value of I’s winnings (II’ s losses).
There is only one feature lacking from the above figure. From the tree we should be
able to reconstruct all the essential rules of the game. That is not the case with the tree of
Figure 1 since we have not indicated that at the time II makes her decision she does not
know which card I has received. That is, when it is II’s turn to move, she does not know at
which of her two possible positions she is. We indicate this on the diagram by encircling the
II – 50
3 1
1
−3 1
−1
N
I
II II
I
winning losing
1/4 3/4
bet check bet check
call fold call fold
Figure 1.
two positions in a closed curve, and we say that these two vertices constitute an information
set. The two vertices at which I is to move constitute two separate information sets since
he is told the outcome of the chance move. To be complete, this must also be indicated
on the diagram by drawing small circles about these vertices. We may delete one of the
labels indicating II’s vertices since they belong to the same information set. It is really
the information set that must be labeled. The completed game tree becomes
3 1
1
−3 1
−1
N
I I
II
winning losing
1/4 3/4
bet check bet check
call fold call fold
Figure 2.
The diagram now contains all the essential rules of the game.
5.3 The Kuhn Tree. The game tree with all the payoffs, information sets, and labels
for the edges and vertices included is known as the Kuhn Tree. We now give the formal
definition of a Kuhn tree.
Not every set of vertices can form an information set. In order for a player not to
be aware of which vertex of a given information set the game has come to, each vertex in
that information set must have the same number of edges leaving it. Furthermore, it is
important that the edges from each vertex of an information set have the same set of labels.
The player moving from such an information set really chooses a label. It is presumed that
a player makes just one choice from each information set.
II – 51
Definition. A finite two-person zero-sum game in extensive form is given by
1) a finite tree with vertices T,
2) a payoff function that assigns a real number to each terminal vertex,
3) a set T0 of non-terminal vertices (representing positions at which chance moves
occur) and for each t ∈ T0, a probability distribution on the edges leading from t,
4) a partition of the rest of the vertices (not terminal and not in T0) into two groups
of information sets T11, T12,...,T1k1 (for Player I) and T21, T22,...,T2k2 (for Player II),
and
5) for each information set Tjk for player j, a set of labels Ljk, and for each t ∈ Tjk,
a one-to-one mapping of Ljk onto the set of edges leading from t.
The information structure in a game in extensive form can be quite complex. It may
involve lack of knowledge of the other player’s moves or of some of the chance moves. It
may indicate a lack of knowledge of how many moves have already been made in the game
(as is the case With Player II in Figure 3).
0
1 –1 1 –1 0 –2 2 1
1 2
I
II II
I
A B
a b c d
e
DE de DE a b c
D E
Figure 3.
It may describe situations in which one player has forgotten a move he has made
earlier (as is the case With Player I in Figures 3 or 4). In fact, one way to try to model
the game of bridge as a two-person zero-sum game involves the use of this idea. In bridge,
there are four individuals forming two teams or partnerships of two players each. The
interests of the members of a partnership are identical, so it makes sense to describe this
as a two-person game. But the members of one partnership make bids alternately based
on cards that one member knows and the other does not. This may be described as a
single player who alternately remembers and forgets the outcomes of some of the previous
random moves. Games in which players remember all past information they once knew
and all past moves they made are called games of perfect recall.
A kind of degenerate situation exists when an information set contains two vertices
which are joined by a path, as is the case with I’s information set in Figure 5.
II – 52
1
–1 0 0 –1
2
I
II
I
f g
ab ab
cd cd
Figure 4.
We take it as a convention that a player makes one choice from each information set
during a game. That choice is used no matter how many times the information set is
reached. In Figure 5, if I chooses option a there is no problem. If I chooses option b, then
in the lower of I’s two vertices the a is superfluous, and the tree is really equivalent to
Figure 6. Instead of using the above convention, we may if we like assume in the definition
of a game in extensive form that no information set contains two vertices joined by a path.
2 0
0 1
2
I
II
a b
cd cd
a b
2 0
1
2
I
I
II
a b
cd cd
b
Figure 5. Figure 6.
Games in which both players know the rules of the game, that is, in which both players
know the Kuhn tree, are called games of complete information. Games in which one or
both of the players do not know some of the payoffs, or some of the probabilities of chance
moves, or some of the information sets, or even whole branches of the tree, are called
games with incomplete information, or pseudogames. We assume in the following
that we are dealing with games of complete information.
5.4 The Representation of a Strategic Form Game in Extensive Form. The
notion of a game in strategic form is quite simple. It is described by a triplet (X, Y, A) as in
Section 1. The extensive form of a game on the other hand is quite complex. It is described
II – 53
by the game tree with each non-terminal vertex labeled as a chance move or as a move
of one of the players, with all information sets specified, with probability distributions
given for all chance moves, and with a payoff attached to each terminal vertex. It would
seem that the theory of games in extensive is much more comprehensive than the theory
of games in strategic form. However, by taking a game in extensive form and considering
only the strategies and average payoffs, we may reduce the game to strategic form.
First, let us check that a game in strategic form can be put into extensive form. In the
strategic form of a game, the players are considered to make their choices simultaneously,
while in the extensive form of a game simultaneous moves are not allowed. However,
simultaneous moves may be made sequentially as follows. We let one of the players, say
Player I, move first, and then let player II move without knowing the outcome of I’s move.
This lack of knowledge may be described by the use of an appropriate information set.
The example below illustrates this.
 3 01
−120
3 0 1 –1 2 0
I
II
1 2
1 23 1 23
Matrix Form Equivalent Extensive Form
Player I has 2 pure strategies and Player II has 3. We pretend that Player I moves first by
choosing row 1 or row 2. Then Player II moves, not knowing the choice of Player I. This is
indicated by the information set for Player II. Then Player II moves by choosing column
1, 2 or 3, and the appropriate payoff is made.
5.5 Reduction of a Game in Extensive Form to Strategic Form. To go in
the reverse direction, from the extensive form of a game to the strategic form, requires the
consideration of pure strategies and the usual convention regarding random payoffs.
Pure strategies. Given a game in extensive form, we first find X and Y , the sets of
pure strategies of the players to be used in the strategic form. A pure strategy for Player
I is a rule that tells him exactly what move to make in each of his information sets. Let
T11,...,T1k1 be the information sets for Player I and let L11,...,L1k1 be the corresponding
sets of labels. A pure strategy for I is a k1-tuple x = (xl, ..., xk1 ) where for each i, xi is one
of the elements of L1i. If there are mi elements in L1i, the number of such kl-tuples and
hence the number of I s pure strategies is the product m1m2 ··· mk. The set of all such
strategies is X. Similarly, if T21,...,T2k2 represent II’s information sets and L21,...,L2k2
the corresponding sets of labels, a pure strategy for II is a k2-tuple, y = (y1,...,yk2 ) where
yj ∈ L2j for each j. Player II has n1n2 ··· nk2 pure strategies if there are nj elements in
L2j . Y denotes the set of these strategies.
Random payoffs. A referee, given x ∈ X and y ∈ Y , could play the game, playing the
appropriate move from x whenever the game enters one of I’s information sets, playing the
II – 54
appropriate move from y whenever the game enters one of II’s information sets, and playing
the moves at random with the indicated probabilities at each chance move. The actual
outcome of the game for given x ∈ X and y ∈ Y depends on the chance moves selected,
and is therefore a random quantity. Strictly speaking, random payoffs were not provided
for in our definition of games in normal form. However, we are quite used to replacing
random payoffs by their average values (expected values) when the randomness is due to
the use of mixed strategies by the players. We adopt the same convention in dealing with
random payoffs when the randomness is due to the chance moves. The justification of this
comes from utility theory.
Convention. If for fixed pure strategies of the players, x ∈ X and y ∈ Y , the payoff is
a random quantity, we replace the payoff by the average value, and denote this average
value by A(x, y).
For example, if for given strategies x ∈ X and y ∈ Y , Player I wins 3 with probability
1/4, wins 1 with probability 1/4, and loses 1 with probability 1/2, then his average payoff
is 1
4 (3) + 1
4 (1) + 1
2 (−1) = 1/2 so we let A(x, y)=1/2.
Therefore, given a game in extensive form, we say (X, Y, A) is the equivalent strategic
form of the game if X and Y are the pure strategy spaces of players I and II respectively,
and if A(x, y) is the average payoff for x ∈ X and y ∈ Y .
5.6 Example. Let us find the equivalent strategic form to Basic Endgame in Poker
described in the Section 5.2, whose tree is given in Figure 2. Player I has two information
sets. In each set he must make a choice from among two options. He therefore has 2·2=4
pure strategies. We may denote them by
(b, b): bet with a winning card or a losing card.
(b, c): bet with a winning card, check with a losing card.
(c, b): check with a winning card, bet with a losing card.
(c, c): check with a winning card or a losing card.
Therefore, X = {(b, b),(b, c),(c, b),(c, c)}. We include in X all pure strategies whether
good or bad (in particular, (c, b) seems a rather perverse sort of strategy.)
Player II has only one information set. Therefore, Y = {c, f}, where
c: if I bets, call.
f: if I bets, fold.
Now we find the payoff matrix. Suppose I uses (b, b) and II uses c. Then if I gets a
winning card (which happens with probability 1/4), he bets, II calls, and I wins 3 dollars.
But if I gets a losing card (which happens with probability 3/4), he bets, II calls, and I
loses 3 dollars. I’s average or expected winnings is
A((b, b), c) = 1
4
(3) + 3
4
(−3) = −3
2
.
II – 55
This gives the upper left entry in the following matrix. The other entries may be computed
similarly and are left as exercises.
⎛
⎜⎜⎝
c f
(b, b) −3/2 1
(b, c) 0 −1/2
(c, b) −2 1
(c, c) −1/2 −1/2
⎞
⎟⎟⎠
Let us solve this 4 by 2 game. The third row is dominated by the first row, and the
fourth row is dominated by the second row. In terms of the original form of the game, this
says something you may already have suspected: that if I gets a winning card, it cannot
be good for him to check. By betting he will win at least as much, and maybe more. With
the bottom two rows eliminated the matrix becomes  −3/2 1
0 −1/2

, whose solution is
easily found. The value is V = −1/4. I’s optimal strategy is to mix (b, b) and (b, c) with
probabilities 1/6 and 5/6 respectively, while II’s optimal strategy is to mix c and f with
equal probabilities 1/2 each. The strategy (b, b) is Player I’s bluffing strategy. Its use
entails betting with a losing hand. The strategy (b, c) is Player I’s “honest” strategy, bet
with a winning hand and check with a losing hand. I’s optimal strategy requires some
bluffing and some honesty.
In Exercise 4, there are six information sets for I each with two choices. The number
of I’s pure strategies is therefore 26 = 64. II has 2 information sets each with two choices.
Therefore, II has 22 = 4 pure strategies. The game matrix for the equivalent strategic
form has dimension 64 by 4. Dominance can help reduce the dimension to a 2 by 3 game!
(See Exercise 10(d).)
5.7 Games of Perfect Information. Now that a game in extensive form has been
defined, we may make precise the notion of a game of perfect information.
Definition. A game of perfect information is a game in extensive form in which each
information set of every player contains a single vertex.
In a game of perfect information, each player when called upon to make a move knows
the exact position in the tree. In particular, each player knows all the past moves of the
game including the chance ones. Examples include tic-tac-toe, chess, backgammon, craps,
etc.
Games of perfect information have a particularly simple mathematical structure. The
main result is that every game of perfect information when reduced to strategic form has
a saddle point; both players have optimal pure strategies. Moreover, the saddle point can
be found by removing dominated rows and columns. This has an interesting implication
for the game of chess for example. Since there are no chance moves, every entry of the
game matrix for chess must be either +1 (a win for Player I), or −1 (a win for Player
II – 56
II), or 0 (a draw). A saddle point must be one of these numbers. Thus, either Player
I can guarantee himself a win, or Player II can guarantee himself a win, or both players
can assure themselves at least a draw. From the game-theoretic viewpoint, chess is a very
simple game. One needs only to write down the matrix of the game. If there is a row of all
+1’s, Player I can win. If there is a column of all −1’s, then Player II can win. Otherwise,
there is a row with all +1’s and 0’s and a column with all −1’s and 0’s, and so the game is
drawn with best play. Of course, the real game of chess is so complicated, there is virtually
no hope of ever finding an optimal strategy. In fact, it is not yet understood how humans
can play the game so well.
5.8 Behavioral Strategies. For games in extensive form, it is useful to consider a
different method of randomization for choosing among pure strategies. All a player really
needs to do is to make one choice of an edge for each of his information sets in the game.
A behavioral strategy is a strategy that assigns to each information set a probability
distributions over the choices of that set.
For example, suppose the first move of a game is the deal of one card from a deck of
52 cards to Player I. After seeing his card, Player I either bets or passes, and then Player
II takes some action. Player I has 52 information sets each with 2 choices of action, and
so he has 252 pure strategies. Thus, a mixed strategy for I is a vector of 252 components
adding to 1. On the other hand, a behavioral strategy for I simply given by the probability
of betting for each card he may receive, and so is specified by only 52 numbers.
In general, the dimension of the space of behavioral strategies is much smaller than
the dimension of the space of mixed strategies. The question arises – Can we do as well
with behavioral strategies as we can with mixed strategies? The answer is we can if both
players in the game have perfect recall. The basic theorem, due to Kuhn in 1953 says that
in finite games with perfect recall, any distribution over the payoffs achievable by mixed
strategies is achievable by behavioral strategies as well.
To see that behavioral strategies are not always sufficient, consider the game of imperfect
recall of Figure 4. Upon reducing the game to strategic form, we find the matrix
⎛
⎜⎜⎝
a b
(f, c) 1 −1
(f, d)10
(g, c)02
(g, d) −1 2
⎞
⎟⎟⎠
The top and bottom rows may be removed by domination, so it is easy to see that the
unique optimal mixed strategies for I and II are (0, 2/3, 1/3, 0) and (2/3, 1/3) respectively.
The value is 2/3. However, Player I’s optimal strategy is not achievable by behavioral
strategies. A behavioral strategy for I is given by two numbers, pf , the probability of choice
f in the first information set, and pc, the probability of choice c in the second information
set. This leads to the mixed strategy, (pf pc, pf (1 − pc),(1 − pf )pc,(1 − pf )(1 − pc)). The
strategy (0, 2/3, 1/3, 0) is not of this form since if the first component is zero, that is if
II – 57
pf pc = 0, then either pf = 0 or pc = 0, so that either the second or third component must
be zero also.
If the rules of the game require players to use behavioral strategies, as is the case for
certain models of bridge, then the game may not have a value. This means that if Player I
is required to announce his behavioral strategy first, then he is at a distinct disadvantage.
The game of Figure 4 is an example of this. (see Exercise 11.)
5.9 Exercises.
1. The Silver Dollar. Player II chooses one of two rooms in which to hide a silver
dollar. Then, Player I, not knowing which room contains the dollar, selects one of the
rooms to search. However, the search is not always successful. In fact, if the dollar is
in room #1 and I searches there, then (by a chance move) he has only probability 1/2 of
finding it, and if the dollar is in room #2 and I searches there, then he has only probability
1/3 of finding it. Of course, if he searches the wrong room, he certainly won’t find it. If
he does find the coin, he keeps it; otherwise the dollar is returned to Player II. Draw the
game tree.
2. Two Guesses for the Silver Dollar. Draw the game tree for problem 1, if
when I is unsuccessful in his first attempt to find the dollar, he is given a second chance
to choose a room and search for it with the same probabilities of success, independent of
his previous search. (Player II does not get to hide the dollar again.)
3. A Statistical Game. Player I has two coins. One is fair (probability 1/2 of heads
and 1/2 of tails) and the other is biased with probability 1/3 of heads and 2/3 of tails.
Player I knows which coin is fair and which is biased. He selects one of the coins and tosses
it. The outcome of the toss is announced to II. Then II must guess whether I chose the
fair or biased coin. If II is correct there is no payoff. If II is incorrect, she loses 1. Draw
the game tree.
4. A Forgetful Player. A fair coin (probability 1/2 of heads and 1/2 of tails) is
tossed and the outcome is shown to Player I. On the basis of the outcome of this toss, I
decides whether to bet 1 or 2. Then Player II hearing the amount bet but not knowing
the outcome of the toss, must guess whether the coin was heads or tails. Finally, Player I
(or, more realistically, his partner), remembering the amount bet and II’s guess, but not
remembering the outcome of the toss, may double or pass. II wins if her guess is correct
and loses if her guess is incorrect. The absolute value of the amount won is [the amount
bet (+1 if the coin comes up heads)] (×2 if I doubled). Draw the game tree.
5. A One-Shot Game of Incomplete Information. Consider the two games
G1 =
 6 0
0 0
and G2 =
 3 0
0 6
. One of these games is chosen to be played at random
with probability 1/3 for G1 and probability 2/3 for G2. The game chosen is revealed to
Player I but not to Player II. Then Player I selects a row, 1 or 2, and simultaneously
Player II chooses a column, 1 or 2, with payoff determined by the selected game. Draw
the game tree. (If the game chosen by nature is played repeatedly with Player II learning
only the pure strategy choices of Player I and not the payoffs, this is called a repeated
II – 58
game of incomplete information. There is a large literature concerning such games; see for
example, the books of Aumann and Maschler (1995) and Sorin (2002).)
6. Basic Endgame in Poker. Generalize Basic Endgame in poker by letting the
probability of receiving a winning card be an arbitrary number p, 0 ≤ p ≤ 1, and by letting
the bet size be an arbitrary number b > 0. (In Figure 2, 1/4 is replaced by p and 3/4 is
replaced by 1 − p. Also 3 is replaced by 1 + b and −3 is replaced by −(1 + b).) Find the
value and optimal strategies. (Be careful. For p ≥ (2 + b)/(2 + 2b) there is a saddle point.
When you are finished, note that for p < (2+b)/(2+2b), Player II’s optimal strategy does
not depend on p!) For other generalizations, see Ferguson and Ferguson (2007).
7. (a) Find the equivalent strategic form of the game with the game tree:
3
3 0 3 –3 –3 0 0 0 3
N
I II
II
1/3 2/3
a b c d e
f g d e ab c f g
(b) Solve the game.
8. (a). Find the equivalent strategic form of the game with the game tree:
0 2 1 –1 2 –2 0 4
N
I I
II II
1/2 1/2
A B C D
ab cd ab cd
(b). Solve the game.
9. Coin A has probability 1/2 of heads and 1/2 of tails. Coin B has probability 1/3 of
heads and 2/3 of tails. Player I must predict “heads” or “tails”. If he predicts heads, coin
A is tossed. If he predicts tails, coin B is tossed. Player II is informed as to whether I’s
prediction was right or wrong (but she is not informed of the prediction or the coin that
was used), and then must guess whether coin A or coin B was used. If II guesses correctly
II – 59
she wins 1 dollar from I. If II guesses incorrectly and I’s prediction was right, I wins 2
dollars from II. If both are wrong there is no payoff.
(a) Draw the game tree.
(b) Find the equivalent strategic form of the game.
(c) Solve.
10. Find the equivalent strategic form and solve the game of
(a) Exercise 1.
(b) Exercise 2.
(c) Exercise 3.
(d) Exercise 4.
(e) Exercise 5.
11. Suppose, in the game of Figure 4, that Player I is required to use behavioral
strategies. Show that if Player I is required to announce his behavioral strategy first, he
can only achieve a lower value of 1/2. Whereas, if Player II is required to announce her
strategy first, Player I has a behavioral strategy reply that achieves the upper value of 2/3
at least.
12. (Beasley (1990), Chap. 6.) Player I draws a card at random from a full deck of
52 cards. After looking at the card, he bets either 1 or 5 that the card he drew is a face
card (king, queen or jack, probability 3/13). Then Player II either concedes or doubles. If
she concedes, she pays I the amount bet (no matter what the card was). If she doubles,
the card is shown to her, and Player I wins twice his bet if the card is a face card, and
loses twice his bet otherwise.
(a) Draw the game tree. (You may argue first that Player I always bets 5 with a face card
and Player II always doubles if Player I bets 1.)
(b) Find the equivalent normal form.
(c) Solve.
II – 60
6. Recursive and Stochastic Games
6.1 Matrix Games with Games as Components. We consider now matrix games
in which the outcome of a particular choice of pure strategies of the players may be that
the players have to play another game. Let us take a simple example.
Let G1 and G2 denote 2 × 2 games with matrices
G1 =
 0 3
2 –1
and G2 =
 0 1
4 3
and let G denote the 2 × 2 game whose matrix is represented by
G =
 G1 4
5 G2

.
The game G is played in the usual manner with Player I choosing a row and Player II
choosing a column. If the entry in the chosen row and column is a number, II pays I that
amount and the game is over. If I chooses row 1 and II chooses column 1, then the game
G1 is played. If I chooses row 2 and II chooses column 2, then G2 is played.
We may analyze the game G by first analyzing G1 and G2.
G1 : Optimal for I is (1/2,1/2)
Optimal for II is (2/3,1/3)
Val(G1)=1.
G2 : Optimal for I is (0,1)
Optimal for II is (0,1)
Val(G2)=3.
If after playing the game G the players end up playing G1, then they can expect a payoff
of the value of G1, namely 1, on the average. If the players end up playing G2, they can
expect an average payoff of the value of G2, namely 3. Therefore, the game G can be
considered equivalent to the game with matrix
 1 4
5 3
G :
Optimal for I is (2/5,3/5)
Optimal for II is (1/5,4/5)
Val(G)=17/5.
This method of solving the game G may be summarized as follows. If the matrix of
a game G has other games as components, the solution of G is the solution of the game
whose matrix is obtained by replacing each game in the matrix of G by its value.
II – 61
Decomposition. This example may be written as a 4×4 matrix game. The four pure
strategies of Player I may be denoted {(1, 1),(1, 2),(2, 1),(1, 2)}, where (i, j) represents:
use row i in G, and if this results in Gi being played use row j. A similar notation may
be used for Player II. The 4 × 4 game matrix becomes
G =
⎛
⎜⎝
0 3
2 –1
4 4
4 4
5 5
5 5
0 1
4 3
⎞
⎟⎠
We can solve this game by the methods of Chapter 4.
Conversely, suppose we are given a game G and suppose after some rearrangement of
the rows and of the columns the matrix may be decomposed into the form
G =
 G11 G12
G21 G22
where G11 and G22 are arbitrary matrices and G12 and G21 are constant matrices. (A
constant matrix has the same numerical value for all of its entries.) Then we can solve
G by the above method, pretending that as the first move the players choose a row and
column from the 2 × 2 decomposed matrix. See Exercise 1(b).
6.2 Multistage Games. Of course, a game that is the component of some matrix
game may itself have other games as components, in which case one has to iterate the
above method to obtain the solution. This works if there are a finite number of stages.
Example 1. The Inspection Game. (M. Dresher (1962)) Player II must try to perform
some forbidden action in one of the next n time periods. Player I is allowed to inspect II
secretly just once in the next n time periods. If II acts while I is inspecting, II loses 1 unit
to I. If I is not inspecting when II acts, the payoff is zero.
Let Gn denote this game. We obtain the iterative equations
Gn =

act wait
inspect 1 0
wait 0 Gn−1

for n = 2, 3,...
with boundary condition G1 = (1). We may solve iteratively.
Val(G1)=1
Val(G2) = Val  1 0
0 1
= 1/2
Val(G3) = Val  1 0
0 1/2

= 1/3
.
.
.
Val(Gn) = Val  1 0
0 1/(n − 1)
= 1/n
II – 62
since inductively, Val(Gn) = 1
n−1 /(1 + 1
n−1 )=1/n. The optimal strategy in the game Gn
for both players is (1/n,(n − 1)/n). For other games of this sort, see the book by Garnaev
(2000).
Example 2. Guess it! (Rufus Isaacs (1955); see also Martin Gardner (1978), p. 40.) As
a more complex example of a multistage game, consider the following game loosely related
to the game Cluedo. From a deck with m + n + 1 distinct cards, m cards are dealt to
Player I, n cards are dealt to Player II, and the remaining card, called the “target card”,
is placed face down on the table. Players knows their own cards but not those of their
opponent. The objective is to guess correctly the target card. Players alternate moves,
with Player I starting. At each move, a player may either
(1) guess at the target card, in which case the game ends, with the winner being the player
who guessed if the guess is correct, and his opponent if the guess is incorrect, or
(2) ask if the other player holds a certain card. If the other player has the card, that card
must be shown and is removed from play.
With a deck of say 11 cards and each player receiving 5 cards, this is a nice playable
game that illustrates need for bluffing in a clear way. If a player asks about a card that is
in his own hand, he knows what the answer will be. We call such a play a bluff . If a player
asks about a card not in his hand, we say he is honest. If a player is always honest and
the card he asks about is the target card, the other player will know that the requested
card is the target card and so will win. Thus a player must bluff occasionally. Bluffing
may also lure the opponent into a wrong guess at the target card.
Let us denote this game with Player I to move by Gm,n. The game Gm,0 is easy to
play. Player I can win immediately. Since his opponent has no cards, he can tell what the
target card is. Similarly, the game G0,n is easy to solve. If Player I does not make a guess
immediately, his opponent will win on the next move. However, his probability of guessing
correctly is only 1/(n + 1). Valuing 1 for a win and 0 for a loss from Player I’s viewpoint,
the value of the game is just the probability I wins under optimal play. We have
Val(Gm,0) = 1 for all m ≥ 0, and Val(G0,n) = 1
n + 1
for all n ≥ 0. (1)
If Player I asks for a card that Player II has, that card is removed from play and it is
Player II’s turn to move, holding n − 1 cards to her opponent’s m cards. This is exactly
the game Gn−1,m but with Player II to move. We denote this game by Gn−1,m. Since the
probability that Player I wins is one minus the probability that Player II wins, we have
Val(Gn,m)=1 − Val(Gn,m) for all m and n. (2)
Suppose Player I asks for a card that Player II does not have. Player II must immediately
decide whether or not Player I was bluffing. If she decides Player I was honest, she
will announce the card Player I asked for as her guess at the target card, and win if she
was right and lose if she was wrong. If she decides Player I was bluffing and she is wrong,
Player I will win on his turn. If she is correct, the card Player I asked for is removed from
his hand, and the game played next is Gn,m−1.
II – 63
Using such considerations, we may write the game as a multistage game in which
a stage consists of three pure strategies for Player I (honest, bluff, guess) and two pure
strategies for Player II (ignore the asked card, call the bluff by guessing the asked card).
The game matrix becomes, for m ≥ 1 and n ≥ 1,
Gm,n =
⎛
⎝
ignore call
honest n
n+1Gn−1,m + 1
n+1
n
n+1Gn−1,m
bluff Gn,m−1 1
guess 1
n+1
1
n+1
⎞
⎠ (3)
This assumes that if Player I asks honestly, he chooses among the n + 1 unknown
cards with probability 1/(n +1) each; also if he bluffs, he chooses among his m cards with
probability 1/m each. That this may be done follows from the invariance considerations
of Section 3.6.
As an example, the upper left entry of the matrix is found as follows. With probability
n/(n + 1), Player I asks a card that is in Player II’s hand and the game becomes Gn−1,m;
with probability 1/(n + 1), Player I asks the target card, Player II ignors it and Player I
wins on his next turn, i.e. gets 1. The upper right entry is similar, except this time if the
asked card is the target card, Player II guesses it and Player I gets 0.
It is reasonable to assume that if m ≥ 1 and n ≥ 1, Player I should not guess,
because the probability of winning is too small. In fact if m ≥ 1 and n ≥ 1, there is a
strategy for Player I that dominates guessing, so that the last row of the matrix may be
deleted. This strategy is: On the first move, ask any of the m + n + 1 cards with equal
probability 1/(m + n + 1) (i.e. use row 1 with probability (n + 1)/(m + n + 1) and row
2 with probability m/(m + n + 1)), and if Player II doesn’t guess at her turn, then guess
at the next turn. We must show that Player I wins with probability at least 1/(n + 1)
whether or not Player II guesses at her next turn. If Player II guesses, her probability of
win is exactly 1/(m + 1) whether or not the asked card is one of hers. So Player I’s win
probability is m/(m + 1) ≥ 1/2 ≥ 1/(n + 1). If Player II does not guess, then at Player
I’s next turn, Player II has at most n cards (she may have n − 1) so again Player I’s win
probability is at least 1/(n + 1).
So the third row may be removed in (3) and the games reduce to
Gm,n =

ignore call
honest n
n+1Gn−1,m + 1
n+1
n
n+1Gn−1,m
bluff Gn,m−1 1

(4)
for m ≥ 1 and n ≥ 1. These 2 by 2 games are easily solved recursively, using the boundary
conditions (1). One can find the value and optimal strategies of Gm,n after one finds the
values of Gn,m−1 and Gn−1,m and uses (2). For example, the game G1,1 reduces to the
game with matrix  3/4 1/4
0 1
. The value of this game is 1/2, an optimal strategy for
II – 64
Player I is (2/3,1/3) (i.e. bluff with probability 1/3), and the optimal strategy of player
II is (1/2,1/2).
One can also show that for all m ≥ 1 and n ≥ 1 these games do not have saddle points.
In fact, one can show more: that Val(Gm,n) is nondecreasing in m and nonincreasing in n.
(The more cards you have in your hand, the better.) Let Vm,n = Val(Gm,n). Then using
Val  a b
c d
= (ad − bc)/(a + d − b − c), we have after some algebra
Vm,n = Val  n
n+1 (1 − Vn−1,m) + 1
n+1
n
n+1 (1 − Vn−1,m)
(1 − Vn,m−1) 1
= 1 + n(1 − Vn−1,m)Vn,m−1
1+(n + 1)Vn,m−1
.
for m ≥ 1 and n ≥ 1. This provides a simple direct way to compute the values recursively.
The following table gives the computed values as well as the optimal strategies for the
players for small values of m and n.
m\n 123456
0.5000 0.5000 0.4000 0.3750 0.3333 0.3125
1 0.3333 0.2500 0.2000 0.1667 0.1429 0.1250
0.5000 0.5000 0.4000 0.3750 0.3333 0.3125
0.6667 0.5556 0.5111 0.4500 0.4225 0.3871
2 0.5000 0.3333 0.2667 0.2143 0.1818 0.1563
0.3333 0.3333 0.2889 0.2500 0.2301 0.2055
0.6875 0.6250 0.5476 0.5126 0.4667 0.4411
3 0.5000 0.3750 0.2857 0.2361 0.1967 0.1701
0.3750 0.3250 0.2762 0.2466 0.2167 0.1984
0.7333 0.6469 0.5966 0.5431 0.5121 0.4749
4 0.5556 0.3947 0.3134 0.2511 0.2122 0.1806
0.3333 0.3092 0.2634 0.2342 0.2118 0.1899
0.7500 0.6809 0.6189 0.5810 0.5389 0.5112
5 0.5714 0.4255 0.3278 0.2691 0.2229 0.1917
0.3333 0.2908 0.2566 0.2284 0.2062 0.1885
0.7714 0.6972 0.6482 0.6024 0.5704 0.5353
6 0.6000 0.4410 0.3488 0.2808 0.2362 0.2003
0.3143 0.2834 0.2461 0.2236 0.2028 0.1854
Table of values and optimal strategies of Gm,n for 1 ≤ m, n ≤ 6. The top number in each
box is the value, the middle number is the probability with which Player I should bluff,
and the bottom number is the probability with which Player II should call the asked card.
II – 65
6.3 Recursive Games. -Optimal Strategies. In some games with games as
components, it may happen that the original game comes up again. Such games are called
recursive. A simple example is
G =
 G 1
1 0
This is an infinite game. If the players always play the first row and first column, the game
will be played forever. No matter how unlikely such a possibility is, the mathematical
definition is not complete until we say what the payoff is if G is played forever. Let us say
that II pays I Q units if they both choose their first pure strategy forever, and write
G =
 G 1
1 0
, Q.
We are not automatically assured the existence of a value or the existence of optimal
strategies in infinite games. However, it is easy to see that the value of G exists and is
equal to 1 no matter what the value of the number Q is. The analysis can be made as
follows.
II can restrict her losses to at most 1 by choosing the second column. If Q ≥ 1, I can
guarantee winning at least 1 by playing his first row forever. But if Q < 1, this won’t work.
It turns out that an optimal strategy for I, guaranteeing him at least 1, does not exist in
this case. However, for any  > 0 there is a strategy for I that guarantees him an average
gain of at least 1 − . Such a strategy, that guarantees a player an average payoff within
 of the value, is called -optimal. In this case, the strategy that continually uses the
mixed strategy (1 − , ) (top row with probability 1 −  and bottom row with probability
) is -optimal for I. The use of such a strategy by I insures that he will eventually choose
row 2, so that the payoff is bound to be 0 or 1 and never Q. The best that Player II can
do against this strategy is to choose column 2 immediately, hoping that I chooses row 2.
The expected payoff would then be 1 · (1 − )+0 ·  = 1 − .
In summary, for the game G above, the value is 1; Player II has an optimal strategy,
namely column 2; If Q ≥ 1, the first row forever is optimal for I; if Q < 1, there is no
optimal strategy for I, but the strategy (1 − , ) forever is -optimal for I.
Consider now the game
G0 =
 G0 5
1 0
, Q.
For this game, the value depends on Q. If Q ≥ 1, the first row forever is optimal for I,
and the value is Q if 1 ≤ Q ≤ 5, and the value is 5 if Q ≥ 5. For Q < 1, the value is 1;
however, in contrast to the game G above, I has an optimal strategy for the game G0, for
example (1/2, 1/2) forever. II’s optimal strategy is the first column forever if Q < 5, the
second column if Q > 5 and anything if Q = 5.
In analogy to what we did for games with games as components, we might attempt
to find the value v of such a game by replacing G0 by v in the matrix and solving the
II – 66
equation
v = Val  v 5
1 0
for v. Here there are many solutions to this equation. The set of all solutions to this
equation is the set of numbers v in the interval 1 ≤ v ≤ 5. (Check this!)
This illustrates a general result that the equation, given by equating v to the value of
the game obtained by replacing the game in the matrix by v, always has a solution equal
to the value of the game. It may have more solutions but the value of the game is that
solution that is closest to Q. For more information on these points, consult the papers of
Everett (1957) and of Milnor and Shapley (1957).
Example 3. Let
G =
⎛
⎝
G 1 0
1 0 G
0 G 1
⎞
⎠ , Q.
Then, if the value of G is v,
v = Val
⎛
⎝
v 1 0
1 0 v
0 v 1
⎞
⎠ = 1 + v
3 .
This equation has a unique solution, v = 1/2. This must be the value for all Q. The
strategy (1/3,1/3,1/3) forever is optimal for both players.
Example 4. The basic game of Dare is played as follows. Player I, the leader, and Player
II, the challenger, simultaneously “pass” or “dare”. If both pass, the payoff is zero (and
the game is over). If I passes and II dares, I wins 1. If I dares and II passes, I wins 3. If
both dare, the basic game is played over with the roles of the players reversed (the leader
becomes the challenger and vice versa). If the players keep daring forever, let the payoff
be zero. We might write
G =

pass dare
pass 0 1
dare 3 −GT

where −GT represents the game with the roles of the players reversed. (Its matrix is the
negative of the transpose of the matrix G.) The value of −GT is the negative of the value
of G.
If v represents the value of G, then v ≥ 0 because of the top row. Therefore the matrix
for G with −GT replaced by −v does not have a saddle point, and we have
v = Val  0 1
3 −v

= 3
4 + v
.
This gives the quadratic equation, v2 + 4v − 3 = 0. The only nonnegative solution is
v = √7 − 2 = .64575 ···. The optimal strategy for I is ((5 − √7)/3, (
√7 − 2)/3) and the
optimal strategy for II is (3 − √7,
√
7 − 2).
II – 67
Example 5. Consider the following three related games.
G1 =
 G2 0
0 G3

G2 =
 G1 1
1 0
G3 =
 G1 2
2 0
and suppose the payoff if the games are played forever is Q. Let us attempt to solve these
games. Let v1 = Val(G1), v2 = Val(G2), and v3 = Val(G3). Player I can guarantee that
v1 > 0, v2 > 0 and v3 > 0 by playing (1/2, 1/2) forever. In addition, v2 ≤ 1 and v3 ≤ 2,
which implies v1 < 1. Therefore none of the games has a saddle point and we may write
v1 = v2v3
v2 + v3
, v2 = 1
2 − v1
, v3 = 4
4 − v1
.
Substituting the latter two equations into the former, we obtain
v1
2 − v1
+
4v1
4 − v1
= 4
(2 − v1)(4 − v1)
5v2
1 − 12v1 +4=0
(5v1 − 2)(v1 − 2) = 0
Since 0 < v1 < 1, this implies that v1 = 2/5. Hence
Game value opt. for I= opt. for II
G1 2/5 (16/25, 9/25)
G2 5/8 (5/8, 3/8)
G3 10/9 (5/9, 4/9)
independent of the value of Q.
6.4 Stochastic Movement Among Games. We may generalize the notion of a
recursive game by allowing the choice of the next game played to depend not only upon the
pure strategy choices of the players, but also upon chance. Let G1,...,Gn be games and
let p1,...,pn be probabilities that sum to one. We use the notation, p1G1 +···+pnGn, to
denote the situation where the game to be played next is chosen at random, with game Gi
being chosen with probability pi, i = 1,...,n. Since, for a given number z, the 1×1 matrix
(z) denotes the trivial game in which II pays I z, we may, for example, use 1
2G1 + 1
2 (3) to
represent the situation where G1 is played if a fair coin comes up heads, and II pays I 3
otherwise.
Example 6. Let G1 and G2 be related as follows.
G1 =
 1
2G2 + 1
2 (0) 1
2 0
G2 =
 2
3G1 + 1
3 (−2) 0
0 −1

The game must eventually end (with probability 1). In fact, the players could not play
forever even if they wanted to. Even if they choose the first row and first column forever,
II – 68
eventually the game would end with a payoff of 0 or −2. Thus we do not need to specify
any payoff if play continues forever. To solve, let vi = Val(Gi) for i = 1, 2. Then 0 ≤ v1 ≤ 1
and −1 ≤ v2 ≤ 0, so neither game has a saddle point. Hence,
v1 = Val  1
2 v2 1
2 0
= 4
6 − v2
and
v2 = Val  2
3 v1 − 2
3 0
0 −1

= −2(1 − v1)
5 − 2v1
Thus
v1 = 4
6 + 2(1−v1)
5−2v1
= 2(5 − 2v1)
16 − 7v1
.
This leads to the quadratic equation, 7v2
1−20v1+10 = 0, with solution, v1 = (10−
√30)/7 =
.646 ···. We may substitute back into the equation for v2 to find v2 = −(2√30 − 10)/5 =
−.191 ···. From these values one can easily find the optimal strategies for the two games.
Example 7. A coin with probability 2/3 of heads is tossed. Both players must guess
whether the coin will land heads or tails. If I is right and II is wrong, I wins 1 if the coin
is heads and 4 if the coin is tails and the game is over. If I is wrong and II is right, there
is no payoff and the game is over. If both players are right, the game is played over. But
if both players are wrong, the game is played over with the roles of the players reversed.
If the game never ends, the payoff is Q.
If we denote this game by G, then
G =
 2
3G + 1
3 (−GT ) 2
3 (1) + 1
3 (0)
2
3 (0) + 1
3 (4) 2
3 (−GT ) + 1
3G

If we let its value be denoted by v, then
v = Val  1
3 v 2
3 4
3 −1
3 v

If v ≥ 2, then there is a saddle at the upper right corner with v = 2/3. This contradiction
shows that v < 2 and there is no saddle. Therefore,
v = 8 + v2
18
or v2 − 18v +8=0.
This has a unique solution less than two,
v = 9 − √
73 = .456 ···
from which we may calculate the optimal strategy for I:
(
13 − √73
6 ,
√
73 − 7
6 )=(.743 ··· , .256 ···)
II – 69
and the optimal strategy for II:
(
11 − √
73
6 ,
√
73 − 5
6 )=(.409 ··· , .591 ···).
The value and optimal strategies are independent of Q.
6.5 Stochastic Games. If to the features of the games of the previous section is
added the possibility of a payoff at each stage until the game ends, the game is called
a Stochastic Game. This seems to be the proper level of generality for theoretical
treatment of multistage games. It is an area of intense contemporary research. See for
example the books of Filar and Vrieze (1997) and Maitra and Sudderth (1996). Stochastic
games were introduced by Shapley in (1953) in a beautiful paper that has been reprinted
in Raghavan et al. (1991), and more recently in Kuhn (1997). In this section, we present
Shapley’s main result.
A Stochastic Game, G, consists of a finite set of positions or states, {1, 2,...,N}, one
of which is specified as the starting position. We denote by G(k) the game in which k is the
starting position. Associated with each state, k, is a matrix game, A(k) = 
a(k)
ij 

. If the
stochastic game is in state k, the players simultaneously choose a row and column of A(k),
say i and j. As a result, two things happen. First, Player I wins the amount a(k)
ij from
Player II. Second, with probabilities that depend on i, j and k, the game either stops, or
it moves to another state (possibly the same one). The probability that the game stops is
denoted by s
(k)
ij , and the probability that the next state is  is denoted by P(k)
ij (), where
s
(k)
ij +

N
=1
P(k)
ij () = 1 (5)
for all i, j and k.
The payoffs accumulate throughout the game until it stops. To make sure the game
eventually stops, we make the assumption that all the stopping probabilities are positive.
Let s denote the smallest of these probabilities.
s = min
i,j,k
s
(k)
ij > 0 (6)
Under this assumption, the probability is one that the game ends in a finite number of
moves. This assumption also makes the expected accumulated payoff finite no matter how
the game is played, since if M denotes the largest of the absolute values of the payoffs,
M = maxi,j,k |a(k)
ij |, then the total expected payoff to either player is bounded by
M + (1 − s)M + (1 − s)
2M + ··· = M/s. (7)
Player I wishes to maximize the total accumulated payoff and Player II to minimize
it. We use a modification of the notation of the previous section to describe this game.
II – 70
G(k) =

a
(k)
ij +

N
=1
P(k)
ij ()G()

. (8)
Note that the probabilities in each component of this matrix sum to less than one. It is
understood that with the remaining probability, s
(k)
ij , the game ends. It should be noted
that in contrast to the previous section, a payoff does not end the game. After a payoff is
made, it is then decided at random whether the game ends and, if not, which state should
be played next.
Since no upper bound can be placed on the length of the game, this is an infinite
game. A strategy for a player must specify for every n how to choose an action if the game
reaches stage n. In general, theory does not guarantee a value. Moreover, the choice of
what to do at stage n may depend on what happened at all previous stages, so the space
of possible strategies is extremely complex.
Nevertheless, in stochastic games, the value and optimal strategies for the players
exist for every starting position. Moreover, optimal strategies exist that have a very simple
form. Strategies that prescribe for a player a probability distribution over his choices that
depends only on the game, Gk, being played and not on the stage n or past history are
called stationary strategies. The following theorem states that there exist stationary
optimal strategies.
Theorem 1. (Shapley (1952)) Each game G(k) has a value, v(k). These values are the
unique solution of the set of equations,
v(k) = Val 
a(k)
ij +

N
=1
P(k)
ij () v()

for k = 1,...,N. (9)
Each player has a stationary optimal strategy that in state k uses the optimal mixed
strategy for the game with matrix
A(k)
(v) = 
a(k)
ij +

N
=1
P(k)
ij () v()

(10)
where v represents the vector of values, v = (v(1),...,v(N)).
In equations (9), we see the same principle as in the earlier sections: the value of a
game is the value of the matrix game (8) with the games replaced by their values. A proof
of this theorem may be found in Appendix 2.
Example 8. As a very simple example, consider the following stochastic game with
one state, call it G.
G =
 1 + (3/5)G 3 + (1/5)G
1 + (4/5)G 2 + (2/5)G

II – 71
From Player II’s viewpoint, column 1 is better than column 2 in terms of immediate payoff,
but column 2 is more likely to end the game sooner than column 1, so that it should entail
smaller future payoffs. Which column should she choose?
Assume that all strategies are active, i.e. that the game does not have a saddle point.
We must check when we are finished to see if the assumption was correct. Then
v = Val  1 + (3/5)v 3 + (1/5)v
1 + (4/5)v 2 + (2/5)v

= (1 + (4/5)v)(3 + (1/5)v) − (1 + (3/5)v)(2 + (2/5)v)
1 + (4/5)v + 3 + (1/5)v − 1 − (3/5)v − 2 − (2/5)v
=1+ v − (2/25)v2
This leads to
(2/25)v2 = 1.
Solving this quadratic equation gives two possible solutions v = ±
25/2 = ±(5/2)√
2.
Since the value is obviously positive, we must use the plus sign. This is v = (5/2)√
2 =
3.535. If we put this value into the matrix above, it becomes
 1 + (3/2)√2 3 + (1/2)√2
1+2√2 2+ √2

The optimal strategy for Player I in this matrix is p = (√
2 − 1, 2 − √
2) = (.414, .586),
and the optimal strategy for Player II is q = (1 − √2/2,
√2/2) = (.293, .707). Since these
are probability vectors, our assumption is correct and these are the optimal strategies, and
v = (5/2)√2 is the value of the stochastic game.
6.6 Approximating the solution. For a general stochastic game with many states,
equations (9) become a rather complex system of simultaneous nonlinear equations. We
cannot hope to solve such systems in general. However, there is a simple iterative method
of approximating the solution. This is based on Shapley’s proof of Theorem 1, and is called
Shapley iteration.
First we make a guess at the solution, call it v0 = (v0(1),...,v0(N)). Any guess will
do. We may use all zero’s as the initial guess, v0 = 0 = (0,..., 0). Then given vn, we
define inductively, vn+1, by the equations,
vn+1(k) = Val 
a
(k)
ij +

N
=1
P(k)
ij () vn()

for k = 1,...,N. (11)
With v0 = 0, the vn(k) have an easily understood interpretation. vn(k) is the value of the
stochastic game starting in state k if there is forced stopping if the game reaches stage n.
In particular, v1(k) = Val(Ak) for all k.
The proof of Theorem 1 shows that vn(k) converges to the true value, v(k), of the
stochastic game starting at k. Two useful facts should be noted. First, the convergence
II – 72
is at an exponential rate: the maximum error goes down at least as fast as (1 − s)n.
(See Corollary 1 of Appendix 2.) Second, the maximum error at stage n + 1 is at most
the maximum change from stage n to n + 1 multiplied by (1 − s)/s. (See Corollary 2 of
Appendix 2.)
Let us take an example of a stochastic game with two positions. The corresponding
games G(1) and G(2), are related as follows.
G(1) =
 4 + .3G(1) 0 + .4G(2)
1 + .4G(2) 3 + .5G(1)
G(2) =
 0 + .5G(1) −5
−4 1+ .5G(2)
Using v0 = (0, 0) as the initial guess, we find v1 = (2, −2), since
v1(1) = Val  4 0
1 3
= 2 v1(2) = Val  0 −5
−4 1
= −2.
The next iteration gives
v2(1) = Val  4.6 −.8
.2 4
= 2.0174 v2(2) = Val  1 −5
−4 0
= −2.
Continuing, we find
v3(1) = 2.0210 v3(2) = −1.9983
v4(1) = 2.0220 v4(2) = −1.9977
v5(1) = 2.0224 v5(2) = −1.9974
v6(1) = 2.0225 v6(2) = −1.9974
The smallest stopping probability is .5, so the rate of convergence is at least (.5)n and the
maximum error of v6 is at most .0002.
The optimal strategies using v6 are easily found. For game G(1), the optimal strategies
are p(1) = (.4134, .5866) for Player I and q(1) = (.5219, .4718) for Player II. For game G(2),
the optimal strategies are p(2) = (.3996, .6004) for Player I and q(2) = (.4995, .5005) for
Player II.
6.7 Exercises
1.(a) Solve the system of games
G =
 0 G1
G2 G3

G1 =
 4 3
1 2
G2 =
 0 6
5 1
G3 =
 0 −2
−2 0
.
(b). Solve the games with matrices
(b1) (b2)
⎛
⎜⎝
0602
0305
5020
1040
⎞
⎟⎠
⎛
⎜⎜⎜⎝
31522
13522
44122
11163
11147
⎞
⎟⎟⎟⎠
II – 73
2. The Inspection Game. Let Gm,n denote the inspection game in which I is
allowed m inspections in the n time periods. (Thus, for 1 ≤ n ≤ m, Val(Gm,n) = 1, while
for n ≥ 1, Val(G0,n) = 0.) Find the iterative structure of the games and solve.
3. A Game of Endurance. II must count from n down to zero by subtracting
either one or two at each stage. I must guess at each stage whether II is going to subtract
one or two. If I ever guesses incorrectly at any stage, the game is over and there is no
payoff. Otherwise, if I guesses correctly at each stage, he wins 1 from II. Let Gn denote
this game, and use the initial conditions G0 = (1) and G1 = (1). Find the recursive
structure of the games and solve. (In the solution, you may use the notation Fn to denote
the Fibonacci sequence, 1, 1, 2, 3, 5, 8, 13,... , with definition F0 = 1, F1 = 1, and for
n ≥ 2, Fn = Fn−1 + Fn−2.)
4. Solve the sequence of games, G0, G1,..., where
G0 =
 3 2
1 G1

,...,Gn =
 n + 3 n + 2
n + 1 Gn+1
,...
Assume that if play continues forever, the payoff is zero.
5. (a) In the game “Guess it!”, G1,n, with m = 1 and arbitrary n, show that Player
I’s optimal strategy if to bluff with probability 1/(n + 2).
(b) Show that Player II’s optimal strategy in G1,n is to call the asked card with
probability V1,n, the value of G1,n.
6. Recursive Games. (a) Solve the game G =
 G 2
0 1
, Q.
(b) Solve the game G =
⎛
⎝
G 1 1
1 0 G
1 G 0
⎞
⎠ , Q.
7. Consider the following three related games.
G1 =
 G2 1
1 0
G2 =
 G3 0
0 2
G3 =
 G1 1
1 0
and suppose the payoff is Q if the games are played forever. Solve.
8. Consider the following three related games.
G1 =
⎛
⎝
G1 G2 G3
G2 G3 G1
G3 G1 G2
⎞
⎠ G2 =
 G1 0
0 2
G3 =
 G2 1
1 0
and suppose the payoff is Q if the games are played forever. Solve.
II – 74
9. There is one point to go in the match. The player that wins the last point while
serving wins the match. The server has two strategies, high and low. The receiver has
two strategies, near and far. The probability the server wins the point is given in the
accompanying table.
near far
high .8 .5
low .6 .7
If the server misses the point, the roles of the players are interchanged and the win probabilities
for given pure strategies are the same for the new server. Find optimal strategies
for server and receiver, and find the probability the server wins the match.
10. Player I tosses a coin with probability p of heads. For each k = 1, 2,..., if I tosses
k heads in a row he may stop and challenge II to toss the same number of heads; then II
tosses the coin and wins if and only if he tosses k heads in a row. If I tosses tails before
challenging II, then the game is repeated with the roles of the players reversed. If neither
player ever challenges, the game is a draw.
(a) Solve when p = 1/2.
(b) For arbitrary p, what are the optimal strategies of the players? Find the limit as p → 1
of the probability that I wins.
11. Solve the following stochastic game.
G =
 4 1 + (1/3)G
0 1 + (2/3)G

.
12. Consider the following stochastic game with two positions.
G(1) =
 2 2 + (1/2)G(2)
0 4 + (1/2)G(2)
G(2) =
 −4 0
−2 + (1/2)G(1) −4 + (1/2)G(1)
(a) Solve the equations (9) exactly for the values v(1) and v(2).
(b) Carry out Shapley iteration to find v2 starting with the initial guess v0 = (0, 0),
and compare with the exact values found in (a).
II – 75
7. Infinite Games.
In this Chapter, we treat infinite two-person, zero-sum games. These are games
(X, Y, A), in which at least one of the strategy sets, X and Y , is an infinite set. The
famous example of Exercise 4.7.3, he-who-chooses-the-larger-integer-wins, shows that an
infinite game may not have a value. Even worse, the example of Exercise 4.7.5 shows
that the notion of a value may not even make sense in infinite games without further
restrictions. This latter problem will be avoided when we assume that the function A(x, y)
is either bounded above or bounded below.
7.1 The Minimax Theorem for Semi-Finite Games. The minimax theorem for
finite games states that every finite game has a value and both players have optimal mixed
strategies. The first theorem below generalizes this result to the case where only one of
the players has a finite number of pure strategies. The conclusion is that the value exists
and the player with a finite number of pure strategies has an optimal mixed strategy. But
first we must discuss mixed strategies and near optimal strategies for infinite games.
Mixed Strategies for Infinite Games: First note that for infinite games, the notion
of a mixed strategy is somewhat open to choice. Suppose the strategy set, Y , of Player
II is infinite. The simplest choice of a mixed strategy is a finite distribution over Y .
This is a distribution that gives all its probability to a finite number of points. Such a
distribution is described by a finite number of points of Y , say y1, y2,...,yn, and a set of
probabilities, q1, q2,...,qn summing to one with the understanding that point yj is chosen
with probability qj . We will denote the set of finite distributions on Y by Y ∗
F .
When Y is an interval of the real line, we may allow as a mixed strategy any distribution
over Y given by its distribution function, F(z). Here, F(z) represents the probabiity
that the randomly chosen pure strategy, y, is less than or equal to z. The advantage of
enlarging the set of mixed strategies is that it then becomes more likely that an optimal
mixed strategy will exist. The payoff for using such a strategy is denoted by A(x, F) for
x ∈ X.
Near Optimal Strategies for Infinite Games: When a game has a finite value and
an optimal strategy for a player does not exist, that player must be content to choosing
a strategy that comes within  of achieving the value of the game for some small  > 0.
Such a strategy is called an -optimal strategy and was discussed in Chapter 6.
In infinite games, we allow the value to be +∞ or −∞. For example, the value is +∞
if for every number B, however large, there exists a mixed strategy p for Player I such
that A(p, y) ≥ B for all y ∈ Y . A simple example would be: X = [0,∞), Y arbitrary, and
A(x, y) = x independent of y. The value is +∞, since for any B, Player I can guarantee
winning at least B by choosing any x ≥ B. Such a strategy might be called B-optimal.
We will refer to both -optimal and B-optimal strategies as near optimal strategies.
The Semi-Finite Minimax Theorem. For finite X = {x1,...,xm}, we denote the set
of mixed strategies of Player I as usual by X∗. If Player I uses p ∈ X∗ and Player II uses
II – 76
q ∈ Y ∗
F , then the average payoff is denoted by
A(p, q) = 

i


j
piA(xi , yj )qj . (1)
We denote the set of mixed strategies of Player II by Y ∗, but we shall always assume that
Y ∗
F ⊂ Y ∗.
Consider the semi-finite two-person zero-sum game, (X, Y, A), in which X is a finite
set, Y is an arbitrary set, and A(x, y) is the payoff function — the winnings of Player I
if he chooses x ∈ X and Player II chooses y ∈ Y . To avoid the the possibility that the
average payoff does not exist or that the value might be −∞, we assume that the payoff
function, A(x, y), is bounded below. By bounded below, we mean that there is a number
M such that A(x, y) > M for all x ∈ X and all y ∈ Y . This assumption is weak from
the point of view of utility theory because, as mentioned in Appendix 1, it is customary
to assume that utility is bounded.
It is remarkable that the minimax theorem still holds in this situation. Specifically
the value exists and Player I has a minimax strategy. In addition, for every  > 0, Player
II has an -minimax strategy within Y ∗
F .
Theorem 7.1. If X is finite and A is bounded below, then the game (X, Y, A) has a finite
value and Player I has an optimal mixed strategy. In addition, if X has m elements, then
Player II has near optimal strategies that give weight to at most m points of Y .
This theorem is valid without the asumption that A is bounded below provided Player
II is restricted to finite strategies, i.e. Y ∗ = Y ∗
F . See Exercise 1. However, the value may
be −∞, and the notion of near optimal strategies must be extended to this case.
By symmetry, if Y is finite and A is bounded above, then the game (X, Y, A) has a
value and Player II has an optimal mixed strategy.
Solving Semi-Finite Games. Here are two methods that may be used to solve semi-
finite games. We take X to be the finite set, X = {x1,...,xm}.
METHOD 1. The first method is similar to the method used to solve 2 × n games
presented in Section 2.2. For each fixed y ∈ Y , the payoff, A(p, y), is a linear function of p
on the set X∗ = {p = (p1,...,pm) : pi ≥ 0,
pi = 1}. The optimal strategy for Player I is
that value of p that maximizes the lower envelope, f(p) ≡ infy∈Y A(p, y). Note that f(p),
being the infimum of a collection of concave continuous (here linear) functions, is concave
and continuous on X∗. Since X∗ is compact, there exists a p at which the maximum of f(p)
is attained. General methods for solving concave maximization problems are available.
Example 1. Player I chooses x ∈ {x1, x2}, Player II chooses y ∈ [0, 1], and the payoff
is
A(x, y) =  y if x = x1
(1 − y)2 if x = x2
II – 77
A(p, y)
0 1 .553
0
1
p
y=0
y=1/4
y=.382
y=1/2
y=2/3
y=1
Figure 7.1
Let p denote the probability that Player I chooses x1 and let p = (p, 1−p). For a given
choice of y ∈ Y by Player II, the expected payoff to Player I is A(p, y) = py+(1−p)(1−y)2.
The minimum of A(p, y) over y occurs at p − (1 − p)2(1 − y) = 0, or y = (2 − 3p)/(2 − 2p);
except that for p > 2/3, the minimum occurs at y = 0. So, the lower envelope is
f(p) = min
y
A(p, y) =  p 4−5p
4−4p
if p ≤ 2/3
1 − p if p ≥ 2/3
The maximum of this function occurs for p ≤ 2/3, and is easily found to be p = 1 −
(1/
√5) = .553 .... The optimal strategy for Player II occurs at that value of y for which
the slope of A(p, y) (as a function of p) is zero. This occurs when y = (1 − y)2. We find
y = (3 − √5)/2 = .382 ... is an optimal pure strategy for Player II. This is also the value
of the game. See Figure 7.1, in which the lower envelope is shown as the thick line.
METHOD 2: S-GAMES. (Blackwell and Girshick (1954).) Let X = {1, 2,...,m},
and let S be a non-empty convex subset of m-dimensional Euclidean space, Rm, and
assume that S is bounded below. Player II chooses a point s = (s1,...,sm) in S, and
simultaneously Player I chooses a coordinate i ∈ X. Then Player II pays si to Player I.
Such games are called S-games.
This game arises from the semi-finite game (X, Y, A) with X = {1, 2,...,m} by letting
S0 = {s = (A(1, y),...,A(m, y)) : y ∈ Y }. Choosing y ∈ Y is equivalent to choosing
s ∈ S0. Although S0 is not necessarily convex, Player II can, by using a mixed strategy,
choose a probability mixture of points in S0. This is equivalent to choosing a point s in
S, where S is the convex hull of S0.
To solve the game, let Wc denote the “wedge” at the point (c, . . . , c) on the diagonal
in Rm,
Wc = {s : si ≤ c for all i = 1,...,m}.
II – 78
Start with some c such that the wedge Wc contains no points of s, i.e. Wc ∩ S = ∅. Such
a value of c exists from the assumption that S is bounded below. Now increase c and so
push the wedge up to the right until it just touches S. See Figure 7.2(a). This gives the
value of the game:
v = sup{c : Wc ∩ S = ∅}.
Any point s ∈ Wv ∩ S is an optimal pure strategy for Player II. It guarantees that II will
lose no more than v. Such a strategy will exist if S is closed. If Wv ∩ S is empty, then
any point s ∈ Wv+ ∩ S is an -optimal strategy for Player II. The point (v,... , v) is not
necessarily an optimal strategy for Player II. The optimal strategy could be on the side of
the wedge as in Figure 7.2(b).
S
Wv
(v,v)
(0,0)
W S v
(v,v)
(0,0)
Figure 7.2(a) Figure 7.2(b)
To find an optimal strategy for Player I, first find a plane that separates Wv and S,
i.e. that keeps Wv on one side and S on the other. Then find the vector perpendicular to
this plane, i.e. the normal vector. The optimal strategy of Player I is the mixed strategy
with components proportional to this normal vector.
Example 2. Let S be the set S = {(y1, y2) : y1 ≥ 0, y2 ≥ (1−y1)2}. This is essentially
the same as Example 1. The wedge first hits S at the vertex (v, v) when v = (1−v)2. The
solution to this equation gives the value, v = (3 − √5)/2. The point (v, v) is optimal for
Player II. To find Player I’s optimal strategy, we find the slope of the curve y2 = (1−y1)2 at
the point (v, v). The slope of the curve is −2(1−y1), which at y1 = v is −2(1−v)=1−√5.
The slope of the normal is the negative of the reciprocal of this, namely 1/(
√5 − 1). So
p2/p1 = 1/(
√5 − 1), and since p1 + p2 = 1, we find p2(
√5 − 1) = 1 − p2, or p2 = 1/
√5 and
p1 = 1 − (1/
√
5). as found in Example 1.
7.2 Continuous Games. The simplest extension of the minimax theorem to a more
general case is to assume that X and Y are compact subsets of Euclidean spaces, and that
A(x, y) is a continuous function of x and y. To conclude that optimal strategies for the
players exist, we must allow arbitrary distribution functions on X and Y . Thus if X is a
compact subset of m-dimensional space Rm, X∗ is taken to be the set of all distributions
II – 79
on Rm that give probability 0 to the complement of X. Similarly if Y is n-dimensional,
Y ∗ is taken to be the set of all distributions on Rn giving weight 0 to the complement of
Y . Then A is extended to be defined on X∗ × Y ∗ by
A(P, Q) =   A(x, y) dP(x) dQ(y)
Theorem 7.2. If X and Y are compact subsets of Euclidean space and if A(x, y) is a
continuous function of x and y, then the game has a value, v, and there exist optimal
strategies for the players; that is, there is a P0 ∈ X∗ and a Q0 ∈ Y ∗ such that
A(P, Q0) ≤ v ≤ A(P0, Q) for all P ∈ X∗ and Q ∈ Y ∗.
Example 1. Consider the game (X, Y, A), where X = Y = [0, 1], the unit interval,
and
A(x, y) =  g(x − y) if 0 ≤ y ≤ x
g(1 + x − y) if 0 ≤ x<y ≤ 1,
where g is a continuous function defined on [0, 1], with g(0) = g(1). Here, both X∗ and
Y ∗ are the set of probability distributions on the unit interval.
Since X and Y are compact and A(x, y) is continuous on [0, 1]2, we have by Theorem
7.2, that the game has a value and the players have optimal strategies. Let us check that
the optimal strategies for both players is the uniform distribution on [0, 1]. If Player I uses
a uniform on [0,1] to choose x and Player II uses the pure strategy y ∈ [0, 1], the expected
payoff to Player I is
 1
0
A(x, y) dx =
 y
0
g(1 + x − y) dx +
 1
y
g(x − y) dx
=
 1
1−y
g(u) du +
 1−y
0
g(u) du =
 1
0
g(u) du
Since this is independent of y, Player I’s strategy is an equalizer strategy, guaranteeing
him an average payoff of  1
0 g(u) du. Clearly, the same analysis gives Player II this same
amount if he chooses y at random according to a uniform distribution on [0,1]. So these
strategies are optimal and the value is v =  1
0 g(u) du. It may be noticed that this example
is a continuous version of a Latin square game. In fact the same solution holds even if g
in not continuous. One only needs g to be integrable on [0, 1].
A One-Sided Minimax Theorem. In the way that Theorem 7.1 generalized the
finite minimax theorem by allowing Y to be an arbitrary set, Theorem 7.2 may be generalized
to allow Y to be arbitrary, provided we keep the compactness condition on X.
The continuity condition may be weakened to assuming only that A(x, y) is a continuous
function of x for every y ∈ Y . And even this can be weakened to assuming that A(x, y) is
only upper semi-continuous in x for every y ∈ Y .
II – 80
A function f(x) defined on X is upper semi-continuous at a point x0 ∈ X, if for any
sequence x1, x2,... of points in X such that limn→∞ xn = x0, we have limn→∞ f(xn) ≤
f(x0). It is upper semi-continuous (usc) on X if it is upper semicontinuous at every point
of X. A function f(x) is lower semi-continuous (lsc) if the above inequality is changed to
limn→∞ f(xn) ≥ f(x0), or equivalently, if the function, −f(x), is upper semi-continuous.
As an example, the function
f(x) =  0 if x < 0
a if x = 0
1 if x > 0
is usc if a ≥ 1 and lsc if a ≤ 0. It is neither usc nor lsc if 0 <a< 1.
Theorem 7.3. If X is a compact subset of Euclidean space, and if A(x, y) is an upper
semi-continuous function of x ∈ X for all y ∈ Y and if A is bounded below (or if Y ∗ is the
set of finite mixtures), then the game has a value, Player I has an optimal strategy in X∗,
and for every  > 0 Player II has an -optimal strategy giving weight to a finite number of
points.
Similarly from Player II’s viewpoint, if Y is a compact subset of Euclidean space, and
if A(x, y) is a lower semi-continuous function of y ∈ Y for all x ∈ X and if A is bounded
above (or if X∗ is the set of finite mixtures), then the game has a value and Player II has
an optimal strategy in Y ∗.
Example 2. Player I chooses a number in [0,1] and Player II tries to guess what it
is. Player I wins 1 if Player II’s guess is off by at least 1/3; otherwise, there is no payoff.
Thus, X = Y = [0, 1], and A(x, y) =  1 if |x − y| ≥ 1/3
0 if |x − y| < 1/3. Although the payoff
function is not continuous, it is upper semi-continuous in x for every y ∈ Y . Thus the
game has a value and Player I has an optimal mixed strategy.
If we change the payoff so that Player I wins 1 if Player II’s guess is off by more than
1/3, then A(x, y) =  1 if |x − y| > 1/3
0 if |x − y| ≤ 1/3. This is no longer upper semi-continuous in x
for fixed y; instead it is lower semi-continuous in y for each x ∈ X. This time, the game
has a value and Player II has an optimal mixed strategy.
7.3 Concave Games and Convex Games. If in Theorem 7.2, we add the assumption
that the payoff function A(x, y) is concave in x for all y or convex in y for all x, then
we can conclude that one of the players has an optimal pure strategy, which is usually easy
to find. Here is a one-sided version that complements Theorem 7.3. A good reference for
these ideas is the book of Karlin (1959), vol. 2.
Theorem 7.4. Let (X, Y, A) be a game with Y arbitrary, X a compact, convex subset of
Rm, and A(x, y) bounded below. If A(x, y) is a concave function of x ∈ X for all y ∈ Y ,
then the game has a value and Player I has an optimal pure strategy. Moreover, Player II
has an -optimal strategy that is a mixture of at most m + 1 pure strategies.
II – 81
The dual statement for convex functions is: If Y is compact and convex in Rn, and
if A is bounded above and is convex in y ∈ Y for all x ∈ X, then the game has a value,
Player II has an optimal pure strategy and Player I has -optimal strategies giving weight
to at most n + 1 points..
These games may be solved by a method similar to Method 1 of Section 7.1. Let’s
see how to find the optimal strategy of Player II in the convex functio case. Let g(y) =
supx A(x, y) be the upper envelope. Then g(y) is finite since A is bounded above. It is also
convex since the supremum of any set of convex functions is convex. Then since convex
functions defined on a compact set attain their maximum, there exists a point y∗ at which
g(y) takes on its maximum value, so that
A(x, y∗) ≤ max x A(x, y∗) = g(y∗) for all x ∈ X.
Any such point is an optimal pure strategy for Player II. By choosing y∗, Player II will
lose no more than g(y∗) no matter what Player I does. Player I’s optimal strategy is
more complex to describe in general; it gives weight only to points that play a role in the
upper envelope at the point y∗. These are points x such that A(x, y) is tangent (or nearly
tangent if only -optimal strategies exist) to the surface g(y) at y∗. It is best to consider
examples.
Example 1. Estimation. Player I chooses a point x ∈ X = [0, 1], and Player II
tries to choose a point y ∈ Y = [0, 1] close to x. Player II loses the square of the distance
from x to y: A(x, y)=(x − y)2. This is a convex function of y ∈ [0, 1] for all x ∈ X.
For any x, A(x, y) is bounded above by either A(0, y) or A(1, y) so the upper envelope
is g(y) = max{A(0, y), A(1, y)} = max{y2,(1 − y)2}. This is minimized at y∗ = 1/2. If
Player II uses y∗, she is guaranteed to lose no more than g(y∗)=1/4.
Since x = 0 and x = 1 are the only two pure strategies influencing the upper envelope,
and since y2 and (1−y)2 have slopes at y∗ that are equal in absolute value but opposite in
sign, Player I should mix 0 and 1 with equal probability. This mixed strategy has convex
payoff (1/2)(A(0, y) + A(1, y)) with slope zero at y∗. Player I is guaranteed winning at
least 1/4, so v = 1/4 is the value of the game. The pure strategy y∗ is optimal for Player
II and the mixed strategy, 0 with probability 1/2 and 1 with probability 1/2, is optimal for
Player I. In this example, n = 1, and Player I’s optimal strategy mixes 2 = n + 1 points.
Theorem 7.4 may also be stated with the roles of the players reversed. If Y is arbitrary,
and if X is a compact subset of Rm and if A(x, y) is bounded below and concave in x ∈ X
for all y ∈ Y , then Player I has an optimal pure strategy, and Player II has an -optimal
strategy mixing at most m +1 pure strategies. It may also happen that A(x, y) is concave
in x for all y, and convex in y for all x. In that case, both players have optimal pure
strategies as in the following example.
Example 2. A Convex-Concave Game. Suppose X = Y = [0, 1], and A(x, y) =
−2x2 +4xy+y2 −2x−3y+1. The payoff is convex in y for all x and concave in x for all y.
Therefore, both players have pure optimal strategies, say x0 and y0. If Player II uses y0,
II – 82
then A(x, y0) must be maximized by x0. To find maxx∈[0,1] A(x, y0) we take a derivative
with respect to x: ∂
∂xA(x, y0) = −4x + 4y0 − 2. So
x0 =
 y0 − (1/2) if y0 > 1/2
0 if y0 ≤ 1/2
Similarly, if Player I uses x0, then A(x0, y) is minimized by y0. Since ∂
∂y A(x0, y)=4x0 +
2y − 3, we have
y0 =
⎧
⎨
⎩
1 if x0 ≤ 1/4
(1/2)(3 − 4x0) if 1/4 ≤ x0 ≤ 3/4
0 if x0 ≥ 3/4.
These two equations are satisfied only if x0 = y0 − (1/2) and y0 = (1/2)(3 − 4x0). It is
then easily found that x0 = 1/3 and y0 = 5/6. The value is A(x0, y0) = −7/12.
It may be easier here to find the saddle-point of the surface, z = −2x2 + 4xy + y2 −
2x − 3y + 1, and if the saddle-point is in the unit square, then that is the solution. But
the method used here shows what must be done in general.
7.4 Solving Games. There are many interesting games that are more complex and
that require a good deal of thought and ingenuity to find solutions. There is one tool
for solving such games that is basic. This is the infinite game analog of the principle of
indifference given in Chapter 3: Search for strategies that make the opponent indifferent
among all his “good” pure strategies.
To be more specific, consider the game (X, Y, A) with X = Y = [0, 1] and A(x, y)
continuous. Let v denote the value of the game and let P denote the distribution that
represents the optimal strategy for Player I. Then, A(P, y) must be equal to v for all
“good” y, which here means for all y in the support of Q for any Q that is optimal for
Player II. (A point y is in the support of Q if the Q probability of the interval (y −, y +)
is positive for all  > 0.) So to attempt to find the optimal P, we guess at the set, S, of
“good” points y for Player II and search for a distribution P such that A(P, y) is constant
on S. Such a strategy, P, is called an equalizer strategy on S. The first example shows
what is involved in this.
Example 1. Meeting Someone at the Train Station. A young lady is due to
arrive at a train station at some random time, T, distributed uniformly between noon and
1 PM. She is to wait there until one of her two suitors arrives to pick her up. Each suitor
chooses a time in [0,1] to arrive. If he finds the young lady there, he departs immediately
with her; otherwise, he leaves immediately, disappointed. If either suitor is successful in
meeting the young lady, he receives 1 unit from the other. If they choose the same time
to arrive, there is no payoff. Also, if they both arrive before the young lady arrives, the
payoff is zero. (She takes a taxi at 1 PM.)
Solution: Denote the suitors by I and II, and their strategy spaces by X = [0, 1]
and Y = [0, 1]. Let us find the function A(x, y) that represents I’s expected winnings if I
chooses x ∈ X and II chooses y ∈ Y . If x<y, I wins 1 if T <x and loses 1 if x<T <y.
II – 83
The probability of the first is x and the probability of the second is y − x, so A(x, y) is
x − (y − x)=2x − y when x<y. When y<x, a similar analysis shows A(x, y) = x − 2y,
Thus,
A(x, y) =  2x − y if x<y
x − 2y if x>y
0 if x = y.
(1)
This payoff function is not continuous, nor is it upper semicontinuous or lower semicontinuous.
It is symmetric in the players so if it has a value, the value is zero and the players
have the same optimal strategy.
Let us search for an equalizer strategy for Player I and assume it has a density f(x)
on [0,1]. We would have
A(f,y) =  y
0
(2x − y)f(x) dx +
 1
y
(x − 2y)f(x) dx
=
 y
0
(x + y)f(x) dx +
 1
0
(x − 2y)f(x) dx = constant
(2)
Taking a derivative with respect to y yields the equation
2yf(y) +  y
0
f(x) dx − 2
 1
0
f(x) dx = 0 (3)
and taking a second derivative gives
2f(y)+2yf
(y) + f(y) = 0 or f
(y)
f(y) = − 3
2y
. (4)
This differential equation has the simple solution,
log f(y) = −3
2 log(y) + c or f(y) = ky−3/2 (5)
for some constants c and k. Unfortunately,  1
0 y−3/2 dy = ∞, so this cannot be used as a
density on [0,1].
If we think more about the problem, we can see that it cannot be good to come in very
early. There is too little chance that the young lady has arrived. So perhaps the “good”
points are only those from some point a > 0 on. That is, we should look for a density
f(x) on [a, 1] that is an equalizer from a on. So in (2) we replace the integrals from 0 to
integrals from a and assume y>a. The derivative with respect to y gives (3) with the
integrals starting from a rather than 0. And the second derivative is (4) exactly. We have
the same solution (5) but for y>a. This time the resulting f(y) on [a, 1] is a density if
k−1 =
 1
a
x−3/2 dx = −2
 1
a
dx−1/2 = 2(1 − √a)
√a (6)
II – 84
We now need to find a. That may be done by solving equation (3) with the integrals
starting at a.
2yky−3/2 +
 y
a
kx−3/2 dx − 2=2ky−1/2 − 2k(y−1/2 − a−1/2) − 2=2ka−1/2 − 2=0
So ka−1/2 = 1, which implies 1 = 2(1 − √a) or a = 1/4, which in turn implies k = 1/2.
The density
f(x) =  0 if 0 <x< 1/4
(1/2)x−3/2 if 1/4 <x< 1 (7)
is an equalizer for y > 1/4 and is therefore a good candidate for the optimal strategy. We
should still check at points y less than 1/4. For y < 1/4, we have from (2) and (7)
A(f,y) =  1
1/4
(x − 2y)(1/2)x−3/2 dx =
 1
1/4
1
2
√x
dx − 2y = 1
2 − 2y.
So
A(f,y) =  (1 − 4y)/2 for y < 1/4
0 for y > 1/4 (8)
This guarantees I at least 0 no matter what II does. Since II can use the same strategy,
The value of the game is 0 and (7) is an optimal strategy for both players.
Example 2. Competing Investors. Two investors compete to see which of them,
starting with the same initial fortune, can end up with the larger fortune. The rules of
the competition require that they invest only in fair games. That is, they can only invest
non-negative amounts in games whose expected return per unit invested is 1.
Suppose the investors start with 1 unit of fortune each (and we assume money is
infinitely divisible). Thus no matter what they do, their expected fortune at the end is
equal to their initial fortune, 1.
Thus the players have the same pure strategy sets. They both choose a distribution
on [0,∞) with mean 1, say Player I chooses F with mean 1, and Player II chooses G with
mean 1. Then Z1 is chosen from F and Z2 is chosen from G independently, and I wins
if Z1 > Z2, II wins if Z2 > Z1 and it is a tie if Z1 = Z2. What distributions should the
investors choose?
The game is symmetric in the players, so the value if it exists is zero, and both players
have the same optimal strategy. Here the strategy spaces are very large, much larger than
in the Euclidean case. But it turns out that the solution is easy to describe. The optimal
strategy for both players is the uniform distribution on the interval (0,2):
F(z) =  z/2 for 0 ≤ z ≤ 2
1 for z > 2
This is a distribution on [0,∞] with mean 1 and so it is an element of the strategy space
of both players. Suppose Player I uses F. Then the probability that I loses is
P(Z1 < Z2) = E[P(Z1 < Z2|Z2)] ≤ E[Z2/2] = (1/2)E[Z2]=1/2.
II – 85
So the probability I wins is at least 1/2. Since the game is symmetric, Player II by using
the same strategy can keep Player I’s probability of winning to at most 1/2.
7.5 Uniform[0,1] Poker Models. The study of two-person Uniform[0,1] poker
models goes back to Borel (1938) and von Neumann (1944). We present these two models
here. In these models, the set of possible “hands” of the players is the interval, [0, 1].
Players I and II are dealt hands x and y respectively in [0, 1] according to a uniform
distribution over the interval [0, 1]. Throughout the play, both players know the value of
their own hand, but not that of the opponent. We assume that x and y are independent
random variables; that is, learning the value of his own hand gives a player no information
about the hand of his opponent.
There follows some rounds of betting in which the players take turns acting. After
the dealing of the hands, all actions that the players take are announced. Except for the
dealing of the hands at the start of the game, this would be a game of perfect information.
Games of this sort, where, after an initial random move giving secret information to the
players, the game is played with no further random moves of nature, are called games of
almost perfect information (See Sorin and Ponssard (1980).
It is convenient to study the action part of games of almost complete information by
what we call the betting tree. This is distinct from the Kuhn tree in that it neglects the
information sets that may arise from the initial distribution of hands. The examples below
illustrate this concept.
The Borel Model: La Relance. Both players contribute an ante of 1 unit into the
pot and receive independent uniform hands on the interval [0, 1]. Player I acts first either
by folding and thus conceding the pot to Player II, or by betting a prescribed amount
β > 0 which he adds to the pot. If Player I bets, then Player II acts either by folding and
thus conceding the pot to Player I, or by calling and adding β to the pot. If Player II calls
the bet of Player I, the hands are compared and the player with the higher hand wins the
entire pot. That is, if x>y then Player I wins the pot; if x<y then Player II wins the
pot. We do not have to consider the case x = y since this occurs with probability 0.
The betting tree is
I
II
bet fold
call fold
+1
−1
±(β +1)
In this diagram, the plus-or-minus sign indicates that the hands are compared, and
the higher hand wins the amount β + 1.
II – 86
It is easy to see that the optimal strategy for Player II must be of the form for some
number b in the interval [0,1]: fold if y<b and call if y>b. The optimal value of b may be
found using the principle of indifference. Player II chooses b to make I indifferent between
betting and folding when I has some hand x<b. If I bets with such an x, he, wins 2 (the
pot) if II has y<b and loses β if II has y>b. His expected winnings are in this case,
2b − β(1 − b). On the other hand, if I folds he wins nothing. (This views the game as a
constant-sum game. It views the money already put into the pot as a sunk cost, and so
the sum of the payoffs of the players is 2 whatever the outcome. This is a minor point but
it is the way most poker players view the pot.) He will be indifferent between betting and
folding if
2b − β(1 − b)=0
from which we conclude
b = β/(2 + β). (1)
Player I’s optimal strategy is not unique, but all of his optimal strategies are of the
form: if x>b, bet; and if x<b, do anything provided the total probability that you fold
is b2. For example, I may fold with his worst hands, i.e. with x<b2, or he may fold with
the best of his hands less than b, i.e. with b − b2 <x<b, or he may, for all 0 <x<b,
simply toss a coin with probability b of heads and fold if the coin comes up heads.
The value of the game may be computed as follows. Suppose Player I folds with any
x<b2 and bets otherwise and suppose Player II folds with y<b. Then the payoff in the
unit square has the values given in the following diagram. The values in the upper right
corner cancel and the rest is easy to evaluate. The value is v(β) = −(β + 1)(1 − b)(b −
b2) + (1 − b2)b − b2, or, recalling b = β/(2 + β),
v(β) = −b2 = − β2
(2 + β)2 . (2)
Thus, the game is in favor of Player II.
0
0
0
0
1
1
y
x
b
b b 2
−1
+1
−(β+1)
−(β+1)
β+1
II – 87
We summarize in
Theorem 7.5. The value of la relance is given by (2). An optimal strategy for Player I
is to bet if x>b − b2 and to fold otherwise, where b is given in (1). An optimal strategy
for Player II is to call if y>b and to fold otherwise.
As an example, suppose β = 2, where the size of the bet is the size of the pot. Then
b = 1/2. An optimal strategy for Player I is to bet if x > 1/4 and fold otherwise; the
optimal strategy of Player II is to call if y > 1/2. The game favors Player II, whose
expected return is 1/4 unit each time the game is played.
If I bets when x<b, he knows he will lose if called, assuming II is using an optimal
strategy. Such a bet is called a bluff. In la relance, it is necessary for I to bluff with
probability b2. Which of the hands below b he chooses to bluff with is immaterial as far as
the value of the game is concerned. However, there is a secondary advantage to bluffing
(betting) with the hands just below b, that is, with the hands from b2 to b. Such a strategy
takes maximum advantage of a mistake the other player may make.
A given strategy σ for a player is called a mistake if there exists an optimal strategy
for the opponent when used against σ gives the opponent an expected payoff better than
the value of the game. In la relance, it is a mistake for Player II to call with some y<b
or to fold with some y>b. If II calls with some y<b, then I can gain from the mistake
most profitably if he bluffs only with his best hands below b.
A strategy is said to be admissible for a player if no other strategy for that player
does better against one strategy of the opponent without doing worse against some other
strategy of the opponent. The rule of betting if and only if x>b2 is the unique admissible
optimal strategy for Player I.
The von Neumann Model. The model of von Neumann differs from the model of
Borel in one small but significant respect. If Player I does not bet, he does not necessarily
lose the pot. Instead the hands are immediately compared and the higher hand wins the
pot. We say Player I checks rather than folds. This provides a better approximation to
real poker and a clearer example of the concept of “bluffing” in poker. The betting tree of
von Neumann’s poker is the same as Borel’s except that the −1 payoff on the right branch
is changed to ±1.
I
II
bet check
call fold
+1
±1
±(β +1)
II – 88
This time it is Player I that has a unique optimal strategy. It is of the form for some
numbers a and b with a<b: bet if x<a or if x>b, and check otherwise. Although there
are many optimal strategies for Player II (and von Neumann finds all of them), one can
show that there is a unique admissible one and it has the simple form: call if y>c for
some number c. It turns out that 0 <a<c<b< 1.
I: | bet | check | bet |
0a b 1
II: | fold | call |
0 c1
The region x<a is the region in which Player I bluffs. It is noteworthy that Player
I must bluff with his worst hands, and not with his moderate hands. It is a mistake for
Player I to do otherwise. Here is a rough explanation of this somewhat counterintuitive
feature. Hands below c may be used for bluffing or checking. For bluffing it doesn’t matter
much which hands are used; one expects to lose them if called. For checking though it
certainly matters; one is better off checking with the better hands.
Let us apply the principle of indifference to find the optimal values of a, b and c. This
will lead to three equations in three unknowns, known as the indifference equations (not
to be confused with difference equations). First, Player II should be indifferent between
folding and calling with a hand y = c. Again we use the gambler’s point of view of the
game as a constant sum game, where winning what is already in the pot is considered as
a bonus. If II folds, she wins zero. If she calls with y = c, she wins (β + 2) if x<a and
loses β if x>b. Equating her expected winnings gives the first indifference equation,
(β + 2)a − β(1 − b)=0. (3)
Second, Player I should be indifferent between checking and betting with x = a. If
he checks with x = a, he wins 2 if y<a, and wins nothing otherwise, for an expected
return of 2a. If he bets, he wins 2 if y<c and loses β if y>c, for an expected return of
2c − β(1 − c). Equating these gives the second indifference equation,
2c − β(1 − c)=2a. (4)
Third, Player I should be indifferent between checking and betting with x = b. If he
checks, he wins 2 if y<b. If he bets, he wins 2 if y<c and wins β + 2 if c<y<b, and
loses β if y>b, for an expected return of 2c + (β + 2)(b − c) − β(1 − b). This gives the
third indifference equation,
2c + (β + 2)(b − c) − β(1 − b)=2b,
which reduces to
2b − c = 1. (5)
II – 89
The optimal values of a, b and c can be found by solving equations (4) (5) and (6) in
terms of β. The solution is
a = β
(β + 1)(β + 4) b = β2 + 4β + 2
(β + 1)(β + 4) c = β(β + 3)
(β + 1)(β + 4). (6)
The value is
v(β) = a = β/((β + 1)(β + 4)). (7)
This game favors Player I. We summarize this in
Theorem 7.6. The value of von Neumann’s poker is given by (7). An optimal strategy
for Player I is to check if a<x<b and to bet otherwise, where a and b are given in (6).
An optimal strategy for Player II is to call if y>c and to fold otherwise, where c is given
in (6).
For pot-limit poker where β = 2, we have a = 1/9, b = 7/9, and c = 5/9, and the
value is v(2) = 1/9.
It is interesting to note that there is an optimal bet size for Player I. It may be found
by setting the derivative of v(β) to zero and solving the resulting equation for β. It is
β = 2. In other words, the optimal bet size is the size of the pot, as in pot-limit poker!
7.6 Exercises.
1. Let X = {−1, 1}, let Y = {..., −2, −1, 0, 1, 2,...} be the set of all integers, and let
A(x, y) = xy.
(a) Show that if we take Y ∗ = Y ∗
F , the set of all finite distributions on Y , then the
value exists, is equal to zero and both players have optimal strategies.
(b) Show that if Y ∗ is taken to be the set of all distributions on Y , then we can’t speak
of the value, because Player II has a strategy, q, for which the expected payoff, A(x, q)
doesn’t exist for any x ∈ X.
2. Simultaneously, Player I chooses x ∈ {x1, x2}, and Player II chooses y ∈ [0, 1]; then
I receives
A(x, y) =  y if x = x1
e−y if x = x2
from II. Find the value and optimal strategies for the players.
3. Player II chooses a point (y1, y2) in the ellipse (y1 − 3)2 + 4(y2 − 2)2 ≤ 4. Simultaneously,
Player I chooses a coordinate k ∈ {1, 2} and receives yk from Player II. Find the
value and optimal strategies for the players.
4. Solve the two games of Example 2. Hint: Use domination to remove some pure
strategies.
II – 90
5. Consider the game with X = [0, 1], Y = [0, 1], and
A(x, y) =
⎧
⎪⎪⎪⎨
⎪⎪⎪⎩
0 if x = y
−1 if x = 0 and y > 0
+1 if y = 0 and x > 0
−1 if 0 <y<x
+1 if 0 <x<y
Note that A(x, y) is not usc in x for all y nor lsc in y for all x. Show the game does not
have a value.
6. The Greedy Game. Each player can demand from the other as much as desired
between zero and one, but there is a penalty for being too greedy. The player who demands
more than his opponent must pay a fine of b to the other, where b is a fixed number,
0 ≤ b ≤ 1/2. Thus we have the game (X, Y, A) where X = Y = [0, 1], and
A(x, y) = x − y +
 +b if x<y
0 if x = y
−b if x>y
Solve.
7. Find optimal strategies and the value of the following games.
(a) X = Y = [0, 1] and A(x, y) =  (x − y)2 if x ≤ y
2(x − y)2 if x ≥ y. (Underestimation is the more
serious error of Player II.)
(b) X = Y = [0, 1] and A(x, y) = xe−y + (1 − x)y.
8. Hide and Seek in a Compact, Convex Set. (a) Let S be the triangle in the
plane with vertices (−1, 0), (1, 0), and (0, 2). Player I chooses a point x in S in which to
hide, and Player II chooses a point y in S to seek. The payoff to Player I is the square
of Euclidean distance between x and y. Thus, X = S, Y = S, and A(x, y) = x − y2.
Solve.
(b) See if you can formulate a procedure for solving the above game of Hide and Seek
if S is an arbitrary compact, convex set in Rn.
9. The Wallet Game. Two players each put a random amount with mean one into
their wallets. The player whose wallet contains the smaller amount wins the larger amount
from the opponent.
Carroll, Jones and Rykken (2001) show that this game does not have a value. But
suppose we restrict the players to putting at most some amount b in their wallets. Here is
the game:
Player I, resp. Player II, chooses a distribution F, resp. G, on the interval [0, b] with
mean 1, where b > 1. Then independent random variables, X from F and Y from G, are
II – 91
chosen. If X<Y , Player I wins Y from Player II. If X>Y , Player II wins X from Player
I, and if X = Y , there is no payoff. So the payoff function is
A(F, G) = E(Y I(X<Y ) − XI(X>Y ))
= E((Y + X)I(X<Y )) − 1 + E(XI(X = Y )). (1)
The game is symmetric, so if the value exists, the value is zero, and the players have the
same optimal strategies. Find an optimal strategy for the players. Hint: Search among
distributions F having a density f on the interval (a, b) for some a < 1. Note that the last
term on the right of Equation (1) disappears for such distributions.
10. The Multiplication Game. (See Kent Morrison (2010).) Players I and II
simutaneously select positive numbers x and y. Player I wins +1 if the product xy, written
in decimal form has initial significant digit 1, 2 or 3. Thus, the pure strategy spaces are
X = Y = (0,∞) and the payoff function is
A(x, y) =  +1 if the initial significant digit is 1, 2 or 3
0 otherwise.
Solve.
Hint:(1) First note that both players may restrict their pure strategy sets to X =
Y = [1, 10). so that
A(x, y) = I{1 ≤ xy < 4 or 10 ≤ xy < 40}.
(2) Take logs to the base 10. Let u = log10(x) and v = log10(y). Now, players I and
II choose u and v in [0,1) with payoff
B(u, v) = I{0 ≤ u + v<c or 1 ≤ u + v < 1 + c}
where c = log10(4) = .60206 .... Solve the game in this form and translate back to the
original game.
11. Suppose, in La Relance, that when Player I checks, Player II is given a choice
between checking in which case there is no payoff, and calling in which case the hands are
compared and the higher hand wins the antes.
(a) Draw the betting tree.
(b) Assume optimal strategies of the following form. I checks if and only if 0 <a<
x<b< 1 for some a and b. If I bets, then II calls iff y>c, and if Player I checks, Player
II calls iff y>d, where a ≤ c ≤ b and a ≤ d ≤ b. Find the indifference equations.
(c) Solve the equations when β = 2, and find the value in this case. Which player has
the advantage?
12. Last Round Betting. Here is a game that occurs in the last round of blackjack
or baccarat tournaments, and also in the television game show, Final Jeopardy. For the
general game, see Ferguson and Melolidakis (1997).
II – 92
In the last round of betting in a contest to see who can end up with the most money,
Player I starts with $70 and Player II starts with $100. Simultaneously, Player I must
choose an amount to bet between $0 and $70 and Player II must choose an amount between
$0 and $100. Then the players independently play games with probability .6 of winning
the bet and .4 of losing it. The player who has the most money at the end wins a big prize.
If they end up with the same amount of money, they share the prize.
We may set this up as a game (X, Y, A), with X = [0, 0.7], Y = [0, 1.0], measured in
units of $100, and assuming money is infinitely divisible. We assume the payoff, A(x, y),
is the probability that Player I wins the game plus one-half the probabiity of a tie, when
I bets x and II bets y. The probability that both players win their bets is .6 ∗ .6 = .36,
the probability that both players lose their bets is .4 ∗ .4 = .16, and the probability that I
wins his bet and II loses her bet is .6 ∗ .4 = .24. Therefore,
P(I wins) = .36 I(.7 + x > 1 + y) + .24 I(.7 + x > 1 − y) + .16 I(.7 − x > 1 − y)
= .36 I(x − y>.3) + .24 I(x + y>.3) + .16 I(y − x>.3)
P(a tie) = .36 I(.7 + x =1+ y) + .24 I(.7 + x = 1 − y) + .16 I(.7 − x = 1 − y)
= .36 I(x − y = .3) + .24 I(x + y = .3) + .16 I(y − x = .3)
where I(·) represents the indicator function. This gives
A(x, y) = P(I wins) + 1
2
P(a tie) =
⎧
⎪⎪⎪⎪⎪⎪⎪⎪⎪⎪⎪⎨
⎪⎪⎪⎪⎪⎪⎪⎪⎪⎪⎪⎩
.60 if y<x − .3
.40 if y>x + .3
.00 if y + x<.3
.24 if y>x − .3,y<x + .3, y + x>.3
.42 if 0 < y = x − .3
.32 if 0 < x = y − .3
.12 if x + y = .3,x> 0,y> 0
.30 if x = .3, y = 0
.20 if x = 0, y = .3
Find the value of the game and optimal strategies for both players. (Hint: Both players
have an optimal strategy that give probability to only two points.)
II – 93
References.
Robert J. Aumann and Michael B. Maschler (1995) Repeated Games of Incomplete Information,
The MIT Press, Cambridge, Mass.
A. Ba˜nos (1968) On pseudo-games. Ann. Math. Statist. 39, 1932-1945.
V. J. Baston, F. A. Bostock and T. S. Ferguson (1989) The number hides game, Proc.
Amer. Math. Soc. 107, 437-447.
John D. Beasley (1990) The Mathematics of Games, Oxford University Press.
Emile Borel (1938) Trait´ ´ e du Calcul des Probabilit´es et ses Applications Volume IV, Fascicule
2, Applications aux jeux des hazard, Gautier-Villars, Paris.
G. W. Brown (1951) ”Iterative Solutions of Games by Fictitious Play” in Activity Analysis
of Production and Allocation, T.C. Koopmans (Ed.), New York: Wiley.
G. S. Call and D. J. Velleman (1993) Pascal’s matrices, Amer. Math. Mo. 100, 372-376.
M. T. Carroll, M. A. Jones, E. K.Rykken (2001) The Wallet Paradox Revisited, Math.
Mag. 74, 378-383.
W. H. Cutler (1975) An optimal strategy for pot-limit poker, Amer. Math. Monthly 82,
368-376.
W. H. Cutler (1976) End-Game Poker, Preprint.
Melvin Dresher (1961) Games of Strategy: Theory and Applications, Prentice Hall, Inc.
N.J.
Melvin Dresher (1962) A sampling inspection problem in arms control agreements: a gametheoretic
analysis, Memorandum RM-2972-ARPA, The RAND Corporation, Santa
Monica, California.
R. J. Evans (1979) Silverman’s game on intervals, Amer. Math. Mo. 86, 277-281.
H. Everett (1957) Recursive games, Contrib. Theor. Games III, Ann. Math. Studies 39,
Princeton Univ. Press, 47-78.
C. Ferguson and T. Ferguson (2007) The Endgame in Poker, in Optimal Play: Mathematical
Studies of Games and Gambling, Stuart Ethier and William Eadington, eds.,
Institute for the Study of Gambling and Commercial Gaming, 79-106.
T. S. Ferguson (1967) Mathematical Statistics — A Decision-Theoretic Approach, Academic
Press, New York.
T. S. Ferguson and C. Melolidakis (1997) Last Round Betting, J. Applied Probability 34
974-987.
J. Filar and K. Vrieze (1997) Competitive Markov Decision Processes, Springer-Verlag,
New York.
L. Friedman (1971) Optimal bluffing strategies in poker, Man. Sci. 17, B764-B771.
II – 94
S. Gal (1974) A discrete search game, SIAM J. Appl. Math. 27, 641-648.
M. Gardner (1978) Mathematical Magic Show, Vintage Books, Random House, New York.
Andrey Garnaev (2000) Search Games and Other Applications of Game Theory, Lecture
Notes in Economics and Mathematical Systems 485, Springer.
G. A. Heuer and U. Leopold-Wildburger (1991) Balanced Silverman Games on General
Discrete Sets, Lecture Notes in Econ. & Math. Syst., No. 365, Springer-Verlag.
R. Isaacs (1955) A card game with bluffing, Amer. Math. Mo. 62, 99-108.
S. M. Johnson (1964) A search game, Advances in Game Theory, Ann. Math. Studies 52,
Princeton Univ. Press, 39-48.
Samuel Karlin (1959) Mathematical Methods and Theory in Games, Programming and
Economics, in two vols., Reprinted 1992, Dover Publications Inc., New York.
H. W. Kuhn, (1950) A simplified two-person poker, Contrib. Theor. Games I, Ann. Math.
Studies 24, Princeton Univ. Press, 97-103.
H. W. Kuhn (1997) Classics in Game Theory, Princeton University Press.
A. Maitra and W. Sudderth (1996) Discrete Gambling and Stochastic Games, in the Series
Applications in Mathematics 32, Springer.
J. J. C. McKinsey (1952) Introduction to the Theory of Games, McGraw-Hill, New York.
N. S. Mendelsohn (1946) A psychological game, Amer. Math. Mo. 53, 86-88.
J. Milnor and L. S. Shapley (1957) On games of survival, Contrib. Theor. Games III, Ann.
Math. Studies 39, Princeton Univ. Press, 15-45.
Kent E. Morrison (2010) The Multiplication Game, Math. Mag. 83, 100-110.
J. F. Nash and L. S. Shapley (1950) A simple 3-person poker game, Contrib. Theor. Games
I, Ann. Math. Studies 24, Princeton Univ. Press, 105-116.
D. J. Newman (1959) A model for “real” poker, Oper. Res. 7, 557-560.
Guillermo Owen (1982) Game Theory, 2nd Edition, Academic Press.
T. E. S. Raghavan, T. S. Ferguson, T. Parthasarathy and O. J. Vrieze, eds. (1991) Stochastic
Games and Related Topics, Kluwer Academic Publishers.
J. Robinson (1951) An Iterative Method of Solving a Game, Annals of Mathematics 54,
296-301.
W. H. Ruckle (1983) Geometric games and their applications, Research Notes in Mathematics
82, Pitman Publishing Inc.
L. S. Shapley (1953) Stochastic Games, Proc. Nat. Acad. Sci. 39, 1095-1100.
L. S. Shapley and R. N. Snow (1950) Basic solutions of discrete games, Contrib. Theor.
Games I, Ann. Math. Studies 24, Princeton Univ. Press, 27-35.
II – 95
S. Sorin and J. P. Ponssard (1980) The LP formulation of finite zero-sum games with
incomplete information, Int. J. Game Theory 9, 99-105.
Philip D. Straffin (1993) Game Theory and Strategy, Mathematical Association of America.
John Tukey (1949) A problem in strategy, Econometrica 17, 73.
J. von Neumann and O. Morgenstern (1944) The Theory of Games and Economic Behavior,
Princeton University Press.
J. D. Williams, (1966) The Compleat Strategyst, 2nd Edition, McGraw-Hill, New York.
II – 96
