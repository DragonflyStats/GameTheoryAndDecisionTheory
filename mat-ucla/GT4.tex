
%====================================================================%

\section{4. Solving Finite Games.}
Consider an arbitrary finite two-person zero-sum game, (X, Y, A), with m × n matrix,
A. Let us take the strategy space X to be the first m integers, X = {1, 2,...,m}, and
similarly, Y = {1, 2,...,n}. A mixed strategy for Player I may be represented by a column
vector, (p1, p2,...,pm)
T of probabilities that add to 1. Similarly, a mixed strategy for Player
II is an n-tuple q = (q1, q2,...,qn)
T. The sets of mixed strategies of players I and II will
be denoted respectively by X∗ and Y ∗,
X∗ = {p = (p1,...,pm)
T : pi ≥ 0, for i = 1,...,m and m
1 pi = 1}
Y ∗ = {q = (q1,...,qn)
T : qj ≥ 0, for j = 1,...,n and n
1 qj = 1}
The m-dimensional unit vector ek ∈ X∗ with a one for the kth component and zeros
elsewhere may be identified with the pure strategy of choosing row k. Thus, we may
consider the set of Player I’s pure strategies, X, to be a subset of X∗. Similarly, Y may
be considered to be a subset of Y ∗. We could if we like consider the game (X, Y, A) in
which the players are allowed to use mixed strategies as a new game (X∗, Y ∗, A), where
A(p, q) = pTAq, though we would no longer call this game a finite game.

%====================================================================%
In this section, we give an algorithm for solving finite games; that is, we show how to
find the value and at least one optimal strategy for each player. Occasionally, we shall be
interested in finding all optimal strategies for a player.
4.1 Best Responses. Suppose that Player II chooses a column at random using
q ∈ Y ∗. If Player I chooses row i, the average payoff to I is

n
j=1
aij qj = (Aq)i, (1)
the ith component of the vector Aq. Similarly, if Player I uses p ∈ X∗ and Player II
chooses column j, Then I’s average payoff is

n
i=1
piaij = (pTA)j , (2)
the jth component of the vector pTA. More generally, if Player I uses p ∈ X∗ and Player
II uses q ∈ Y ∗, the average payoff to I becomes

m
i=1
⎛
⎝
n
j=1
aij qj
⎞
⎠ pi = 
m
i=1

n
j=1
piaij qj = p
TAq. (3)
Suppose it is known that Player II is going to use a particular strategy q ∈ Y ∗. Then
Player I would choose that row i that maximizes (1); or, equivalently, he would choose
that p ∈ X∗ that maximizes (3). His average payoff would be
max
1≤i≤m

n
j=1
aij qj = max
p∈X∗ pTAq (4)
II – 35
To see that these quantities are equal, note that the left side is the maximum of pTAq over
p ∈ X∗, and so, since X ⊂ X∗ , must be less than or equal to the right side. The reverse
inequality follows since (3) is an average of the quantities in (1) and so must be less than
or equal to the largest of the values in (1).
Any p ∈ X∗ that achieves the maximum of (3) is called a best response or a Bayes
strategy against q. In particular, any row i that achieves the maximum of (1) is a (pure)
Bayes strategy against q. There always exist pure Bayes strategies against q for every
q ∈ Y ∗ in finite games.

%====================================================================%
Similarly, if it is known that Player I is going to use a particular strategy p ∈ X∗, then
Player II would choose that column j that minimizes (2), or, equivalently, that q ∈ Y ∗
that minimizes (3). Her average payoff would be
min
1≤j≤n

m
i=1
piaij = min
q∈Y ∗ p
TAq. (5)
Any q ∈ Y ∗ that achieves the minimum in (5) is called a best response or a Bayes strategy
for Player II against p.
The notion of a best response presents a practical way of playing a game: Make a
guess at the probabilities that you think your opponent will play his/her various pure
strategies, and choose a best response against this. This method is available in quite
complex situations. In addition, it allows a player to take advantage of an opponent’s
perceived weaknesses. Of course this may be a dangerous procedure. Your opponent may
be better at this type of guessing than you. (See Exercise 1.)

%====================================================================%
\subsection{4.2 Upper and Lower Values of a Game.} Suppose now that II is required to
announce her choice of a mixed strategy q ∈ Y ∗ before I makes his choice. This changes
the game to make it apparently more favorable to I. If II announces q, then certainly I
would use a Bayes strategy against q and II would lose the quantity (4) on the average.
Therefore, II would choose to announce that q that minimizes (4). The minimum of (4)
over all q ∈ Y ∗ is denoted by V and called the upper value of the game (X, Y, A).
V = min
q∈Y ∗ max
1≤i≤m

n
j=1
aij qj = min
q∈Y ∗ max
p∈X∗ p
TAq. (6)
Any q ∈ Y ∗ that achieves the minimum in (6) is called a minimax strategy for II. It
minimizes her maximum loss. There always exists a minimax strategy in finite games: the
quantity (4), being the maximum of m linear functions of q, is a continuous function of q
and since Y ∗ is a closed bounded set, this function assumes its minimum over Y ∗ at some
point of Y ∗.
In words, V as the smallest average loss that Player II can assure for herself no matter
what I does.
A similar analysis may be carried out assuming that I must announce his choice of a
mixed strategy p ∈ X∗ before II makes her choice. If I announces p, then II would choose
II – 36
that column with the smallest average payoff, or equivalently that q ∈ Y ∗ that minimizes
the average payoff (5). Given that (5) is the average payoff to I if he announces p, he
would therefore choose p to maximize (5) and obtain on the average
V = max
p∈X∗ min
1≤j≤n

m
i=1
piaij = max
p∈X∗ min
q∈Y ∗ p
TAq. (7)
The quantity V is called the lower value of the game. It is the maximum amount that I can
guarantee himself no matter what II does. Any p ∈ X∗ that achieves the maximum in (7)
is called a minimax strategy for I. Perhaps maximin strategy would be more appropriate
terminology in view of (7), but from symmetry (either player may consider himself Player
II for purposes of analysis) the same word to describe the same idea may be preferable
and it is certainly the customary terminology. As in the analysis for Player II, we see that
Player I always has a minimax strategy. The existence of minimax strategies in matrix
games is worth stating as a lemma.

%====================================================================%
Lemma 1. In a finite game, both players have minimax strategies.
It is easy to argue that the lower value is less than or equal to the upper value. For if
V < V and if I can assure himself of winning at least V , Player II cannot assure herself of
not losing more than V , an obvious contradiction. It is worth stating this fact as a lemma
too.
Lemma 2. The lower value is less than or equal to the upper value,
V ≤ V.
This lemma also follows from the general mathematical principle that for any realvalued
function, f(x, y), and any sets, X∗ and Y ∗,
max
x∈X∗ min
y∈Y ∗ f(x, y) ≤ min
y∈Y ∗ max
x∈X∗ f(x, y).
To see this general principle, note that miny f(x, y
) ≤ f(x, y) ≤ maxx f(x
, y) for every
fixed x and y. Then, taking maxx on the left does not change the inequality, nor does
taking miny on the right, which gives the result.
If V < V , the average payoff should fall between V and V . Player II can keep it from
getting larger than V and Player I can keep it from getting smaller than V . When V = V ,
a very nice stable situation exists.
Definition. If V = V , we say the value of the game exists and is equal to the common
value of V and V , denoted simply by V . If the value of the game exists, we refer to
minimax strategies as optimal strategies.
The Minimax Theorem, stated in Chapter 1, may be expressed simply by saying that
for finite games, V = V .
II – 37
The Minimax Theorem. Every finite game has a value, and both players have minimax
strategies.
We note one remarkable corollary of this theorem. If the rules of the game are changed
so that Player II is required to announce her choice of a mixed strategy before Player I
makes his choice, then the apparent advantage given to Player I by this is illusory. Player
II can simply announce her minimax strategy.
4.3 Invariance under Change of Location and Scale. Another simple observation
is useful in this regard. This concerns the invariance of the minimax strategies under
the operations of adding a constant to each entry of the game matrix, and of multiplying
each entry of the game matrix by a positive constant. The game having matrix A = (aij )
and the game having matrix A = (a
ij ) with a
ij = aij + b, where b is an arbitrary real
number, are very closely related. In fact, the game with matrix A is equivalent to the
game in which II pays I the amount b, and then I and II play the game with matrix A.
Clearly any strategies used in the game with matrix A give Player I b plus the payoff
using the same strategies in the game with matrix A. Thus, any minimax strategy for
either player in one game is also minimax in the other, and the upper (lower) value of the
game with matrix A is b plus the upper (lower) value of the game with matrix A.
Similarly, the game having matrix A = (a
ij ) with a
ij = caij , where c is a positive
constant, may be considered as the game with matrix A with a change of scale (a change
of monetary unit if you prefer). Again, minimax strategies do not change, and the upper
(lower) value of A is c times the upper (lower) value of A. We combine these observations
as follows. (See Exercise 2.)
Lemma 3. If A = (aij ) and A = (a
ij ) are matrices with a
ij = caij + b, where c > 0,
then the game with matrix A has the same minimax strategies for I and II as the game
with matrix A
. Also, if V denotes the value of the game with matrix A, then the value
V  of the game with matrix A satisfies V  = cV + b.

%====================================================================%
\subsection{4.4 Reduction to a Linear Programming Problem.} There are several nice proofs
of the Minimax Theorem. The simplest proof from scratch seems to be that of G. Owen
(1982). However, that proof is by contradiction using induction on the size of the matrix.
It gives no insight into properties of the solution or on how to find the value and optimal
strategies. Other proofs, based on the Separating Hyperplane Theorem or the Brouwer
Fixed Point Theorem, give some insight but are based on nontrivial theorems not known
to all students.
Here we use a proof based on linear programming. Although based on material not
known to all students, it has the advantage of leading to a simple algorithm for solving
finite games. For a background in linear programming, the book by Chv´atal (1983) can be
recommended. A short course on Linear Programming more in tune with the material as
it is presented here may be found on the web at http://www.math.ucla.edu/˜tom/LP.pdf.

%====================================================================%
A Linear Program is defined as the problem of choosing real variables to maximize or
minimize a linear function of the variables, called the objective function, subject to linear
II – 38
constraints on the variables. The constraints may be equalities or inequalities. A standard
form of this problem is to choose y1,... ,yn, to
maximize b1y1 + ··· + bnyn, (8)
subject to the constraints
a11y1 + ··· + a1nyn ≤ c1
.
.
.
am1y1 + ··· + amnyn ≤ cm
(9)
and
yj ≥ 0 for j = 1,...,n.
Let us consider the game problem from Player I’s point of view. He wants to choose
p1,..., pm to maximize (5) subject to the constraint p ∈ X∗. This becomes the mathematical
program: choose p1,...,pm to
maximize min
1≤j≤n

m
i=1
piaij (10)
subject to the constraints
p1 + ··· + pm = 1 (11)
and
pi ≥ 0 for i = 1, . . . , m.
Although the constraints are linear, the objective function is not a linear function of
the p’s because of the min operator, so this is not a linear program. However, it can be
changed into a linear program through a trick. Add one new variable v to Player I’s list of
variables, restrict it to be less than the objective function, v ≤ min1≤j≤n
m
i=1 piaij , and
try to make v as large as possible subject to this new constraint. The problem becomes:
Choose v and p1,...,pm to
maximize v (12)
subject to the constraints
v ≤ 
m
i=1
piai1
.
.
.
v ≤ 
m
i=1
piain
p1+ ··· + pm = 1
(13)
II – 39
and
pi ≥ 0 for i = 1, . . . , m.
This is indeed a linear program. For solving such problems, there exists a simple algorithm
known as the simplex method.
In a similar way, one may view the problem from Player II’s point of view and arrive
at a similar linear program. II’s problem is: choose w and q1,...,qn to
minimize w (14)
subject to the constraints
w ≥ 
n
j=1
a1j qj
.
.
.
w ≥ 
n
j=1
amj qj
q1+ ··· + qn = 1
(15)
and
qj ≥ 0 for j = 1, . . . , n.
In Linear Programming, there is a theory of duality that says these two programs,
(12)-(13), and (14)-(15), are dual programs. And there is a remarkable theorem, called the
Duality Theorem, that says dual programs have the same value. The maximum Player I
can achieve in (14) is equal to the minimum that Player II can achieve in (12). But this is
exactly the claim of the Minimax Theorem. In other words, the Duality Theorem implies
the Minimax Theorem.
There is another way to transform the linear program, (12)-(13), into a linear program
that is somewhat simpler for computations when it is known that the value of the game
is positive. So suppose v > 0 and let xi = pi/v. Then the constraint p1 + ··· + pm = 1
becomes x1 + ··· + xm = 1/v, which looks nonlinear. But maximizing v is equivalent to
minimizing 1/v, so we can remove v from the problem by minimizing x1 +···+xm instead.
The problem, (12)-(13), becomes: choose x1,...,xm to
minimize x1 + ··· + xm (16)
subject to the constraints
1 ≤ 
m
i=1
xiai1
.
.
.
1 ≤ 
m
i=1
xiain
(17)
II – 40
and
xi ≥ 0 for i = 1, . . . , m.
When we have solved this problem, the solution of the original game may be easily found.
The value will be v = 1/(x1 + ··· + xm) and the optimal strategy for Player I will be
pi = vxi for i = 1, . . . .m.
4.5 Description of the Pivot Method for Solving Games The following algorithm
for solving finite games is essentially the simplex method for solving (16)-(17) as
described in Williams (1966).
Step 1. Add a constant to all elements of the game matrix if necessary to insure that the
value is positive. (If you do, you must remember at the end to subtract this constant from
the value of the new matrix game to get the value of the original matrix game.)
Step 2. Create a tableau by augmenting the game matrix with a border of −1’s along
the lower edge, +1’s along the right edge, and zero in the lower right corner. Label I’s
strategies on the left from x1 to xm and II’s strategies on the top from y1 to yn.
y1 y2 ··· yn
x1 a11 a12 ··· a1n 1
x2 a21 a22 ··· a2n 1
.
.
. .
.
. .
.
. .
.
. .
.
.
xm am1 am2 ··· amn 1
−1 −1 ··· −1 0
Step 3. Select any entry in the interior of the tableau to be the pivot, say row p column q,
subject to the properties:
a. The border number in the pivot column, a(m + 1, q), must be negative.
b. The pivot, a(p, q), itself must be positive.
c. The pivot row, p, must be chosen to give the smallest of the ratios the border
number in the pivot row to the pivot, a(p, n+1)/a(p, q), among all positive pivots for that
column.
Step 4. Pivot as follows:
a. Replace each entry, a(i, j), not in the row or column of the pivot by a(i, j)−a(p, j)·
a(i, q)/a(p, q).
b. Replace each entry in the pivot row, except for the pivot, by its value divided by
the pivot value.
c. Replace each entry in the pivot column, except for the pivot, by the negative of its
value divided by the pivot value.
d. Replace the pivot value by its reciprocal.
This may be represented symbolically by
p r
c q −→ 1/p r/p
−c/p q − (rc/p)
II – 41
where p stands for the pivot, r represents any number in the same row as the pivot, c
represents any number in the same column as the pivot, and q is an arbitrary entry not in
the same row or column as the pivot.
Step 5. Exchange the label on the left of the pivot row with the label on the top of the
pivot column.
Step 6. If there are any negative numbers remaining in the lower border row, go back to
step 3.
Step 7. Otherwise, a solution may now be read out:
a. The value, v, is the reciprocal of the number in the lower right corner. (If you
subtracted a number from each entry of the matrix in step 1, it must be added to v here.)
b. I’s optimal strategy is constructed as follows. Those variables of Player I that end
up on the left side receive probability zero. Those that end up on the top receive the value
of the bottom edge in the same column divided by the lower right corner.
c. II’s optimal strategy is constructed as follows. Those variables of Player II that
end up on the top receive probability zero. Those that end up on the left receive the value
of the right edge in the same row divided by the lower right corner.
4.6 A Numerical Example. Let us illustrate these steps using an example. Let’s
take a three-by-three matrix since that is the simplest example one we cannot solve using
previous methods. Consider the matrix game with the following matrix,
B =
⎛
⎝
2 −1 6
0 1 −1
−22 1
⎞
⎠
We might check for a saddle point (there is none) and we might check for domination
(there is none). Is the value positive? We might be able to guess by staring at the matrix
long enough, but why don’t we simply make the first row positive by adding 2 to each
entry of the matrix:
B =
⎛
⎝
418
231
043
⎞
⎠
The value of this game is at least one since Player I can guarantee at least 1 by using the
first (or second) row. We will have to remember to subtract 2 from the value of B to get
the value of B. This completes Step 1 of the algorithm.
In Step 2, we set up the tableau for the matrix B as follows.
y1 y2 y3
x1 418 1
x2 231 1
x3 043 1
−1 −1 −1 0
II – 42
In Step 3, we must choose the pivot. Since all three columns have a negative number
in the lower edge, we may choose any of these columns as the pivot column. Suppose we
choose column 1. The pivot row must have a positive number in this column, so it must be
one of the top two rows. To decide which row, we compute the ratios of border numbers
to pivot. For the first row it is 1/4; for the second row it is 1/2. the former is smaller, so
the pivot is in row 1. We pivot about the 4 in the upper left corner.
Step 4 tells us how to pivot. The pivot itself gets replaced by its reciprocal, namely
1/4. The rest of the numbers in the pivot row are simply divided by the pivot, giving 1/4,
2, and 1/4. Then the rest of the numbers in the pivot column are divided by the pivot and
changed in sign. The remaining nine numbers are modified by subtracting r · c/p for the
corresponding r and c. For example, from the 1 in second row third column we subtract
8 × 2/4 = 4, leaving −3. The complete pivoting operation is
y1 y2 y3
x1 418 1
x2 231 1
x3 043 1
−1 −1 −1 0
−→
x1 y2 y3
y1 1/4 1/4 2 1/4
x2 −1/2 5/2 −3 1/2
x3 043 1
1/4 −3/4 1 1/4
In the fifth step, we interchange the labels of the pivot row and column. Here we
interchange x1 and y1. This has been done in the display.
For Step 6, we check for negative entries in the lower edge. Since there is one, we
return to Step 3.
This time, we must pivot in column 2 since it has the unique negative number in the
lower edge. All three numbers in this column are positive. We find the ratios of border
numbers to pivot for rows 1, 2 and 3 to be 1, 1/5, and 1/4. The smallest occurs in the
second row, so we pivot about the 5/2 in the second row, second column. Completing
Steps 4 and 5, we obtain
x1 y2 y3
y1 1/4 1/4 2 1/4
x2 −1/2 5/2 −3 1/2
x3 043 1
1/4 −3/4 1 1/4
−→
x1 x2 y3
y1 .3 −.1 2.3 .2
y2 −.2 .4 −1.2 .2
x3 .8 −1.6 7.8 .2
.1 .3 .1 .4
At Step 6 this time, all values on the lower edge are non-negative so we pass to Step
7. We may now read the solution to the game with matrix B
.
The value is the reciprocal of .4, namely 5/2.
Since x3 is on the left in the final tableau, the optimal p3 is zero. The optimal p1
and p2 are the ratios, .1/.4 and .3/.4, namely 1/4 and 3/4. Therefore, I’s optimal mixed
strategy is (p1, p2, p3)=(.25, .75, 0).
II – 43
Since y3 is on the top in the final tableau, the optimal q3 is zero. The optimal q1
and q2 are the ratios, .2/.4 and .2/.4, namely 1/2 and 1/2. Therefore, II’s optimal mixed
strategy is (q1, q2, q3)=(.5, .5, 0).
The game with matrix B has the same optimal mixed strategies but the value is
5/2 − 2=1/2.
Remarks. 1. The reason that the pivot row is chosen according to the rule in Step
3(c) is so that the numbers in the resulting right edge of the tableau stay non-negative. If
after pivoting you find a negative number in the last column, you have made a mistake,
either in the choice of the pivot row, or in your numerical calculations for pivoting.
2. There may be ties in comparing the ratios to choose the pivot row. The rule given
allows you to choose among those rows with the smallest ratios. The smallest ratio may
be zero.
3. The value of the number in the lower right corner never decreases. (Can you see
why this is?) In fact, the lower right corner is always equal to the sum of the values in the
lower edge corresponding to Player I’s labels along the top. Similarly, it is also the sum
of the values on the right edge corresponding to Player II’s labels on the left. This gives
another small check on your arithmetic.
4. One only pivots around numbers in the main body of the tableau, never in the
lower or right edges.
5. This method gives one optimal strategy for each player. If other optimal strategies
exist, there will be one or more zeros in the bottom edge or right edge in the final tableau.
Other optimal basic strategies can be found by pivoting further, in a column with a zero
in the bottom edge or a row with a zero in the right edge, in such a way that the bottom
row and right edge stay nonnegative.
4.7 Approximating the Solution: Fictitious Play.
As an alternative to the simplex method, the method of fictitious play may be used to
approximate the value and optimal strategies of a finite game. It is a sequential procedure
that approximates the value of a game as closely as desired, giving upper and lower bounds
that converge to the value and strategies for the players that achieve these bounds.
The advantage of the simplex method is that it gives answers that are accurate,
generally to machine accuracy, and for small size problems is extremely fast. The advantage
of the method of fictitious play is its simplicity, both to program and understand, and the
fact that you can stop it at any time and obtain answers whose accuracy you know. The
simplex method only gives answers when it is finished. For large size problems, say a
matrix 50 by 50 or greater, the method of fictitious play will generally give a sufficiently
accurate answer in a shorter time than the simplex method. For very large problems, it is
the only way to proceed.
Let A(i, j) be an m by n payoff matrix. The method starts with an arbitrary initial
pure strategy 1 ≤ i1 ≤ m for Player I. Alternatively from then on, each player chooses his
II – 44
next pure strategy as a best reply assuming the other player chooses among his previous
choices at random equally likely. For example, if i1,...,ik have already been chosen by
Player I for some k ≥ 1, then jk is chosen as that j that minimizes the expectation
(1/k)
k
=1 A(i, j). Similarly, if j1,...,jk have already been chosen, ik+1 is then chosen
as that i that maximizes the expectation (1/k)
k
=1 A(i, j). To be specific, we define
sk(j) = 

k
=1
A(i, j) and tk(i) = 

k
=1
A(i, j) (1)
and then define
jk = argmin sk(j) and ik+1 = argmax tk(i) (2)
If the maximum of tk(i) is assumed at several different values of i, then it does not matter
which of these is taken as ik+1. To be specific, we choose ik+1 as the smallest value of i
that maximizes tk(i). Similarly jk is taken as the smallest j that minimizes sk(j). In this
way, the sequences ik and jk are defined deterministically once i1 is given.
Notice that Vk = (1/k)tk(ik+1) is an upper bound to the value of the game since
Player II can use the strategy that chooses j randomly and equally likely from j1,...,jk
and keep Player I’s expected return to be at most Vk. Similarly, Vk = (1/k)sk(jk) is a
lower bound to the value of the game. It is rather surprising that these upper and lower
bounds to the value converge to the value of the game as k tends to infinity.
Theorem. If V denotes the value of the game, then Vk → V , Vk → V , and Vk ≤ V ≤ Vk,
for all k.
This approximation method was suggested by George Brown (1951), and the proof
of convergence was provided by Julia Robinson (1951). The convergence of Vk and Vk to
V is slow. It is thought to be of order at least 1/
√
k. In addition, the convergence is not
monotone. See the example below.
A modification of this method in which i1 and j1 are initially arbitrarily chosen, and
then the selection of future ik and jk is made simultaneously by the players rather than
sequentially, is often used, but it is not as fast.
It should be mentioned that as a practical matter, choosing at each stage a best reply
to an opponent’s imagined strategy of choosing among his previous choices at random is
not a good idea. See Exercise 7. On the other hand, Alfredo Ba˜nos (1968) describes a
sequential method for Player I, say, to choose mixed strategies such that liminf of the
average payoff is at least the value of the game no matter what Player II does. This
choice of mixed strategies is based only upon Player I’s past pure strategy choices and
the past observed payoffs, but not otherwise on the payoff matrix or upon the opponent’s
pure strategy choices. It would be nice to devise a practical method of choosing a mixed
strategy depending on all the information contained in the previous plays of the game that
performs well.
II – 45
EXAMPLE. Take as an example the game with matrix
A =
⎛
⎝
2 −1 6
0 1 −1
−22 1
⎞
⎠
This is the game solved in Section 4.6. It has value .5, and optimal mixed strategies,
(.25, .75, 0) and (.5, .5, 0) for Player I and Player II respectively. It is easy to set up a
program to perform the calculations. In particular, the computations, (1), may be made
recursively in the simpler form
sk(j) = sk−1(j) + A(ik, j) and tk(i) = tk−1(i) + A(i, jk) (3)
We take the initial i1 = 1, and find
k ik sk(1) sk(2) sk(3) Vk jk tk(1) tk(2) tk(3) Vk
1 1 2 −1 6 −1 2 −1 1 2 2
2 3 0 1 70 1 1 1 00.5
3 1 2 0 13 0 2 0 2 2 0.6667
4 2 2 1 12 0.25 2 −1 3 4 1
5 3 0 3 13 0 1 1 3 2 0.6
6 2 0 4 12 0 1 3 3 00.5
7 1 2 3 18 0.2857 1 5 3 −2 0.7143
8 1 4 2 24 0.25 2 4 4 00.5
9 1 6 1 30 0.1111 2 3 5 2 0.5556
10 2 6 2 29 0.2 2 2 6 4 0.6
11 2 6 3 28 0.2727 2 1 7 6 0.6364
12 2 6 4 27 0.3333 2 0 8 8 0.6667
13 2 6 5 26 0.3846 2 −1 9 10 0.7692
14 3 4 7 27 0.2857 1 1 9 8 0.6429
15 2 4 8 26 0.2667 1 3 9 6 0.6
The initial choice of i1 = 1 gives (s1(1), s1(2), s1(3)) as the first row of A, which has a
minimum at s1(2). Therefore, j1 = 2. The second column of A has t1(3) as the maximum,
so i2 = 3. Then the third row of A is added to the s1 to produce the s2 and so on. The
minimums of the sk and the maximums of the tk are indicated in boldface. The largest
of the Vk found so far occurs at k = 13 and has value sk(jk)/k = 5/13 = 0.3846 .... This
value can be guaranteed to Player I by using the mixed strategy (5/13,6/13,2/13), since in
the first 13 of the ik there are 5 1’s, 6 2’s and 2 3’s. The smallest of the Vk occurs several
times and has value .5. It can be achieved by Player II using the first and second columns
equally likely. So far we know that .3846 ≤ V ≤ .5, although we know from Section 4.6
that V = .5.
Computing further, we can find that V91 = 44/91 = .4835 ... and is achieved by
the mixed strategy (25/91, 63/91, 3/91). From row 9 on, the difference between the boldface
numbers in each row seems to be bounded between 4 and 6. This implies that the
convergence is of order 1/k.
II – 46
4.8 Exercises.
1. Consider the game with matrix A. Past experience in playing the game with Player
II enables Player I to arrive at a set of probabilities reflecting his belief of the column that II
will choose. I thinks that with probabilities 1/5, 1/5, 1/5, and 2/5, II will choose columns
1, 2, 3, and 4 respectively.
A =
⎛
⎝
0724
1482
9 3 −1 6
⎞
⎠ .
(a) Find for I a Bayes strategy (best response) against (1/5, 1/5, 1/5, 2/5).
(b) Suppose II guesses correctly that I is going to use a Bayes strategy against
(1/5, 1/5, 1/5, 2/5). Instruct II on the strategy she should use - that is, find II’s Bayes
strategy against I’s Bayes strategy against (1/5, 1/5, 1/5, 2/5).
2. The game with matrix A has value zero, and (6/11, 3/11, 2/11) is optimal for I.
A =
⎛
⎝
0 −1 1
2 0 −2
−330
⎞
⎠ B =
⎛
⎝
537
951
−1 11 5
⎞
⎠
(a) Find the value of the game with matrix B and an optimal strategy for I.
(b) Find an optimal strategy for II in both games.
3. A game without a value. Let X = {1, 2, 3,...}, let Y = {1, 2, 3,...} and
A(i, j) =  +1 if i>j
0 if i = j
−1 if i<j
This is the game “The player that chooses the larger integer wins”. Here we may take for
the space X∗ of mixed strategies of Player I
X∗ = {(p1, p2,...) : pi ≥ 0 for all i, and 
∞
i=1
pi = 1}.
Similarly,
Y ∗ = {(q1, q2,...) : qj ≥ 0 for all j, and 
∞
j=1
qj = 1}.
The payoff for given p ∈ X∗ and q ∈ Y ∗ is
A(p, q) = 
∞
i=1

∞
j=1
piA(i, j)qj .
II – 47
(a) Show that for all q ∈ Y ∗, sup1≤i<∞
∞
j=1 A(i, j)qj = +1.
(b) Conclude that V = +1.
(c) Using symmetry, argue that V = −1.
(d) What are I’s minimax strategies in this game?
4. Use the method presented in Section 4.5 to solve the game with matrix
A =
⎛
⎝
012
2 −1 −2
3 −3 0
⎞
⎠ .
Either argue that the value is positive, or add +1 to the elements of the matrix. To go
easy on the homework grader, make the first pivot in the second column.
5. An Example In Which the Lower Value is Greater than the Upper
Value? Consider the infinite game with strategy spaces X = Y = {0, 1, 2,...}, and payoff
function,
A(i, j) =  0 if i = j
4j if i>j
−4i if i<j.
Note that the game is symmetric. Let p = (p0, p1, p2,...)
T = (1/2, 1/4, 1/8,...)
T be be a
mixed strategy for Player I, pi = 2−(i+1).
(a) Show that if Player I uses this strategy, his average return, ∞
i=0 piA(i, j), is equal
to 1/2 for all pure strategies j for Player II.
(b) So p is an equalizer strategy that guarantees Player I at least 1/2. So the lower
value is at least 1/2. Perhaps he can do better. In fact he can, but ...Wait a minute! The
game is symmetric. Shouldn’t the value be zero? Worse, suppose Player II uses the same
strategy. By symmetry, she can keep Player I’s winnings down to −1/2 no matter what
pure strategy he chooses. So the upper value is at most −1/2. What is wrong? What
if both players use the mixed strategy, p? We haven’t talked much about infinite games,
but what restrictions would you place on infinite games to avoid such absurd examples?
Should the restrictions be placed on the payoff function, A, or on the notion of a mixed
strategy?

%====================================================================%
6. Carry out the fictitious play algorithm on the matrix A =
 1 −1
0 2
through step
k = 4. Find the upper and lower bounds on the value of the game that this gives.
7. Suppose the game with matrix,  √
2 0
0 1
is played repeatedly. On the first round
the players make any choices.
(a) Thereafter Player I makes a best response to his opponent’s imagined strategy of
choosing among her previous choice at random. If Player II knows this, what should she
do? What are the limiting average frequencies of the choices of the players?
(b) Suppose Player II is required to play a best response to her opponent’s previous
choices. What should Player I do, and what would his limiting average payoff be?
II – 48

