\documentclass[]{article}
\voffset=-1.5cm
\oddsidemargin=0.0cm
\textwidth = 480pt


\usepackage{amsmath}
\usepackage{graphicx}
\usepackage{amssymb}
\usepackage{framed}
\usepackage{multicol}
%\usepackage[paperwidth=21cm, paperheight=29.8cm]{geometry}
%\usepackage[angle=0,scale=1,color=black,hshift=-0.4cm,vshift=15cm]{background}
%\usepackage{multirow}
\usepackage{enumerate}

\usepackage{amsmath,amsfonts,amssymb}
\usepackage{color}
\usepackage{multirow}
\usepackage{eurosym}
\usepackage{framed}

%\graphicspath{ {images/} }
%\input def.tex
%\input dsdef.tex
%\input rgb.tex

%\newcommand \la{\lambda}
%\newcommand \al{a}
%\newcommand \be{b}
\newcommand \x{\overline{x}}
\newcommand \y{\overline{y}}
\begin{document}


%- III – 15
 

\section{Models of Duopoly}
The examples given of the noncooperative theory of equilibrium have generally shown
the theory to have poor predictive power. This is mainly because there may be multiple
equilibria with no way to choose among them. Alternately, there may be a unique equilibrium
with a poor outcome, even one found by iterated elimination of dominated strategies
as in the prisoner’s dilemma or the centipede game. But there are some situations, such
as the prisoner’s dilemma played once, in which strategic equilibria are quite reasonable
predictive indicators of behavior. We begin with a model of duopoly due to A. Cournot
(1838).

%==============================================================%
\section*{Cournot Duopoly}

Cournot competition is an economic model used to describe an industry structure in which companies compete on the amount of output they will produce, which they decide on independently of each other and at the same time. It is named after Antoine Augustin Cournot ($1801-1877$) who was inspired by observing competition in a spring water duopoly.
\\ \smallskip
The analysis of competition
among firms was proposed by Augustine Cournot in a book published in 1838. In fact,
Cournot is the one who invented what we now call Nash equilibrium, although his
analysis was restricted to a small class of games.

\subsection*{Features}
Cournot Duopoly has the following features:

\begin{enumerate}
\item There is more than one firm and all firms produce a homogeneous product, i.e. there is no product differentiation;
\item Firms do not cooperate, i.e. there is no collusion;
\item Firms have market power, i.e. each firm's output decision affects the good's price;
\item The number of firms is fixed;
\item Firms compete in quantities, and choose quantities simultaneously;
\end{enumerate}

The firms are economically rational and act strategically, usually seeking to maximize profit given their competitors' decisions.
An essential assumption of this model is the "not conjecture" that each firm aims to maximize profits, based on the expectation that its own output decision will not have an effect on the decisions of its rivals. Price is a commonly known decreasing function of total output. All firms know ${\displaystyle N}$ , the total number of firms in the market, and take the output of the others as given. Each firm has a cost function ${\displaystyle c_{i}(q_{i})}$. 

Normally the cost functions are treated as common knowledge. The cost functions may be the same or different among firms. The market price is set at a level such that demand equals the total quantity produced by all firms. Each firm takes the quantity set by its competitors as a given, evaluates its residual demand, and then behaves as a monopoly.


\subsection{The Cournot Model of Duopoly.} 
\begin{itemize}
    \item There are two competing firms producing a
single homogeneous product. 
\item These firms must choose how much of the good to produce.
The cost of producing one unit of the good is a constant c, the same for both firms. m 
\iteIf
a Firm i produces the quantity $q_i$ units of the good, then the cost to Firm i is cqi, for
i = 1, 2. (There is no setup cost.) 
\item The price of a unit of the good is negatively related to
the total amount produced. If Firm 1 produces q1 and Firm 2 produces q2 for a total of
$Q = q1 + q2$, the price is
\end{itemize}
\[P(Q) = XXX a − Q \mbox{ if } 0 \leq Q \leq a\]
\[0 if Q>a = (a − Q)\]
+ (1)
for some constant a. 
\\
\textit{(This is not a realistic assumption, but the price will be approximately linear near the equilibrium point, and that is the main thing.) }

We assume the firms must
choose their production quantities simultaneously; no collusion is allowed.
The pure strategy spaces for this game are the sets $X = Y = [0,\infty)$. Note these are
infinite sets, so the game is not a finite game. It would not hurt to restrict the strategy
spaces to [0, a]; no player would like to produce more than a units because the return is
zero. 
\subsection*{Payoffs}
The payoffs for the two players are the profits,
\[u1(q1, q2) = q1P(q1 + q2) − cq1 = q1(a − q1 − q2)
+ − cq1 (2)\]
\[u2(q1, q2) = q2P(q1 + q2) − cq2 = q2(a − q1 − q2)
+ − cq2 (3)\]
\begin{itemize}
    \item This defines the strategic form of the game. We assume that $c<a$, since otherwise the
cost of production would be at least as great as any possible return.
\end{itemize}
%=========================================================%
\subsection{Monppoly}
\begin{itemize}


\item First, let us find out what happens in the monopolistic case when there is only one
producer. 
\item That is, suppose q2 = 0. Then the return to Firm 1 if it produces q1 units is
$u(q1) = q1(a − q1)+ − cq1$. The firm will choose q1 to maximize this quantity. 
\item Certainly
the maximum will occur for $0 < q1 < a$; in this case, u(q1) = q1(a − c) − q2
1, and we may
find the point at which the maximum occurs by taking a derivative with respect to q1,
setting it to zero and solving for q1.
\item The resulting equation is u
(q1) = a − c − 2q1 = 0,
whose solution is $q1 = (a − c)/2$. 
\item The monopoly price is P((a − c)/2) = (a + c)/2, and the
monopoly profit is $u((a − c)/2) = (a − c)2/4$.
\end{itemize}

\subsection{Duopoly}
%%--- III – 16
To find a duopoly PSE, we look for a pure strategy for each player that is a best
response to the other’s strategy. We find simultaneously the value of q1 that maximizes
(2) and the value of q2 that maximizes (3) by setting the partial derivatives to zero.
\[\frac{\partial u1(q1, q2)}{\partial q1} =
 = a − 2q1 − q2 − c = 0 (4)\]
\[
\frac{\partial u2(q1, q2)}{\partial q2}  = a − q1 − 2q2 − c = 0 (5)\]
(u1 is a quadratic function of q1 with a negative coefficient, so this root represents a point
of maximum.) 
\\

Solving these equations simultaneously and denoting the result by $q^{\ast}_1$ and $q^{\ast}_2$, we find

\[q^{\ast}_1 = (a − c)/3\] and \[q^{\ast}
2 = (a − c)/3. (6)\]
Therefore, $(q^{\ast}_1 , q^{\ast}_2$ ) is a PSE for this problem.

In this SE, each firm produces less than the monopoly production, but the total
produced is greater than the monopoly production. The payoff each player receives from
this SE is
\[
u1(q^{\ast}_1 , q^{\ast}_2 ) = a − c
3 (a − a − c
3 − a − c
3 ) − c
a − c
3 = (a − c)2
9 . (7)\]
\begin{itemize}
    \item 

Note that the total amount received by the firms in this equilibrium is (2/9)(a − c)2.
\item 
This is less than (1/4)(a − c)2, which is the amount that a monopoly would receive using
the monopolistic production of (a − c)/2.
\item This means that if the firms were allowed to
cooperate, they could improve their profits by agreeing to share the production and profits.
    \item Thus each would produce less, (a−c)/4 rather than (a−c)/3, and receive a greater profit,
(a − c)2/8 rather than (a − c)2/9.
\itmm On the other hand, the duopoly price is $P(q^{\ast}
1 +q^{\ast}
2)=(a+2c)/3$, which is less than the
monopoly price, (a + c)/2 (since c<a). Thus, the consumer is better off under a duopoly
than under a monopoly.
\item This PSE is in fact the unique SE. This is because it can be attained by iteratively
deleting strictly dominated strategies. To see this, consider the points at which the function
u1 has positive slope as a function of $q1 \geq 0$ for fixed $q2 \geq 0$. The derivative (4) is positive
provided $2q1 + q2 < a − c$. See Figure 3.1.
\item For all values of $q2 ≥ 0$, the slope is negative for all q1 > (a − c)/2. Therefore, all
$q1 > (a − c)/2$ are strictly dominated by $q1 = (a − c)/2$.
\item But since the game is symmetric in the players, we automatically have all $q2 > (a−c)/2$
are strictly dominated and may be removed. 
\item When all such points are removed from
consideration in the diagram, we see that for all remaining q2, the slope is positive for all
$q1 < (a − c)/4$. Therefore, all $q1 < (a − c)/4$ are strictly dominated by $q1 = (a − c)/4$.
\end{itemize}

Again symmetrically eliminating all $q2 < (a − c)/4$, we see that for all remaining q2,
the slope is negative for all q1 > 3(a − c)/8. Therefore, all q1 > 3(a − c)/8 are strictly
III – 17
Area of
Negative
slope of u1
Area of
Positive
slope of u1
Figure 3.1
0 q1 (a-c)/4 3(a-c)/8 (a-c)/2
0
(a-c)/4
(a-c)/2
a-c
q2
dominated by $q_1 = 3(a − c)/8$. And so on, chipping a piece off from the lower end and
then one from the upper end of the interval of the remaining q1 not yet eliminated.

\begin{itemize}
    \item If
this is continued an infinite number of times, all $q_1$ are removed by iterative elimination
of strictly dominated strategies except the point $q^{\ast}$
1 , and by symmetry $q^{\ast}$
2 for Player II.
\item Note that the prisoner’s dilemma puts in an appearance here. 
\item Instead of using the
SE obtained by removing strictly dominated strategies, both players would be better off if
they could cooperate and produce (a − c)/4 each.
\newpage
    
\end{itemize}
\newpage
A Duopoly Example.
%-https://www0.gsb.columbia.edu/faculty/nsicherman/B7006-002/duopoly.pdf
Consider an industry with two firms. Firms are identical and produce an
homogenous product. Firms have to select outputs (capacity) in order to maximize
profits. Each firm knows its own total cost of production, the total cost of production of
the competitor and the industry demand.

The following data are known by both firms and describe the industry
situation:
\begin{enumerate}
    \item p = 140 - (Q1+Q2) (industry demand)
\item TC1 = 20Q1 (total cost of firm 1),
\item TC2 = 20Q2 (total cost of firm 2).
\end{enumerate}


Observe that the industry price, equation 1, depends on the output of both firms. ThiP
feature has two implications: 
\begin{enumerate}
    \item since the profits of each firm depend on the price, they
depend on the choice of the competitor (strategic interaction),
\item in order to establish
profit maximizing decisions, each firm has to guess what the competitor will do.
\end{enumerate}
\subsection{One shot case.}
We analyze and compare two different situations. In the first, firms compete
strategically. In order to maximize their profits, they guess and take into account what the
competitor does (Cournot - Nash). In the second, firms collude and coordinate their
actions by forming a Cartel.

\subsection{Cournot-Nash Competition.}
We know that firms with market power maximize profits by selecting a level of
output at which their marginal revenue is equal to their marginal costs. However, contrary
to what we have seen up to now, in a situation of strategic interaction, the marginal
revenue of a firm depends on the competitors choices.

We divide the analyses in three steps. 
\begin{itemize}
    \item In the first, we compute the marginal revenue
curves.
\item In the second, the profit maximizing outputs. 
\item In the third, equilibrium price and
profits.
\end{itemize}
\subsubsection{Step 1: Compute the marginal revenues for both firms.}
Start with firm 1. The total revenues of firm 1, R1, are:
2
2
\[R1= pQ1 = (140-(Q1+Q2))Q1 = 140 Q1 - Q1
2 - Q2Q1.\]
The marginal revenue of firm 1, MR1, is just the derivative of the total revenues with
respect to Q1. Hence
5) MR1 = 140- 2Q1 - Q2 (Marginal revenue of firm 1).
Similarly, the total revenues of firm 2, R2 are:
\[R2= pQ2 = (140-(Q1+Q2))Q2 = 140 Q1-Q2
2
-Q2Q1.\]
Hence, the marginal revenue for firm 2, MR2, is:
6) MR2 = 140- 2Q2 - Q1, (Marginal revenue of firm 2)
\subsubsection{Step 2: Compute the profit maximizing outputs for both firms.}
\begin{itemize}
    \item To start with observe that equations 2) and 3) imply that MC1 =MC2 = 20.
Start with firm 1. 
\item Profit maximization for both firms entails selecting an output at which
the marginal revenue equates the marginal cost. 
\item Hence for firm 1, MR1 = MC1 implies by
equation 4):
 140- 2Q1 - Q2 = 20 (Marginal revenue of firm 1)
or equivalently
6) Q1 = (120 - Q2 )/2 (Reaction function of firm 1, RF1)
\item Equation 6) is called a reaction function: it sets the profit maximizing value of firm 1’ s
output, Q1, for any value of the competitor output, Q2 .
\item Using the same argument, for firm 2, MR2 =MC2 implies by equation 4)
 140- 2Q2 - Q1 = 20 (Marginal revenue of firm 2)
or equivalently
7) Q2 = (120 - Q1 )/2 (Reaction function of firm 2)
\item Hence, for given choices of the competitors, both firms use their reaction
functions to set profit maximizing quantities of their outputs. However, the game is
simultaneous and firms do not observe the actions selected by the competitor. 
\item Therefore,
how are they going to guess the competitor output? Each firm knows that everybody in
the market will try to maximize profits or, equivalently, that every firm will use its own
reaction function.
\end{itemize}
 Both firms can compute the solution (Q1
^{\ast}
, Q2
^{\ast}
 ) to equations 6 and 7. 
3
3
Given $Q2^{\ast}$
, firm 1 will maximize its profit by choosing $Q1^{\ast}$ and, given $Q1^{\ast}$, firm 2 will
maximize its profit by choosing $Q2^{\ast}$. 
In other words, the pair $(Q1^{\ast}, Q2^{\ast})$ constitutes a
Nash equilibrium: no firm has an incentive to take unilateral deviations.

In order to compute the pair $(Q1
^{\ast}, Q1^{\ast})$, we need to solve equations 6 and 7.
However, a simple observation will simplify the computations. The two firms are
identical and, therefore, it must be that $Q1^{\ast}= Q2^{\ast}$. (more precisely, 6 and 7 are linear
equations symmetric in the indexes 1 and 2 ). Substituting the equation 

$Q1^{\ast}= Q2^{\ast}$
 into
equation 6 (or 7) we get:
$Q1^{\ast}= (120 - Q1^{\ast})/2$
which together with $Q1^{\ast}= Q2^{\ast}$ implies $Q1^{\ast}= Q2^{\ast} = 40$
Hence, substituting Q1
^{\ast}
= Q2
^{\ast}
 = 40 in the demand expression (equation 1)), we get
` p^{\ast}=140 -( Q1
^{\ast}
+ Q2
^{\ast}
 ) = 60
Profits are:
\[
A1^{\ast}  = p^{\ast} Q1^{\ast}- 20 Q1^{\ast}\\
 = 60^{\ast}40 - 20^{\ast}40\\ = 1,600
\]
\[A2
^{\ast}
 = p^{\ast} Q2
^{\ast}
- 20 Q2
^{\ast}
\\ = 60^{\ast}40 - 20^{\ast}40 \\ = 1,600\]
\end{document}