Game Theory

Dutch Auctions

English Auction

Winner Curse

Nash Equilibrium

%=========================================================================%
Topics covered are:
\begin{multicols}{2}
\begin{itemize}
\item Combinatorial games and Nim.

\item Game trees with perfect information, backward induction.

\item Extensive and strategic (normal) form of a game.

\item Nash equilibrium.

\item Commitment.

\item Mixed strategies and Nash equilibria in mixed strategies.

\item Finding mixed-strategy equilibria for twoperson games.

\item Zero sum games, maxmin strategies.

\item Extensive games with information sets,

\item behaviour strategies, perfect recall.

\item The Nash bargaining solution.

\item Multistage bargaining.
\end{itemize}

\end{multicols}

%=========================================================================%
\section{What is Game Theory}

Game theory is a branch of applied mathematics that uses models to study interactions with formalised incentive structures ("games"). 
Unlike decision theory, which also studies formalised incentive structures, game theory encompasses decisions that are made in an environment where various players interact strategically.
In other words, game theory studies choice of optimal behavior when costs and benefits of each option are not fixed, but depend upon the choices of other individuals.

Game theory has applications in a variety of fields, including economics,  international relations, evolutionary biology, political science, and  military strategy. Game theorists study the predicted and actual  behaviour of individuals in games, as well as optimal strategies.  Seemingly different situations can have similar incentive  structures, thus all exemplifying one particular game. 

%=========================================================================%
\subsection{Cooperative game}
A cooperative game is a game wherein two or more players do not compete, but rather strive toward a unique objective and therefore win or lose as a group.

Cooperative games are rare, but still many exist. One example is "Stand Up", where a number of individuals sit down, link arms (all facing away from each other) and attempt to stand up. This objective becomes more difficult as the number of players increases.

Another is the counting game, where the players, as a group, attempt to count to 20 with no two participants saying the same number twice. In a cooperative version of volleyball, the emphasis is on keeping the ball in the air for as long as possible.

Cooperative games are rare in recreational gaming, where conflict between players is a powerful force. However, such scenarios can occur in real life (when the sense of the word "game" is extended beyond recreational games). For example, operation of a successful business is, at least in theory, a cooperative game, since all participants benefit if the business succeeds and suffer if it fails.

Role-playing games are the most common form of cooperative game, though these games are not always purely cooperative. In such games, the players (who act through personae called "characters") usually strive toward intertwined and similar goals. However, each character has his or her own ambitions, and ultimately, individual goals. Hence conflict between characters often occurs in these games.

%=========================================================================%
\subsection{Axiomatic bargaining}
Two players may bargain how much share they want in a contract. The theory of axiomatic bargaining tells you how much share is reasonable for you. For example, Nash bargaining solution demands that the share is fair and efficient (see an advanced textbook for the complete formal description).

However, you may not be concerned with fairness and may demand more. How does Nash bargaining solution deal with this problem? Actually, there is a non-cooperative game of alternating offers (by Rubinstein) supporting Nash bargaining solution as the unique Nash equilibrium.


%=========================================================================%
% Game Theory

\subsection{Nash Equilibrium}

Nash Equilibrium recommnds a strategy to each player that the player cannot improve upon unilaterally, as long as the other players follow the recommendation.

Since the other olayers are assumed to be rational, it is reasonable to expectt the opponents to follow the recommendation as well.
Game Theory

%=========================================================================%
\subsection{Winner's Curse}
The winner's curse is a phenomenon that may occur in common value auctions with incomplete information. In short, the winner's curse says that in such an auction, the winner will tend to overpay. The winner may overpay or be "cursed" in one of two ways: 1) the winning bid exceeds the value of the auctioned asset such that the winner is worse off in absolute terms; or 2) the value of the asset is less than the bidder anticipated, so the bidder may still have a net gain but will be worse off than anticipated.[1] However, an actual overpayment will generally occur only if the winner fails to account for the winner's curse when bidding (an outcome that, according to the revenue equivalence theorem, need never occur).
