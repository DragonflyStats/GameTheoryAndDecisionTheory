Game Theory

Dutch Auctions

English Auction

Winner Curse

Nash Equilibrium

%=========================================================================%
Topics covered are:
\begin{multicols}{2}
\begin{itemize}
\item Combinatorial games and Nim.

\item Game trees with perfect information, backward induction.

\item Extensive and strategic (normal) form of a game.

\item Nash equilibrium.

\item Commitment.

\item Mixed strategies and Nash equilibria in mixed strategies.

\item Finding mixed-strategy equilibria for twoperson games.

\item Zero sum games, maxmin strategies.

\item Extensive games with information sets,

\item behaviour strategies, perfect recall.

\item The Nash bargaining solution.

\item Multistage bargaining.
\end{itemize}

\end{multicols}

%=========================================================================%
\section{What is Game Theory}

Game theory is a branch of applied mathematics that uses models to study interactions with formalised incentive structures ("games"). 
Unlike decision theory, which also studies formalised incentive structures, game theory encompasses decisions that are made in an environment where various players interact strategically.
In other words, game theory studies choice of optimal behavior when costs and benefits of each option are not fixed, but depend upon the choices of other individuals.

Game theory has applications in a variety of fields, including economics,  international relations, evolutionary biology, political science, and  military strategy. Game theorists study the predicted and actual  behaviour of individuals in games, as well as optimal strategies.  Seemingly different situations can have similar incentive  structures, thus all exemplifying one particular game. 

Game theory

Game theory is the science of strategic reasoning, in such a way that it studies the behaviour of rational game players who are trying to maximise their utility, profits, gains, etc., in interaction with other players, and therefore in a context of strategic interdependence.

 

\subsection{Historical Framework}

The initial game theory discussions and analysis can be traced back a long time before the 20th century, with works from authors such as James Waldegrave and his minimax mixed strategy solution for a two-person game in 1713, James Madison and his theoretical analysis of what would the effect of different tax systems be, or Antoine Cournot’s solution to a duopoly that resembles to what would later be known as Nash equilibria. However, it is not until the 20th century, that game theory is broadly developed. We can differentiate several periods in the game theory evolution through the 20th century. However, many concepts were continuously developed through the years.

During the earliest years of the century, from 1910 to 1930, the main focus of game theory was on  strictly competitive games commonly referred to as two-person zero-sum games. In this kind of games, the preferences (and payoffs) of a player will always be opposite to the other player’s and therefore cooperation between both players will be pointless. This kind of games has been extremely productive, as they have set the bases for future development of game theory, becoming its first main milestones. During these years, John von Neumann’s contribution was especially important, and thus, he is considered as one of the fathers of game theory.  In his “On the Theory of Parlor Games”, 1928, he introduced concepts such as the extensive (or tree) form for the description of sequential games and the minimax theorem, providing empirical support to everything he wrote. Another important contribution during these years is the introduction of the strategic form (game matrix) of a game that represents each player’s profile in a matrix form.


The following 20 years were heavily influenced by von Neumann and Oskar Morgenstern’s work, which would culminate in the joint publication of their book “Theory of Games and Economic Behaviour”, 1944.  With this book, game theory gained the status of an independent scientific discipline. One of the major contributions of this book is the application of strategy game concepts instead of random ones, the introduction of a notion of a cooperative game, its coalitional form, and the ‘Neumann-Morgenstern stable sets’. They were the first to extensively apply game theory for practical reasons, especially for analysing economic behaviour. It is also important to stress that the impact of this book has gone beyond its own time.  Many future developments emerged from it, such as the notion of core and transferable utility, and the further development of expected utility theory. Other developments from this period include games with continuum of pure strategies, the computation of minimax strategies and advances in mathematical methods that would be instrumental to later work.

The 1950s were a key period for game theory. The initial phase of incubation had reached its end and the economic perspective and application took the lead, overtaking other perspectives such as the military one. It was during this period when John Nash laid the essential groundwork that would result in the general non-cooperative theory and for cooperative bargaining theory; Lloyd Shapley was another key figure as he defined the value for coalitional games (cooperative games), he initiated the theory of stochastic games, co-invented the core with D.B. Gillies, and, along with John Milnor, developed the first game models with a continuum of players; Harold Kuhn worked on behaviour strategies and perfect recall. The prisoner’s dilemma, which is probably the best known game, was formalized by Al Tucker also during this period. It was also during these years, when the Folk theorem appeared as a result of the interest and study of repeated games, used to analyse punishment strategies in collusion agreements.

The period from 1960 to 1970 was also important in the development of game theory. The discipline expanded, not only theoretically, but also geographically. With the extension of games such as those with incomplete information and non-transferable utility coalitional games, game theory became more widely applicable, and new research centres were established outside the U. S.

From 1970 on, game theory acquired maturity. Up till then, most of game theory concepts and discoveries had been spread word to mouth. However, during this period economic theory journals published many articles related with game theory and even, new journals entirely dedicated to game theory became popular. The interest and study in politics and political economic models grew. The use of non-cooperative game theory to a large variety of economic models brought also the study and refinement of the equilibrium concept. Other advances were made in other areas of research such as complete and incomplete information repeated games, stochastic games, value, core, nucleolus bargaining theory, games with many players and etc. The use of game theory proved to be useful in a wide diversity of areas that include biology, computer science, moral philosophy and cost allocation, to name a few.

It is worth to note, that game theory should be regarded as a tool to find where incentives will lead, but it makes no moral recommendation to what choices should be taken. This science could be pictured as the study of selfishness, but without recommending it.  Welfare economics and its main tools (such as Pareto optimality and compensation criteria) study these problems outside the selfishness condition.
%=========================================================================%
\newpage
\subsection{Cooperative game}
A cooperative game is a game wherein two or more players do not compete, but rather strive toward a unique objective and therefore win or lose as a group.

Cooperative games are rare, but still many exist. One example is "Stand Up", where a number of individuals sit down, link arms (all facing away from each other) and attempt to stand up. This objective becomes more difficult as the number of players increases.

Another is the counting game, where the players, as a group, attempt to count to 20 with no two participants saying the same number twice. In a cooperative version of volleyball, the emphasis is on keeping the ball in the air for as long as possible.

Cooperative games are rare in recreational gaming, where conflict between players is a powerful force. However, such scenarios can occur in real life (when the sense of the word "game" is extended beyond recreational games). For example, operation of a successful business is, at least in theory, a cooperative game, since all participants benefit if the business succeeds and suffer if it fails.

Role-playing games are the most common form of cooperative game, though these games are not always purely cooperative. In such games, the players (who act through personae called "characters") usually strive toward intertwined and similar goals. However, each character has his or her own ambitions, and ultimately, individual goals. Hence conflict between characters often occurs in these games.

%=========================================================================%
\subsection{Axiomatic bargaining}
Two players may bargain how much share they want in a contract. The theory of axiomatic bargaining tells you how much share is reasonable for you. For example, Nash bargaining solution demands that the share is fair and efficient (see an advanced textbook for the complete formal description).

However, you may not be concerned with fairness and may demand more. How does Nash bargaining solution deal with this problem? Actually, there is a non-cooperative game of alternating offers (by Rubinstein) supporting Nash bargaining solution as the unique Nash equilibrium.


%=========================================================================%
% Game Theory

\subsection{Nash Equilibrium}

Nash Equilibrium recommnds a strategy to each player that the player cannot improve upon unilaterally, as long as the other players follow the recommendation.

Since the other olayers are assumed to be rational, it is reasonable to expectt the opponents to follow the recommendation as well.
Game Theory

%=========================================================================%
\subsection{Winner's Curse}
The winner's curse is a phenomenon that may occur in common value auctions with incomplete information. In short, the winner's curse says that in such an auction, the winner will tend to overpay. The winner may overpay or be "cursed" in one of two ways: 1) the winning bid exceeds the value of the auctioned asset such that the winner is worse off in absolute terms; or 2) the value of the asset is less than the bidder anticipated, so the bidder may still have a net gain but will be worse off than anticipated.[1] However, an actual overpayment will generally occur only if the winner fails to account for the winner's curse when bidding (an outcome that, according to the revenue equivalence theorem, need never occur).
