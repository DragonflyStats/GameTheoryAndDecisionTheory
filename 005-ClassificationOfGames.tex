% !TEX TS-program = pdflatex
% !TEX encoding = UTF-8 Unicode

% This is a simple template for a LaTeX document using the "article" class.
% See "book", "report", "letter" for other types of document.

\documentclass[11pt]{article} % use larger type; default would be 10pt

\usepackage[utf8]{inputenc} % set input encoding (not needed with XeLaTeX)

%%% Examples of Article customizations
% These packages are optional, depending whether you want the features they provide.
% See the LaTeX Companion or other references for full information.

%%% PAGE DIMENSIONS
\usepackage{geometry} % to change the page dimensions
\geometry{a4paper} % or letterpaper (US) or a5paper or....
% \geometry{margin=2in} % for example, change the margins to 2 inches all round
% \geometry{landscape} % set up the page for landscape
%   read geometry.pdf for detailed page layout information

\usepackage{graphicx} % support the \includegraphics command and options

% \usepackage[parfill]{parskip} % Activate to begin paragraphs with an empty line rather than an indent

%%% PACKAGES
\usepackage{booktabs} % for much better looking tables
\usepackage{array} % for better arrays (eg matrices) in maths
\usepackage{paralist} % very flexible & customisable lists (eg. enumerate/itemize, etc.)
\usepackage{verbatim} % adds environment for commenting out blocks of text & for better verbatim
\usepackage{subfig} % make it possible to include more than one captioned figure/table in a single float
% These packages are all incorporated in the memoir class to one degree or another...

%%% HEADERS & FOOTERS
\usepackage{fancyhdr} % This should be set AFTER setting up the page geometry
\pagestyle{fancy} % options: empty , plain , fancy
\renewcommand{\headrulewidth}{0pt} % customise the layout...
\lhead{}\chead{}\rhead{}
\lfoot{}\cfoot{\thepage}\rfoot{}

%%% SECTION TITLE APPEARANCE
\usepackage{sectsty}
\allsectionsfont{\sffamily\mdseries\upshape} % (See the fntguide.pdf for font help)
% (This matches ConTeXt defaults)

%%% ToC (table of contents) APPEARANCE
\usepackage[nottoc,notlof,notlot]{tocbibind} % Put the bibliography in the ToC
\usepackage[titles,subfigure]{tocloft} % Alter the style of the Table of Contents
\renewcommand{\cftsecfont}{\rmfamily\mdseries\upshape}
\renewcommand{\cftsecpagefont}{\rmfamily\mdseries\upshape} % No bold!

%%% END Article customizations
\begin{document}

\section*{1.3 Classification of Games}
Games can be categorized according to several criteria:
\begin{enumerate}
    \item 
How many players are there in the game? Usually there should be more than one player. However,
you can play roulette alone—the casino doesn’t count as player since it doesn’t make any decisions. It
collects or gives out money. Most books on game theory do not treat one-player games, but I will allow
them provided they contain elements of randomness.
\item Is play simultaneous or sequential? In a simultaneous game, each player has only one move, and all
moves are made simultaneously. \\ In a sequential game, no two players move at the same time, and
players may have to move several times. There are games that are neither simultaneous nor sequential.
\item Does the game have random moves? Games may contain random events that influence its outcome. They
are called random moves.
\item Do players have perfect information? A sequential game has perfect information if every player, when
about to move, knows all previous moves.
\item Do players have complete information? This means that all players know the structure of the game—the
order in which the players move, all possible moves in each position, and the payoffs for all outcomes.\\

Real-world games usually do not have complete information. In our games we assume complete information
in most cases, since games of incomplete information are more difficult to analyze.
\item Is the game zero-sum? Zero-sum games have the property that the sum of the payoffs to the players
equals zero. A player can have a positive payoff only if another has a negative payoff. Poker and chess
are examples of zero-sum games. Real-world games are rarely zero-sum.
\item Is communication permitted? Sometimes communication between the players is allowed before the game
starts and between the moves and sometimes it is not.
\item Is the game cooperative or non-cooperative? Even if players negotiate, the question is whether the results
of the negotiations can be enforced. If not, a player can always move differently from what was promised
in the negotiation. Then the communication is called “cheap talk”. A cooperative game is one where
the results of the negotiations can be put into a contract and be enforced. 
\end{enumerate}
There must also be a way of
distributing the payoff among the members of the coalition. I treat cooperative games in Chapter 35.

\end{document}