\section{Cournot Duopoly with Homogeneous items}


\subsubsection{Linear Demand and Linear Costs}

Let x1 and x2 be the quantities of homogeneous items produced by two firms with associated
costs C1(x1) = c1x1 and C2(x2) = c2x2 respectively.
Items sell at P = a − b(x1 + x2) each and it is assumed that all items produced are sold.
The profits made by the firms are then
π1 = P x1 − c1x1 = (a − c1 − b(x1 + x2)) x1
π2 = P x2 − c2x2 = (a − c2 − b(x1 + x2)) x2
respectively.
Let’s consider the specific example: P = 5−
x1+x2
2000 where the firms have costs C1(x1) = 2x1
and C(x2) = 2.5x2 respectively.
The profits made by the firms are then
π1 =

3 −
x1 + x2
2000 
x1
π2 =

2.5 −
x1 + x2
2000 
x2
respectively.
Maximising π1 with respect to x1
∂π1
∂x1
= 3 −
x1 + x2
2000
−
1
2000
x1
set = 0
⇒ x1 = 3000 −
1
2
x2 (1)
Similarly maximising π2 with respect to x2 yields
x2 = 2500 −
1
2
x1 (2)
Equations 1 and 2 are referred to as Reaction Functions or Best Response Functions -
provided their solutions are nonnegative, which I’ll assume in the following.
Solving equations 1 and 2 simultaneously gives the equilibrium values
x
∗
1 = 2333.33, x∗
2 = 1333.33
At these equilibrium values
P
∗ = 3.17
and
π
∗
1 = 2722.22, π∗
2 = 888.89 (3)
Cournot duopoly is an example of a 2-player matrix form game with an infinite number
of strategies available to both players (firms), i.e. the choice of x1 and x2 respectively.
hx
∗
1
, x∗
2
i is then a Nash equilibrium with payoffs π
∗
1
and π
∗
2
respectively.
%===========================================================%
\newpage
The Stackelberg leadership model is a strategic game in economics in which the leader firm moves first and then the follower firms move sequentially. It is named after the German economist Heinrich Freiherr von Stackelberg who published Market Structure and Equilibrium (Marktform und Gleichgewicht) in 1934 which described the model.

In game theory terms, the players of this game are a leader and a follower and they compete on quantity. The Stackelberg leader is sometimes referred to as the Market Leader.

Firms may engage in Stackelberg competition if one has some sort of advantage enabling it to move first. More generally, the leader must have commitment power. Moving observably first is the most obvious means of commitment: once the leader has made its move, it cannot undo it - it is committed to that action. Moving first may be possible if the leader was the incumbent monopoly of the industry and the follower is a new entrant. Holding excess capacity is another means of commitment.


The profit of firm 2 (the follower) is revenue minus cost. Revenue is the product of price and quantity and cost is given by the firm's cost structure, so profit is: \[{\displaystyle \Pi _{2}=P(q_{1}+q_{2})\cdot q_{2}-C_{2}(q_{2})} \Pi _{2}=P(q_{1}+q_{2})\cdot q_{2}-C_{2}(q_{2}).\] The best response is to find the value of {\displaystyle q_{2}} q_{2} that maximises {\displaystyle \Pi _{2}} \Pi _{2} given {\displaystyle q_{1}} q_{1}, i.e. given the output of the leader (firm 1), the output that maximises the follower's profit is found. Hence, the maximum of {\displaystyle \Pi _{2}} \Pi _{2} with respect to {\displaystyle q_{2}} q_{2} is to be found. First differentiate {\displaystyle \Pi _{2}} \Pi _{2} with respect to {\displaystyle q_{2}} q_{2}:

\[{\displaystyle {\frac {\partial \Pi _{2}}{\partial q_{2}}}={\frac {\partial P(q_{1}+q_{2})}{\partial q_{2}}}\cdot q_{2}+P(q_{1}+q_{2})-{\frac {\partial C_{2}(q_{2})}{\partial q_{2}}}.} {\frac  {\partial \Pi _{2}}{\partial q_{2}}}={\frac  {\partial P(q_{1}+q_{2})}{\partial q_{2}}}\cdot q_{2}+P(q_{1}+q_{2})-{\frac  {\partial C_{2}(q_{2})}{\partial q_{2}}}\].
Setting this to zero for maximisation:

\[{\displaystyle {\frac {\partial \Pi _{2}}{\partial q_{2}}}={\frac {\partial P(q_{1}+q_{2})}{\partial q_{2}}}\cdot q_{2}+P(q_{1}+q_{2})-{\frac {\partial C_{2}(q_{2})}{\partial q_{2}}}=0.} {\frac  {\partial \Pi _{2}}{\partial q_{2}}}={\frac  {\partial P(q_{1}+q_{2})}{\partial q_{2}}}\cdot q_{2}+P(q_{1}+q_{2})-{\frac  {\partial C_{2}(q_{2})}{\partial q_{2}}}=0.\]
The values of {\displaystyle q_{2}} q_{2} that satisfy this equation are the best responses. Now the best response function of the leader is considered. This function is calculated by considering the follower's output as a function of the leader's output, as just computed.

The profit of firm 1 (the leader) is {\displaystyle \Pi _{1}=P(q_{1}+q_{2}(q_{1})).q_{1}-C_{1}(q_{1})} \Pi _{1}=P(q_{1}+q_{2}(q_{1})).q_{1}-C_{1}(q_{1}), where {\displaystyle q_{2}(q_{1})} q_{2}(q_{1}) is the follower's quantity as a function of the leader's quantity, namely the function calculated above. The best response is to find the value of {\displaystyle q_{1}} q_{1} that maximises {\displaystyle \Pi _{1}} \Pi _{1} given {\displaystyle q_{2}(q_{1})} q_{2}(q_{1}), i.e. given the best response function of the follower (firm 2), the output that maximises the leader's profit is found. Hence, the maximum of {\displaystyle \Pi _{1}} \Pi _{1} with respect to {\displaystyle q_{1}} q_{1} is to be found. First, differentiate {\displaystyle \Pi _{1}} \Pi _{1} with respect to {\displaystyle q_{1}} q_{1}:

{\displaystyle {\frac {\partial \Pi _{1}}{\partial q_{1}}}={\frac {\partial P(q_{1}+q_{2})}{\partial q_{2}}}\cdot {\frac {\partial q_{2}(q_{1})}{\partial q_{1}}}\cdot q_{1}+{\frac {\partial P(q_{1}+q_{2})}{\partial q_{1}}}\cdot q_{1}+P(q_{1}+q_{2}(q_{1}))-{\frac {\partial C_{1}(q_{1})}{\partial q_{1}}}.} {\frac  {\partial \Pi _{1}}{\partial q_{1}}}={\frac  {\partial P(q_{1}+q_{2})}{\partial q_{2}}}\cdot {\frac  {\partial q_{2}(q_{1})}{\partial q_{1}}}\cdot q_{1}+{\frac  {\partial P(q_{1}+q_{2})}{\partial q_{1}}}\cdot q_{1}+P(q_{1}+q_{2}(q_{1}))-{\frac  {\partial C_{1}(q_{1})}{\partial q_{1}}}.
Setting this to zero for maximisation:

{\displaystyle {\frac {\partial \Pi _{1}}{\partial q_{1}}}={\frac {\partial P(q_{1}+q_{2})}{\partial q_{2}}}\cdot {\frac {\partial q_{2}(q_{1})}{\partial q_{1}}}\cdot q_{1}+{\frac {\partial P(q_{1}+q_{2})}{\partial q_{1}}}\cdot q_{1}+P(q_{1}+q_{2}(q_{1}))-{\frac {\partial C_{1}(q_{1})}{\partial q_{1}}}=0.} {\frac  {\partial \Pi _{1}}{\partial q_{1}}}={\frac  {\partial P(q_{1}+q_{2})}{\partial q_{2}}}\cdot {\frac  {\partial q_{2}(q_{1})}{\partial q_{1}}}\cdot q_{1}+{\frac  {\partial P(q_{1}+q_{2})}{\partial q_{1}}}\cdot q_{1}+P(q_{1}+q_{2}(q_{1}))-{\frac  {\partial C_{1}(q_{1})}{\partial q_{1}}}=0.


\subsection{Stackelberg Duopoly}
Stackelberg duopoly is an example of a 2-player extensive form game in which Firm 1
moves first (the “Leader”) and Firm 2 responds (the “Follower”). Irrespective of what
the leader does, the follower will use the reaction function (Eq. 2) as it is its best response.
Knowing this, the leader seeks to maximise
Π1 =

3 −

x1 + 2500 −
1
2
x1

2000 !
x1 =

1.75 −
x1
4000

x1
as a function of x1.
dΠ1
dx1
= 1.75 −
x1
2000
set = 0
⇒ x1 = 3500 (4)
Denoting this optimal value by X∗
1
and the corresponding value of x2 by X∗
2
(substitute
Eq. 4 into Eq. 2) gives
X
∗
1 = 3500, X∗
2 = 750
At these equilibrium values
P
∗ = 2.875
and
Π
∗
1 = 3062.5, Π
∗
2 = 281.25 (5)

\subsection{Exercises}
Exercise: Redo the analysis if Firm 2 is the leader?
