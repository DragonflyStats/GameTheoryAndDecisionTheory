Cournot Duopoly with Homogeneous items: Linear Demand and Linear
Costs
Let x1 and x2 be the quantities of homogeneous items produced by two firms with associated
costs C1(x1) = c1x1 and C2(x2) = c2x2 respectively.
Items sell at P = a − b(x1 + x2) each and it is assumed that all items produced are sold.
The profits made by the firms are then
π1 = P x1 − c1x1 = (a − c1 − b(x1 + x2)) x1
π2 = P x2 − c2x2 = (a − c2 − b(x1 + x2)) x2
respectively.
Let’s consider the specific example: P = 5−
x1+x2
2000 where the firms have costs C1(x1) = 2x1
and C(x2) = 2.5x2 respectively.
The profits made by the firms are then
π1 =

3 −
x1 + x2
2000 
x1
π2 =

2.5 −
x1 + x2
2000 
x2
respectively.
Maximising π1 with respect to x1
∂π1
∂x1
= 3 −
x1 + x2
2000
−
1
2000
x1
set = 0
⇒ x1 = 3000 −
1
2
x2 (1)
Similarly maximising π2 with respect to x2 yields
x2 = 2500 −
1
2
x1 (2)
Equations 1 and 2 are referred to as Reaction Functions or Best Response Functions -
provided their solutions are nonnegative, which I’ll assume in the following.
Solving equations 1 and 2 simultaneously gives the equilibrium values
x
∗
1 = 2333.33, x∗
2 = 1333.33
At these equilibrium values
P
∗ = 3.17
and
π
∗
1 = 2722.22, π∗
2 = 888.89 (3)
Cournot duopoly is an example of a 2-player matrix form game with an infinite number
of strategies available to both players (firms), i.e. the choice of x1 and x2 respectively.
hx
∗
1
, x∗
2
i is then a Nash equilibrium with payoffs π
∗
1
and π
∗
2
respectively.
Stackelberg Duopoly
Stackelberg duopoly is an example of a 2-player extensive form game in which Firm 1
moves first (the “Leader”) and Firm 2 responds (the “Follower”). Irrespective of what
the leader does, the follower will use the reaction function (Eq. 2) as it is its best response.
Knowing this, the leader seeks to maximise
Π1 =

3 −

x1 + 2500 −
1
2
x1

2000 !
x1 =

1.75 −
x1
4000

x1
as a function of x1.
dΠ1
dx1
= 1.75 −
x1
2000
set = 0
⇒ x1 = 3500 (4)
Denoting this optimal value by X∗
1
and the corresponding value of x2 by X∗
2
(substitute
Eq. 4 into Eq. 2) gives
X
∗
1 = 3500, X∗
2 = 750
At these equilibrium values
P
∗ = 2.875
and
Π
∗
1 = 3062.5, Π
∗
2 = 281.25 (5)
Exercise: Redo the analysis if Firm 2 is the leader?
