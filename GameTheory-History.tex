\documentclass[]{report}
\voffset=-1.5cm
\oddsidemargin=0.0cm
\textwidth = 480pt


\usepackage{amsmath}
\usepackage{graphicx}
\usepackage{amssymb}
\usepackage{framed}
\usepackage{multicol}
%\usepackage[paperwidth=21cm, paperheight=29.8cm]{geometry}
%\usepackage[angle=0,scale=1,color=black,hshift=-0.4cm,vshift=15cm]{background}
%\usepackage{multirow}
\usepackage{enumerate}

\usepackage{amsmath,amsfonts,amssymb}
\usepackage{color}
\usepackage{multirow}
\usepackage{eurosym}
\usepackage{framed}

%\input def.tex
%\input dsdef.tex
%\input rgb.tex

%\newcommand \la{\lambda}
%\newcommand \al{a}
%\newcommand \be{b}
\newcommand \x{\overline{x}}
\newcommand \y{\overline{y}}

\begin{document}
	
\subsection{Historical Framework}

The initial game theory discussions and analysis can be traced back a long time before the 20th century, with works from authors such as James Waldegrave and his minimax mixed strategy solution for a two-person game in 1713, James Madison and his theoretical analysis of what would the effect of different tax systems be, or Antoine Cournot’s solution to a duopoly that resembles to what would later be known as Nash equilibria. However, it is not until the 20th century, that game theory is broadly developed. We can differentiate several periods in the game theory evolution through the 20th century. However, many concepts were continuously developed through the years.

During the earliest years of the century, from 1910 to 1930, the main focus of game theory was on  strictly competitive games commonly referred to as two-person zero-sum games. In this kind of games, the preferences (and payoffs) of a player will always be opposite to the other player’s and therefore cooperation between both players will be pointless. This kind of games has been extremely productive, as they have set the bases for future development of game theory, becoming its first main milestones. During these years, John von Neumann’s contribution was especially important, and thus, he is considered as one of the fathers of game theory.  In his “\textit{On the Theory of Parlor Games}”, 1928, he introduced concepts such as the extensive (or tree) form for the description of sequential games and the minimax theorem, providing empirical support to everything he wrote. Another important contribution during these years is the introduction of the strategic form (game matrix) of a game that represents each player’s profile in a matrix form.


The following 20 years were heavily influenced by von Neumann and Oskar Morgenstern’s work, which would culminate in the joint publication of their book “\textit{Theory of Games and Economic Behaviour}”, 1944.  With this book, game theory gained the status of an independent scientific discipline. One of the major contributions of this book is the application of strategy game concepts instead of random ones, the introduction of a notion of a cooperative game, its coalitional form, and the ‘Neumann-Morgenstern stable sets’. They were the first to extensively apply game theory for practical reasons, especially for analysing economic behaviour. It is also important to stress that the impact of this book has gone beyond its own time.  

Many future developments emerged from it, such as the notion of core and transferable utility, and the further development of expected utility theory. Other developments from this period include games with continuum of pure strategies, the computation of minimax strategies and advances in mathematical methods that would be instrumental to later work.

The 1950s were a key period for game theory. The initial phase of incubation had reached its end and the economic perspective and application took the lead, overtaking other perspectives such as the military one. It was during this period when John Nash laid the essential groundwork that would result in the general non-cooperative theory and for cooperative bargaining theory; Lloyd Shapley was another key figure as he defined the value for coalitional games (cooperative games), he initiated the theory of stochastic games, co-invented the core with D.B. Gillies, and, along with John Milnor, developed the first game models with a continuum of players; Harold Kuhn worked on behaviour strategies and perfect recall. The prisoner’s dilemma, which is probably the best known game, was formalized by Al Tucker also during this period. It was also during these years, when the Folk theorem appeared as a result of the interest and study of repeated games, used to analyse punishment strategies in collusion agreements.

The period from 1960 to 1970 was also important in the development of game theory. The discipline expanded, not only theoretically, but also geographically. With the extension of games such as those with incomplete information and non-transferable utility coalitional games, game theory became more widely applicable, and new research centres were established outside the U. S.

From 1970 on, game theory acquired maturity. Up till then, most of game theory concepts and discoveries had been spread word to mouth. However, during this period economic theory journals published many articles related with game theory and even, new journals entirely dedicated to game theory became popular. The interest and study in politics and political economic models grew. The use of non-cooperative game theory to a large variety of economic models brought also the study and refinement of the equilibrium concept. Other advances were made in other areas of research such as complete and incomplete information repeated games, stochastic games, value, core, nucleolus bargaining theory, games with many players and etc. The use of game theory proved to be useful in a wide diversity of areas that include biology, computer science, moral philosophy and cost allocation, to name a few.

It is worth to note, that game theory should be regarded as a tool to find where incentives will lead, but it makes no moral recommendation to what choices should be taken. This science could be pictured as the study of selfishness, but without recommending it.  Welfare economics and its main tools (such as Pareto optimality and compensation criteria) study these problems outside the selfishness condition.

\end{document}