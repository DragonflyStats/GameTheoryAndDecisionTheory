\documentclass[]{report}
\voffset=-1.5cm
\oddsidemargin=0.0cm
\textwidth = 480pt


\usepackage{amsmath}
\usepackage{graphicx}
\usepackage{amssymb}
\usepackage{framed}
\usepackage{multicol}
%\usepackage[paperwidth=21cm, paperheight=29.8cm]{geometry}
%\usepackage[angle=0,scale=1,color=black,hshift=-0.4cm,vshift=15cm]{background}
%\usepackage{multirow}
\usepackage{enumerate}

\usepackage{amsmath,amsfonts,amssymb}
\usepackage{color}
\usepackage{multirow}
\usepackage{eurosym}
\usepackage{framed}

%\input def.tex
%\input dsdef.tex
%\input rgb.tex

%\newcommand \la{\lambda}
%\newcommand \al{a}
%\newcommand \be{b}
\newcommand \x{\overline{x}}
\newcommand \y{\overline{y}}

\begin{document}

\section{Stackelberg duopoly}
%- http://policonomics.com/stackelberg-duopoly-model/
Stackelberg duopoly, also called Stackelberg competition, is a model of imperfect competition based on a non-cooperative game. It was developed in 1934 by Heinrich Stackelbelrg in his \textit{“Market Structure and Equilibrium”} and represented a breaking point in the study of market structure, particularly the analysis of duopolies, since it was a model based on different starting assumptions and gave different conclusions to those of the Cournot’s and Bertrand’s duopoly models.

\begin{itemize}
\item In game theory, a Stackelberg duopoly is a \textbf{sequential} game (not simultaneous as in Cournot’s model). There are two firms, which sell homogeneous products, and are subject to the same demand and cost functions. 
\item One firm, the \textbf{leader}, is perhaps better known or has greater brand equity, and is therefore better placed to decide first which quantity $q_1$ to sell.
\item The other firm, the \textbf{follower}, observes this and decides on its production quantity q2. 
\end{itemize}

\subsection*{Solving Stackleberg Problems}
\begin{itemize}

\item To find the Nash equilibrium of the game we need to use backward induction, as in any sequential game. That is, start analyzing the decision of the follower.

\item 

Therefore, the quantities sold by each firm at equilibrium are:
\[ Equation \]

Formula - Stackelberg duopoly - Equilibrium quantity	Formula - Stackelberg duopoly - Equilibrium quantity 2


\item The perfect equilibrium of the game is the \textbf{Stackelberg equilibrium}. In this game, the leader has decided not to behave as in the Cournot’s model, however, we cannot ensure that the leader is going to produce more and make more profits than the follower (production will be larger for the firm with lower marginal costs). 
\item Total production will be greater and prices lower, but player one will be better off than player two, which serves to highlight two things: the importance of accurate market information when defining a strategy, and the interdependence of each player’s strategies, especially when there is a market leader (with the benefit of moving first) and a follower.
%\item  When it comes to economic efficiency, the result is similar to Cournot’s duopoly model. 
%\item The Nash equilibrium is not Pareto efficient (isoprofit curves, green curves, are not tangent to each other) and therefore, there is a loss in economic efficiency. Nevertheless, the loss is lower in the Stackelberg duopoly than in Cournot’s.
\item 
Stackelberg and Cournot equilibria are stable in a static model of just one period. In a dynamic context (repeated games), the models need to be reconsidered.
\end{itemize}
\end{document}