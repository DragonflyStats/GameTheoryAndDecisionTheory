% !TEX TS-program = pdflatex
% !TEX encoding = UTF-8 Unicode

% This is a simple template for a LaTeX document using the "article" class.
% See "book", "report", "letter" for other types of document.

\documentclass[11pt]{article} % use larger type; default would be 10pt

\usepackage[utf8]{inputenc} % set input encoding (not needed with XeLaTeX)

%%% Examples of Article customizations
% These packages are optional, depending whether you want the features they provide.
% See the LaTeX Companion or other references for full information.

%%% PAGE DIMENSIONS
\usepackage{geometry} % to change the page dimensions
\geometry{a4paper} % or letterpaper (US) or a5paper or....
% \geometry{margin=2in} % for example, change the margins to 2 inches all round
% \geometry{landscape} % set up the page for landscape
%   read geometry.pdf for detailed page layout information

\usepackage{graphicx} % support the \includegraphics command and options

% \usepackage[parfill]{parskip} % Activate to begin paragraphs with an empty line rather than an indent

%%% PACKAGES
\usepackage{booktabs} % for much better looking tables
\usepackage{array} % for better arrays (eg matrices) in maths
\usepackage{paralist} % very flexible & customisable lists (eg. enumerate/itemize, etc.)
\usepackage{verbatim} % adds environment for commenting out blocks of text & for better verbatim
\usepackage{subfig} % make it possible to include more than one captioned figure/table in a single float
% These packages are all incorporated in the memoir class to one degree or another...

%%% HEADERS & FOOTERS
\usepackage{fancyhdr} % This should be set AFTER setting up the page geometry
\pagestyle{fancy} % options: empty , plain , fancy
\renewcommand{\headrulewidth}{0pt} % customise the layout...
\lhead{}\chead{}\rhead{}
\lfoot{}\cfoot{\thepage}\rfoot{}

%%% SECTION TITLE APPEARANCE
\usepackage{sectsty}
\allsectionsfont{\sffamily\mdseries\upshape} % (See the fntguide.pdf for font help)
% (This matches ConTeXt defaults)

%%% ToC (table of contents) APPEARANCE
\usepackage[nottoc,notlof,notlot]{tocbibind} % Put the bibliography in the ToC
\usepackage[titles,subfigure]{tocloft} % Alter the style of the Table of Contents
\renewcommand{\cftsecfont}{\rmfamily\mdseries\upshape}
\renewcommand{\cftsecpagefont}{\rmfamily\mdseries\upshape} % No bold!

%%% END Article customizations
\begin{document}
%%% The "real" document content comes below...



\section{Introduction}
 Consider $n \geq 2$ firms which produce
an identical product. Let i q be the quantity produced by Firm i (i =1,…,n). For
Firm i the cost of producing i q units of output is i i c q , where i c is a positive
constant. 

For simplicity we will restrict attention to the case of two firms (n = 2)
and identical cost functions: $c1 = c2 = c$ . 

Let $Q$ be total industry output, that is,
1 2 Q = q + q . 

%=========================================================%

The price at which each firm can sell each unit of output is given by
the inverse demand function $P = a − bQ$ where a and b are positive constants.

Cournot assumed that each firm was only interested in its own profit and preferred higher profit to lower profit (that is, each firm is “selfish and greedy”). 

The profit function of Firm 1 is given by
[ ] 2
1 1 2 1 1 1 2 1 1 1 1 1 2 π (q ,q ) = Pq − cq = a − b(q + q ) q − cq = (a − c)q −b(q ) −bq q .
Similarly, the profit function of Firm 2 is given by
2
2 1 2 2 2 1 2 π (q ,q ) = (a − c)q − b(q ) −bq q
Cournot defined an equilibrium as a pair ( ) 1 2 q , q such that
( ) ( )
( ) ( )
1 1 2 1 1 2 1
2 1 2 2 1 2 2
, , , for every 0 ( )
and
, , , for every 0 ( )
q q q q q
q q q q q
π π
π π
≥ ≥
≥ ≥
♣
♦
%=========================================================%

Of course, this is the same as saying that ( ) 1 2 q , q is a Nash equilibrium of the game
where the players are the two firms, the strategy sets are 1 2 S = S = [0,∞) and the
payoff functions are the profit functions. How do we find a Nash equilibrium?
First of all, note that the profit functions are differentiable. Secondly note that (♣)
says that, having fixed the value of 2 2 q at q , the function ( ) 1 1 2 π q ,q viewed as a
function of 1 q alone is maximized at the point 1 1 q = q . A necessary condition for
this (if 1 q > 0 ) is that the derivative of this function be zero at the point 1 1 q = q ,
that is, it must be that 1 ( )
1 2
1
q , q 0.
q
∂π
=
∂

%==================================================%
% GAME THEORY – Giacomo Bonanno
% 39
This condition is also sufficient since the

second derivative of this function is always negative ( ( ) 2
1
2 1 2
1
q ,q 2b
q
∂ π
= −
∂
for every
1 2 (q ,q ) ). Similarly, by (♦) , it must be that 2 ( )
1 2
2
q , q 0.
q
∂π
=
∂
Thus the Nash
equilibrium is found by solving the system of two equations
1 2
2 1
2 0
2 0
a c bq bq
a c bq bq
− − − = 

 − − − =
. The solution is 1 2 3
a c
q q
b
−
= = . The corresponding price is
2
2
3 3
a c a c
P a b
b
 −  + = −   =
 
and the corresponding profits are
2
1 3 3 2 3 3
( )
( , ) ( , )
9
a c a c a c a c
b b b b
a c
b
π − − π − − −
= = . For example, if a = 25, b = 2, c =1 then the
Nash equilibrium is given by (4,4) with corresponding profits of 32 for each firm.

%==========================================================%
The analysis can easily be extended to the case of more than two firms. The reader
who is interested in further exploring the topic of competition among firms can
consult any textbook on Industrial Organization.
This is a good time to test your understanding of the concepts introduced in
this section, by going through the exercises in Section 1.E.7 of Appendix 1.E at the
end of this chapter.

\end{document}
