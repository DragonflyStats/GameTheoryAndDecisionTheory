\section{Game Theory}
Game Theory (GT) is the study of strategic interdependence. The typical “game” consists
of players, actions, strategies and payoffs. The standard modes of analysis are once-played
games in either
\begin{itemize}
\item[(i)] matrix or strategic form where players’ choice of actions are made simultaneously,
or
\item[(ii)] extensive form where choice of actions are made sequentially.
\end{itemize}
1 Analysis of (finite) Matrix Games
One of the best known games is
Prisoner’s Dilemma (PD)
Player 2
Keep Quiet Confess
Player 1 Keep Quiet (-1,-1) (-12,0)
Confess (0,-12) (-8,-8)
The solution is h confess, confess i. (See IESDS).
Why don’t the players coordinate to get h Keep Quiet, Keep Quiet i ?
The exact payoffs are irrelevant, the game can also be represented by the order of players’
preferences - most preferred (p1) to least preferred (p4) :
Player 2
Keep Quiet Confess
Player 1 Keep Quiet (p2,p2) (p4,p1)
Confess (p1,p4) (p3,p3)

Other examples of PD-like games are

\begin{itemize}
\item War with strategies Defend, Attack respectively.
\item Arms Race with strategies Pass, Build respectively.
\item Free Trade/ Protection with strategies No Tax, Tax respectively.
\item Advertising with strategies No Ads, Ads respectively.
\end{itemize}

Deadlock is another game ( success is to fail!):
Player 2
Try Fail
Player 1 Try (0,0) (-1,1)
Fail (1,-1) (0,0)
Using preferences, we can consider the more general version of deadlock
Player 2
Left Right
Player 1 Up (p2,p2) (p1,p4)
Down (p4,p1) (p3,p3)
The solution is h Up, Left i. (See IESDS). Neither player gets hir first choice unless the
other makes a mistake.

\subsection{1.1 Strict Dominance}
Strategy X strictly dominates strategy Y for a player if X gives a bigger (more preferred)
payoff than Y no matter what the other players do. Players never rationally choose strictly
dominated strategies.
Reduce the matrix by Iterated Elimination of Strictly Dominated Strategies
(IESDS) - see PD and Deadlock above. The order of elimination is irrelevant. Another
example is the Dance Club game:
Boon docks
Salsa Hip Hop
Downtown Salsa (80,0) (60,40)
Hip Hop (40,60) (40,0)

The solution is h Salsa, Hip Hop i. Salsa dominates Hip Hop for Club Downtown, then
Boonies choses Hip Hop.
Some more examples
Player 2
Left Centre Right
Player 1
Up (13,3) (1,4) (7, 3)
Middle (4,1) (3,3) (6,2)
Down (-1,9) (2,8) (8,-1)
Denoting strict dominance by >, then in the order given C > R, M > D, C > L, M > U.
Thus the solution is h Middle, Centre i.

Cournot Duopoly game. Firm 1 can produce i units at a cost of e1 each. Similarly firm
2 can produce j units at a cost of e1 each. The units sell on the market at a price of
e[8−2(i+j)]+ each, where [.]
+ is the positive part of [.] The payoff to firm 1 is the profit
gained which is e[8−2(i+j)]+i−i. Similarly the payoff to firm 2 is e[8−2(i+j)]+j −j.
The game matrix is
Firm 2
j = 0 j = 1 j = 2 j = 3
Firm 1
i = 0 (0,0) (0,5) (0, 6) (0,1)
i = 1 (5,0) (3,3) (1,2) (-1,-3)
i = 2 (6,0) (2,1) (-2,-2) (-2,-3)
i = 3 (1,0) (-3,-1) (-3,-2) (-3,-3)
The solution is hi = 1, j = 1i. What is the sequence of eliminations that leads to this?
All routes lead to the same result (proof?):
Player 2
Left Right
Player 1
Up (1,-1) (4,2)
Middle (0,2) (3,3)
Down (-2,-2) (2,-1)
The solution is h Up, Right i.
%=====================================================%
\subsection{1.2 Weak Dominance}
Strategy X weakly dominates strategy Y for a player if X gives at least as big a payoff as Y
no matter what the other players do and there is one at least one X payoff than is strictly
greater than the corresponding Y payoff.
Iterated Elimination of Weakly Dominated Strategies (IEWDS) is in general not
a valid technique. Consider the game:
Player 2
Left Right
Player 1
Up (0,1) (-4,2)
Middle (0,3) (3,3)
Down (-2,2) (3,-1)
If we proceed as before and denoting weak dominance by ≥, we might argue that M ≥
U, L ≥ R. The solution is then h Middle, Left i. Alternatively we might argue that M ≥
D, R ≥ L which leads to the solution h Middle, Right i.
1.3 Best Response & Nash Equilibrium
Stag Hunt(SH) - it requires cooperation to catch a stag!
Player 2
Stag Hare
Player 1 Stag (3∗
, 3
∗
) (0,2)
Hare (2,0) (1∗
, 1
∗
)
For this game there is no SDS nor WDS.
A Nash equilibrium (NE) is a set of strategies, one for each player, from which there
is no incentive for any one player to deviate if all the other players play these strategies,
i.e. no player can gain by changing, also called a “No regrets” choice. The Best Response
of a player to another player’s choice of strategy is the strategy that gives the largest or
best payoff. We’ll denote this by placing an ∗ beside the payoff, e.g. in the stag game
above , “stag” with associated payoff 3 is the best response of player 1 to player 2 playing
“stag”. Hence if both parts of a payoff pair have asterisks beside them, this must be a pure
strategy Nash equilibrium (PSNE) - a pair of strategies where both players are playing
deterministic strategies as opposed to a mixed strategy Nash equilibrium (MSNE), where
players are randomly mixing between the strategies available to them.
Thus in the stag game, the PSNE solutions are h stag, stag i and h hare, hare i. Notice
that without efficient coordination, either solution is possible.
Consider the “Good buddies prisoner’s dilemma” game:
Player 2
Keep Quiet Confess
Player 1 Keep Quiet (p1,p1) (p4,p2)
Confess (p2,p4) (p3,p3)
This is identical to the stag hunt game.
An alternative definition of a NE is “mutual best response”. Some further examples:
Traffic Lights(TL)
Car 2
Go Stop
Car 1 Go (-5,-5) (1,0)
Stop (0,1) (-1,-1)
The PSNE solutions are h go, stop i and h stop, go i.
Generals, Armies & Battles. Each general commands 3 armies. No battle occurs if either
general puts 0 armies in the field. Otherwise the general with more armies wins the day.
With i andj standing for the number of armies of General 1 and General 2 respectively
in the battlefield, one possible game matrix is
General 2
j = 0 j = 1 j = 2 j = 3
General 1
i = 0 (0,0) (0,0) (0, 0) (0,0)
i = 1 (0,0) (0,0) (-1,1) (-1,1)
i = 2 (0,0) (1,-1) (0,0) (-1,1)
i = 3 (0,0) (1,-1) (1,-1) (0,0)
What are the PSNE(s) ?
1.4 Dominance & NE
If IESDS results in a unique solution then it is a (unique) NE. [proof by appeal to “no
regrets”]. IESDS does not remove any NE.
IEWDS may lose NE. It is necessary to check using e.g. Best Responses. If you have a
choice eliminate SDS before WDS. Examples:
• The example of Section 1.2 has two NE, each obtained by a different sequence of
IEWDS.
• Consider the game
Player 2
Left Right
Player 1 Up (2,3) (4,3)
Down (3,3) (1,1)
Using IEWDS, L ≥ R, then D > U. Hence the solution is h Down, Left i. But using
Best Responses, another NE is h Up, Right i.
• The game
Player 2
Left Centre Right
Player 1
Up (2,2) (4,2) (4, 3)
Middle (2,4) (5,5) (7,3)
Down (3,4) (3,7) (6,6)
Using IEWDS C ≥ L, then M > U, M > D and C> R. Hence the solution is
h Middle, Centre i. This is the only NE.
%----------------------------------------------------------%
\subsection{1.5 MSNE}
Matching Pennies(MP) is an example of a game with no PSNE.
Player 2
Heads Tails
Player 1 Heads (1,-1) (-1,1)
Tails (-1,1) (1,-1)

Other names for this game are
\begin{itemize}
\item Goalkeeeper v. Penalty Taker
\item Offense v. Defense (American Football)
\item Fastball v. Curveball (Baseball)
\item Attack A or B v. Defend A or B
\end{itemize}
Nash proved that every finite 1 game has a NE.
This game has the MSNE h 1/2 Heads + 1/2 Tails, 1/2 Heads + 1/2 Tails i.
More generally, MSNEs can be found using the Bishop-Cannings theorem. Consider the
following game:
Player 2
Left Right
Player 1 Up (3,-3) (-2,2)
Down (-1,1) (0,0)
Again note that there is no PSNE. We’ll use the notation R1hs1, s2i and R2hs1, s2i to
stand for the payoffs to Player 1 and 2 respectively when Player 1 plays strategy s1 and
Player 2 strategy s2.
In the mixed strategy game let p be the probability that Player 1 plays Up, and similarly
q the probability that Player 2 plays Left. Then the payoff to player 2 by playing Left
against Player 1’s mixed strategy is
R2hpUp + (1 − p)Down, Lefti = −3p + 1(1 − p)
Again the payoff to player 2 by playing Right against Player 1’s mixed strategy is
R2hpUp + (1 − p)Down, Righti = 2p + 0(1 − p)

If the two payoff values are different then Player 2 has a definite preference between the two and so would not want to randomise. If for instance Player 2 were to choose Left, then Player 1 would abandon his mixed strategy and choose Up instead. Similarly if Player 2
were to choose Right instead, Player 1 would change from randomising to playing Down.

In either case, the mixed strategies go out the window and no NE exists. To get a MSNE requires that Player 2 be indifferent between the two payoffs i.e. requires that the two payoffs be the same and since the the payoff at any mixed strategy is a linear combination
of the pure strategy payoffs, this must also have the same value. Equating the two payoffs
gives
−3p + 1(1 − p) = 2p + 0(1 − p)
⇒ p = 1/6
and the value of the payoff is
v2 = R2

1
6
Up + 5
6
Down, Left or Right
=
1
3
Similarly the payoffs to Player 1 by playing Up (respectively Down) against Player 2’s
mixed strategy are
R1hUp, qLeft + (1 − q)Righti = 3q + (−2)(1 − q)
R1hDown, qLeft + (1 − q)Righti = (−1)q + 0(1 − q)
1finite number of players, finite number of pure strategies
respectively. To be indifferent between the two requires
3q + (−2)(1 − q) = (−1)q + 0(1 − q)
⇒ q = 1/3
and the value of the payoff is
v1 = R1

Up or Down,
1
3
Left + 2
3
Right
= −
1
3
Exercise: Show that the stag hunt game has a MSNE at h 1/2 stag + 1/2 hare, 1/2 stag
+ 1/2 hare i.
We can have partial MSNE where (at least) one player has a pure strategy and (at least)
one player has a mixed strategy.
Examples of games with MSNE:
• Chicken (aka Snowdrift) :
Player 2
Continue / Stay in car Swerve / Shovel
Player 1 Continue / Stay in car (-10,-10) (2,-2)
Swerve / Shovel (-2,2) (0,0)
PSNEs occur at h Continue, Swerve i and at h Swerve, Continue i. There is also
a MSNE at h 1/5 continue + 4/5 swerve, 1/5 continue + 4/5 swerve i. Show that
v1 = v2 = −2/5.
• Battle of the Sexes:
Her
Ballet Fight
Him Ballet (1,2) (-1,1)
Fight (-1,1) (1,-1)
PSNEs occur at h Ballet, Ballet i and at h Fight, Fight i. Show that there is a
MSNE at h 1/3 ballet + 2/3 fight, 2/3 ballet + 1/3 fight i with v1 = 2/3 = v2.
Compare and contast this with the payoffs of the PSNEs.
1.6 MSNE and Dominance
A SDS cannot be played with positive probability in a MSNE (otherwise a higher payoff
can be obtained by not playing the SDS when the strategy say to play it).
1.7 Strict Dominance in Mixed Strategies
Consider the game:
Player 2
Left Right
Player 1
Up (3,-1) (-1,1)
Middle (0,0) (0,0)
Down (-1,2) (2,-1)
This game has no SDS dominated by a pure strategy nor any PSNE. If a mixture of two
pure strategies dominates another, then that strategy is a SDS.
In the above game 1/2 Up + 1/2 Down > Middle. Remove Middle to get
Player 2
Left Right
Player 1 Up (3,-1) (-1,1)
Down (-1,2) (2,-1)
Show that a MSNE exists at h (3/5)Up + (2/5)Down, (3/7)Left + (4/7)Right i.
Another example:
Player 2
Left Centre Right
Player 1
Up (-3,6) (9,1) (0, 2)
Middle (3,-4) (2,4) (4,1)
Down (4,7) (3,2) (-3,2)
Using IESDS, we get (1/4)L + (3/4)C > R, D > M, L > C, D > U. Thus the solution is
h Down, Left i.
1.8 Atypical Matrix Games
Almost all matrix games have an odd number of solutions (Wilson 1971). Examples of
non generic games follow. Weak dominance is usually the culprit.
•
Player 2
Left Right
Player 1 Up (1,1) (0,0)
Down (0,0) (0,0)
There are two PSNEs.
•
Player 2
Left Right
Player 1 Up (2,2) (9,0)
Down (2,3) (5,-1)
Using IESDS, L > R. Player 1 can choose Up or Down as pure strategies or any
mixture of the two. All strategies yield a payoff of 2: an infinite number of strategies.
If tempted to use IEWDS, U ≥ D, leading to the solution h Up, Right i. Caveat
emptor!
• Example 2 of Section 1.4 had two PSNEs. It also has an infinite number of MSNEs
at h Up, qLeft + (1 − q)Right i for q ≤ 3/4.
• Selten’s Horse:
Player 2
Left Right
Player 1 Up (3,1) (0,0)
Down (2,2) (2,2)
There are two PSNEs at h Up,Left i and h Down,Right i and an infinite number of
MSNEs at h Down, qLeft + (1 − q)Right i for q ≤ 2/3.
• Take or Share - TV game show:
Player 2
Share Take
Player 1 Share (4,4) (0,8)
Take (8,0) (0,0)
There are three PSNEs at all but h Share, Share i. If one player chooses Take, then
the other is indifferent between Share and Take which leads to an infinity of MSNEs.
