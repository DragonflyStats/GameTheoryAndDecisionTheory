\documentclass[a4paper,12pt]{article}

\usepackage{amsmath}
\usepackage{graphicx}
\usepackage{amssymb}
\usepackage{framed}
\usepackage{multicol}
%\usepackage[paperwidth=21cm, paperheight=29.8cm]{geometry}
%\usepackage[angle=0,scale=1,color=black,hshift=-0.4cm,vshift=15cm]{background}
%\usepackage{multirow}
\usepackage{enumerate}

\usepackage{amsmath,amsfonts,amssymb}
\usepackage{color}
\usepackage{multirow}
\usepackage{eurosym}
\usepackage{framed}
\usepackage{fancyhdr}
\usepackage{listings}
\usepackage{eurosym}
\usepackage{vmargin}
\usepackage{amsmath}
\usepackage{fancyhdr}
\usepackage{listings}
\usepackage{framed}
\usepackage{graphics}
\usepackage{epsfig}
\usepackage{subfigure}
\usepackage{fancyhdr}

%\input def.tex
%\input dsdef.tex
%\input rgb.tex

%\newcommand \la{\lambda}
%\newcommand \al{a}
%\newcommand \be{b}
\newcommand \x{\overline{x}}
\newcommand \y{\overline{y}}

\pagestyle{fancy}
\setmarginsrb{20mm}{0mm}{20mm}{25mm}{12mm}{11mm}{0mm}{11mm}
\lhead{Operations Research 2} \rhead{Kevin O'Brien}
\chead{MS4315}
%\input{tcilatex}

\begin{document}
%==============================================================%
\section{Cournot Duopoly}

Cournot competition is an economic model used to describe an industry structure in which companies compete on the amount of output they will produce, which they decide on independently of each other and at the same time. It is named after Antoine Augustin Cournot (1801–1877) who was inspired by observing competition in a spring water duopoly.It has the following features:

\begin{enumerate}
\item There is more than one firm and all firms produce a homogeneous product, i.e. there is no product differentiation;
\item Firms do not cooperate, i.e. there is no collusion;
\item Firms have market power, i.e. each firm's output decision affects the good's price;
\item The number of firms is fixed;
\item Firms compete in quantities, and choose quantities simultaneously;
\end{enumerate}
The firms are economically rational and act strategically, usually seeking to maximize profit given their competitors' decisions.
An essential assumption of this model is the "not conjecture" that each firm aims to maximize profits, based on the expectation that its own output decision will not have an effect on the decisions of its rivals. Price is a commonly known decreasing function of total output. All firms know N ${\displaystyle N}$ N, the total number of firms in the market, and take the output of the others as given. Each firm has a cost function ${\displaystyle c_{i}(q_{i})}$. 

Normally the cost functions are treated as common knowledge. The cost functions may be the same or different among firms. The market price is set at a level such that demand equals the total quantity produced by all firms. Each firm takes the quantity set by its competitors as a given, evaluates its residual demand, and then behaves as a monopoly.

\section{Stackenberg Duopoly}
%==============================================================%
\end{document}
%=============================================================%
