\documentclass[a4paper,12pt]{article}

\usepackage{amsmath}
\usepackage{graphicx}
\usepackage{amssymb}
\usepackage{framed}
\usepackage{multicol}
%\usepackage[paperwidth=21cm, paperheight=29.8cm]{geometry}
%\usepackage[angle=0,scale=1,color=black,hshift=-0.4cm,vshift=15cm]{background}
%\usepackage{multirow}
\usepackage{enumerate}

\usepackage{amsmath,amsfonts,amssymb}
\usepackage{color}
\usepackage{multirow}
\usepackage{eurosym}
\usepackage{framed}
\usepackage{fancyhdr}
\usepackage{listings}
\usepackage{eurosym}
\usepackage{vmargin}
\usepackage{amsmath}
\usepackage{fancyhdr}
\usepackage{listings}
\usepackage{framed}
\usepackage{graphics}
\usepackage{epsfig}
\usepackage{subfigure}
\usepackage{fancyhdr}

%\input def.tex
%\input dsdef.tex
%\input rgb.tex

%\newcommand \la{\lambda}
%\newcommand \al{a}
%\newcommand \be{b}
\newcommand \x{\overline{x}}
\newcommand \y{\overline{y}}

\pagestyle{fancy}
\setmarginsrb{20mm}{0mm}{20mm}{25mm}{12mm}{11mm}{0mm}{11mm}
\lhead{Operations Research 2} \rhead{Kevin O'Brien}
\chead{MS4315}
%\input{tcilatex}

\begin{document}
%==============================================================%
\section{Cournot Duopoly}

Cournot competition is an economic model used to describe an industry structure in which companies compete on the amount of output they will produce, which they decide on independently of each other and at the same time. It is named after Antoine Augustin Cournot (1801–1877) who was inspired by observing competition in a spring water duopoly.

Cournot Duopoly has the following features:

\begin{enumerate}
\item There is more than one firm and all firms produce a homogeneous product, i.e. there is no product differentiation;
\item Firms do not cooperate, i.e. there is no collusion;
\item Firms have market power, i.e. each firm's output decision affects the good's price;
\item The number of firms is fixed;
\item Firms compete in quantities, and choose quantities simultaneously;
\end{enumerate}
The firms are economically rational and act strategically, usually seeking to maximize profit given their competitors' decisions.
An essential assumption of this model is the "not conjecture" that each firm aims to maximize profits, based on the expectation that its own output decision will not have an effect on the decisions of its rivals. Price is a commonly known decreasing function of total output. All firms know ${\displaystyle N}$ , the total number of firms in the market, and take the output of the others as given. Each firm has a cost function ${\displaystyle c_{i}(q_{i})}$. 

Normally the cost functions are treated as common knowledge. The cost functions may be the same or different among firms. The market price is set at a level such that demand equals the total quantity produced by all firms. Each firm takes the quantity set by its competitors as a given, evaluates its residual demand, and then behaves as a monopoly.



\begin{center}
\textbf{Cournot Duopoly with Heterogeneous items: Linear Demand and Linear Costs %(  au Spaniel)
}
\end{center}
Let $x_1$ and $x_2$ be the quantities of heterogeneous or non-identical items produced by two firms with associated costs $C_1(x_1) = c_1x_1$ and $C_2(x_2)= c_2x_2$ respectively.\\
Firm 1's items sell at $P_1 = a_1 - b_{11}x_1-b_{12}x_2$ each, firm 2's at $P_2 = a_2 - b_{21}x_1-b_{22}x_2$ each and it is assumed that all items produced are sold. \\ The profits made by the firms are then
$$ \pi_1 = P_1 x_1 - c_1 x_1 = \left(a_1-c_1 - b_{11}x_1 - b_{12}x_2\right)x_1$$
$$ \pi_2 = P_2 x_2 - c_2 x_2 =\left(a_2-c_2 - b_{21}x_1 - b_{22}x_2\right)x_2$$
respectively.\\
Maximising $\pi_1$ with respect to $x_1$
\begin{eqnarray}
 \frac{\partial \pi_1} {\partial x_1} &=& a_1-c_1 - b_{11}x_1 - b_{12}x_2 - b_{11}x_1 \nonumber \\
 & \stackrel{set}{=} & 0 \nonumber \\
 \Rightarrow x_1 &=& \frac{a_1-c_1}{2b_{11}} - \frac{1}{2}\frac{b_{12}}{b_{11}} x_2 \label{r1}
 \end{eqnarray}
 Similarly maximising $\pi_2$ with respect to $x_2$ yields
 \begin{equation} x_2 =  \frac{a_2-c_2}{2b_{22}} - \frac{1}{2}\frac{b_{21}}{b_{22}} x_1 \label{r2} \end{equation}
 We'll assume that Equations \ref{r1} and \ref{r2} give nonnegative values for $x_1$ and $x_2$ and so represent Reaction Functions.
 Solving equations \ref{r1} and \ref{r2} simultaneously gives the \textit{equilibrium} values
 \begin{eqnarray*}
 % \nonumber to remove numbering (before each equation)
   x^*_1 &=& \frac{2b_{22}(a_1-c_1)-b_{12}(a_2-c_2)}{4b_{11}b_{22}-b_{12}b_{21}} \\
   x^*_2 &=& \frac{-b_{21}(a_1-c_1)+2b_{11}(a_2-c_2)}{4b_{11}b_{22}-b_{12}b_{21}}
 \end{eqnarray*}
 At these equilibrium values

$$P^*_1  = \frac{2a_1b_{11} b_{22} - a_2 b_{11} b_{12} + b_{11} b_{12} c_2
  + 2 b_{11} b_{22} c_1 - b_{12} b_{21} c_1}{4 b_{11} b_{22} - b_{12} b_{21}}$$

$$P^*_2  = \frac{-a_1b_{21} b_{22} + 2a_2 b_{11} b_{22} +2 b_{11} b_{22} c_2
  -  b_{12} b_{21} c_2 + b_{21} b_{22} c_1}{4 b_{11} b_{22} - b_{12} b_{21}}$$

and
 \begin{equation} \pi_1^* =\frac{ (2a_1b_{22}-a_2b_{12}+b_{12}c_2-2b_{22}c_1)^2b_{11}}{(4b_{11}b_{22}-b_{12}b_{21})^2} \label{cp1} \end{equation}
  \begin{equation} \pi_2^* = \frac{(a_1b_{21}-2a_2b_{11}+2b_{11}c_2-b_{21}c_1)^2b_{22}}{(4b_{11}b_{22}-b_{12}b_{21})^2} \label{cp2} \end{equation}


 \begin{center}
\textbf{Stackelberg Duopoly %(  au Spaniel)
}
\end{center}
We'll consider a \textit{Stackelberg} duopoly in which Firm 1 is the Leader and Firm 2 is the Follower. Irrespective of what the leader does, the follower will use the reaction function (Eq. \ref{r2}) as its best response.
\newpage
Knowing this, the leader seeks to maximise $$ \Pi_1 = \left(a_1-c_1 - b_{11}x_1 -b_{22}\left( \frac{a_2-c_2}{2b_{22}} - \frac{1}{2}\frac{b_{21}}{b_{22}} x_1\right)\right) x_1 = \left( a_1 - c_1 - \frac{a_2 - c_2}{2} - \left(b_{11} - \frac{b_{21}}{2}\right)x_1\right)x_1$$
as a function of $x_1$.
\begin{eqnarray}
 \frac{d \Pi_1} {d x_1} &=&a_1 - c_1 - \frac{ a_2 - c_2}{2} - \left(b_{11} - \frac{b_{21}}{2}\right)x_1 - \left(b_{11} - \frac{b_{21}}{2}\right)x_1 \nonumber \\
 & \stackrel{set}{=} & 0 \nonumber \\
 \Rightarrow x_1 &=& \frac{a_1 - c_1 - \frac{ a_2 - c_2}{2}}{2b_{11}-b_{21}} \label{r3}
 \end{eqnarray}
 Denoting this optimal value by $X_1^*$ and the corresponding value of $x_2$ by $X_2^*$,(substitute Eq. \ref{r3} into Eq. \ref{r2}) gives
$$ X_1^* = \frac{a_1 - c_1 - \frac{ a_2 - c_2}{2}}{2b_{11}-b_{21}}, \hspace{10mm} X_2^* = \frac{(a_2-c_2)(2b_{11}-b_{21}/2)-b_{21}(a_1-c_1)}{2b_{22}(2b_{11}-b_{21})} $$
At these equilibrium values
$$ P_1^* = , \hspace{10mm} P_2^*= $$
 and
 \begin{equation} \Pi_1^* = , \hspace{10mm} \Pi_2^* = \label{sp} \end{equation}

\newpage

\subsection{2012 Past Paper Queston}

Question 6 

Consider the (symmetric) Cournot duopoly game: Firm i, i = 1, 2 produces
xi
items at a cost of
C(xi) = x
2
i
1000 + 3xi + 20
. The items sell at a price of
p(x1, x2) = 5 −
x1 + x2
500
each.
\begin{itemize}
\item[(a)] Find the equilibrium of this game, and prove that it is a Nash equilibrium.
8 %
\item[(b)] Investigate this game if a collusive strategy is used. Contrast its solution
with that of part (a). 8 %
\item[(c)] If the game is to be played repeatedly, does it ever pay to defect from
the collusive strategy? In particular, consider the stern strategy: a firm
produces the collusive number of items until the other firm defects,
after which it reverts to producing the Cournot number of items. Using
the discount factor ω per period, when is this stern strategy a Nash
equilibrium ? 9 %
\end{itemize}
\end{document}

\begin{figure}[h]
\centering
%\begin{minipage}[l]{80mm}
{\includegraphics[height=80mm,width=90mm]{Log-OGY.jpg}}
%\end{minipage}
%\qquad
%\begin{minipage}[l]{80mm}
%{\includegraphics[height=85mm,width=90mm]{rec-fb3.pdf}}
%\end{minipage}
\caption{The trajectory of the Logistic Map with OGY control starting from $x_0 = 0.43$} \label{fig:trajOGY}
\vspace{5mm}
\end{figure}
\section{Stackenberg Duopoly}
%==============================================================%
\end{document}
%=============================================================%
