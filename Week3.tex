\documentclass[a4paper,12pt]{article}

\usepackage{amsmath}
\usepackage{graphicx}
\usepackage{amssymb}
\usepackage{framed}
\usepackage{multicol}
%\usepackage[paperwidth=21cm, paperheight=29.8cm]{geometry}
%\usepackage[angle=0,scale=1,color=black,hshift=-0.4cm,vshift=15cm]{background}
%\usepackage{multirow}
\usepackage{enumerate}

\usepackage{amsmath,amsfonts,amssymb}
\usepackage{color}
\usepackage{multirow}
\usepackage{eurosym}
\usepackage{framed}
\usepackage{fancyhdr}
\usepackage{listings}
\usepackage{eurosym}
\usepackage{vmargin}
\usepackage{amsmath}
\usepackage{fancyhdr}
\usepackage{listings}
\usepackage{framed}
\usepackage{graphics}
\usepackage{epsfig}
\usepackage{subfigure}
\usepackage{fancyhdr}

%\input def.tex
%\input dsdef.tex
%\input rgb.tex

%\newcommand \la{\lambda}
%\newcommand \al{a}
%\newcommand \be{b}
\newcommand \x{\overline{x}}
\newcommand \y{\overline{y}}

\pagestyle{fancy}
\setmarginsrb{20mm}{0mm}{20mm}{25mm}{12mm}{11mm}{0mm}{11mm}
\lhead{Operations Research 2} \rhead{Kevin O'Brien}
\chead{MS4315}
%\input{tcilatex}

\begin{document}

\section{Nash Equilibrium}

\begin{framed}
\begin{itemize}
\item Nash Equilibrium recommends a strategy to each player that the player cannot improve upon unilaterally, as long as the other players follow the recommendation.
\item Since the other olayers are assumed to be rational, it is reasonable to expect the opponents to follow the recommendation as well.
\end{itemize}
\end{framed}
Game Theory


Nash equilibrium is a solution concept of a non-cooperative game involving two or more players in which each player is assumed to know the equilibrium strategies of the other players, and no player has anything to gain by changing only his or her own strategy.[1] If each player has chosen a strategy and no player can benefit by changing strategies while the other players keep theirs unchanged, then the current set of strategy choices and the corresponding payoffs constitutes a Nash equilibrium. 

Nash equilibrium is one of the foundational concepts in game theory.


%===========================================================%

\subsection{Informal definition}
Informally, a strategy profile is a Nash equilibrium if no player can do better by unilaterally changing his or her strategy. To see what this means, imagine that each player is told the strategies of the others. Suppose then that each player asks themselves: "Knowing the strategies of the other players, and treating the strategies of the other players as set in stone, can I benefit by changing my strategy?"

If any player could answer "Yes", then that set of strategies is not a Nash equilibrium. But if every player prefers not to switch (or is indifferent between switching and not) then the strategy profile is a Nash equilibrium. Thus, each strategy in a Nash equilibrium is a best response to all other strategies in that equilibrium.

The Nash equilibrium may sometimes appear non-rational in a third-person perspective. This is because it may happen that a Nash equilibrium is not Pareto optimal.

The Nash equilibrium may also have non-rational consequences in sequential games because players may "threaten" each other with non-rational moves. For such games the subgame perfect Nash equilibrium may be more meaningful as a tool of analysis.

\newpage
\begin{itemize}
\item A Nash equilibrium of a strategic game is an action profile (list of actions, one for each player) with the property that no player can increase her payoff by choosing a different action, given the other players' actions.
\item Note that nothing in the definition suggests that a strategic game necessarily has a Nash equilibrium, or that if it does, it has a single Nash equilibrium. 
\item A strategic game may have no Nash equilibrium, may have a single Nash equilibrium, or may have many Nash equilibria.
\end{itemize}
Finding Nash equilibria: games with finitely many actions for each player

Consider, for example, the game
Player 2
L	R
Player 1	T	
2,2	0,3
3,0	1,1
B
There are four action profiles ((T,L), (T,R), (B,L), and (B,R)); we can examine each in turn to check whether it is a Nash equilibrium.
\begin{itemize}
\item (T,L)
By choosing B rather than T, player 1 obtains a payoff of 3 rather than 2, given player 2's action. Thus (T,L) is not a Nash equilibrium. [Player 2 also can increase her payoff (from 2 to 3) by choosing R rather than L.]
\item (T,R)
By choosing B rather than T, player 1 obtains a payoff of 1 rather than 0, given player 2's action. Thus (T,R) is not a Nash equilibrium.
\item (B,L)
By choosing R rather than L, player 2 obtains a payoff of 1 rather than 0, given player 1's action. Thus (B,L) is not a Nash equilibrium.
item (B,R)
Neither player can increase her payoff by choosing an action different from her current one. Thus this action profile is a Nash equilibrium.
\end{itemize}
We conclude that the game has a unique Nash equilibrium, (B,R).
Notice that in this equilibrium both players are worse off than they are in the action profile (T,L). Thus they would like to achieve (T,L); but their individual incentives point them to (B,R).
\newpage
%----------------------------------------%
\section{The Nash Equilibrium}
Nash Equilibrium is an outcome reached that, once achieved, means no player can increase payoff by changing decisions unilaterally. It can also be thought of as "no regrets," in the sense that once a decision is made, the player will have no regrets concerning decisions considering the consequences.

The Nash Equilibrium is reached over time, in most cases. However, once the Nash Equilibrium is reached, it will not be deviated from. After we learn how to find the Nash Equilibrium, take a look at how a unilateral move would affect the situation. Does it make any sense? It shouldn't, and that's why the Nash Equilibrium is described as "no regrets."

%----------------------------------------%
\subsection{Finding a Nash Equilibria}
Step One: Determine player one's best response to player two's actions.
When examining the choices that may maximize a player's payout, we must look at how player one should respond to each of the options player two has. An easy way to do this visually is to cover up the choices of player two. Consider the matrix portrayed at the beginning of this article as we apply this method.

Player one / Player two	Left	Right
Up	(1, -)	(4, -)
Down	(3, -)	(3, -)


	\begin{center}
		{\color{blue}
			\begin{tabular}{c|c|c|c|}
				\multicolumn{2} {c} {} & \multicolumn{2}{c} {{\color{red}Player 2}} \\
				\cline{3-4}
				\multicolumn{2}{c|}{} &   Left       &  Right       \\
				\cline{2-4}
				\multirow{2} {*} {{\color{red}Player 1}}& Left & (80,0) & (60,40) \\
				\cline{2-4}
				& Right &(40,60)& (40,0) \\
				\cline{2-4}
				%C & (2,6) & (4,7)& (0,8) \\
				%\hline
			\end{tabular}
		}
	\end{center}
\begin{itemize}
	\item Player one has two possible choices to play: "up" or "down." Player two also has two choices to play: "left" or "right." In this step of determining Nash Equilibrium, we look at responses to player two's actions.
	\item  If player two chooses to play "left," we can play "up" with the payoff of one, or play "down" with the payoff of three. Since three is greater than one, we will bold the 3 indicating the option to play "down" here.
	
	\item If player two chooses to play "right," we can either choose to play 'up' for a payoff of four or play "down" for a playoff of three. Since four is greater than three, we bold the four to indicate the option to play "up" here. The bold outcomes are shown below on the full matrix.
	
\end{itemize}

Player one / Player two	Left	Right
Up	(1, 3)	(4, 2)
Down	(3, 2)	(3, 1)

	\begin{center}
		{\color{blue}
			\begin{tabular}{c|c|c|c|}
				\multicolumn{2} {c} {} & \multicolumn{2}{c} {{\color{red}Player 2}} \\
				\cline{3-4}
				\multicolumn{2}{c|}{} &   Left       &  Right       \\
				\cline{2-4}
				\multirow{2} {*} {{\color{red}Player 1}}& Left & (80,0) & (60,40) \\
				\cline{2-4}
				& Right &(40,60)& (40,0) \\
				\cline{2-4}
				%C & (2,6) & (4,7)& (0,8) \\
				%\hline
			\end{tabular}
		}
	\end{center}
	
%----------------------------------------%
Step Two: Determine player two's best response to player one's actions.
As we did before with the player two payoffs for player one, we will hide the payoffs of player one when determining the best responses for player two. (To learn more about behavioral finance, check out Leading Indicators Of Behavioral Finance.)

Player one / Player two	Left	Right
Up	(-, 3)	(-, 2)
Down	(-, 2)	(-, 1)

	\begin{center}
		{\color{blue}
			\begin{tabular}{c|c|c|c|}
				\multicolumn{2} {c} {} & \multicolumn{2}{c} {{\color{red}Player 2}} \\
				\cline{3-4}
				\multicolumn{2}{c|}{} &   Left       &  Right       \\
				\cline{2-4}
				\multirow{2} {*} {{\color{red}Player 1}}& Left & (80,0) & (60,40) \\
				\cline{2-4}
				& Right &(40,60)& (40,0) \\
				\cline{2-4}
				%C & (2,6) & (4,7)& (0,8) \\
				%\hline
			\end{tabular}
		}
	\end{center}
	
	
Just as when looking at player one, each player has two choices to play. If player one chooses to play "up," we can play "left," with a payoff of three, or "right," with a payoff of two. Since three is greater than two, we bold the three to show the option to play "left" here. If player one chooses to play "down," we can play "left," for a payoff of two, or "right," for a payoff of one. Since two is greater than one, we bold the two indicating the option to play "left" here. The bold outcomes are shown below on the full matrix.

Player one / Player two	Left	Right
Up	(1, 3)	(4, 2)
Down	(3, 2)	(3, 1)
Step Three: Determine which outcomes have both payoffs bold. That particular outcome is the Nash Equilibrium.
Now, we combine the bold options for both players onto the full matrix.

	\begin{center}
		{\color{blue}
			\begin{tabular}{c|c|c|c|}
				\multicolumn{2} {c} {} & \multicolumn{2}{c} {{\color{red}Player 2}} \\
				\cline{3-4}
				\multicolumn{2}{c|}{} &   Left       &  Right       \\
				\cline{2-4}
				\multirow{2} {*} {{\color{red}Player 1}}& Left & (80,0) & (60,40) \\
				\cline{2-4}
				& Right &(40,60)& (40,0) \\
				\cline{2-4}
				%C & (2,6) & (4,7)& (0,8) \\
				%\hline
			\end{tabular}
		}
	\end{center}
	


Look for intersections where both payoffs are bold. In this case, we find the intersection of (Down , Left) with the payoff of (3, 2) fits our criteria. This indicates our Nash Equilibrium.

This method of finding Nash Equilibrium is well-suited to finding equilibria in games that are simultaneous since we are looking at how a player would respond independently of how the other acts. This scenario of a simultaneous game is often played out in businesses such as airlines. Below is an example, similar to the game above, of how airline pricing may play out. The payouts are in thousands of dollars. Remember, these are the payouts, not the prices. The method we applied previously is already applied to show where the Nash Equilibrium appears.

%%Airline one / Airline two	Low Price	High Price
%%Low Price	(3,000, 3,000) &	(4,000, 2,000) \\
%%High Price	(2,000, 4,000) &	(3,500, 3,500) \\

	\begin{center}
		{\color{blue}
			\begin{tabular}{c|c|c|c|}
				\multicolumn{2} {c} {} & \multicolumn{2}{c} {{\color{red}Airline 2}} \\
				\cline{3-4}
				\multicolumn{2}{c|}{} &   Low Price       &  High Price      \\
				\cline{2-4}
				\multirow{2} {*} {{\color{red}Player 1}}& Low Price & (3,000, 3,000) &	(4,000, 2,000) \\
				\cline{2-4}
				& High Price & (2,000, 4,000) &	(3,500, 3,500) \\
				\cline{2-4}
				%C & (2,6) & (4,7)& (0,8) \\
				%\hline
			\end{tabular}
		}
	\end{center}
\begin{itemize}	
\item Looking at just A1's choices we can see that if A2 chooses to play low price, we choose between Low Price for 3,000 or high price for 2,000. We choose "low," since 3,000>2,000. 
\item We do the same thing for A2 playing High Price and see that we play "low" because 4,000>3,500. Conversely, looking just at A2's choices, we can see that if A1 chooses to play low price, we choose between "low price" for 3,000 and "high price" for 2,000. Since 3,000>2,000, we choose the "low price" option here.\item  If A1 plays high price, we can charge a low price for 4,000 or high price for 3,500. Since 4,000>3,500, we choose to play \textit{"low price"} here.
\end{itemize}

%=======================================%
\begin{itemize}
	\item The Nash Equilibrium is that both airlines will charge a low price (shown when choices for each party are highlighted). If both airlines charged a high price, they would each be better off than they are at the Nash Equilibrium.
	
\item So why don't they agree to do this? First off, it's illegal to collude. Second, if this were to occur, a unilateral action on behalf of one airline to charge a low price would be beneficial, resulting in that airline making more money in turn. This logic also shows how the Nash Equilibrium is reached, and why it is not beneficial to deviate from it once it is reached. 
\end{itemize}


%----------------------------------------%
\subsection{Multiple Nash Equilibria \& How The Nash Equilibrium Plays Out}
\begin{itemize}
\item Generally, there can be more than one equilibrium in a game. However, this usually occurs in games with more complex elements than two choices by two players. In simultaneous games that are repeated over time, one of these multiple equilibria is reached after some trial and error. 
\item This scenario of differing choices over time before reaching equilibrium is the most often played out in the business world when two firms are determining prices for very interchangeable products, such as airfare or soda pop.

With these advanced methods, more real-world situations can be modeled and solved. 
\item The different kinds of Nash Equilibrium we discussed are the most commonly found solutions to real-world modeled games. A working knowledge of Game Theory can help you form a strategy, whether playing a friend playing tic-tac-toe or vying for the largest profits.
\end{itemize}
\end{document}
