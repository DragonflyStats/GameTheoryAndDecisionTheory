\subsection*{Introduction to Game Theory/Matrix Notation}
%- https://en.wikibooks.org/wiki/Introduction_to_Game_Theory/Matrix_Notation

If you remember, the game we've looked at—the Prisoner's Dilemma—had to be explained with the use of a story. However this is not only very verbose and imprecise but also impossible to do for many games that are simply too complicated. One simple way of showing a game is by using a game matrix 
This is really a table of utility. Utility is the amount of happiness an agent (player) gets from a particular outcome, or payoff. 
In order to create a game matrix, we first need to work out the utility values. We assign the payoffs that are least attractive to a player low values and payoffs that are attractive to the player high payoffs. Initially[1], these are what we call "ordinal" utility values, not "cardinal" utility values. This means that a payoff of 10 isn't necessarily twice as good as one of 5. In fact, there is no difference between the two following utility values lists when talking about ordinal values: 
Setup 1
Setup 2 
Event A = Utility of 1
Event B = Utility of 2
Event C = Utility of 3
Event A = Utility of 1
Event B = Utility of 1,000,000
Event C = Utility of 230,000,000,000,000

It's just that the first list is far more concise. Remembering that we go from lowest level of attraction to highest, let's assign payoffs to the Prisoner's Dilemma game. 

10 years in jail = 1
7 years in jail  = 2
2 years in jail  = 3
Get off free   = 4

Now we can arrange a table that shows what happens when each player chooses different options. 
%%%%%%%%%%%%%%%%%%%%%%%%%%%%%

\subsection*{Prisoner's Dilemma}
Player 2 
Confess
Stay Silent 
Player 1 
Confess 
(2,2)
(4,1) 
Stay Silent 
(1,4)
(3,3) 
It should be clear soon how to read this table. Player 1 has two rows, "Confess" and "Stay Silent", and Player 2 has two columns marked the same. Where a column and a row intersect are the payoffs. Thus when Player 1's "Confess" row and Player 2's "Stay Silent" column intersect (which means in terms of the game, when Player 1 confesses and Player 2 stays silent) the payoff (4,1) is awarded. This means that Player 1 (whose personal payoff comes first in the brackets) gets a payout of 4—his highest payoff—and Player 2 gets a payout of 1—the lowest. 
