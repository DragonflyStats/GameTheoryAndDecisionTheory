

%%- C8

Decision Theory

%%%%%%%%%%%%%%%%%%%%%%%%%%%%%%%%%%%%%%%%%%%%%%%%%%%%%%%%%%%%%%%%%%%%%%%
\subsection*{8.1}

There are many situations in which we have to make decisions based on observations or data
that are random variables.
The theory behind the solutions for these situations is known as \textit{decision theory}
or \textit{hypothesis testing}.
In communications technology or radar technology, this area is also known as signal detection theory.

In this section we will provide a brief overview of decision theory and various decision tests.

%%%%%%%%%%%%%%%%%%%%%%%%%%%%%%%%%%%%%%%%%%%%%%%%%%%%%%%%%%%%%%%%%%%%%%%
\subsection*{8.2 Hypothesis Testing and Types of Error}


%---------------------------------------------------------%
\begin{enumerate}[(i)]
\item $H_0$ true; accept $H_0$. 
\item $H_0$ true; 
\item $H_1$ true; accept $H_1$. 
\item $H_1$ true; 
\end{enumerate}

%---------------------------------------------------------%

The first and third actions correspond to correct decisions, and the second and fourth correspond to errors.
These errors are classified as follows

%---------------------------------------------------------%
\begin{enumerate}[(i)]
\item Type I Error:
\item Type II Error: 
\end{enumerate}

Let $P_{1}$ and $P_{II}$ denote the probabilities of Type I and Type II errors

\[ P_{I} = P(D|H) = P(x \in \mathbb{R}) \]
\[ P_{II} = P(D|H) = P(x \in \mathbb{R}) \]

%Page 265 Middle

The probabilities of correct decisions (actions 1 and 3) may be expressed as

\[ P_{I} = P(D|H) = P(x \in \mathbb{R}) \]
\[ P_{II} = P(D|H) = P(x \in \mathbb{R}) \]

In radar signal detection, the two hypotheses are

\[ H_0 = \]
\[ H_1 = \]

%%%%%%%%%%%%%%%%%%%%%%%%%%%%%%%%%%%%%%%%%%%%%%%%%%%%%%%%%%%%%%%%%%%%%%%
\subsection*{8.3 Decision Tests}

\subsubsection*{A. Maximum-Likelihood Test}

Note that the likelihood ratio $\Lambda(X)$ is often expressed as
\[ \Lambda(X) =  \frac{f(x|H_1)}{f(x|H_0)} \]
\subsubsection*{B. MAP Test}

\subsubsection*{C. Neyman-Pearson Test}

It is not possible to simultaneously minimize both $\alpha$ and $\beta$.

The Neyman-Pearson Test provides a workable solution to this problem in that the tests minimized $beta$ for a given level of $\alpha$.

This is a classical problem in optimization: maxmizing a function subject to a constraint, which can be solved by using the Lagrange multiplier Method.


Then the critical region $R_1$ is chosen to maximize $J$. It can be shown that the Neyman-Pearson test can be expressed in terms of the likelihood test as follow:

\[  Equation \]

\subsubsection*{D. Bayes Test}


\subsubsection*{E. Minimum Probability of Error Test}


\subsubsection*{F. Minimax Test}

%%%%%%%%%%%%%%%%%%%%%%%%%%%%%%%%%%%%%%%%%%%%%%%%%%%%%%%%%%%%%%%%%%%%%%%

\subsection*{Solved Problems}

\begin{enumerate}
\item
% - Example 8.1

Suppose a manufacturer of memory chips observes that the probability of chip failure is $p=0.05$. A new procedure is introduced to improve the design of chips.
To test this new procedure, 200 chips could be produced using this new procedures and tests.

Let the random variable $X$ denote the number of chips that fail.

%------------------------------------------------------%
\item
% - Example 8.2


We would reject the new procedure if $X>5$. Find the probability of a Type II error.
%------------------------------------------------------%
\item 

Let $(X_1,X_2,\ldots,X_n)$ be a random sample of a normal random variable X with mean $mu$ and variance $\sigma^2=100)


% - Example 8.3

\begin{itemize}
\item[(a)]
\item[(b)]
\item[(c)]
\end{itemize}

\noindent \textbf{Solutions}
\begin{itemize}
\item[(a)]
\item[(b)]
\item[(c)]
\end{itemize}

%------------------------------------------------------%
\item
% - Example 8.4
Consider the binary decision theory problem of example 3. We modify the decision rule such that we reject $H_0$ if $x \geq c$

%------------------------------------------------------%
\item
% - Example 8.5

\item
% - Example 8.6

In a simple binary communications system, during every T seconds, one of two possible signals $s_0(t)$ and
$s_0(t)$ is transmitted.

\begin{itemize}
\item[(a)] Using the maximum likelihood test, determine which signal has been transmitted.
\item[(b)] Find $P_{I}$ and $P_{II}$.
\end{itemize}

\noindent \textbf{Solutions}
\begin{itemize}
\item[(a)]
\item[(b)]
\end{itemize}

%------------------------------------------------%

\item 
% - Example 8.7


\begin{itemize}
\item[(a)] Using the MAP Test, determinethe signal that is transmitted when x = 0.6.
\item[(b)]
\end{itemize}

\noindent \textbf{Solutions}
\begin{itemize}
\item[(a)]
\item[(b)]
\end{itemize}
%------------------------------------------------%

\item
% - Example 8.8

%------------------------------------------------%
\item
% - Example 8.9


%------------------------------------------------%
\item
% - Example 8.10


%------------------------------------------------%

\item 
% - Example 8.11

\begin{itemize}
\item[(a)]
\item[(b)]
\end{itemize}

\noindent \textbf{Solutions}
\begin{itemize}
\item[(a)]
\item[(b)]
\end{itemize}

%------------------------------------------------%
\item
% - Example 8.12


%------------------------------------------------%
\item
% - Example 8.13


%------------------------------------------------%
\item
% - Example 8.14


%------------------------------------------------%
\item 
% - Example 8.15

\begin{itemize}
\item[(a)] Determine the Maximum Likelihood Test.
\item[(b)] Find P_1 and P_{II} for n=5 and n= 10.
\end{itemize}

\noindent \textbf{Solutions}
\begin{itemize}
\item[(a)] Setting $\mu_0 = 0$ and $\mu_1 = 1$, the maximum likelihood test is
\[  \frac{1}{n}  \]

\item[(b)]
\end{itemize}


%------------------------------------------------%

\item
% - Example 8.16

\end{enumerate}

%====================%
\section{Supplementary Problems}

\begin{enumerate}
%------------------------------------------------%
\item
% - Example 8.17


%------------------------------------------------%
\item
% - Example 8.18


%------------------------------------------------%
\item 
% - Example 8.19

%------------------------------------------------%
\item
% - Example 8.20


%------------------------------------------------%
\item
% - Example 8.21


%------------------------------------------------%
\item 
% - Example 8.22


%------------------------------------------------%
\item 
% - Example 8.23

\end{itemize}
\end{document}
