
%%%%%%%%%%%%%%%%%%%%%%%%%%%%%%%%%%%%%%%%%%%%%%%%
%B25 - Page 321
Every matrix A defines a game as follows
\begin{itemize}
\item[(1)] There are two players , one called R, and the other called C.
\item[(2)] A play of the game  consists of R choosing a row of A and, simultaneously, C's choosing of a column of A.
\item[(3)] After each play of the game, R receives from C an amoun equal to the entry in the chosen cell . A negative entry denotes a payment from R to C.

\end{itemize}


\subsection{Example 1.2}

\begin{itemize}
\item[(i)]
\item[(ii)]
\item[(iii)]
\item[(iv)] 
\end{itemize}
%%%%%%%%%%%%%%%%%%%%%%%%%%%%%%%%%%%%%%%%%%%%%%%%
%B25 - Page 322

We remark that matrix games and the above competition games which can be represented
by matrix games are termed two-person zero sum games.
The zero sum corresponds to the fact that the sum of the winnings and losses of all of the players
is zero after each play.
There also exists a theory for n-person games and non-zero sum games but that lies beyond the scope of this course.
\subsection{Strategies}

By a strategy for R in a matrix game, we mean a decision by R to play the various rows with a given probability distribution, for example
 to play row 1 with a probability of $p_1$, to play row 2 with a probability of $p_2$ and so on.
The strategy for R is formally denoted by the probability vector $\boldsymbol{p} = {p_1,p_2,\ldots,p_n\}$.

\subsection{Optimum Strategies and the Value of the game}
%%%%%%%%%%%%%%%%%%%%%%%%%%%%%%%%%%%%%%%%%%%%%%%%
%B25 - Page 323

\subsection{Definition}
In a matrix game A, a strategy $P^0$ for player R is an optimum strategy, a strategy $q^0$ for player C is an optimum strategy.

We point out that  the solution of a matrix means find optimum strategies for the players and the value of the game.
Before we give the solution to an arbitrary matrix game by means of the simplex method

We shall discuss two special cases which occur frequently and which offer simple solutions: striclty determined games and 2 \times 2 games.
%%%%%%%%%%%%%%%%%%%%%%%%%%%%%%%%%%%%%%%%%%%%%%%%
%B25 - Page 324

\subsection{$2 \times 2$ Matrix Games}

Lemma

Lemma
Suppos that the above matrix A is strictly etermined. Then $P^0 = (x_1,x_2)$ is an optimum strategy for player $R$, $Q^0 = (y_1,y_2)$ is an optimum strategy for player C, and y is the value of the game.

%%%%%%%%%%%%%%%%%%%%%%%%%%%%%%%%%%%%%%%%%%%%%%%%
%B25 - Page 325

\subsection*{Example 3.1}
Consider the following matrix game (see Example 1.2).

Thus the game is not fair and is in favour of the column player $C$.

\subsection{Recessive Rows and Columns}

Let A be a matrix game. Suppose A contains a row $r_i$ such that $r_i \leq rj$ for some other
row $r_j$.
Recall that $r_i\leq r_j$ means that every entry of $r_i$ is less than or equal to the corresponding entry of $r_j$.
The $r_i$ is called a recessive row, and $r_j$ is said to domninate it.
Cleary player $R$ would always play row $j$ than row $i$ since they are guaranteed to win the same or a greater amount in each possible play of the game.
Accordingly a recessive row can be omitted from the game.
%%%%%%%%%%%%%%%%%%%%%%%%%%%%%%%%%%%%%%%%%%%%%%%
%B25 - Page 326

Solution of a matrix game by the simplex method.

\begin{itemize}
\item[(1)]
\item[(2)]
\item[(3)]
\end{itemize}

\subsection*{Example 5.1}
%%%%%%%%%%%%%%%%%%%%%%%%%%%%%%%%%%%%%%%%%%%%%%%%
%B25 - Page 327

\subsection*{Example 5.2}
%%%%%%%%%%%%%%%%%%%%%%%%%%%%%%%%%%%%%%%%%%%%%%%%
%B25 - Page 328


The value of the game is therefore $-4/13$.

%%%%%%%%%%%%%%%%%%%%%%%%%%%%%%%%%%%%%%%%%%%%%%%%
%B25 - Page 329

\section{Summary}

In finding a soultion to the matrix game, you should follow the following steps.
\begin{itemize}
\item[(1)] Test to see if the game is strictly determined.
\item[(2)] Eliminate all recessive rows and columns.
\item[(3)] In the case of $2 \times 2$, use the following formula in Theorem 25.3
\item[(4)] In the case of a $2 \times m$ or an $m \times 2$, reduce the game to a $2 \times 2$ game.
\item[(5)] Use the simplex method in all other cases.
\end{itemize}



\section*{Solved Problems}


\subsection{Exercise 25.1}

\begin{itemize}
\item[(1)]
\item[(2)]
\item[(3)]
\end{itemize}
%%%%%%%%%%%%%%%%%%%%%%%%%%%%%%%%%%%%%%%%%%%%%%%

\subsection{Exercise 25.2}
Find the solution to the following $2 \times 2$ matrix game.

\begin{itemize}
\item[(1)]
\item[(2)]
\item[(3)]
\item[(4)] 
\end{itemize}

\subsection{Exercise 25.3}
Find the solution to the following matrix game.

First circle each row minimum and box each column maxmimum.

Observe there is no saddle point and so the game is not strictly determined.
We now test for recessive rows and columns.
Not that the third column is recessive since each entry is larger than the corresponding entry in the
second column. Hence the third column can be omitted to obtain the game.
