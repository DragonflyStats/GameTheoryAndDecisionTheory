\documentclass[]{report}
\voffset=-1.5cm
\oddsidemargin=0.0cm
\textwidth = 480pt


\usepackage{amsmath}
\usepackage{graphicx}
\usepackage{amssymb}
\usepackage{framed}
\usepackage{multicol}
%\usepackage[paperwidth=21cm, paperheight=29.8cm]{geometry}
%\usepackage[angle=0,scale=1,color=black,hshift=-0.4cm,vshift=15cm]{background}
%\usepackage{multirow}
\usepackage{enumerate}

\usepackage{amsmath,amsfonts,amssymb}
\usepackage{color}
\usepackage{multirow}
\usepackage{eurosym}
\usepackage{framed}
\usepackage{tikz}
\newcommand*\circled[1]{\tikz[baseline=(char.base)]{
            \node[shape=circle,draw,inner sep=2pt] (char) {#1};}}

%\input def.tex
%\input dsdef.tex
%\input rgb.tex

%\newcommand \la{\lambda}
%\newcommand \al{a}
%\newcommand \be{b}
\newcommand \x{\overline{x}}
\newcommand \y{\overline{y}}
\begin{document}





\section{Introduction to Matrix Games}
%%%%%%%%%%%%%%%%%%%%%%%%%%%%%%%%%%%%%%%%%%%%%%%%
%B25 - Page 321
Every matrix A defines a game as follows
\begin{itemize}
\item[(1)] There are two players, one called R (or player 1), and the other called C (or player 2).
\item[(2)] A play of the game consists of R choosing a row of matrix A and, simultaneously, C's choosing of a column of matrix A.
\item[(3)] After each play of the game, R receives from C an amount equal to the entry in the chosen cell. A negative entry denotes a payment from R to C.

\end{itemize}
%-----------------------------------%


%-----------------------------------%
In the first game, if R consistently plays the first row, hoping to win an amount 3 or 4, then C can play the second column
and so win an amount 1.t

However if C consistently plays the second column then R can play the second row and so win an amount 1.

The we see that if either player should consistently play a particular row or column, the other player can take advantage of that fact.

On the other hand, suppose a competitive situation is as follows:
\begin{itemize},j
\item[(i)] there are two people (oe organizations, or competitors etc)
\item[(ii)] each competitor has a finite number of courses of action.
\item[(iii)] both competitors choose a course of action simultaneously.
\item[(iv)] for each pair $i,j$ of choices, there is a payment $b_{ij}$ from one given player to the other.
\end{itemize}

The the above competitive situation is equivalent to the matrix game determined by the matrix $B=b_{i}$
\subsection{Example 1.2}
Each of the two competitors R and C simultaneously show 1 or 2 fingers. If the sum of the fingers is even, then R wins the sum from C, of the sum is odd, the R loses the sum to C. The matrix of this game is as follows:

\[Matrix\]

We shall show later that this game is favourable to C.
%%%%%%%%%%%%%%%%%%%%%%%%%%%%%%%%%%%%%%%%%%%%%%%%
%B25 - Page 322

We remark that matrix games and the above competition games which can be represented
by matrix games are termed \textit{two-person zero sum games}.
The zero sum corresponds to the fact that the sum of the winnings and losses of all of the players
is zero after each play.
There also exists a theory for n-person games and non-zero sum games but that lies beyond the scope of this course.
%-----------------------------------%
\subsection{Strategies}
\begin{itemize}
\item By a strategy for R in a matrix game, we mean a decision by R to play the various rows with a given probability distribution, for example
 to play row 1 with a probability of $p_1$, to play row 2 with a probability of $p_2$ and so on.
The strategy for R is formally denoted by the probability vector $\boldsymbol{p} = {p_1,p_2,\ldots,p_n\}$.
\item For example, if the matrix game has two rows and r tosses a coin to decide which row to play, then their strategy is the probability vector $p=(1/2,1/2)$.
\item Similarly By a strategy for C in a matrix game, we mean a decision by C to play the various columns with a given probability distribution, for example
 to play columns 1 with a probability of $q_1$, to play column 2 with a probability of $q_2$ and so on.
\item The strategy for C is formally denoted by the probability vector $\boldsymbol{q} = {q_1,_2,\ldots,q_n\}$.
\end{itemize}
%-----------------------------------%
A strategy which contains a 1 as a component and 0 everywhere else, i.e. where R decides to play consistently a given 
row or C decides to play consistently a given column, is called a \textbf{\textit{pure strategy}}, otherwise it is called a \textbf{\textit{mixed strategy}}.
%-----------------------------------%
\section{Optimum Strategies and the Value of the game}
For simplicity, we shall first consider a $2 \times 2$ matrix game, say 
\[ A = \begin{array}{|c|c|} \hline
a & b \\ \hline
c & d \\ \hline
\end{array} \]
Suppose player R adopts the strategy $p=(p_1,p_2)$ and player C adopts a strategy $q=(q_1,q_2)$. Then R play row 1 with 
probability $p_1$ and C plays column 1 with probability $q_1$, and so the entry $a$ will occur with probability $p_1 \time q_1$, entry $b$ will occur with probability $p_1 \time q_2$ and so on.
The expected winnings for player R
\[ E(p,q) = a(p_1 \time q_1) + b(p_1 \time q_2) + c(p_2 \time q_1) + d(p_2 \time q_2) \]
\[ E(p,q) = (p_1,p_2) \left(\begin{array}{cc} 
a & b \\
c & d \\ \end{array} \right)\left(\begin{array}{c} 
q_1\\
q_2\\ \end{array} \right) = (p A q^t) \]
You can show that E(p,q) can be expressed as $q A^t p^t$.
\smallskip
Similarly if $A$ is any $k \times m$ matrix and player R adopts the strategy $p=()$ and C adopts the strategy $q=()$ then the expected winding of R is given b
\[ E(p,q) =  pAq^t = q A^t p^t \]
Now suppose that $v_1$ is the maximum numer for which there is a strategy $p^0$ for R such that expected winnings of R is at least $v_1$ for any strategy adopted by C:
\[E  \mbox{ for every strategy q of C}\]
%%%%%%%%%%%%%%%%%%%%%%%%%%%%%%%%%%%%%%%%%%%%%%%%
%B25 - Page 323
Since $v_1$ is the maximum of such numbers of such numbers, the strategy $p^0$ is in a certain sense a best strategy that R can adopt; we called $p^0$ an optimum strategy for R.
We show in problem 25.9 page 233 that (1) is equivalent to the condition
\[ A \geq \{ v_1,v1,\ldots,\v_1 \}\]

On the other hand, suppose $v_2$ is the minimum number for which there is a strategy $q^0$ for player $C$ such that 
the expected loss of $C$ is not greater than $v_2$ for any strategy adopted by$R$
Then the strategy $q^0$ is in a certain sense a best strategy that $C$ can adopt, we call $q^0$ an optimum strategy
or C.
%-------------------------------------------------------------%
The fundamental theorem on game theory by Von Neumann states that $v_1$ and $v_2$ exist for any matrix game
and are equal to each other : $v = v_1 = v_2$. This number $v$ is called the value of the game, and the game is said to be fair if $v=0$.ann's th
We observe that (1) and (2) imply that $v_1 leq E(p^0,q^0) \leq v_2$, hence by von Neumann's Theorem we have the 
equality relation
\[ E(p^0,q^0) = p^0 A (q^0)^{t} = q^0 A^t (p^0)^{t} = v\]
WE summarize and formally define the terms introduced above.
%-------------------------------------------------------------%
\subsection{Definition}
In a matrix game A, a strategy $P^0$ for player R is an optimum strategy, a strategy $q^0$ for player C is an optimum strategy.
We point out that  the solution of a matrix means find optimum strategies for the players and the value of the game.
Before we give the solution to an arbitrary matrix game by means of the simplex method
We shall discuss two special cases which occur frequently and which offer simple solutions: striclty determined games and $2 \times 2$ games.

%-------------------------------------------------------------%
\section{Strictly Determined Games}

A matrix game is called strictly determined if the matrix has an entry which is a minimum in its row and a maximum in its column.
Such an entry is called a "saddle point".

Theorem Let $v$ be a saddle point of a strictly determined game. Then an optimum strategy for R is always to play the row 
containing $v$, an optimum strategy for C is always to play the column containing $v$, and $v$ is the value of the game.

Thus in a strictly determined game a pure strategy for each player is an optimum strategy.

\[
A = 
\begin{array}{cccc}
3 & 0 & -4 & -3 \\
2 & 3 & 1 & 2 \\
-4 & 2 & -1 & 3 \\
end{array}
\]

%%%%%%%%%%%%%%%%%%%%%%%%%%%%%%%%%%%%%%%%%%%%%%%%
%B25 - Page 324
We now show that $p^0,q^0$ and $v$ do satisfy the required properties to be optimum strategies and the value of the game:
\begin{enumerate}
\item
\item
\end{enumerate}
We remark that the proof of Theorem 25.1 is similar to the proof above the special case.
\subsection{$2 \times 2$ Matrix Games}
If A is strictly determined, then the solution is indicated in the preceding section. Thus we
need only considered the case in which A is non-strictly determined.
We first state a useful criterion to determine whether or not a $2 \times 2$ matrix game is strictly determined.
Lemma
The following theorem applies
Lemma
Suppose that the above matrix A is strictly determined. Then $P^0 = (x_1,x_2)$ is an optimum strategy for player $R$, $Q^0 = (y_1,y_2)$ is an optimum strategy for player C, and y is the value of the game.
\[ \frac{d-c}{a+d-b-c}, \frac{a-b}{a+d-b-c}\]
\[ \frac{d-b}{a+d-b-c}, \frac{a-c}{a+d-b-c}\]
%%%%%%%%%%%%%%%%%%%%%%%%%%%%%%%%%%%%%%%%%%%%%%%%
%B25 - Page 325
\subsection*{Example 3.1}
Consider the following matrix game (see Example 1.2).
Thus the game is not fair and is in favour of the column player $C$.
\subsection{Recessive Rows and Columns}
Let A be a matrix game. Suppose A contains a row $r_i$ such that $r_i \leq rj$ for some other
row $r_j$.
Recall that $r_i\leq r_j$ means that every entry of $r_i$ is less than or equal to the corresponding entry of $r_j$.
The $r_i$ is called a recessive row, and $r_j$ is said to domninate it.
Cleary player $R$ would always play row $j$ than row $i$ since they are guaranteed to win the same or a greater amount in each possible play of the game.
Accordingly a recessive row can be omitted from the game.
We emphasize the a recessive row contains numbers smaller than those of another
row, whereas a recessive column contains numbers than those of another column.
%------------------------------------------------------%
\subsection{Example 4.1}
Consider the game A.
\[
\begin{array}{ccc}
2 & -1 & 2 \\ \hline
-2 & 3 & 4 \\ \hline
-2 & 3 & 4 \\ \hline
\end{array}.
\]
Note the $(-5,-3,1) \leq (2,-1,2)$, i.e. every entry is the first row is $\leq$ the corresponding entry in
the second row. Thus the first row is recessive and can be omitted from the game may be reduced to 
\[
\begin{array}{ccc}
2 & -1 & 2 \\ \hline
-2 & 3 & 4 \\ \hline
\end{array}.
\]
Now observe that the third row is recessive since each entry is greater than corresponding entry in the second column.
Thus the game can be reduced to a $2 \times 2$ game.

\begin{array}{cc}
2 & -1  \\ \hline
-2 & 3 \\ \hline
\end{array}.

The solution to $A^{\ast}$ can be found by using the Theorem and is

\[ v = \frac{4}{8} = {1}{2} ; p^{\ast}_{0} =(5/8,3/8); q^{\ast}_{0} =(1/2,1/2) \]

Then the solution to the original game $A$ is


\[ v= \frac{1}{2} ; p^{\ast}_{0} =(0,5/8,3/8); q^{\ast}_{0} =(1/2,1/2,0) \]

%%%%%%%%%%%%%%%%%%%%%%%%%%%%%%%%%%%%%%%%%%%%%%%
%B25 - Page 326
\subsection{Solution of a matrix game by the simplex method.}
Suppose we are given a matrix game $A$. We assume that $A$ is non-strictly determined and does not contain any 
recessive rows or columns. We obtain the solution to $A$ as follows:
\begin{itemize}
\item[(1)] Add a sufficiently large number
(The purpose of this step is to guarantee that the value of the new matrix game is positive.)
\item[(2)] Form the following initial tableau:
The tableay corresponds to the linear programming problem:
\item[(3)] Solve the above linear programming problem by the simplex method. Let $P$ be the
optimum solution to the maximum problem, $Q$ the optimum solution to the dual minimum problem.
\begin{multicols}{3}
\begin{enumerate}[(i)]
\item Add a sufficiently large number $k$ to every entry of A to form,say, the following matrix game which has only positive entries:
(The purpose of the step is to guarantee that the value of the new matrix game is positive).
\item
\item
\end{enumerate}
\end{multicols}
\end{itemize}
\subsection*{Example 5.1}
%%%%%%%%%%%%%%%%%%%%%%%%%%%%%%%%%%%%%%%%%%%%%%%%
%B25 - Page 327
\subsection*{Example 5.2}
%%%%%%%%%%%%%%%%%%%%%%%%%%%%%%%%%%%%%%%%%%%%%%%%
%B25 - Page 328
The value of the game is therefore $-4/13$.
%%%%%%%%%%%%%%%%%%%%%%%%%%%%%%%%%%%%%%%%%%%%%%%%
%B25 - Page 329
\section{Summary}
In finding a soultion to the matrix game, you should follow the following steps.
\begin{itemize}
\item[(1)] Test to see if the game is strictly determined.
\item[(2)] Eliminate all recessive rows and columns.
\item[(3)] In the case of $2 \times 2$, use the following formula in Theorem 25.3
\item[(4)] In the case of a $2 \times m$ or an $m \times 2$, reduce the game to a $2 \times 2$ game.
\item[(5)] Use the simplex method in all other cases.
\end{itemize}
\section*{Solved Problems}
\subsection{Exercise 25.1}
\begin{itemize}
\item[(1)]
\item[(2)]
\item[(3)]
\end{itemize}
%%%%%%%%%%%%%%%%%%%%%%%%%%%%%%%%%%%%%%%%%%%%%%%
\subsection{Exercise 25.2}
Find the solution to the following $2 \times 2$ matrix game.
\begin{itemize}
\item[(1)] The game is non-strictly
\item[(2)] The game is strictly determined with saddle point -1
\item[(3)] The game is non-strictly determined.
\item[(4)] The game is non-strictly determined. First compute $a+d-b-c = -3-3-2-5 = -13$
Then $x_1 = 5/13, x_2 = 8/13,y_1=8/13,y_2=5/13$
\end{itemize}
\subsection{Exercise 25.3}
Find the solution to the following matrix game.
First circle each row minimum and box each column maxmimum.
Observe there is no saddle point and so the game is not strictly determined.
We now test for recessive rows and columns.
Not that the third column is recessive since each entry is larger than the corresponding entry in the
second column. Hence the third column can be omitted to obtain the game.
%%%%%%%%%%%%%%%%%%%%%%%%%%%%%%%%%%%%%%%%%%%%%%%
\subsection{Exercise 25.4}
\subsection{Exercise 25.5}
%%%%%%%%%%%%%%%%%%%%%%%%%%%%%%%%%%%%%%%%%%%%%%%
\subsection{Exercise 25.6}
\subsection{Exercise 25.7}
%%%%%%%%%%%%%%%%%%%%%%%%%%%%%%%%%%%%%%%%%%%%%%%
\subsection{Exercise 25.8}
Two players $R$ and $C$ match pennies. If the coins match, then R wins, if the coins
do not match then C wins. Determine optimum strategies
\subsection{Exercise 25.9}
\subsection{Exercise 25.10}
%%%%%%%%%%%%%%%%%%%%%%%%%%%%%%%%%%%%%%%%%%%%%%%
\subsection{Exercise 25.11}
%%%%%%%%%%%%%%%%%%%%%%%%%%%%%%%%%%%%%%%%%%%%%%%
%%%%%%%%%%%%%%%%%%%%%%%%%%%%%%%%%%%%%%%%%%%%%%%
\end{document}
