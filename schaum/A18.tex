% A18

%%%%%%%%%%%%%%%%%%%%%%%%%%%%%%%%%%%%%%

\section{Decision Processes}

A decision process is a process requiring either a single or sequential set 
of decisions for its completion


\subsection{Example 18.1}

A major energy company offers a landowner


%%%%%%%%%%%%%%%%%%%%%%%%%%%%%%%%%%%%%%
%Page 326

Decision Trees

Naive Decision Criteria

A Priori Criterion

A Posteriori criterion

Utility
%%%%%%%%%%%%%%%%%%%%%%%%%%%%%%%%%%%%%


% Decision Theory
% Page 327


Lotteries

a Lottery P(A,B;p) is a random event having two outcomes, A and B, occuring with probabilities
$p$ and $1-p$ resspectively.

Von Neuman Utilities

\begin{description}
\item[Step 1]
\item[Step 2]
\item[Step 3]
\end{description}

A utility is normalized if $u(e_1)=1$ and $u(E_p)=0$, making the utilities identical to the
equivalence probabilites.

%%%%%%%%%%%%%%%%%%%%%%%%%%%%%%%%%

18.2


18.3


18.4 Determine the recommended decision under the \textit{a priori} criterion for the
decision process in problem 18.2

%%%%%%%%%%%%%%%%%%%%%%%%%%%%%%%%%%%%%%%%%%%%%%%%%%

State and Prove Bayes Theorem


Consider a single purpose space $\mathcal{P}$ consisting of all possible outcomes
of a conceptual experiment.

If A and B are two events (subsets ) of P, then the conditional probabilityof A giventhen B has soocured


18.10
18.11
Determine the recommended decision under the a priori criterion for the landowner example in problem 18.1
18.12
18.13
18.14
18.15
18.16
18.17
18.18
18.19
18.20
