\documentclass[]{article}
\voffset=-1.5cm
\oddsidemargin=0.0cm
\textwidth = 480pt


\usepackage{amsmath}
\usepackage{graphicx}
\usepackage{amssymb}
\usepackage{framed}
\usepackage{multicol}
%\usepackage[paperwidth=21cm, paperheight=29.8cm]{geometry}
%\usepackage[angle=0,scale=1,color=black,hshift=-0.4cm,vshift=15cm]{background}
%\usepackage{multirow}
\usepackage{enumerate}

\usepackage{amsmath,amsfonts,amssymb}
\usepackage{color}
\usepackage{multirow}
\usepackage{eurosym}
\usepackage{framed}

%\graphicspath{ {images/} }
%\input def.tex
%\input dsdef.tex
%\input rgb.tex

%\newcommand \la{\lambda}
%\newcommand \al{a}
%\newcommand \be{b}
\newcommand \x{\overline{x}}
\newcommand \y{\overline{y}}
\begin{document}


%- III – 15
\newpage
\section{The Bertrand Model of Duopoly.} In 1883, J. Bertrand proposed a different
model of competition between two duopolists, based on allowing the firms to set prices
rather than to fix production quantities. In this model, demand is a function of price
rather than price a function of quantity available.
First consider the case where the two goods are identical and price information to the
consumer is perfect so that the firm that sets the lower price will corner the market. We
use the same price/demand function (1) solved for demand Q in terms of price P,
\[Q(P) = a − P if 0 \leq P \leq a\]
0 if P>a = (a − P)
+. (8)
%%- III – 18
\begin{itemize}
    \item The actual demand is Q(P) where P is the lowest price. The monopoly behavior under this
model is the same as for the Cournot model of the previous section. 
\item The monopolist sets
the price at (a + c)/2 and produces the quantity (a − c)/2, receiving a profit of (a − c)2/4.
\item Suppose firms 1 and 2 choose prices p1 and p2 respectively. We assume that if p1 = p2
the firms share the market equally. We take the cost for a unit production again to be
$c > 0$ so that the profit is pi − c times the amount sold.
\end{itemize}
 Then the payoff functions are
u1(p1, p2) =
⎧
⎨
⎩
(p1 − c)(a − p1)+ if p1 < p2
(p1 − c)(a − p1)+/2 if p1 = p2
0 if p1 > p2
(9)
and
u2(p1, p2) =
⎧
⎨
⎩
(p2 − c)(a − p2)+ if p2 < p1
(p2 − c)(a − p2)+/2 if p2 = p1
0 if p2 > p1
(10)
Here there is a unique PSE but it is rather disappointing. Both firms charge the production
cost, p^{\ast}
1 = p^{\ast}
2 = c, and receive a payoff of zero. This is the safety level for each player. It
is easy to check that this is an equilibrium. No other pair of prices can be an equilibrium
because either firm could capture the entire market by slightly undercutting the other’s
price.
\begin{itemize}
    \item This feature of capturing the entire market by undercutting the other’s price is not
entirely reasonable for a number of reasons. Usually the products of two firms are not
entirely interchangeable so some consumers may prefer one product to the other even if
it costs somewhat more.
\item In addition there is the problem of the consumer getting the
information on the prices, and there is the feature of brand loyalty by consumers. We may
modify the model in an attempt to take this into account.
\end{itemize}
\end{document}
\subsubsection{The Bertrand Model with Differentiated Products.}
\begin{itemize}
    \item \item 
\end{itemize}Again we assume that the firms
choose prices p1 and p2 and that the cost of unit production is c > 0. \item Since the profits per
unit produced are $p1 − c$ and $p2 − c$, we may assume that the prices will satisfy $p1 \geq c$ and
$p2 \geq c$. 
\end{itemize}
This time we assume that the demand functions of the products of the firms for
given price selections are given by
q1(p1, p2)=(a − p1 + bp2)
+
q2(p1, p2)=(a − p2 + bp1)
+, (11)
where $b > 0$ is a constant representing how much the product of one firm is a substitute
for the product of the other. We assume $b \leq 1$ for simplicity. These demand functions
are unrealistic in that one firm could conceivably charge an arbitrarily high price and still
have a positive demand provided the other firm also charges a high enough price. However,
this function is chosen to represent a linear approximation to the “true” demand function,
appropriate near the usual price settings where the equilibrium is reached.
Under these assumptions, the strategy sets of the firms are X = [0,∞) and $Y = [0,\infty)$,
and the payoff functions are
\[u1(p1, p2) = q1(p1, p2)(p1 − c)=(a − p1 + bp2)
+(p1 − c)\]
\[u2(p1, p2) = q2(p1, p2)(p2 − c)=(a − p2 + bp1)
+(p2 − c). (12)\]
%%- III – 19
To find the equilibrium prices, we must find points (p^{\ast}
1, p^{\ast}
2) at which u1 is maximized in p1
and u2 is maximized in p2 simultaneously. Assuming a−p1 +bp2 > 0 and a−p2 +bp1 > 0,
we find
∂
∂p1
u1(p1, p2) = a − 2p1 + bp2 + c = 0
\[ ∂
∂p2
u2(p1, p2) = a − 2p2 + bp1 + c = 0.\]
Again the functions are quadratic in the variable of differentiation with a negative coeffi-
cient, so the resulting roots represent maxima. Solving simultaneously and denoting the
result by p^{\ast}
1 and p^{\ast}
2, we find
p^{\ast}
1 = p^{\ast}
2 = a + c
2 − b
.
\newpage
\section{The Stackelberg Model of Duopoly.} In the Cournot and Bertrand models
of duopoly, the players act simultaneously. H. von Stackelberg (1934) proposed a model
of duopoly in which one player, called the dominant player or leader, moves first and the
outcome of that player’s choice is made known to the other player before the other player’s
choice is made. An example might be General Motors, at times big enough and strong
enough in U.S. history to play such a dominant role in the automobile industry. Let us
analyze the Cournot model from this perspective.
\begin{itemize}
    \item Firm 1 chooses an amount to produce, $q_1$, at a cost c per unit. This amount is then
told to Firm 2 which then chooses an amount q2 to produce also at a cost of c per unit.
\item Then the price P per unit is determined by equation (1), P = (a − q1 − q2)+, and the
players receive $u1(q1, q2)$ and $u2(q1, q2)$ of equations (2) and (3).
\item Player I’s pure strategy space is X = $[0,\infty)$. From the mathematical point of view, the
only difference between this model and the Cournot model is that Firm 2’s pure strategy
space, Y , is now a set of functions mapping q1 into q2. However, this is now a game of
perfect information that can be solved by backward induction.
\item Since Firm 2 moves last,
we first find the optimal q2 as a function of q1. That is, we solve equation (5) for q2. This
gives us Firm 2’s strategy as
q2(q1)=(a − q1 − c)/2. (13)
Since Firm 1 now knows that Firm 2 will choose this best response, Firm 1 now wishes to
choose q1 to maximize
u1(q1, q2(q1)) = q1(a − q1 − (a − q1 − c)/2) − cq1
= −1
2
q2
1 +
a − c
2 q1. (14)
\end{itemize}

This quadratic function is maximized by q1 = q^{\ast}
1 = (a−c)/2. Then Firm 2’s best response
is \[q^{\ast}
2 = q2(q^{\ast}
1 )=(a − c)/4.\]
Let us analyze this SE and compare its payoff to the payoff of the SE in the Cournot
duopoly. Firm 1 produces the monopoly quantity and Firm 2 produces less than the
%- III – 20
Cournot SE. The payoff to Firm 1 is u1(q^{\ast}
1 , q^{\ast}
2 )=(a − c)2/8 and the payoff to Firm 2
is 

\[u2(q^{\ast}
1 , q^{\ast}
2 )=(a − c)2/16.\] Therefore Firm 1’s profits are greater than that given by
the Cournot equilibrium, and Firm 2’s are less. Note that the total amount produced is
(3/4)(a − c), which is greater than $(2/3)(a − c)$, the total amount produced under the
Cournot equilibrium. This means the Stackelberg price is lower than the Cournot price,
and the consumer is better off under the Stackelberg model.
The information that Firm 2 received about Firm 1’s production has been harmful.
Firm 1 by announcing its production has increased its profit. This shows that having more
information may make a player worse off. More precisely, being given more information
and having that fact be common knowledge may make you worse off.
%==================================================%
\newpage
\subsection{Entry Deterrence.} 
\begin{itemize}
    \item Even if a firm acts as a monopolist in a certain market, there
may be reasons why it is in the best interests of the firm to charge less than the monopoly
price, or equivalently, produce more than the monopoly production. 
\item One of these reasons
is that the high price of the good achieved by monopoly production may attract another
firm to enter the market.
\item We can see this in the following example. Suppose the price/demand relationship can
be expressed as
\[P(Q) =  17 − Q if 0 \leq Q \leq 17\]
0 otherwise, (15)
where Q represents the total amount produced, and P represents the price per unit amount.
\item Suppose additionally, that the cost to the firm of producing q1 items is q1 + 9. That is,
there is a fixed cost of 9 and a constant marginal cost of 1 per unit quantity. 
\item The profit
to the firm of producing quantity q1 of the good is
\[u(q1) = (17 − q1)q1 − (q1 + 9) = 16q1 − q2
1 − 9. \]
\end{itemize}
The value of q1 that maximizes the profit is found by setting the derivative of u(q1) to zero
and solving for q1:
$u (q1) = 16 − 2q1 = 0.$$
So the monopoly production is
q1 = 8,
the monopoly price is 9, and the monopoly profit is
\[u(8) = 9 \times 8 − 17 = 55\].
Suppose now a competing firm observes this market and thinks of producing a small
amount, q2, to hopefully make a small profit. Suppose also that this firm also has the same
cost, q2+9, as the monopoly firm. On producing q2 the price will drop to $P(8+q2)=9−q2$,
and the competing firm’s profit will be
\[u2 = (9 − q2)q2 − (q2 + 9) = 8q2 − q2
2 − 9.\]
%%- III – 21
\begin{itemize}
    \item This is maximized at q2 = 4 and the profit there is u2 = 7. Since this is positive, the firm
has an incentive to enter the market.
\item Of course, the incumbent monopolist can foresee this possibility and can calculate the
negative effect it will have on the firm’s profits.
\item If the challenger enters the market with a
production of 4, the price will drop to P(8+4) = 5, and the monopolist’s profits will drop
from 55 to 5 · 8 − 17 = 23.
\item It seems reasonable that some preventative measures might be
worthwhile.
\end{itemize}

If the monopolist produces a little more than the monopoly quantity, it might deter
the challenger from entering the market. How much more should be produced? If the
monopolist produces q1, then the challenger’s firm’s profits may be computed as in (17)
by
\[u2(q1, q2) = (17 − q1 − q2)q2 − (q2 + 9)\].
This is maximized at q2 = (16 − q1)/2 for a profit of
u2(q1,(16 − q1)/2) = (16 − q1)
2/4 − 9.
The profit is zero if (16 − q1)2 = 36, or equivalently, if q1 = 10.
\begin{itemize}
    \item This says that if the monopolist produces 10 rather than 8, then the challenger can
see that it is not profitable to enter the market.
However, the monopolist’s profits are reduced by producing 10 rather than 8. From
(16) we see that the profits to the firm when q1 = 10 are
u1(10) = 7 · 10 − 19 = 51
instead of 55. 
\item This is a relatively small amount to pay as insurance against the much
bigger drop in profits from 55 to 23 the monopolist would suffer if the challenger should
enter the market.
\item The above analysis assumes that the challenger believes that, even if the challenger
should enter the market, the monopolist will continue with the monopoly production,
or the pre-entry production. This would be the case if the incumbent monopolist were
considered as the dominant player in a Stackelberg model. 
\item Note that the entry deterrence
strategy pair, q1 = 10 and q2 = 0, does not form a strategic equilibrium in this Stackelberg
model, since q1 = 10 is not a best response to q2 = 0. To analyze the situation properly,
we should enlarge the model to allow the game to be played sequentially several times.
If this problem were analyzed as a Cournot duopoly, we would find that, at equilibrium,
each firm would produce 5 1
3 , the price would drop to 6 1
3 , and each firm would realize a
profit of 19 4
9 . This low profit is another reason that the incumbent firm should make strong
efforts to deter the entry of a challenger.
III – 22
\end{itemize}

\section{3.5 Exercises.}
\begin{enumerate}
    \item 1.
    \begin{itemize}
        \item[(a)] Suppose in the Cournot model that the firms have different production costs.
Let c1 and c2 be the costs of production per unit for firms 1 and 2 respectively, where both
c1 and c2 are assumed less than a/2. Find the Cournot equilibrium.
\item[(b)] What happens, if in addition, each firm has a set up cost? Suppose Player I’s
cost of producing x is x + 2, and II’s cost of producing y is 3y + 1. Suppose also that the
price function is p(x, y) = 17 − x − y, where x and y are the amounts produced by I and
II respectively. What is the equilibrium production, and what are the players’ equilibrium
payoffs?
\end{itemize}
\item  Extend the Cournot model of Section 3.1 to three firms. Firm i chooses to produce
qi at cost cqi where c > 0. The selling price is P(Q)=(a − Q)+ where Q = q1 + q2 + q3.
What is the strategic equilibrium?
\item  Modify the Bertrand model with differentiated products to allow sequential selection
of the price as in Stackelberg’s variation of Cournot’s model. The dominant player
announces a price first and then the subordinate player chooses a price. Solve by backward
induction and compare to the SE for the simultaneous selection model.
\item Consider the Cournot duopoly model with the somewhat more realistic price function,
P(Q) =  1
4Q2 − 5Q + 26 for 0 ≤ Q ≤ 10,
1 for $Q \geq 10$.
This price function starts at 26 for Q = 0 and decreases down to 1 at Q = 10 and then
stays there. 
%===============================================%
Assume that the cost, c, of producing one unit is $c = 1$ for both firms. No
firm would produce more than 10 because the return for selling a unit would barely pay
for the cost of producing it. Thus we may restrict the productions q1, q2, to the interval
[0, 10].
\begin{enumerate}[(a)]
\item Find the monopoly production, and the optimal monopoly return.
\item Show that if q2 = 5/2, then u1(q1, 5/2) is maximized at q1 = 5/2. Show that this
implies that q1 = q2 = 5/2 is an equilibrium production in the duopoly.
\item (a) Suppose in the Stackelberg model that the firms have different production costs.
Let c1 and c2 be the costs of production per unit for firms 1 and 2 respectively. Find the
Stackelberg equilibrium. For simpicity, you may assume that both c1 and c2 are small, say
less than a/3.
(b) Suppose in the Stackelberg model, Player I’s cost of producing x is x +2, and II’s
cost of producing y is $3y + 1$. 

Suppose also that the price function is $p(x, y) = 17 − x − y$,
where x and y are the amounts produced by I and II respectively. What is the equilibrium
production, and what are the players’ equilibrium payoffs?
\item Extend the Stackelberg model to three firms. For i = 1, 2, 3, Firm i chooses to
produce qi at cost cqi where $c > 0$. 

Firm 1 acts first in announcing the production q1. Then
%=======================================%

Firm 2 announces q2, and finally Firm 3 announces q3. The selling price is P(Q)=(a−Q)+
where $Q = q1 + q2 + q3$. What is the strategic equilibrium?
\item  An Advertising Campaign. Two firms may compete for a given market of
total value, V , by investing a certain amount of effort into the project through advertising,
securing outlets, etc. Each firm may allocate a certain amount for this purpose. If firm 1
allocates x > 0 and firm 2 allocates y > 0, then the proportion of the market that firm 1
corners is $x/(x+ y)$. 
%===========================%
The firms have differing difficulties in allocating these resources. The
cost per unit allocation to firm i is ci, i = 1, 2. Thus the profits to the two firms are
M1(x, y) = V · x
x + y − c1x
M2(x, y) = V · y
x + y − c2y
If both x and y are zero, the payoffs to both are zero.
(a) Find the equilibrium allocations, and the equilibrium profits to the two firms, as
a function of V , c1 and c2.
(b) Specialize to the case V = 1, c1 = 1, and c2 = 2.
\end{enumerate}

\end{document}