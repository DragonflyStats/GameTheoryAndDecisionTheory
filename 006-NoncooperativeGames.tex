\documentclass{article}
\usepackage[utf8]{inputenc}
\usepackage{framed}
\title{Prisoners Dilemma}
\author{Kevin O'Brien}
\date{September 2017}

\begin{document}


\section{Noncooperative Games}
%- III – 6
%- 2. 
%==========================================================%
Two-person general-sum games and n-person games for n > 2 are more difficult to
analyze and interpret than the zero-sum two-person games of Part II. The notion of “optimal”
behavior does not extend to these more complex situations. In the noncooperative
theory, it is assumed that the players cannot overtly cooperate to attain higher payoffs. If
communication is allowed, no binding agreements may be formed. One possible substitute
for the notion of a “solution” of a game is found in the notion of a strategic equilibrium.

\subsection{Strategic Equilibria.} A finite n-person game in strategic form is given by n
nonempty finite sets, X1, X2, ... , Xn, and n real-valued functions u1, u2, ... , un, defined
on X1 × X2 × ··· × Xn. The set Xi represents the pure strategy set of player i and
ui(x1, x2,...,xn) represents the payoff to player i when the pure strategy choices of the
players are $$\{x1, x2, ... , xn\}$, with xj ∈ Xj for j = 1, 2,...,n.
\begin{framed}
Definition. A vector of pure strategy choices (x1, x2,...,xn) with xi ∈ Xi for i = 1,...,n
is said to be a pure strategic equilibrium, or PSE for short, if for all i = 1, 2,...,n, and
for all x ∈ Xi,
ui(x1,...,xi−1, xi, xi+1,...,xn) ≥ ui(x1,...,xi−1, x, xi+1,...,xn). (1)
\end{framed}
Equation (1) says that if the players other than player i use their indicated strategies,
then the best player i can do is to use xi. Such a pure strategy choice of player i is called
a best response to the strategy choices of the other players. The notion of strategic
equilibrium may be stated: a particular selection of strategy choices of the players forms
a PSE if each player is using a best response to the strategy choices of the other players.
Consider the following examples with two players,
(a)  (3, 3) (0, 0)
(0, 0) (5, 5) 
(b)  (3, 3) (4, 3)
(3, 4) (5, 5) 
In (a), the first row, first column, denoted 1, 1, is a strategic equilibrium with equilibrium
payoff (3, 3). If each believes the other is going to choose the first strategy, neither player
will want to change to the second strategy. The second row, second column, 2, 2, is
also a strategic equilibrium. Since its equilibrium payoff is (5, 5), both players prefer this
equilibrium. In (b), the first row, first column, 1, 1, is still an equilibrium according to
the definition. Neither player can gain by changing strategy. On the other hand, neither
player can be hurt by changing, and if they both change, they both will be better off. So
the equilibrium 1, 1 is rather unstable.
Example (a) is of a game in which the players receive the same payoff, but are not
allowed to communicate. If they were allowed to communicate, they would choose the
joint action giving the maximum payoff. Other examples of this nature occur in the class
of rendezvous games, in which two players randomly placed on a graph, each not knowing
III – 7
the position of the other, want to meet in minimum time. See the book of Alpern and Gal
(2003).
\begin{itemize}
    \item If players in a noncooperative game are allowed to communicate and do reach some
informal agreement, it may expected to be a strategic equilibrium.
\item Since no binding
agreements may be made, the only agreements that may be expected to occur are those that
are self-enforcing, in which no player can gain by unilaterally violating the agreement.
\item Each player is maximizing his return against the strategy the other player announced he
will use.
It is useful to extend this definition to allow the players to use mixed strategies. We
denote the set of probabilities over k points by Pk:
Pk = {p = (p1,...,pk) : pi ≥ 0 for i = 1,...,k, and k
1 pi = 1}. (2)
\end{itemize}

Let mi denote the number of pure strategy choices of player i, so that the set Xi has mi
elements. Then the set of mixed strategies of player i is just Pmi . It is denoted by X∗
i
where X∗
i = Pmi .
We denote the set of elements of Xi by the first mi integers, Xi = {1, 2,...,mi}.
Suppose that for i = 1, 2,...,n, Player i uses pi = (pi1, pi2,...,pimi ) ∈ X∗
i . Then the
average payoff to player j is
gj(p1,..., pn) = m1
i1=1
··· mn
in=1
p1i1 ··· pnin uj(i1,...,in). (3)
Then the analogous definition of equilibrium using mixed strategies is as follows.
Definition. A vector of mixed strategy choices (p1, p2,..., pn) with pi ∈ X∗
i for i =
1,...,n is said to be a strategic equilibrium, or SE for short, if for all i = 1, 2,...,n, and
for all p ∈ X∗
i ,
gi(p1,..., pi−1, pi, pi+1,..., pn) ≥ gi(p1,..., pi−1, p, pi+1,..., pn). (4)
Any mixed strategy pi that satisfies (4) for all p ∈ X∗
i is a best response of player i
to the mixed strategies of the other players. Thus, a particular selection of mixed strategy
choices of the players forms an SE if and only if each player is using a best response to the
strategy choices of the other players. No player can gain by unilaterally changing strategy.
Note that a PSE is a special case of an SE.
This notion of best response represents a practical way of playing a game: Make
a guess at the probabilities that you think your opponents will play their various pure
strategies, and choose a best response to this. This is an example of the famous Bayesian
approach to decision making. Of course in a game, this may be a dangerous procedure.
Your opponents may be better at this type of guessing than you.
The first question that arises is “Do there always exist strategic equilibria?”. This
question was resolved in 1951 by John Nash in the following theorem which generalizes
von Neumann’s minimax theorem. In honor of this achievement, strategic equilibria are
also called Nash equilibria.
III – 8


%=============================================================%
\begin{framed}
\noindent \textbf{Theorem.} Every finite n-person game in strategic form has at least one strategic equilibrium.
\end{framed}
A proof of this theorem using the Brouwer Fixed Point Theorem is given in Appendix
3. This proof is an existence proof and gives no indication of how to go about finding
equilibria. However, in the case of bimatrix games where n = 2, the Lemke-Howson
algorithm (Journal SIAM, 1964, 12, 413-423) may be used to compute strategic equilibria
in a finite number of steps using a simplex-like pivoting algorithm (see Parthasarathy and
Raghavan (1971) for example). An interesting consequence of this method is that, under
a nondegeneracy condition, the number of SE’s is finite and odd!
One of the difficulties of the noncooperative theory is that there are usually many
equilibria giving different payoff vectors as we shall see in the following examples. Another
difficulty is that even if there is a unique strategic equilibrium, it may not be considered
as a reasonable solution or a predicted outcome. In the rest of this section we restrict
attention to n = 2, the two-person case.


%=============================================================%
\subsubsection{2.2 Examples.} Example 1. A Coordination Game. Consider the game with
bimatrix
 (3, 3) (0, 2)
(2, 1) (5, 5) 
and corresponding payoff matrices
A =
 3 0
2 5 
and B =
 3 2
1 5 
The corresponding maxmin (MM) strategies are (1/2,1/2) for Player I and (3/5,2/5) for
Player II. The safety levels are (vI , vII ) = (5/2, 13/5).
Here there are two obvious pure strategic equilibria (PSE’s) corresponding to the
payoffs (3,3) and (5,5). Both players prefer the second SE because it gives them both
5 instead of 3. If they could coordinate their actions, this outcome would be expected.
However, if they cannot communicate and if both players believe the other is going to
choose the first strategy, then they are both going to choose the first strategy and receive
the payoff 3. One cannot say the outcome (3,3) is irrational. If that’s the way things
have always been, then one who tries to change things hurts oneself. This phenomenon
occurs often, usually with many players. To try to change the structure of a language or
the typewriter keyboard or the system of measurement requires a lot of people to change
simultaneously before any advantage is realized.

There is a third less obvious equilibrium point that sometimes occurs in these games.
If each player has an equalizing strategy for the other player’s matrix, then that pair of
strategies forms an equilibrium. This is because if an opponent uses a strategy that makes
it not matter what you do, then anything you do is a best response, in particular the
equalizing strategy on the opponent’s matrix. (Recall that an equalizing strategy is one
that gives the same average payoff to the opponent no matter what the opponent does.)
III – 9
Let us find this equalizing strategic equilibrium for the above game. Note that
each player uses the matrix of his opponent. 

Player I has the equalizing strategy p =
(4/5, 1/5) for B, and Player II has the equalizing strategy q = (5/6, 1/6) for A. If the
players use these strategies, the average payoff is (5/2, 13/5), the same as the safety levels.
Is it possible that the average payoff from a strategic equilibrium is less than the
safety level for one of the players? The answer is no. (See Exercise 1.) Therefore the
strategic equilibrium (p, q) is as poor a strategic equilibrium as you can get. Moreover,
it is extremely unstable. It is true that it does neither player any good to deviate from
his/her equilibrium strategy, but on the other hand it does not harm a player to change
to another strategy.

In the above example, the payoffs for the three SE’s are all different. The players have
the same preferences as to which of the three outcomes they would prefer. In the next
example, the players have different preferences between the two pure strategic equilibria.
Example 2. The Battle of the Sexes. Suppose the matrices are

a b
a (2, 1) (0, 0)
b (0, 0) (1, 2) 
so that A =

a b
a 2 0
b 0 1 
and B =

a b
a 1 0
b 0 2 
.
The name of this game arises as a description of the game played between a husband
and wife in choosing which movie to see, a or b. They prefer different movies, but going
together is preferable to going alone. Perhaps this should be analyzed as a cooperative
game, but we analyze it here as a noncooperative game.
The pure strategy vectors (a, a) and (b, b) are both PSE’s but Player I prefers the first
and Player II the second.
First note that the safety levels are vI = vII = 2/3, the same for both players. Player
I’s MM strategy is (1/3,2/3), while Player II’s MM strategy is (2/3,1/3). There is a third
strategic equilibrium given by the equalizing strategies p = (2/3, 1/3) and q = (1/3, 2/3).
The equilibrium payoff for this equilibrium point, (vI , vII ) = (2/3, 2/3), is worse for both
players than either of the other two equilibrium points.


%=============================================================%
Example 3. The Prisoner’s Dilemma. It may happen that there is a unique SE
but that there are other outcomes that are better for both players. Consider the game
with bimatrix

cooperate defect
cooperate (3, 3) (0, 4)
defect (4, 0) (1, 1) 
In this game, Player I can see that no matter which column Player II chooses, he will be
better off if he chooses row 2. For if Player I chooses row 2 rather than row 1, he wins
4 rather than 3 if Player II chooses column 1, and he wins 1 rather than 0 if she chooses
column 2. In other words, Player I’s second strategy of choosing the second row strictly
dominates the strategy of choosing the first. On the other hand, the game is symmetric.
%- III – 10
\begin{itemize}
    \item Player II’s second column strictly dominates her first. However, if both players use their
dominant strategies, each player receives 1, whereas if both players use their dominated
strategies, each player receives 3.
\item A game that has this feature, that both players are better off if together they use
strictly dominated strategies, is called the Prisoner’s Dilemma. The story that leads to this
bimatrix and gives the game its name is as follows. Two well-known crooks are captured
and separated into different rooms.
\item The district attorney knows he does not have enough
evidence to convict on the serious charge of his choice, but offers each prisoner a deal. If
just one of them will turn state’s evidence (i.e. rat on his confederate), then the one who
confesses will be set free, and the other sent to jail for the maximum sentence. 
\item If both
confess, they are both sent to jail for the minimum sentence. If both exercise their right
to remain silent, then the district attorney can still convict them both on a very minor
charge.
\item In the numerical matrix above, we take the units of measure of utility to be such
that the most disagreeable outcome (the maximum sentence) has value 0, and the next
most disagreeable outcome (minimum sentence) has value 1. Then we let being convicted
on a minor charge to have value 3, and being set free to have value 4.
This game has abundant economic application. An example is the manufacturing by
two companies of a single good. Both companies may produce either at a high or a low
level. If both produce at a low level, the price stays high and they both receive 3.
\end{itemize}
 If they
both produce at the high level the price drops and they both receive 1. If one produces at
the high level while the other produces at the low level, the high producer receives 4 and
the low producer receives 0. No matter what the other producer does, each will be better
off by producing at a high level.


%=============================================================%

\section{2.3 Finding All PSE’s.} For larger matrices, it is not difficult to find all pure strategic
equilibria. This may be done using an extension of the method of finding all saddle points
of a zero-sum game. With the game written in bimatrix form, put an asterisk after each
of Player I’s payoffs that is a maximum of its column. Then put an asterisk after each of
Player II’s payoffs that is a maximum of its row. Then any entry of the matrix at which
both I’s and II’s payoffs have asterisks is a PSE, and conversely.
An example should make this clear.
⎛
⎜⎜⎝
a b cd e f
A (2, 1) (4, 3) (7∗, 2) (7∗, 4) (0, 5∗) (3, 2)
B (4∗, 0) (5∗, 4) (1, 6∗) (0, 4) (0, 3) (5∗, 1)
C (1, 3∗) (5∗, 3∗) (3, 2) (4, 1) (1∗, 0) (4, 3∗)
D (4∗, 3) (2, 5∗) (4, 0) (1, 0) (1∗, 5∗) (2, 1)
⎞
⎟⎟⎠
In the first column, Player I’s maximum payoff is 4, so both 4’s are given asterisks. In the
first row, Player II’s maximum is 5, so the 5 receives an asterisk. And so on.
When we are finished, we see two payoff vectors with double asterisks. These are the
pure strategic equilibria, (C, b) and (D, e), with payoffs (5, 3) and (1, 5) respectively. 

%=============================================================%
At
III – 11
all other pure strategy pairs, at least one of the players can improve his/her payoff by
switching pure strategies.
In a two-person zero-sum game, a PSE is just a saddle point. Many games have no
PSE’s, for example, zero-sum games without a saddle point. However, just as zero-sum
games of perfect information always have saddle points, non-zero-sum games of perfect
information always have at least one PSE that may be found by the method of backward
induction.
%==========================================================================%
\subsection{Iterated Elimination of Strictly Dominated Strategies.}
\begin{itemize}
    \item Since in generalsum
games different equilibria may have different payoff vectors, it is more important than
in zero-sum games to find all strategic equilibria. We may remove any strictly dominated
row or column without losing any equilibrium points (Exercise 7).
\item We, being rational, would not play a strictly dominated pure strategy, because there
is a (possibly mixed) strategy that guarantees us a strictly better average payoff no matter
what the opponent does. Similarly, if we believe the opponent is as rational as we are, we
believe that he/she will not use a dominated strategy either.
\item Therefore we may cancel any
dominated pure strategy of the opponent before we check if we have any dominated pure
strategies which may now be eliminated.
\item This argument may be iterated. If we believe our opponent not only is rational but
also believes that we are rational, then we may eliminate our dominated pure strategies,
then eliminate our opponent’s dominated pure strategies, and then again eliminate any
of our own pure strategies that have now become dominated. 
\item The ultimate in this line
of reasoning is that if it is common knowledge that the two players are rational, then we
may iteratively remove dominated strategies as long as we like. (A statement is “common
knowledge” between two players if each knows the statement, and each knows the other
knows the statement, and each knows the other knows the other knows the statement, ad
infinitum.)
\item As an example of what this sort of reasoning entails, consider a game of Prisoner’s
Dilemma that is to be played sequentially 100 times. The last time this is to be played it is
clear that rational players will choose to defect. 
\item The other strategy is strictly dominated.
But now that we know what the players will do on the last game we can apply strict
domination to the next to last game to conclude the players will defect on that game too.
\item Similarly all the way back to the first game. The players will each receive 1 at each game.
If they could somehow break down their belief in the other’s rationality, they might receive
3 for each game.

\end{itemize}
Here is another game, called the Centipede Game of Robert Rosenthal (1980), that
illustrates this anomaly more vividly. This is a game of perfect information with no chance
moves, so it is easy to apply the iterated removal of strictly dominated strategies. The
game in extensive form is given in Figure 2.1.
Since this is a game of perfect information, it may be solved by backward induction.
At the last move, Player II will certainly go down instead of across since that gives her
101 instead of 100. Therefore at the next to last move, Player I will go down rather than
III – 12
I II I II I II I II
(1,1) (0,3) (2,2) (1,4) (98,98) (97,100) (99,99) (98,101)
(100,100)
Figure 2.1 The Centipede Game.
across since that gives him 99 instead of the 98. And so forth, back to the initial position,
where Player I will go down rather than across because he receives 1 instead of 0. This is
the unique PSE because all eliminated strategies were strictly dominated.
Empirical evidence acquired by playing similar games shows that this gives a poor
prediction of how people actually play this game. See the book of David M. Kreps (1990)
Game Theory and Economic Modeling, Oxford University Press, for a discussion.
\section{2.5 Exercises.}
1. Strategic Equilibria Are Individually Rational. A payoff vector is said to
be individually rational if each player receives at least his safety level. Show that if (p, q)
is a strategic equilibrium for the game with matrices A and B, then pT
Aq ≥ vI and
pT
Bq ≥ vII . Thus, the payoff vector for a strategic equilibrium is individually rational.
2. Find the safety levels, the MM-strategies, and find all SE’s and associated vector
payoffs of the following games in strategic form.
(a)  (0, 0) (2, 4)
(2, 4) (3, 3) 
. (b)  (1, 4) (4, 1)
(2, 2) (3, 3) 
. (c)  (0, 0) (0, −1)
(1, 0) (−1, 3) 
.

%=============================================================%

3. The Game of Chicken. Two players speed head-on toward each other and a
collision is bound to occur unless one of them chickens out at the last minute. If both
chicken out, everything is okay (they both win 1). If one chickens out and the other does
not, then it is a great success for the player with iron nerves (payoff = 2) and a great
disgrace for the chicken (payoff = −1). If both players have iron nerves, disaster strikes
(both lose 2).
(a) Set up the bimatrix of this game.
(b) What are the safety levels, what are the MM strategies, and what is the average payoff
if the players use the MM strategies?
(c) Find all three SE’s.


%=============================================================%

4. An extensive form non-zero-sum game. A coin with probability 2/3 of heads
and 1/3 of tails is tossed and the outcome is shown to player I but not to player II. Player
I then makes a claim which may be true or false that the coin turned up heads or that the
coin turned up tails. Then, player II, hearing the claim, must guess whether the coin came
up heads or tails. Player II wins $3 if his guess is correct, and nothing otherwise. Player
I wins $3 if I has told the truth in his claim. In addition, Player I wins an additional $6 if
player II guesses heads.
III – 13
(a) Draw the Kuhn tree.
(b) Put into strategic (bimatrix) form.
(c) Find all PSE’s.
5. Find all PSE’s of the following games in strategic form.
(a)
⎛
⎜⎝
(−3, −4) ( 2, −1) ( 0, 6) ( 1, 1)
( 2, 0) ( 2, 2) (−3, 0) ( 1, −2)
( 2, −3) (−5, 1) (−1, −1) ( 1, −3)
(−4, 3) ( 2, −5) ( 1, 2) (−3, 1)
⎞
⎟⎠.
(b)
⎛
⎜⎜⎜⎝
( 0, 0) ( 1, −1) ( 1, 1) (−1, 0)
(−1, 1) ( 0, 1) ( 1, 0) ( 0, 0)
( 1, 0) (−1, −1) ( 0, 1) (−1, 1)
( 1, −1) (−1, 0) ( 1, −1) ( 0, 0)
( 1, 1) ( 0, 0) (−1, −1) ( 0, 0)
⎞
⎟⎟⎟⎠.
6. Consider the bimatrix game:  (0, 0) (1, 2) (2, 0)
(0, 1) (2, 0) (0, 1) 
.
(a) Find the safety levels for the two players.
(b) Find all PSE’s.
(c) Find all SE’s given by mixed equalizing strategies.
7. Strategic Equilibria Survive Elimination of Strictly Dominated Strategies.
Suppose row 1 is strictly dominated (by a probability mixture of rows 2 through m,
i.e. a1j < m
i=2 xiaij for all j where xi ≥ 0 and m
2 xi = 1), and suppose (p∗, q∗) is a
strategic equilibrium. Show that p∗
1 = 0.

%=============================================================%
8. Consider the non-cooperative bimatrix game:
⎛
⎝
(3, 4) (2, 3) (3, 2)
(6, 1) (0, 2) (3, 3)
(4, 6) (3, 4) (4, 5)
⎞
⎠.
(a) Find the safety levels, and the maxmin strategies for both players.
(b) Find as many strategic equilibria as you can.

%=============================================================%
9. A PSE vector of strategies in a game in extensive form is said to be a subgame
perfect equilibrium if at every vertex of the game tree, the strategy vector restricted
to the subgame beginning at that vertex is a PSE. If a game has perfect information, a
subgame perfect equilibrium may be found by the method of backward induction. Figure
2.2 is an example of a game of perfect information that has a subgame perfect PSE and
another PSE that is not subgame perfect.
(a) Solve the game for an equilibrium using backward induction.
(b) Put the game into strategic form.
(c) Find another PSE of the strategic form game, relate it to the extensive form game and
show it is not subgame perfect.
III – 14
I
II
(0,1)
(1,0) (–10,–1)
A B
a b
Figure 2.2 An Extensive Form Game.

%=============================================================%
10. Suppose you are playing the centipede game once as Player I against a person
chosen at random from this class. At what point would you choose the option to go down
ending the game, assuming the opponent has not already ended the game?
Now answer the same question assuming you are Player II.
Your score on this question depends on what the other students in the class do. Do
not reveal your answer to this question to the other students in the class.
\end{document}