\documentclass[]{report}
\voffset=-1.5cm
\oddsidemargin=0.0cm
\textwidth = 480pt


\usepackage{amsmath}
\usepackage{graphicx}
\usepackage{amssymb}
\usepackage{framed}
\usepackage{multicol}
%\usepackage[paperwidth=21cm, paperheight=29.8cm]{geometry}
%\usepackage[angle=0,scale=1,color=black,hshift=-0.4cm,vshift=15cm]{background}
%\usepackage{multirow}
\usepackage{enumerate}

\usepackage{amsmath,amsfonts,amssymb}
\usepackage{color}
\usepackage{multirow}
\usepackage{eurosym}
\usepackage{framed}

%\input def.tex
%\input dsdef.tex
%\input rgb.tex

%\newcommand \la{\lambda}
%\newcommand \al{a}
%\newcommand \be{b}
\newcommand \x{\overline{x}}
\newcommand \y{\overline{y}}

\begin{document}

Cournot duopoly, also called Cournot competition, is a model of imperfect competition in which two firms with identical cost functions compete with homogeneous products in a static setting. It was developed by Antoine A. Cournot in his “Researches Into the Mathematical principles of the Theory of Wealth”, 1838.

%==========================================================================%
\subsection{Cournot duopoly}
% - http://policonomics.com/cournot-duopoly-model/
Cournot duopoly, also called Cournot competition, is a model of imperfect competition in which two firms with identical cost functions compete with homogeneous products in a static setting. It was developed by Antoine A. Cournot in his “Researches Into the Mathematical principles of the Theory of Wealth”, 1838. Cournot’s duopoly represented the creation of the study of oligopolies, more particularly duopolies, and expanded the analysis of market structures which, until then, had concentrated on the extremes: perfect competition and monopolies.

Cournot really invented the concept of game theory almost 100 years before John Nash, when he looked at the case of how businesses might behave in a duopoly. There are two firms operating in a limited market. Market production is: P(Q)=a-bQ, where Q=q1+q2 for two firms. Both companies will receive profits derived from a simultaneous decision made by both on how much to produce, and also based on their cost functions: TCi=C-qi.

\newpage
\section{Cournot-Nash Equilibrium in Duopoly}

%% - https://math.stackexchange.com/questions/327617/cournot-nash-equilibrium-in-duopoly
This is a homework question, but resources online are exceedingly complicated, so I was hoping there was a fast, efficient way of solving the following question:

There are 2 firms in an industry, which have the following total cost functions and inverse demand functions.
%%- Firm 1:Firm 2:C1=50Q_1P1=100–0.5(Q_1+Q_2)C2=24Q_2P2=100–0.5(Q_1+Q_2)
\[Firm 1:C1=50Q_1P1=100–0.5(Q_1+Q_2)Firm 2:C2=24Q_2P2=100–0.5(Q_1+Q_2)\]
\textbf{What is the Cournot-Nash equilibrium for this industry?}

I've tried to solve this dozens of times. My idea was to find the profit equation for both, take the derivative, set equal to zero, and then solve for $Q_1$ and $Q_2$.

Doing this, I get:
%%- Q_1=−5Q_2+500Q_2=−5Q_1+760
\[Q_1=−5Q_2+500 Q_2=−5Q_1+760\]



There is a standard way of solving for $Q_1$ and $Q_2$.

\textbf{Determine the profit functions.}
\textbf{Determine the best response function for the firms.}
Substitute $Q_1$ or $Q_2$ in the other profit function and solve.
All these steps are already mentioned, so you know what to do. Below you can search for your mistake.

\begin{itemize}
	\item The profit function for firm 1 equals \[Π1=P1Q_1−C1=Q_1⋅(100−0.5(Q_1+Q_2))−50Q_1Π1=P1Q_1−C1=Q_1⋅(100−0.5(Q_1+Q_2))−50Q_1\]
	\item	The profit function for firm 2 equals \[Π2=P2Q_2−C2=Q_2⋅(100−0.5(Q_1+Q_2))−24Q_2Π2=P2Q_2−C2=Q_2⋅(100−0.5(Q_1+Q_2))−24Q_2 \]
\end{itemize}

The best response function can be determined by deriving the profit function of firm 1 w.r.t. $Q_1$ and for firm 2 w.r.t. $Q_2$ and set them equal to zero

\[\frac{\partial \pi_1}{ \partial Q_1}=100−Q_1−0.5Q_2−50=50−Q_1−0.5Q_2=0\]
% \[\frac{\part \pi_1}{ \partial Q_1}=100−Q_1−0.5Q_2−50=50−Q_1−0.5Q_2=0\]
\[Q_1=50−0.5Q_2\]
\[ \frac{\partial \Pi_2 }{\partial Q_2}=100−Q_2−0.5Q_1−24=76−Q_2−0.5Q_1=0\]
Now we can make the substitution

\[76−Q_2−0.5⋅(50−0.5Q_2)=0\]
$$51−Q_2+0.25Q_2=0 \rightarrow 0.75Q_2=51$$
And thus we find $Q_2=68$ and can solve easily for $Q_1$
$Q_2=68$ and $Q_1=50−0.5⋅68=16$

%===============================================================%

\section{Cournot Nash Equilibrium Between Two Firms}
%%- https://math.stackexchange.com/questions/139564/cournot-nash-equilibrium-between-two-firms?rq=1
%----------------------------------------------%

Suppose we have two firms with specialized, but similar products. Suppose market demand for the two products is:
\[p1(Q_1,Q_2)=a−bQ_1−dQ_2\]

\[p2(Q_1,Q_2)=a−bQ_2−dQ_1\]

where $d\in(−b,b)$. Suppose that both firms have cost $c(q)=q$
What does $d$ mean intuitively? Is the Cournot Nash Equilibrium for this

\[Q_1=2ba−ad+dc′(Q_2)−c′(Q_1)2b1−d2\]
\[Q_2=2ba−ad+dc′(Q_1)−c′(Q_2)2b1−d2\]

%----------------------------------------------%
\subsection{What does d mean intuitively?}
To answer this question, think about the "vanilla" Cournot competition case, where products p1p1 and p2p2 are identical; they're perfect substitutes. In this case, increases in production from your competitor (i.e. $Q_2$) displaces your own production, so d=bd=b and

\[p1(Q_1,Q_2)=a−b(Q_1+Q_2)p1(Q_1,Q_2)=a−b(Q_1+Q_2).\]

On the other hand, if an increase in production of $Q_2$ increases demand for your own product $Q_1$, then these products are compliments. Be careful about stating they are perfect compliments, because without looking at consumer indifference curves, we can't determine this.

In this case, dd is negative, and is bounded by −b−b.

In short, dd is a measure of the degree to which these two goods are complements or substitutes. Another approach would be to take the derivative of demand with respect to production of the other good, like this:

\[∂p1∂Q_2=−d∂p1∂Q_2=−d.\]

If d>0d>0, $∂p1∂Q_2<0$ and $Q_2$ is a complement to $Q_1$. Likewise, if $d<0$, $∂p1∂Q_2>0$ and $Q_2$ is a substitute for $Q_1$. Because of the symmetry of the problem, both will either be complements or substitutes. However, in the real world this is not always the case.

%----------------------------------------------%
\subsection{What is the Cournot-Nash equilibrium?}
The Cournot-Nash equilibrium is the output $\{Q_1, Q_2\}$ from which neither firm can profitably deviate. To answer this, you need to find the best response function for each firm by solving for the optimal output, given the production of the other firm. This is accomplished by equating Marginal Revenue = Marginal Cost. Note that the marginal cost of production is zero; i.e.$ c′(Q_1)=c′(Q_2)=0c′(Q_1)=c′(Q_2)=0$.

\[BR1(Q_2)=a−dQ_22bBR1(Q_2)=a−dQ_22b\] and 
\[BR2(Q_1)=a−dQ_12bBR2(Q_1)=a−dQ_12b.\]

The Cournot-Nash equilibrium is located where these two Best Response functions intersect. Solving the system of two equations and two unknowns, I get:

\[q^{\ast}_{1}=q^{\ast}_{2}=a(12−d4b)b−d24b\]
\[Q^{\ast}_{1}=Q^{\ast}_{2}=a(12−d4b)b−d24b.\]

%----------------------------------------------%
\newpage
\subsection{2012}
Question 6 

Consider the (symmetric) Cournot duopoly game: Firm i, i = 1, 2 produces
xi
items at a cost of
C(xi) = x
2
i
1000 + 3xi + 20
. The items sell at a price of
p(x1, x2) = 5 −
x1 + x2
500
each.
\begin{itemize}
\item[(a)] Find the equilibrium of this game, and prove that it is a Nash equilibrium.
8 \%
\item[(b)] Investigate this game if a collusive strategy is used. Contrast its solution
with that of part (a). 8 \%
\item[(c)] If the game is to be played repeatedly, does it ever pay to defect from
the collusive strategy? In particular, consider the stern strategy: a firm
produces the collusive number of items until the other firm defects,
after which it reverts to producing the Cournot number of items. Using
the discount factor $\omega$ per period, when is this stern strategy a Nash
equilibrium ? 9 \%
\end{itemize}
\subsection{2013}
Consider the asymmetric duopoly game: Firm i, i = 1, 2 produces xi
items
at a cost of
C(xi) = 1
i
xi + 20.
The items sell at a price of
p(x1, x2) = 5 −
x1 + x2
500
each.

\begin{itemize}
\item[(a)] Find the equilibrium of the game if it is played as a Cournot game,
and prove that it is a Nash equilibrium. 8
\item[(b)] Find the equilibrium if it played as a Stackelberg game with Firm 1 as
leader. 8
\item[(c)] Contrast and comment on the two solutions. 4
\item[(d)] Firm 1 receives an injection of capital and initiates a “leader strategy”.
However Firm 2 persists with its Cournot strategy. What happens and
what should each firm do in future interactions? 5
\end{itemize}

\subsection{2015}
3 Consider the following asymmetric duopoly game with isoelastic demand:
Two firms sell equivalent items at a price of
p(x1, x2) = 600
x1 + x2
per item, where Firm i,(i = 1, 2), produces xi
items at a cost of
C(xi) = mixi
.
The marginal costs are given by mi = i + 1 respectively.

(a) If the game is played as a Cournot game, show that the best response
xi = B(xj) of Firm i to Firm j is given by
B(xj) = r
600xj
mi
− xj
and hence find the Nash equilibrium. 8
(b) Find the equilibrium if it played as a Stackelberg game with Firm 1 as
leader. 8
(c) Contrast and comment on the two solutions. 4
(d) Firm 1 decides it will stick with the “leader strategy” in further production
cycles. However Firm 2 decides to ignore this and persist with
its Cournot strategy of part(a). What happens and what should each
firm do in future interactions?
\end{document} 
