\documentclass[]{report}
\voffset=-1.5cm
\oddsidemargin=0.0cm
\textwidth = 480pt


\usepackage{amsmath}
\usepackage{graphicx}
\usepackage{amssymb}
\usepackage{framed}
\usepackage{multicol}
%\usepackage[paperwidth=21cm, paperheight=29.8cm]{geometry}
%\usepackage[angle=0,scale=1,color=black,hshift=-0.4cm,vshift=15cm]{background}
%\usepackage{multirow}
\usepackage{enumerate}

\usepackage{amsmath,amsfonts,amssymb}
\usepackage{color}
\usepackage{multirow}
\usepackage{eurosym}
\usepackage{framed}

%\input def.tex
%\input dsdef.tex
%\input rgb.tex

%\newcommand \la{\lambda}
%\newcommand \al{a}
%\newcommand \be{b}
\newcommand \x{\overline{x}}
\newcommand \y{\overline{y}}

\begin{document}

Cournot duopoly, also called Cournot competition, is a model of imperfect competition in which two firms with identical cost functions compete with homogeneous products in a static setting. It was developed by Antoine A. Cournot in his “Researches Into the Mathematical principles of the Theory of Wealth”, 1838.

%==========================================================================%
\subsection{Cournot duopoly}
% - http://policonomics.com/cournot-duopoly-model/
Cournot duopoly, also called Cournot competition, is a model of imperfect competition in which two firms with identical cost functions compete with homogeneous products in a static setting. It was developed by Antoine A. Cournot in his “Researches Into the Mathematical principles of the Theory of Wealth”, 1838. Cournot’s duopoly represented the creation of the study of oligopolies, more particularly duopolies, and expanded the analysis of market structures which, until then, had concentrated on the extremes: perfect competition and monopolies.

Cournot really invented the concept of game theory almost 100 years before John Nash, when he looked at the case of how businesses might behave in a duopoly. There are two firms operating in a limited market. Market production is: P(Q)=a-bQ, where Q=q1+q2 for two firms. Both companies will receive profits derived from a simultaneous decision made by both on how much to produce, and also based on their cost functions: TCi=C-qi.

\end{document} 
