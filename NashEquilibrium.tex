the Nash equilibrium is a solution concept of a non-cooperative game involving two or more players in which each player is assumed to know the equilibrium strategies of the other players, and no player has anything to gain by changing only his or her own strategy.[1] If each player has chosen a strategy and no player can benefit by changing strategies while the other players keep theirs unchanged, then the current set of strategy choices and the corresponding payoffs constitutes a Nash equilibrium. 

The Nash equilibrium is one of the foundational concepts in game theory.


%===========================================================%

\subsection{Informal definition}
Informally, a strategy profile is a Nash equilibrium if no player can do better by unilaterally changing his or her strategy. To see what this means, imagine that each player is told the strategies of the others. Suppose then that each player asks themselves: "Knowing the strategies of the other players, and treating the strategies of the other players as set in stone, can I benefit by changing my strategy?"

If any player could answer "Yes", then that set of strategies is not a Nash equilibrium. But if every player prefers not to switch (or is indifferent between switching and not) then the strategy profile is a Nash equilibrium. Thus, each strategy in a Nash equilibrium is a best response to all other strategies in that equilibrium.

The Nash equilibrium may sometimes appear non-rational in a third-person perspective. This is because it may happen that a Nash equilibrium is not Pareto optimal.

The Nash equilibrium may also have non-rational consequences in sequential games because players may "threaten" each other with non-rational moves. For such games the subgame perfect Nash equilibrium may be more meaningful as a tool of analysis.
