\section{Cournot-Nash Equilibrium in Duopoly}

%% - https://math.stackexchange.com/questions/327617/cournot-nash-equilibrium-in-duopoly
This is a homework question, but resources online are exceedingly complicated, so I was hoping there was a fast, efficient way of solving the following question:

There are 2 firms in an industry, which have the following total cost functions and inverse demand functions.
%%- Firm 1:Firm 2:C1=50Q_1P1=100–0.5(Q_1+Q_2)C2=24Q_2P2=100–0.5(Q_1+Q_2)
\[Firm 1:C1=50Q_1P1=100–0.5(Q_1+Q_2)Firm 2:C2=24Q_2P2=100–0.5(Q_1+Q_2)\]
\textbf{What is the Cournot-Nash equilibrium for this industry?}

I've tried to solve this dozens of times. My idea was to find the profit equation for both, take the derivative, set equal to zero, and then solve for $Q_1$ and Q_2Q_2.

Doing this, I get:
%%- Q_1=−5Q_2+500Q_2=−5Q_1+760
\[Q_1=−5Q_2+500Q_2=−5Q_1+760\]



There is a standard way of solving for $Q_1$ and $Q_2$.

\textbf{Determine the profit functions.}
\textbf{Determine the best response function for the firms.}
Substitute $Q_1$ or $Q_2$ in the other profit function and solve.
All these steps are already mentioned, so you know what to do. Below you can search for your mistake.

\begin{itemize}
	\item The profit function for firm 1 equals \[Π1=P1Q_1−C1=Q_1⋅(100−0.5(Q_1+Q_2))−50Q_1Π1=P1Q_1−C1=Q_1⋅(100−0.5(Q_1+Q_2))−50Q_1\]
	\item	The profit function for firm 2 equals \[Π2=P2Q_2−C2=Q_2⋅(100−0.5(Q_1+Q_2))−24Q_2Π2=P2Q_2−C2=Q_2⋅(100−0.5(Q_1+Q_2))−24Q_2 \]
\end{itemize}

The best response function can be determined by deriving the profit function of firm 1 w.r.t. Q_1Q_1 and for firm 2 w.r.t. Q_2Q_2 and set them equal to zero

\[\frac{\part \pi_1}{ \partial Q_1}=100−Q_1−0.5Q_2−50=50−Q_1−0.5Q_2=0\]
% \[\frac{\part \pi_1}{ \partial Q_1}=100−Q_1−0.5Q_2−50=50−Q_1−0.5Q_2=0\]
⟹Q_1=50−0.5Q_2
⟹Q_1=50−0.5Q_2
∂Π2∂Q_2=100−Q_2−0.5Q_1−24=76−Q_2−0.5Q_1=0
∂Π2∂Q_2=100−Q_2−0.5Q_1−24=76−Q_2−0.5Q_1=0
Now we can make the substitution

76−Q_2−0.5⋅(50−0.5Q_2)=0
76−Q_2−0.5⋅(50−0.5Q_2)=0
⟹51−Q_2+0.25Q_2=0⟹0.75Q_2=51
⟹51−Q_2+0.25Q_2=0⟹0.75Q_2=51
And thus we find Q_2=68Q_2=68 and can solve easily for Q_1Q_1
Q_2=68 and Q_1=50−0.5⋅68=16

%===============================================================%

\section{Cournot Nash Equilibrium Between Two Firms}
%%- https://math.stackexchange.com/questions/139564/cournot-nash-equilibrium-between-two-firms?rq=1
%----------------------------------------------%

Suppose we have two firms with specialized, but similar products. Suppose market demand for the two products is:
p1(Q_1,Q_2)=a−bQ_1−dQ_2
p1(Q_1,Q_2)=a−bQ_1−dQ_2
p2(Q_1,Q_2)=a−bQ_2−dQ_1
p2(Q_1,Q_2)=a−bQ_2−dQ_1
where d∈(−b,b)d∈(−b,b). Suppose that both firms have cost c(q)=qc(q)=q
What does dd mean intuitively? Is the Cournot Nash Equilibrium for this

Q_1=2ba−ad+dc′(Q_2)−c′(Q_1)2b1−d2
Q_1=2ba−ad+dc′(Q_2)−c′(Q_1)2b1−d2
Q_2=2ba−ad+dc′(Q_1)−c′(Q_2)2b1−d2

%----------------------------------------------%
What does d mean intuitively?
To answer this question, think about the "vanilla" Cournot competition case, where products p1p1 and p2p2 are identical; they're perfect substitutes. In this case, increases in production from your competitor (i.e. $Q_2$) displaces your own production, so d=bd=b and

\[p1(Q_1,Q_2)=a−b(Q_1+Q_2)p1(Q_1,Q_2)=a−b(Q_1+Q_2).\]

On the other hand, if an increase in production of $Q_2$ increases demand for your own product Q_1Q_1, then these products are compliments. Be careful about stating they are perfect compliments, because without looking at consumer indifference curves, we can't determine this.

In this case, dd is negative, and is bounded by −b−b.

In short, dd is a measure of the degree to which these two goods are complements or substitutes. Another approach would be to take the derivative of demand with respect to production of the other good, like this:

∂p1∂Q_2=−d∂p1∂Q_2=−d.

If d>0d>0, ∂p1∂Q_2<0∂p1∂Q_2<0 and $Q_2$ is a complement to $Q_1$. Likewise, if d<0d<0, ∂p1∂Q_2>0∂p1∂Q_2>0 and $Q_2$ is a substitute for $Q_1$. Because of the symmetry of the problem, both will either be complements or substitutes. However, in the real world this is not always the case.

%----------------------------------------------%
What is the Cournot-Nash equilibrium?
The Cournot-Nash equilibrium is the output $\{Q_1, Q_2\}$ from which neither firm can profitably deviate. To answer this, you need to find the best response function for each firm by solving for the optimal output, given the production of the other firm. This is accomplished by equating Marginal Revenue = Marginal Cost. Note that the marginal cost of production is zero; i.e. c′(Q_1)=c′(Q_2)=0c′(Q_1)=c′(Q_2)=0.

\[BR1(Q_2)=a−dQ_22bBR1(Q_2)=a−dQ_22b and BR2(Q_1)=a−dQ_12bBR2(Q_1)=a−dQ_12b.\]

The Cournot-Nash equilibrium is located where these two Best Response functions intersect. Solving the system of two equations and two unknowns, I get:

q∗1=q∗2=a(12−d4b)b−d24bQ_1∗=Q_2∗=a(12−d4b)b−d24b.

%----------------------------------------------%
