Operations Research 2
Additional Sheet 4
1. Consider the following game. Player 1 moves first and can take action A or B.
Player 2 observes the action of Player 1 and independently of the action of Player 1 can
take action A or B. If both play A, then Player 1 obtains a payoff of 4 and Player 2
obtains a payoff of 5. If Player 1 plays A and Player 2 plays B, then Player 1 obtains a
payoff of 2 and Player 2 obtains a payoff of 7. If Player 1 plays B and Player 2 plays A,
then Player 1 obtains a payoff of 6 and Player 2 obtains a payoff of 3. If both play B,
then Player 1 obtains a payoff of 0 and Player 2 obtains a payoff of 1.
i) Draw the tree depicting the extensive form of the game.
ii) Solve the game using recursion.
iii) Give the matrix form of the game.
2. Consider the following matrix game (compare with the game from Question 1)
A B
A (4,5) (2,7)
B (6,3) (0,1)
i) Find the minimax solution of this game.
ii) Derive all the Nash equilibria and values of this game.
3. i) By removing all strategies which are dominated by pure or mixed strategies,
derive the reduced version of the following matrix game.
D E F G
A (5,2) (2,4) (3,3) (4,4)
B (3,3) (5,2) (3,5) (2,3)
C (4,4) (4,6) (4,3) (5,4)
ii) Derive the minimax solution of this game.
iii) Derive the Nash equilibria and values of this game.
4. i) Consider the following game in which 2 players must choose the amount of
energy they are willing to put into a project. Suppose they must choose these efforts
simultaneously. Player 1 chooses x from [0,1] and Player 2 chooses y from the same
interval. The payoffs obtained by the two players are given by
[R1(x, y), R2(x, y)] = [3xy − x
2
, 3xy − 2y
2
].
Derive the Nash equilibria and values of this game.
ii) Consider the analogous game in which Player 1 decides on a level of effort that is
communicated to Player 2, who then decides what level of effort to use. Derive the Nash
equilibrium for this game.
1

%===============================================================%
Operations Research 2
Additional Sheet 4
1. Consider the following game. Player 1 moves first and can take action A or B.
Player 2 observes the action of Player 1 and independently of the action of Player 1 can
take action A or B. If both play A, then Player 1 obtains a payoff of 4 and Player 2
obtains a payoff of 5. If Player 1 plays A and Player 2 plays B, then Player 1 obtains a
payoff of 2 and Player 2 obtains a payoff of 7. If Player 1 plays B and Player 2 plays A,
then Player 1 obtains a payoff of 6 and Player 2 obtains a payoff of 3. If both play B,
then Player 1 obtains a payoff of 0 and Player 2 obtains a payoff of 1.
i) Draw the tree depicting the extensive form of the game.
ii) Solve the game using recursion.
iii) Give the matrix form of the game.
2. Consider the following matrix game (compare with the game from Question 1)
A B
A (4,5) (2,7)
B (6,3) (0,1)
i) Find the minimax solution of this game.
ii) Derive all the Nash equilibria and values of this game.
3. i) By removing all strategies which are dominated by pure or mixed strategies,
derive the reduced version of the following matrix game.
D E F G
A (5,2) (2,4) (3,3) (4,4)
B (3,3) (5,2) (3,5) (2,3)
C (4,4) (4,6) (4,3) (5,4)
ii) Derive the minimax solution of this game.
iii) Derive the Nash equilibria and values of this game.
4. i) Consider the following game in which 2 players must choose the amount of
energy they are willing to put into a project. Suppose they must choose these efforts
simultaneously. Player 1 chooses x from [0,1] and Player 2 chooses y from the same
interval. The payoffs obtained by the two players are given by
[R1(x, y), R2(x, y)] = [3xy − x
2
, 3xy − 2y
2
].
Derive the Nash equilibria and values of this game.
ii) Consider the analogous game in which Player 1 decides on a level of effort that is
communicated to Player 2, who then decides what level of effort to use. Derive the Nash
equilibrium for this game.
1
%=======================================================================%



Operations Research 2
Additional Sheet 6
1. Consider the co-ordination game
H D
H (4,6) (0,0)
D (0,0) (5,3)
a) i) Derive the conditions which a correlated equilibrium must satisfy.
ii) Find the correlated equilibrium that maximises the expected reward of Player 1.
iii) Find the correlated equilibrium that maximises the sum of the expected payoffs of
the players.
iv) Find the correlated equilibrium that maximises the minimum expected reward of
the players.
2. Consider the chicken game
A B
A (10,10) (4,12)
B (12,4) (0,0)
a) i) Derive the conditions which a correlated equilibrium must satisfy.
ii) Find the correlated equilibrium that maximises the expected reward of Player 1.
iii) Find the correlated equilibrium that maximises the minimum expected reward of
the players.
b) Now consider the infinitely repeated version of this game with a discount rate of ω.
Let T be the strategy, play A until the other player deviates from A and then always play
B. Find the values of ω for which (T, T) is a Nash equilibrium in this repeated game.
3. a) Consider the following symmetric two-player Cournot game. Firm i produces xi
units and has costs 500 + xi
. The price when the total supply is x satisfies p = 6 −
x
1000 .
i) Find the equilibrium levels of production, profits and price.
ii) Find the level of production that maximises the sum of profits.
iii) Derive the collusive solution and the discount rates for which such collusion is stable
(assume that after any defection by the other firm, a firm will always play the Cournot
equilibrium strategy).
1


%================================================================%
Sheet 4
1. ii) Equilibrium: I - B, II - B after A, A after B.
2. i) Player I - A, Player II - B.
ii) Equilibria (A, B),(B, A),(0.5A + 0.5B, 0.5A + 0.5B)
Values (2, 7),(6, 3),(3, 4).
3. i) D, A and G dominated.
ii) I - C, II - 1/3E+2/3F
iii) I - 0.5B+0.5C, II - 1/3E+2/3F. Value (4,4)
4. i) Equilibria - (0,0), (1,3/4). Values (0,0), (5/4, 9/8)
ii) 2nd equilibrium above.
Sheet 5
1. a) Equilibrium (7000/3, 4000/3), Value (31000/9, 7000/9)
b) Equilibrium (3500,750), Value (4125,-437.5)
2. i) Equilibria - (A,B), (B,A) (0.5A+0.5B,0.5A+0.5B), Values (6,2), (2,6), (3.5,3.5)
ii) No, iii) Risk of any strategy=1/2, No risk dominant.
iii) pn+1 =
pn(6−5pn)
5−2pn−2p
2
n
pn prop. of A players in gen. n.
iv) 0,0.5,1. Central point is an attractor.
3. Mixed equilibrium (1/3R+1/3S+1/3P) is an ESS.
4. i) Nash equilibrium (5/9C+4/9F,0.5L+0.5D).
ii) None (not covered).
Sheet 6
1. ii) (p1, p2, p3, p4) = (0, 0, 0, 1), i.e. (D, D).
iii) (p1, p2, p3, p4) = (1, 0, 0, 0), i.e. (H, H).
iv) (p1, p2, p3, p4) = (0.5, 0, 0, 0.5).
2. ii) (p1, p2, p3, p4) = (0, 0, 1, 0) i.e. (B, A).
iii) (p1, p2, p3, p4) = (0.5, 0.25, 0.25, 0).
b) ω ≥ 0.25.
3. i) p = 8/3, (x1, x2) = (5000/3, 5000/3).
ii) (x1, x2) = (1250, 1250).
iii) ω ≥ 0.529412.
1
