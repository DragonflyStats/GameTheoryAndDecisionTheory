%%- https://brilliant.org/wiki/minimax/

In game theory, minimax is a decision rule used to minimize the worst-case potential loss; in other words, a player considers all of the best opponent responses to his strategies, and selects the strategy such that the opponent's best strategy gives a payoff as large as possible.

The name "minimax" comes from minimizing the loss involved when the opponent selects the strategy that gives maximum loss, and is useful in analyzing the first player's decisions both when the players move sequentially and when the players move simultaneously. In the latter case, minimax may give a Nash equilibrium of the game if some additional conditions hold.

Minimax is also useful in combinatorial games, in which every position is assigned a payoff. The simplest example is assigning a "1" to a winning position and "-1" to a losing one, but as this is difficult to calculate for all but the simplest games, intermediate evaluations (specifically chosen for the game in question) are generally necessary. In this context, the goal of the first player is to maximize the evaluation of the position, and the goal of the second player is to minimize the evaluation of the position, so the minimax rule applies. This, in essence, is how computers approach games like chess and Go, though various computational improvements are possible to the "naive" implementation of minimax.
Contents

Formal definition
Minimax theorem
In combinatorial games
See Also
Formal definition
Suppose player  chooses strategy , and the remaining players choose the strategy profile . If  denotes the utility function for player on strategy profile , the minimax of a game is defined as


Intuitively speaking, the minimax (for player ) is one of two equivalent formulations:

The minimax is the smallest value that the other players can force player  to receive, without knowing player 's strategy
The minimax is the largest value player  can guarantee when he is told the strategies of all other players.
Similarly, the maximin is defined as


which can be intuitively understood as either of:

The maximin is the largest value player  can guarantee when he does not know the strategies of any other player
The maximin is the smallest value the other players can force player  to receive, without knowing player 's strategy
For example, consider the following payoff matrix (the rows represent the first player's choices and the columns represent the second player's choices):

1	2	3
1	-1	-2	-1
2	2	2	1
3	-1	-1	0
where both players have a choice between three strategies. In such a payoff matrix, from the first player's perspective:

The maximin is the largest of the smallest values in each row
The minimax is the smallest of the largest values in each column
so the maximin is the largest of -2, 1, and -1 (i.e. 1), and the minimax is the smaller of 2, 2, and 1 (i.e. 1).

It is extremely important to note that

i.e. the maximin is always at most the minimax.
This can be intuitively understood by noting that in minimax, player  effectively gets to choose his strategy after learning everyone else's, which can only increase his payoff.

In the above example, the maximin and the minimax are in fact equal. In such a case (which does not always happen!), the minimax strategy for both players gives a Nash equilibrium of the game.

This is especially important in zero-sum games, in which the minimax always gives a Nash equilibrium of the game, as the minimax and maximin are necessarily equal.
Minimax theorem
The minimax theorem establishes conditions on when the minimax and maximin of a function are equal. More precisely, the minimax theorem gives conditions on when


Formally,

Minimax theorem:

Let  be two compact convex sets, and  be a continuous function on pairs . If  is convex-concave, i.e.

 is convex for all fixed 
 is concave for all fixed 
then

The application of the minimax theorem to zero-sum games is especially important, as it becomes equivalent to

For a zero-sum game with finitely many strategies, there exists a payoff  and a mixed strategy for each player such that

Player 1 can achieve a payoff of at most , even given player 2's strategy
Player 2 can achieve a payoff of at most , even given player 1's strategy
which is equivalent to establishing a Nash equilibrium.
It is important to note that the minimax strategy may be mixed; in general,

It is not necessarily the case that the pure minimax strategy for each player leads to a Nash equilibrium.
For example, consider the payoff matrix

1	2	3
1	3	-2	2
2	-1	0	4
3	-4	-3	1
The minimax choice for the first player is strategy 2, and the minimax choice for the second player is also strategy 2. But both players choosing strategy 2 does not lead to a Nash equilibrium; either player would choose to change their strategy given knowledge of the other's. In fact, the mixed minimax strategies of:

Player 1 chooses strategy 1 with probability  and strategy 2 with probability 
Player 2 chooses strategy 1 with probability  and strategy 2 with probability 
is stable and represents a Nash equilibrium.
In combinatorial games