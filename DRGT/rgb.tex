% 1. Definition of characteristic points
\figinit{0.6cm}
\def\xB{2} % Abscissa of B
\def\R{3} % Radius of the 3 circles, r < R < r*sqrt(3),
% r radius of the circumscribed circle (r=2*\xB/sqrt(3)=2.3094)
\figpt 1:A(-\xB, 0) \figpt 2:B(\xB, 0) \figptrot 3:C=2/1,-60/
\figptsintercirc 4[1,\R;2,\R] \figptsintercirc 14[3,\R;1,\R]
\figptsintercirc 24[2,\R;3,\R]
% 2. Creation of the graphical file
\figdrawbegin{}
\figset(fillmode=yes)
\figset(color=\Redrgb) \figdrawcirc 1(\R)
\figset(color=\Greenrgb)\figdrawcirc 2(\R)
\figset(color=\Bluergb) \figdrawcirc 3(\R)
\figset(color=\Yellowrgb) \figdrawarccircP1;\R[4,5] \figdrawarccircP2;\R[5,4]
\figset(color=\Magentargb)\figdrawarccircP1;\R[15,14]\figdrawarccircP3;\R[14,15]
\figset(color=\Cyanrgb) \figdrawarccircP3;\R[25,24]\figdrawarccircP2;\R[24,25]
\figset(color=\Whitergb)
\figdrawarccircP2;\R[24,4]\figdrawarccircP1;\R[4,14]\figdrawarccircP3;\R[14,24]
\figdrawline[4,14,24]
\figdrawend
% 3. Inserting the figure
\figvisu{\figBoxA}{\bf RGB colors}{}
\centerline{\box\figBoxA}