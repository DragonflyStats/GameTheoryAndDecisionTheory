
\documentclass[]{report}
\voffset=-1.5cm
\oddsidemargin=0.0cm
\textwidth = 480pt


\usepackage{amsmath}
\usepackage{graphicx}
\usepackage{amssymb}
\usepackage{framed}
\usepackage{multicol}
%\usepackage[paperwidth=21cm, paperheight=29.8cm]{geometry}
%\usepackage[angle=0,scale=1,color=black,hshift=-0.4cm,vshift=15cm]{background}
%\usepackage{multirow}
\usepackage{enumerate}

\usepackage{amsmath,amsfonts,amssymb}
\usepackage{color}
\usepackage{multirow}
\usepackage{eurosym}
\usepackage{framed}

%\input def.tex
%\input dsdef.tex
%\input rgb.tex

%\newcommand \la{\lambda}
%\newcommand \al{a}
%\newcommand \be{b}
\newcommand \x{\overline{x}}
\newcommand \y{\overline{y}}

\begin{document}
%======================================================================%
\subsection{Example 3.8.1}
Consider the co-ordination game given above.
Since there are only 2 possible actions, we need only to consider
the evolution of the proportion of individuals using A.
Let pn be the proportion of individuals using A in generation n (the
remaining individuals use B). The proportion of individuals using A
at the start of the process is p0.
27 / 46

%======================================================================%
\subsection{Example 3.8.1}
The average reward of A players in generation n is given by
RA,n = R(A, pnA + (1 − pn)B) = 5pn.
The average reward of B players in generation n is given by
RB,n = R(B, pnA + (1 − pn)B) = 1 − pn.
28 / 46
%======================================================================%
\subsection{Example 3.8.1}
The average reward of the population as a whole is given by
Rn = pnRA,n + (1 − pn)RB,n = 5p
2
n + (1 − pn)
2 = 1 − 2pn + 6p
2
n
.
Hence, the equation governing the replicator dynamics is
pn+1 =
pnRA,n
Rn
=
5p
2
n
1 − 2pn + 6p
2
n
.
29 / 46

%======================================================================%
\subsection{Example 3.8.1}
A fixed point of these dynamics satisfies pn+1 = pn = p. We have
p =
5p
2
1 − 2p + 6p
2 ⇒ p(1 − 2p + 6p
2
) = 5p
2
.
p = 0 is clearly a root of this equation. Otherwise, dividing by p,
we obtain
(1 − 2p + 6p
2
) = 5p ⇒ 1 − 7p + 6p
2 = 0 ⇒ p = 1 or p =
1
6
.
Hence, there are 3 fixed points of the replicator dynamics 0, 1
6
and
1.
30 / 46

%======================================================================%
\subsection{Example 3.8.1}
Note that 0 and 1 will always be fixed points of the replicator
dynamics in 2×2 games.
This is due to the fact that the replicator dynamics assume that
there is no mutation. Hence, if one of the actions is not played,
then there is no way of introducing it into the population.
We now check which of these fixed points are attractors.
31 / 46

%======================================================================%
\subsection{Example 3.8.1}
To check whether p = 0 is an attractor (this corresponds to the
strategy B), we assume that pn is close to 0 and see whether pn+1
is closer to 0, i.e. let pn = δ where δ is small (δ > 0).
We have
pn+1 − 0 = pn+1 =
5δ
2
1 − 2δ + 6δ
2
= δ
5δ
1 − 2δ + 6δ
2
.
For small δ, pn+1 will be of the order 5δ times pn, i.e. less than pn.
Hence, p = 0 is an attractor.
32 / 46
Example 3.8.1
To check whether p = 1 is an attractor (this corresponds to the
strategy B), we assume that pn is close to 1, i.e. pn = 1 − δ,
where δ is small and positive.
We then look at the distance between 1 and pn+1 (normally as a
multiple of δ). Setting pn = 1 − δ.
pn+1 =
5(1 − δ)
2
1 − 2(1 − δ) + 6(1 − δ)
2
33 / 46

%======================================================================%
\subsection{Example 3.8.1}
Hence,
1 − pn+1 = 1 −
5(1 − δ)
2
1 − 2(1 − δ) + 6(1 − δ)
2
= δ
% 
δ
5 − 10δ + 6δ
2
% 
.
It follows that when pn is δ away from 1, then pn+1 will be of order
δ
5
times further away, i.e. closer to 1.
Hence, p = 1 is an attractor.
34 / 46
%======================================================================%
\subsection{Example 3.8.1}
Finally, we check whether p =
1
6
is an attractor. Let pn+1 =
1
6 + δ
and look at the distance between pn+1 and 1
6
as a multiple of δ.
We have
pn+1 −
1
6
=
5(1/6 + δ)
2
1 − 2(1/6 + δ) + 6(1/6 + δ)
2
−
1
6
This gives
pn+1 −
1
6
= δ
5/3 + 4δ
5/6 + 6δ
2
35 / 46
%======================================================================%
\subsection{Example 3.8.1}
It follows that the difference between pn+1 and 1
6
is of order 2δ,
i.e. twice as far away from 1
6
.
It follows that p =
1
6
is not an attractor.
Hence, the only ESSs in this game are p = 0 (corresponding to B)
and p = 1 (corresponding to A).
36 / 46

%======================================================================%
\subsection{Relation Between the Evolution of the Population and the
Risk Factor}
It should be noted that if p0 >
1
6
, then the population will evolve
to everyone using A.
If p0 <
1
6
, then the population will evolve to everyone using B.
It should be noted that 1
6
is the risk factor associated with the ESS
strategy A. When the probability of an opponent playing A is
greater than 1
6
, then a player should play A.
37 / 46

%======================================================================%
\subsection{3.9 Asymmetric Evolutionary Games}
In general, a Hawk-Dove game will not be symmetric, e.g.
1. One individual may be bigger (stronger) than the
other and hence more likely to win a fight. For
simplicity assume that one player is always bigger and
wins a fight with probability p, where p >
1
2
.
2. Two individuals will not find a resource at the same
time, hence one can be treated as ”an owner” and
the other as ”an intruder”.
In such cases the strategy of an individual should take into account
the role of a player.
38 / 46

%======================================================================%
\subsection{Asymmetric Evolutionary Games}
Suppose that in the Hawk-Dove game one player is ”an owner”
and the other player is ”an intruder”.
Strictly speaking, in such a game since each player can play either
role, the strategy of an individual should be given by a rule
defining how an individual behaves when she is an intruder and a
rule defining how she behaves when she is an owner.
However, assuming that it is clear which role is being played by
each individual, in order to define ESSs for such a game, we only
need to consider the game in which Player 1 is the owner and
Player 2 is the intruder.

39 / 46

%======================================================================%
Asymmetric Evolutionary Games
The payoff matrix is given by
H D
H (0.5[v − c], 0.5[v − c]) (v, 0)
D (0, v) 0.5(v, v)
40 / 46
%======================================================================%
Asymmetric Evolutionary Games
In this case an ESS profile (π
∗
1
, π∗
2
) satisfies the condition that if
the vast majority of the population play π
∗
1 when in Role 1 and π
∗
2
when in Role 2, then an individual who plays a different strategy in
either role will be selected against.
It follows that (π
∗
1
, π∗
2
) is an ESS profile if
R1(π
∗
1
, π∗
2
)>R1(π1, π∗
2
), ∀π1 6= π
∗
1
R2(π
∗
1
, π∗
2
)>R2(π
∗
1
, π2) ∀π2 6= π
∗
2
41 / 46

%======================================================================%
Asymmetric Evolutionary Games
It follows from this definition that any strong Nash equilibrium of
an asymmetric game is an ESS profile.
Note that it is not clear at this stage that an ESS profile has to
satisfy this condition, e.g. no profile with a mixed strategy can
satisfy this condition. It will be argued that no such profile can be
an ESS in an asymmetric game.
42 / 46

%======================================================================%
Example 3.9.1
Consider the asymmetric Hawk-Dove game in which Player 1 is the
owner and Player 2 is an intruder.
Since (H, D) and (D, H) are strong Nash equilibria of this game,
they are ESS profiles.
The first ESS profile is the intuitively appealing ”be aggressive
when you are the owner, but avoid fights when you are an
intruder”.
The second ESS profile is the less intuitive ”avoid fights when you
are the owner, but be aggressive when you are an intruder”.
43 / 46
%======================================================================%

Example 3.9.1
The only other possible ESS profile corresponds to the mixed Nash
equilibrium in which both players play H with probability p
∗ =
v
c
.
Suppose that occupiers play Hawk with probability p and intruders
play H with probability q.
The expected payoff of an occupier who plays H is
R1(H, qH + (1 − q)D) = 0.5q(v − c) + (1 − q)v.
44 / 46
\subsection{Example 3.9.1}
The expected payoff of an occupier who plays D is
R1(D, qH + (1 − q)D) = 0.5(1 − q)v.
Comparing these two expected payoffs, occupiers playing Hawk
obtain a higher payoff than occupiers playing Dove if q <
v
c
.
i.e. if the level of aggression in intruders is low, then owners should
be aggressive.
It follows from the symmetry of the payoffs that intruders playing
Hawk obtain a higher payoff than intruders playing Dove if p <
v
c
(i.e. the level of aggression in owners is low).
%% 45 / 46
%% Example 3.9.1
Suppose occupiers play H with probability v
c + δ and intruders play
H with probability v
% c − , where δ,  > 0.
It follows that selection will favour occupiers playing H and
intruders playing D.
It can be seen that evolution will take the population away from
the mixed Nash equilibrium. Hence, the mixed Nash equilibrium is
not an ESS.
46 / 46
\end{document}