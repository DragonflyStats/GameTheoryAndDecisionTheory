%This is a latex file

\documentclass[12pt]{article}
\usepackage{graphicx}
\usepackage{amsmath,amsfonts,amssymb}
\usepackage{color}
\usepackage{multirow}
\usepackage{eurosym}
\setlength{\oddsidemargin}{0.0mm} \setlength{\textwidth}{160.0mm}
\setlength{\topmargin}{-10.0mm} \setlength{\textheight}{250mm}
\setlength{\parindent}{0mm} \setlength{\parskip}{2mm}
\pagestyle{empty}
%\input def.tex
%\input dsdef.tex
%\input rgb.tex

%\newcommand \la{\lambda}
%\newcommand \al{a}
%\newcommand \be{b}
\newcommand \x{\overline{x}}
\newcommand \y{\overline{y}}

\begin{document}
\begin{center}
\textbf{Cournot Duopoly with Linear Demand and Linear Costs %(  au Spaniel)
}
\end{center}
Let $q_1$ and $q_2$ be the quantities of homogeneous items produced by two firms with associated marginal costs $c_1$ and $c_2$ per item respectively.\\
Items sell at $P = a - b(q_1+q_2)$ each and it is assumed that all items produced are sold. \\ The profits made by the firms are then
$$ \pi_1 = P q_1 - c_1 q_1 = \left(a-c_1 - b(q_1+q_2)\right)q_1$$
$$ \pi_2 = P q_2 - c_2 q_2 = \left(a-c_2 - b(q_1+q_2)\right)q_2$$
respectively.\\
Maximising $\pi_1$ with respect to $q_1$
\begin{eqnarray}
 \frac{\partial \pi_1} {\partial q_1} &=& a-c_1 - b(q_1+q_2) - bq_1 \nonumber \\
 & \stackrel{set}{=} & 0 \nonumber \\
 \Rightarrow q_1 &=& \frac{a-c_1}{2b} - \frac{1}{2} q_2 \label{r1}
 \end{eqnarray}
 Similarly maximising $\pi_2$ with respect to $q_2$ yields
 \begin{equation} q_2 = \frac{a-c_2}{2b} - \frac{1}{2} q_1 \label{r2} \end{equation}
 Equations \ref{r1} and \ref{r2} are referred to as \textit{Reaction Functions} -  provided their solutions are nonnegative, which I'll assume in the following.\\
 Solving equations \ref{r1} and \ref{r2} simultaneously gives the \textit{equilibrium} values
 $$ q_1^* = \frac{a - 2c_1 + c_2}{3b}, \hspace{10mm} q_2^* = \frac{a - 2c_2 + c_1}{3b} $$
 At these equilibrium values
 $$ P^* = \frac{a+c_1+c_2}{3} $$
 and
 \begin{equation} \pi_1^* = \frac{(a-2c_1+ c_2)^2}{9b}, \hspace{10mm} \pi_2^* = \frac{(a-2c_2+ c_1)^2}{9b}\label{cp} \end{equation}
 \textit{Cournot} duopoly is an example of a 2-player matrix form game with an infinite number of strategies available to both players (firms), i.e. the choice of $q_1$ and $q_2$ respectively. $\langle q_1^*, q_2^* \rangle $ is then a \textit{Nash} equilibrium with payoffs $ \pi_1^*$ and $\pi_2^*$ respectively.\\

 \begin{center}
\textbf{Stackelberg Duopoly %(  au Spaniel)
}
\end{center}
\textit{Stackelberg} duopoly is an example of a 2-player extensive form game in which Firm 1 moves first (the ``Leader'') and Firm 2 responds (the ``Follower''). Irrespective of what the leader does, the follower will use the reaction function (Eq. \ref{r2}) as it is its best response.
\newpage
Knowing this, the leader seeks to maximise $$ \Pi_1 = \left(a-c_1 - b\left(q_1 + \frac{a-c_2}{2b} - \frac{1}{2} q_1\right)\right) q_1 = \left(a-c_1 - b\left(\frac{q_1}{2} + \frac{a-c_2}{2b} \right)\right) q_1$$
as a function of $q_1$.
\begin{eqnarray}
 \frac{\partial \Pi_1} {\partial q_1} &=& a-c_1 - b\left(\frac{q_1}{2}+ \frac{a-c_2}{2b}\right) - b\frac{q_1}{2} \nonumber \\
 & \stackrel{set}{=} & 0 \nonumber \\
 \Rightarrow q_1 &=& \frac{a-2c_1 + c_2}{2b} \label{r3}
 \end{eqnarray}
 Denoting this optimal value by $Q_1^*$ and the corresponding value of $q_2$ by $Q_2^*$ (substitute Eq. \ref{r3} into Eq. \ref{r2}) gives
$$ Q_1^* = \frac{a-2c_1 + c_2}{2b}, \hspace{10mm} Q_2^* = \frac{a+2c_1-3c_2}{4b} $$
At these equilibrium values
$$ P^* = \frac{a+2c_1+c_2}{4} $$
 and
 \begin{equation} \Pi_1^* = \frac{(a-2c_1+ c_2)^2}{8b}, \hspace{10mm} \Pi_2^* = \frac{(a+2c_1- 3c_2)^2}{16b}\label{sp} \end{equation}

 \begin{center}
\textbf{Comparing Cournot \& Stackelberg Duopoly Games %(  au Spaniel)
}
\end{center}
From Eqs \ref{cp} and \ref{sp},
$$ \pi_1^* < \Pi_1^* $$
for all parameter values, but it can be shown that
$$ \pi_2^* > \Pi_2^* $$
whenever $ a-2c_1 + c_2 > 0 $. This corresponds to $Q_1^* > 0$. \\ Exercise: prove the second assertion.
\end{document}

\begin{figure}[h]
\centering
%\begin{minipage}[l]{80mm}
{\includegraphics[height=80mm,width=90mm]{Log-OGY.jpg}}
%\end{minipage}
%\qquad
%\begin{minipage}[l]{80mm}
%{\includegraphics[height=85mm,width=90mm]{rec-fb3.pdf}}
%\end{minipage}
\caption{The trajectory of the Logistic Map with OGY control starting from $x_0 = 0.43$} \label{fig:trajOGY}
\vspace{5mm}
\end{figure}
