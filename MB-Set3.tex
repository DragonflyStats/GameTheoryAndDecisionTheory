\subsection{Cournot Duopoly}
Cournot duopoly, also called Cournot competition, is a model of imperfect competition in which two firms with identical cost functions compete with homogeneous products in a static setting. It was developed by Antoine A. Cournot in his “Researches Into the Mathematical principles of the Theory of Wealth”, 1838.

\subsection{Calculating the equilibrium}
In very general terms, let the price function for the (duopoly) industry be {\displaystyle P(q_{1}+q_{2})} P(q_{1}+q_{2}) and firm {\displaystyle i} i have the cost structure {\displaystyle C_{i}(q_{i})} C_{i}(q_{i}). To calculate the Nash equilibrium, the best response functions of the firms must first be calculated.

The profit of firm i is revenue minus cost. Revenue is the product of price and quantity and cost is given by the firm's cost function, so profit is (as described above): {\displaystyle \Pi _{i}=P(q_{1}+q_{2})\cdot q_{i}-C_{i}(q_{i})} \Pi _{i}=P(q_{1}+q_{2})\cdot q_{i}-C_{i}(q_{i}). The best response is to find the value of {\displaystyle q_{i}} q_{i} that maximises {\displaystyle \Pi _{i}} \Pi _{i} given {\displaystyle q_{j}} q_{j}, with {\displaystyle i\neq j} i\neq j, i.e. given some output of the opponent firm, the output that maximises profit is found. Hence, the maximum of {\displaystyle \Pi _{i}} \Pi _{i} with respect to {\displaystyle q_{i}} q_{i} is to be found. First take the derivative of {\displaystyle \Pi _{i}} \Pi _{i} with respect to {\displaystyle q_{i}} q_{i}:

\[{\displaystyle {\frac {\partial \Pi _{i}}{\partial q_{i}}}={\frac {\partial P(q_{1}+q_{2})}{\partial q_{i}}}\cdot q_{i}+P(q_{1}+q_{2})-{\frac {\partial C_{i}(q_{i})}{\partial q_{i}}}} {\frac  {\partial \Pi _{i}}{\partial q_{i}}}={\frac  {\partial P(q_{1}+q_{2})}{\partial q_{i}}}\cdot q_{i}+P(q_{1}+q_{2})-{\frac  {\partial C_{i}(q_{i})}{\partial q_{i}}}\]
Setting this to zero for maximization:

\[{\displaystyle {\frac {\partial \Pi _{i}}{\partial q_{i}}}={\frac {\partial P(q_{1}+q_{2})}{\partial q_{i}}}\cdot q_{i}+P(q_{1}+q_{2})-{\frac {\partial C_{i}(q_{i})}{\partial q_{i}}}=0} {\frac  {\partial \Pi _{i}}{\partial q_{i}}}={\frac  {\partial P(q_{1}+q_{2})}{\partial q_{i}}}\cdot q_{i}+P(q_{1}+q_{2})-{\frac  {\partial C_{i}(q_{i})}{\partial q_{i}}}=0\]

The values of {\displaystyle q_{i}} q_{i} that satisfy this equation are the best responses. The Nash equilibria are where both {\displaystyle q_{1}} q_{1} and {\displaystyle q_{2}} q_{2} are best responses given those values of {\displaystyle q_{1}} q_{1} and {\displaystyle q_{2}} q_{2}.
%=======================================%

\subsection{Cournot Duopoly}
Cournot duopoly, also called Cournot competition, is a model of imperfect competition in which two firms with identical cost functions compete with homogeneous products in a static setting. It was developed by Antoine A. Cournot in his “Researches Into the Mathematical principles of the Theory of Wealth”, 1838.

\subsection{Calculating the equilibrium}
In very general terms, let the price function for the (duopoly) industry be {\displaystyle P(q_{1}+q_{2})} P(q_{1}+q_{2}) and firm {\displaystyle i} i have the cost structure {\displaystyle C_{i}(q_{i})} C_{i}(q_{i}). To calculate the Nash equilibrium, the best response functions of the firms must first be calculated.

The profit of firm i is revenue minus cost. Revenue is the product of price and quantity and cost is given by the firm's cost function, so profit is (as described above): {\displaystyle \Pi _{i}=P(q_{1}+q_{2})\cdot q_{i}-C_{i}(q_{i})} \Pi _{i}=P(q_{1}+q_{2})\cdot q_{i}-C_{i}(q_{i}). The best response is to find the value of {\displaystyle q_{i}} q_{i} that maximises {\displaystyle \Pi _{i}} \Pi _{i} given {\displaystyle q_{j}} q_{j}, with {\displaystyle i\neq j} i\neq j, i.e. given some output of the opponent firm, the output that maximises profit is found. Hence, the maximum of {\displaystyle \Pi _{i}} \Pi _{i} with respect to {\displaystyle q_{i}} q_{i} is to be found. First take the derivative of {\displaystyle \Pi _{i}} \Pi _{i} with respect to {\displaystyle q_{i}} q_{i}:

\[{\displaystyle {\frac {\partial \Pi _{i}}{\partial q_{i}}}={\frac {\partial P(q_{1}+q_{2})}{\partial q_{i}}}\cdot q_{i}+P(q_{1}+q_{2})-{\frac {\partial C_{i}(q_{i})}{\partial q_{i}}}} {\frac  {\partial \Pi _{i}}{\partial q_{i}}}={\frac  {\partial P(q_{1}+q_{2})}{\partial q_{i}}}\cdot q_{i}+P(q_{1}+q_{2})-{\frac  {\partial C_{i}(q_{i})}{\partial q_{i}}}\]
Setting this to zero for maximization:

\[{\displaystyle {\frac {\partial \Pi _{i}}{\partial q_{i}}}={\frac {\partial P(q_{1}+q_{2})}{\partial q_{i}}}\cdot q_{i}+P(q_{1}+q_{2})-{\frac {\partial C_{i}(q_{i})}{\partial q_{i}}}=0} {\frac  {\partial \Pi _{i}}{\partial q_{i}}}={\frac  {\partial P(q_{1}+q_{2})}{\partial q_{i}}}\cdot q_{i}+P(q_{1}+q_{2})-{\frac  {\partial C_{i}(q_{i})}{\partial q_{i}}}=0\]

The values of {\displaystyle q_{i}} q_{i} that satisfy this equation are the best responses. The Nash equilibria are where both {\displaystyle q_{1}} q_{1} and {\displaystyle q_{2}} q_{2} are best responses given those values of {\displaystyle q_{1}} q_{1} and {\displaystyle q_{2}} q_{2}.
%=======================================%

\subsection{Cournot Duopoly with Linear Demand and Linear Costs}
Let q1 and q2 be the quantities of homogeneous items produced by two firms with associated
marginal costs c1 and c2 per item respectively.
Items sell at P = a − b(q1 + q2) each and it is assumed that all items produced are sold.
The profits made by the firms are then
\begin{itemize}
\item π1 = P q1 − c1q1 = (a − c1 − b(q1 + q2)) q1
\item π2 = P q2 − c2q2 = (a − c2 − b(q1 + q2)) q2
\end{itemize}
respectively.


Maximising π1 with respect to q1

∂π1
∂q1
= a − c1 − b(q1 + q2) − bq1
set = 0
⇒ q1 =
a − c1
2b
−
1
2
q2 (1)
Similarly maximising π2 with respect to q2 yields
q2 =
a − c2
2b
−
1
2
q1 (2)
Equations 1 and 2 are referred to as Reaction Functions - provided their solutions are
nonnegative, which I’ll assume in the following.

%====================================%
Solving equations 1 and 2 simultaneously gives the equilibrium values
q
∗
1 =
a − 2c1 + c2
3b
, q∗
2 =
a − 2c2 + c1
3b

At these equilibrium values
P
∗ =
a + c1 + c2
3
and
π
∗
1 =
(a − 2c1 + c2)
2
9b
, π∗
2 =
(a − 2c2 + c1)
2
9b
(3)
Cournot duopoly is an example of a 2-player matrix form game with an infinite number
of strategies available to both players (firms), i.e. the choice of q1 and q2 respectively.
hq
∗
1
, q∗
2
i is then a Nash equilibrium with payoffs π
∗
1
and π
∗
2
respectively.




\newpage


%=================================================================================================%

\subsection{Cournot Duopoly with Linear Demand and Linear Costs}
Let q_1 and q_2 be the quantities of homogeneous items produced by two firms with associated
marginal costs c1 and c2 per item respectively.
Items sell at P = a − b(q_1 + q_2) each and it is assumed that all items produced are sold.
The profits made by the firms are then
\begin{itemize}
\item \pi_1 = P q_1 − c1q_1 = (a − c1 − b(q_1 + q_2)) q_1
\item \pi_2 = P q_2 − c2q_2 = (a − c2 − b(q_1 + q_2)) q_2
\end{itemize}
respectively.


Maximising \pi_1 with respect to q_1 \partial\pi_1
∂q_1
= a − c1 − b(q_1 + q_2) − bq_1
set = 0
⇒ q_1 =
a − c1
2b
−
1
2
q_2 (1)
Similarly maximising \pi_2 with respect to q_2 yields
q_2 =
a − c2
2b
−
1
2
q_1 (2)
Equations 1 and 2 are referred to as Reaction Functions - provided their solutions are
nonnegative, which I’ll assume in the following.

%====================================%
Solving equations 1 and 2 simultaneously gives the equilibrium values
q
∗
1 =
a − 2c1 + c2
3b
, q∗
2 =
a − 2c2 + c1
3b

At these equilibrium values
P
∗ =
a + c1 + c2
3
and
\pi_
∗
1 =
(a − 2c1 + c2)
2
9b
, \pi_∗
2 =
(a − 2c2 + c1)
2
9b
(3)
Cournot duopoly is an example of a 2-player matrix form game with an infinite number
of strategies available to both players (firms), i.e. the choice of q_1 and q_2 respectively.
hq
∗
1
, q∗
2
i is then a Nash equilibrium with payoffs \pi_
∗
1
and \pi_
∗
2
respectively.
%=======================================================================%
\section{Stackelberg Duopoly}
Stackelberg duopoly is an example of a 2-player extensive form game in which Firm 1
moves first (the “Leader”) and Firm 2 responds (the “Follower”). Irrespective of what
the leader does, the follower will use the reaction function (Eq. 2) as it is its best response.
Knowing this, the leader seeks to maximise
\pi_1 =

a − c1 − b

q_1 +
a − c2
2b
−
1
2
q_1
 q_1 =

a − c1 − b

q_1
2
+
a − c2
2b
 q_1
as a function of q_1.
∂\pi_1
∂q_1
= a − c1 − b

q_1
2
+
a − c2
2b

− b
q_1
2
set = 0
⇒ q_1 =
a − 2c1 + c2
2b
(4)
Denoting this optimal value by Q∗
1
and the corresponding value of q_2 by Q∗
2
(substitute
Eq. 4 into Eq. 2) gives
Q
∗
1 =
a − 2c1 + c2
2b
, Q∗
2 =
a + 2c1 − 3c2
4b
At these equilibrium values
P
∗ =
a + 2c1 + c2
4
and
\pi_
∗
1 =
(a − 2c1 + c2)
2
8b
, \pi_
∗
2 =
(a + 2c1 − 3c2)
2
16b
(5)
Comparing Cournot & Stackelberg Duopoly Games
From Eqs 3 and 5,
\pi_
∗
1 < \pi_
∗
1
for all parameter values, but it can be shown that
\pi_
∗
2 > \pi_
∗
2
whenever a − 2c1 + c2 > 0. This corresponds to Q∗
1 > 0.
Exercise: prove the second assertion.

\newpage

%=======================================================================%
\section{Stackelberg Duopoly}
Stackelberg duopoly is an example of a 2-player extensive form game in which Firm 1
moves first (the “Leader”) and Firm 2 responds (the “Follower”). Irrespective of what
the leader does, the follower will use the reaction function (Eq. 2) as it is its best response.
Knowing this, the leader seeks to maximise
Π1 =

a − c1 − b

q1 +
a − c2
2b
−
1
2
q1
 q1 =

a − c1 − b

q1
2
+
a − c2
2b
 q1
as a function of q1.
∂Π1
∂q1
= a − c1 − b

q1
2
+
a − c2
2b

− b
q1
2
set = 0
⇒ q1 =
a − 2c1 + c2
2b
(4)
Denoting this optimal value by Q∗
1
and the corresponding value of q2 by Q∗
2
(substitute
Eq. 4 into Eq. 2) gives
Q
∗
1 =
a − 2c1 + c2
2b
, Q∗
2 =
a + 2c1 − 3c2
4b
At these equilibrium values
P
∗ =
a + 2c1 + c2
4
and
Π
∗
1 =
(a − 2c1 + c2)
2
8b
, Π
∗
2 =
(a + 2c1 − 3c2)
2
16b
(5)
Comparing Cournot & Stackelberg Duopoly Games
From Eqs 3 and 5,
π
∗
1 < Π
∗
1
for all parameter values, but it can be shown that
π
∗
2 > Π
∗
2
whenever a − 2c1 + c2 > 0. This corresponds to Q∗
1 > 0.
Exercise: prove the second assertion.
