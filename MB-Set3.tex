\subsection{Cournot Duopoly with Linear Demand and Linear Costs}
Let q1 and q2 be the quantities of homogeneous items produced by two firms with associated
marginal costs c1 and c2 per item respectively.
Items sell at P = a − b(q1 + q2) each and it is assumed that all items produced are sold.
The profits made by the firms are then
\begin{itemize}
\item π1 = P q1 − c1q1 = (a − c1 − b(q1 + q2)) q1
\item π2 = P q2 − c2q2 = (a − c2 − b(q1 + q2)) q2
\end{itemize}
respectively.


Maximising π1 with respect to q1

∂π1
∂q1
= a − c1 − b(q1 + q2) − bq1
set = 0
⇒ q1 =
a − c1
2b
−
1
2
q2 (1)
Similarly maximising π2 with respect to q2 yields
q2 =
a − c2
2b
−
1
2
q1 (2)
Equations 1 and 2 are referred to as Reaction Functions - provided their solutions are
nonnegative, which I’ll assume in the following.

%====================================%
Solving equations 1 and 2 simultaneously gives the equilibrium values
q
∗
1 =
a − 2c1 + c2
3b
, q∗
2 =
a − 2c2 + c1
3b

At these equilibrium values
P
∗ =
a + c1 + c2
3
and
π
∗
1 =
(a − 2c1 + c2)
2
9b
, π∗
2 =
(a − 2c2 + c1)
2
9b
(3)
Cournot duopoly is an example of a 2-player matrix form game with an infinite number
of strategies available to both players (firms), i.e. the choice of q1 and q2 respectively.
hq
∗
1
, q∗
2
i is then a Nash equilibrium with payoffs π
∗
1
and π
∗
2
respectively.
%=======================================================================%
\section{Stackelberg Duopoly}
Stackelberg duopoly is an example of a 2-player extensive form game in which Firm 1
moves first (the “Leader”) and Firm 2 responds (the “Follower”). Irrespective of what
the leader does, the follower will use the reaction function (Eq. 2) as it is its best response.
Knowing this, the leader seeks to maximise
Π1 =

a − c1 − b

q1 +
a − c2
2b
−
1
2
q1
 q1 =

a − c1 − b

q1
2
+
a − c2
2b
 q1
as a function of q1.
∂Π1
∂q1
= a − c1 − b

q1
2
+
a − c2
2b

− b
q1
2
set = 0
⇒ q1 =
a − 2c1 + c2
2b
(4)
Denoting this optimal value by Q∗
1
and the corresponding value of q2 by Q∗
2
(substitute
Eq. 4 into Eq. 2) gives
Q
∗
1 =
a − 2c1 + c2
2b
, Q∗
2 =
a + 2c1 − 3c2
4b
At these equilibrium values
P
∗ =
a + 2c1 + c2
4
and
Π
∗
1 =
(a − 2c1 + c2)
2
8b
, Π
∗
2 =
(a + 2c1 − 3c2)
2
16b
(5)
Comparing Cournot & Stackelberg Duopoly Games
From Eqs 3 and 5,
π
∗
1 < Π
∗
1
for all parameter values, but it can be shown that
π
∗
2 > Π
∗
2
whenever a − 2c1 + c2 > 0. This corresponds to Q∗
1 > 0.
Exercise: prove the second assertion.
