Introduction to Game Theory/Extensive games
< Introduction to Game Theory
This page may need to be reviewed for quality.

Jump to navigation
Jump to search
<<Back to Introduction to Game Theory 
To model games in which the players take turns the extensive form can be used. 

We can define an extensive game as a tuple with four elements (P, S, F, U) where 
P is a set of players.
S is a set of sequences (terminal histories) with the property that no sequence is the proper subhistory of any other sequence
F is a function (player function) that assigns a player to every move in a terminal history.
U is a set of preferences over the set of terminal histories for each player.

Example (The Entry game) An incumbent faces the possibility of entry by a challenger. This can be an animal competing for the right to mate, a firm entering a new market currently controled by a monopoly, or a politician competing for party backing. The challenger may choose enter or not. If he enters, the incumbent may either acquiesce or fight. For the challenger, he prefers to enter and have the incombent back down, and doesn't like to get into a fight. The incumbent prefers the challenger to stay out, and also doesn't like to fight. 
